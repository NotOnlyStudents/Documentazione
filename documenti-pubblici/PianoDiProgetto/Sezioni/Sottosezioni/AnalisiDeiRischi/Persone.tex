\subsection{Rischi legati ai membri del gruppo}


    \begin{table}
        \begin{tabular}{|c|p{10cm}|}
        \hline
        \multicolumn{2}{|c|}{\textbf{RG1 - Contrasti tra i Componenti}} \\
        \hline
         Descrizione & Il gruppo deve cooperare con professionalità nonostante la poca esperienza, questo può causare tensioni o contrasti.\\ 
         \hline
         Conseguenze & Tensioni e contrasti rallentano e danneggio il corretto svolgimento del progetto\\
         \hline
         Probabilità di Occorrenza & Bassa.\\
         \hline
         Pericolosità & Alta.\\
         \hline
         Precauzioni & Ogni elemento del gruppo cercherà di limitare eventuali tensioni a favore del collettivo.\\
         \hline
         Piano di Contingenza & Il \Responsabile di Progetto riassegnerà i compiti per limitarela vicinanza delle parti interessate, eventualmente insieme al resto del team cercherà di sanare le discordie.  In casi estremi verrà chiamato in causa il Prof. Tullio Vardanega\\ 
         \hline
        \end{tabular}
        \caption{\label{tab:RG1}Analisi dei rischi per contrasti tra i componenti.}
    \end{table}


    \begin{table}
        \begin{tabular}{|c|p{10cm}|}
        \hline
        \multicolumn{2}{|c|}{\textbf{RG2 - Disponibilità dei Membri}} \\
        \hline
         Descrizione & Ogni membro ha impegni universitari e personali oltre all’attività di progetto; possono inoltre insorgere problemi disalute e familiari che potrebbero rendere i membri inattiviin alcuni momenti.\\ 
         \hline
         Conseguenze & Possibilità di ritardi su attività individuali o collettive.\\
         \hline
         Probabilità di Occorrenza & Media.\\
         \hline
         Pericolosità & Medio-Alta.\\
         \hline
         Precauzioni & Ogni elemento del gruppo cercherà di limitare eventuali tensioni a favore del collettivo.\\
         \hline
         Piano di Contingenza & Ogni membro dovrà far presente al resto del gruppo eventuali momenti di inattività in modo da poter organizzare il lavoro\\ 
         \hline
        \end{tabular}
        \caption{\label{tab:RG2}Analisi dei rischi per disponibilità dei membri.}
    \end{table}


    \begin{table}
        \begin{tabular}{|c|p{10cm}|}
        \hline
        \multicolumn{2}{|c|}{\textbf{RG3 - Disponibilità dei Membri}} \\
        \hline
         Descrizione & I membri non hanno esperienza di lavoro che richieda ilcoordinamento di sette persone.\\ 
         \hline
         Conseguenze & Problematiche o ritardi possono derivare da una scarsa organizzazione.\\
         \hline
         Probabilità di Occorrenza & Media.\\
         \hline
         Pericolosità & Medio.\\
         \hline
         Precauzioni & Il \Responsabile, prima di ogni attività di gruppo, deve, insieme a tutti i membri, pianificare le mansioni; non è consentita l’iniziativa personale del singolo e ognimembro dovrà comunicare le proprie conoscenze pregresse al \Responsabile. Con la rotazione dei ruoli si cercherà di far collaborare tutti in modo che ogni attività venga svolta dai membri con più esperienza insieme a quelli che ancora non ne hanno.\\
         \hline
         Piano di Contingenza & Qualunque difficoltà sarà notificata al \Responsabile e affrontata con la collaborazione di tutti. Nella peggiore delle ipotesi il \Responsabile assegnerà un compito più semplice. Il membro coinvolto dovrà eseguire un’autoanalisi per comprendere le motivazioni dei suoi problemi e come migliorare.\\ 
         \hline
        \end{tabular}
        \caption{\label{tab:RG3}Analisi dei rischi per inesperienza nel coordinamento.}
    \end{table}

    \begin{table}
        \begin{tabular}{|c|p{10cm}|}
        \hline
        \multicolumn{2}{|c|}{\textbf{RG4 - Impegno Oneroso Prolungato}} \\
        \hline
         Descrizione & Nessuno dei membri ha mai fatto un’esperienza di lavoroche richiedesse un impegno così prolungato e costante neltempo.\\ 
         \hline
         Conseguenze & Problematiche o ritardi causati dall’inesperienza del gruppo nel mantenere un metodo di lavoro costante nel tempo e adeguato alla realizzazione del progetto.\\
         \hline
         Probabilità di Occorrenza & Media.\\
         \hline
         Pericolosità & Alta.\\
         \hline
         Precauzioni & Il \Responsabile dovrà sempre monitorare il lavoro di ciascun componente così da accertarsi che le mansioni vengano eseguite adeguatamente\\
         \hline
         Piano di Contingenza & Nel caso il \Responsabile rilevasse un approccio lavorativo errato di uno o più componenti del gruppo,provvederà prima ad un richiamo informale verso i soggetti coinvolti (a voce o con gli strumenti di comunicazione interni). Se il comportamento venisse reiterato il \Responsabile programmerà una riunione interna al gruppo per discutere la situazione.\\ 
         \hline
        \end{tabular}
        \caption{\label{tab:RG4}Analisi dei rischi per impegno oneroso prolungato.}
    \end{table}


    \begin{table}
        \begin{tabular}{|c|p{10cm}|}
        \hline
        \multicolumn{2}{|c|}{\textbf{RG5 - Scarsa Comunicazione}} \\
        \hline
         Descrizione & Ognuno dei membri del gruppo non ha molta esperienza nel team work, questo può portare a una cattiva comunicazione tra i membri del gruppo.\\ 
         \hline
         Conseguenze & Problematiche o ritardi dovuti a una cattiva organizzazione derivata da una comunicazione insufficiente.\\
         \hline
         Probabilità di Occorrenza & Bassa.\\
         \hline
         Pericolosità & Media.\\
         \hline
         Precauzioni & Il \Responsabile provvederà a monitorare e promuovere un adeguato livello di comunicazione attiva tratutto il team. Non dovranno insorgere situazioni dove permancata o scarsa comunicazione ci siano ritardi nelle decisioni progettuali importanti, con conseguente insorgenza disituazioni di tensione.\\
         \hline
         Piano di Contingenza & Nel caso si rilevi una scarsa comunicazione sarà compito del \Responsabile ricercarne il motivo, se questo sid imostra non lecito (e.g: senza altri impegni personali notificati precedentemente) provvederà prima ad un richiamo informale (a voce o con gli strumenti di comunicazione interni). Se il comportamento venisse reiterato il \Responsabile programmerà una riunione interna al gruppo per discutere la situazione.\\ 
         \hline
        \end{tabular}
        \caption{\label{tab:RG5}Analisi dei rischi per scarsa comunicazione.}
    \end{table}