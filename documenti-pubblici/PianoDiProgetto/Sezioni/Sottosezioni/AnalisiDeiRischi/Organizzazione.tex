\subsection{Rischi legati all'orgranizzazione}

\begin{table}[H]
    \begin{tabular}{|c | p{10cm}|}
    \hline
    \multicolumn{2}{|c|}{\textbf{RO1 - Costi delle Attività}} \\
    \hline
    Descrizione & Per ogni attività viene calcolato il costo, questo calcolo può essere non corretto (sottostimato o sovrastimato) a causa dell'inesperienza del gruppo\\ 
    \hline
    Conseguenze & Una sottostima provocherebbe ritardi nella pianificazione;una sovrastima porterebbe ad uno spreco di tempo. Entrambi i casi richiederebbero poi una nuova pinificazione delle attività\\
    \hline
    Probabilità di Occorrenza & Medio-Alta.\\
    \hline
    Pericolosità & Medio-Alta.\\
    \hline
    Precauzioni & Il gruppo cercherà di effettuare un backup il più frequentemente possibile in modo da minimizzare possibili danni.\\ 
    \hline
    Piano di Contingenza & Nel caso si rilevino dei malfunzionamenti da parte di serviziesterni ogni membro del gruppo lo comunicherà a tutto ilteam. Nel caso peggiore ilResponsabile di Progettosi im-pegnerà per passare ad altri software il più simili possibilea quelli utilizzati.\\ 
    \hline
    \end{tabular}
    \caption{\label{tab:RO1}Analisi dei rischi per i costi delle attività.}
    
\end{table}