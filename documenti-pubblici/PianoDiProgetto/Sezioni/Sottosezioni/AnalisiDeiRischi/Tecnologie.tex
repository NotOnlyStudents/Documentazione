\subsection{Rischi legati alle tecnologie}

\begin{table}
\begin{tabular}{|c | p{10cm}|}
\hline
\multicolumn{2}{|c|}{\textbf{RT1 - Inesperienza Tecnologica}} \\
\hline
 Descrizione & La maggior parte del gruppo ha conoscenze poco approfondite riguardo le tecnologie da utilizzare durante lo svolgimento del progetto e i tempi di apprendimento sono variabili da persona a persona.\\ 
 \hline
 Conseguenze & Difficoltà nel quantificare il tempo necessario di apprendimento, può causare ritardi e richiedere una revisione della pianificazione.\\
 \hline
 Probabilità di Occorrenza & Alta.\\
 \hline
 Pericolosità & Alta.\\
 \hline
 Precauzioni & Ogni membro comunicherà al \Responsabile il proprio livello di conoscenze così da rendere più equilibrata l'assegnazione dei compiti. In caso di difficoltà durante lo svolgimento ogni membro è tenuto a comunicarlo.\\
 \hline
 Piano di Contingenza & Ciascun componente del gruppo dovrà documentarsi auto-nomamente e grazie anche a materiale fornitogli dagli \Amministratore. L’assegnazione delle attività terràconto del grado di apprendimento raggiunto dai singoli; sicercherà di far collaborare fra loro i membri il più possibi-le in modo da sanare eventuali incomprensioni. I compitipiù complessi verranno inoltre distribuiti su più membri inmodo da favorire lo sviluppo.\\ 
 \hline
\end{tabular}
\caption{\label{tab:RT1}Analisi dei rischi per inesperienza tecnologica.}
\end{table}


\begin{table}
    \begin{tabular}{|c | p{10cm}|}
    \hline
    \multicolumn{2}{|c|}{\textbf{RT2 - Problemi Hardware}} \\
    \hline
     Descrizione & Ogni membro nel gruppo lavora attraverso diversi strumenti e può succedere che qualcuno di questi abbia dei malfunzionamenti.\\ 
     \hline
     Conseguenze & Nel caso di un guasto potrebbero perdersi dati locali non ancora versionati remotamente.\\
     \hline
     Probabilità di Occorrenza & Bassa.\\
     \hline
     Pericolosità & Bassa.\\
     \hline
     Precauzioni & Il gruppo cercherà di effettuare un backup il più frequentemente possibile in modo da minimizzare possibili danni.\\  
     \hline
     Piano di Contingenza & Nel caso di guasti o malfunzionamenti ogni membro del gruppo è tenuto ad avvisare i compagni in modo da poter minimizzare il danno causato al progetto.\\ 
     \hline
    \end{tabular}
    \caption{\label{tab:RT2}Analisi dei rischi per problemi hardware.}
    \end{table}


\begin{table}
    \begin{tabular}{|c | p{10cm}|}
    \hline
    \multicolumn{2}{|c|}{\textbf{RT3 - Problemi Software}} \\
    \hline
    Descrizione & Il gruppo fa utilizzo di software di terze parti, per cui il loromalfunzionamento non dipende dal team.\\ 
    \hline
    Conseguenze & Il loro malfunzionamento causerebbe pesanti perdite di datie ritardi.\\
    \hline
    Probabilità di Occorrenza & Bassa.\\
    \hline
    Pericolosità & Alta.\\
    \hline
    Precauzioni & Il gruppo cercherà di effettuare un backup il più frequentemente possibile in modo da minimizzare possibili danni.\\ 
    \hline
    Piano di Contingenza & Nel caso si rilevino dei malfunzionamenti da parte di serviziesterni ogni membro del gruppo lo comunicherà a tutto ilteam. Nel caso peggiore ilResponsabile di Progettosi im-pegnerà per passare ad altri software il più simili possibilea quelli utilizzati.\\ 
    \hline
    \end{tabular}
    \caption{\label{tab:RT3}Analisi dei rischi per problemi software.}
    
\end{table}