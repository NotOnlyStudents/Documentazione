\section{Introduzione}
\label{introduzione}
\subsection{Scopo del documento}
Questo documento ha lo scopo di mostrare le strategie che il gruppo {\Gruppo} adotta per garantire la \glo{qualità} di \glo{prodotto} e di processo. Per raggiungere questo obiettivo viene applicato un sistema di verifica continua che permette l'individuazione di eventuali errori così da poterli risolvere limitando gli sprechi di tempo. \\Il documento è da considerarsi incompleto in quanto i contenuti verranno aggiornati e modificati avanzando con lo sviluppo del prodotto.

\subsection{Scopo del Prodotto} 
Il progetto {\NomeProgetto} ha come scopo quello di rendere disponibile un servizio \glo{e-commerce} sfruttando tutti i vantaggi di un'architettura \glo{\textit{Serverless}}:
\begin{itemize}
	\item Gli sviluppatori potranno concentrare la propria attenzione sullo sviluppo del prodotto finale invece di focalizzarsi sulla gestione e sul funzionamento di server e di runtime, che siano nel \glo{cloud} o in locale.
	\item Comodità nel costruire un insieme di chiamate \glo{asincrone} che rispondono a diversi clienti contemporaneamente.
	\item Minori costi di sviluppo e di produzione.
	\item Semplicità nel suddividere il progetto in un insieme di \glo{microservizi}.
\end{itemize}

\subsection{Glossario}
Al fine di rendere il documento più chiaro e leggibile si fornisce un \textit{Glossario}. I termini che possono assumere un significato ambiguo sono indicati da una 'G' ad apice e fanno riferimento al documento \Glossariov{1.0.0}.

\subsection{Riferimenti}
\subsubsection{Normativi}
\begin{itemize}
	\item \textit{\NdPv{1.0.0}};
\end{itemize}

\subsubsection{Informativi} \label{riferimenti_info}
\begin{itemize}
	\item \textbf{Capitolato d'appalto C2:}\\
	\url{https://www.math.unipd.it/~tullio/IS-1/2020/Progetto/C2.pdf}
	\item \textbf{Qualità di prodotto:}\\ \url{https://www.math.unipd.it/~tullio/IS-1/2020/Dispense/L12.pdf}
	\item \textbf{Qualità di processo:}\\ \url{https://www.math.unipd.it/~tullio/IS-1/2020/Dispense/L13.pdf}
	\item \textbf{Verifica e validazione:}\\ \url{https://www.math.unipd.it/~tullio/IS-1/2020/Dispense/L14.pdf}
	\item \textbf{\glo{Standard ISO/IEC 15504} (\textit{SPICE}): }\\ \url{https://en.wikipedia.org/wiki/ISO/IEC_15504}
	\item \textbf{\glo{Ciclo di Deming} (\textit{PDCA}): } \\ \url{https://it.wikipedia.org/wiki/Ciclo_di_Deming}
\end{itemize}