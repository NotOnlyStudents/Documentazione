\section{Qualità di processo}
\label{qualità_processo}
Il gruppo {\Gruppo} ha deciso di adottare lo \glo{standard} ISO/IEC 15504, conosciuto anche con il termine \textbf{SPICE}, combinato con il miglioramento continuo. Queste scelte sono state prese per garantire lo sviluppo di un prodotto di \glo{qualità} entro i costi ed i tempi stabiliti nel \PdP{}.\\
Lo standard ISO/IEC 15504 assicura la qualità di tutti i \glo{processi} che compongono lo sviluppo del prodotto finale tramite una definizione chiara degli obiettivi di processo e delle soglie da rispettare. Per garantire invece il principio di miglioramento continuo nella qualità di processo, si è deciso di utilizzare il ciclo di Deming, noto anche come \textbf{PDCA}.\\
Per una descrizione più dettagliata dello standard ISO/IEC 15504 e del PDCA si rimanda rispettivamente alle appendici \S{A} e \S{B} presenti nelle \NdPv{1.0.0}.
\subsection{Processi primari}
\subsubsection{Processo di sviluppo}
\myparagraph{Obiettivi dell'analisi dei requisiti}
\begin{itemize}
	\item \textbf{OPR01: Comprensione requisiti}\\
	L'obiettivo dell'\AdR{} è di trasformare i requisiti dati dallo \glo{stakeholder} in requisiti tecnici che indirizzino il team allo sviluppo del prodotto software.
	\item \textbf{Esiti del processo}\\
	I risultati di un'analisi dei requisiti di qualità sono:
	\begin{itemize}
		\item Definizione di una serie di requisiti che descrivano il problema;
		\item Classificazione dei requisiti in modo da poterne tracciare i cambiamenti;
		\item Realizzazione dei diagrammi dei casi d'uso per iniziare la progettazione;
		\item Valutazione delle richieste degli stakeholders per negoziare modifiche, se necessarie.
	\end{itemize}
\end{itemize}
\myparagraph{Metriche}
Risulta indispensabile definire delle \glo{metriche} in modo che la qualità dei risultati raggiunti sia quantificabile.
\begin{table} [H]
	\begin{center}
		\begin{tabular}{|c| p{12cm}|}
			\rowcolor{darkblue}
			\multicolumn{2}{|c|}{\textcolor{white}{\textbf{MPR01: Soddisfacimento Requisiti Obbligatori}}} \\ \hline
			Descrizione & La metrica che individua i requisiti obbligatori soddisfatti è definita nelle \NdPv{1.0.0} alla sezione \S{D1.7}.\\ \hline
			Risultato & Percentuale di requisiti obbligatori soddisfatti al momento di calcolo nell'indice.\\ \hline
			Valore ottimale & 100\%.\\ \hline
			Valore minimale & 100\%.\\ \hline
		\end{tabular}
	\end{center}
	\caption{\label{tab:MPR01}Metrica relativa al soddisfacimento dei requisiti obbligatori del progetto.}
\end{table}
\subsection{Processi di supporto}
\subsubsection{Processo di documentazione}
\myparagraph{Obiettivi}
\begin{itemize}
	\item \textbf{OPR02: Redazione della documentazione}\\
	Lo scopo del processo di documentazione è di tracciare e mantenere registrate informazioni e prodotti relativi ai processi.
	\item \textbf{Esiti del processo}\\
	Il processo di documentazione di qualità implica la produzione di documentazione:
	\begin{itemize}
		\item Completa;
		\item Leggibile;
		\item Corretta;
		\item Coerente.
	\end{itemize}
\end{itemize}
\myparagraph{Metriche}
\begin{table} [H]
	\begin{center}
		\begin{tabular}{|c| p{12cm}|}
			\rowcolor{darkblue}
			\multicolumn{2}{|c|}{\textcolor{white}{\textbf{MPR02: Indice di Gulpease}}}\\ \hline
			Descrizione & L'indice di Gulpease è definito nelle \NdPv{1.0.0} alla sezione \S{D1.8}.\\ \hline
			Risultato & Indice compreso tra 0 e 100 che stabilisce la leggibilità del testo in base alla lunghezza delle parole.\\ \hline
			Valore ottimale & Compreso tra 80 e 100.\\ \hline
			Valore minimale & Non inferiore a 50.\\ \hline
		\end{tabular}
	\end{center}
	\caption{\label{tab:MPR02}Metrica relativa all'indice di Gulpease.}
\end{table}
\begin{table} [H]
	\begin{center}
		\begin{tabular}{|c| p{12cm}|}
			\rowcolor{darkblue}
			\multicolumn{2}{|c|}{\textcolor{white}{\textbf{MPR03: Indice di correttezza ortografica}}}\\ \hline
			Descrizione & L'indice di correttezza ortografica è definito nelle \NdPv{1.0.0} alla sezione \S{D1.9}.\\ \hline
			Risultato & Numero intero che rappresenta il numero di errori ortografici presenti nel testo.\\ \hline
			Valore ottimale & 0.\\ \hline
			Valore minimale & 0.\\ \hline
		\end{tabular}
	\end{center}
	\caption{\label{tab:MPR03}Metrica relativa all'indice di correttezza ortografica.}
\end{table}
\subsection{Processi organizzativi}
\subsubsection{Processo di gestione}
\myparagraph{Obiettivi}
\begin{itemize}
	\item \textbf{OPR03: Attività principali}\\
	L'obiettivo del processo di pianificazione è di stabilire e organizzare le principali \glo{attività} e compiti di progetto.
	\item \textbf{Esiti del processo}\\
	I risultati di un'attività di pianificazione di qualità sono:
	\begin{itemize}
		\item Individuazione dello scopo del progetto;
		\item Identificazione delle risorse umane e temporali a disposizione e pianificazione delle attività da svolgere;
		\item Definizione di un preventivo e di un consuntivo;
		\item Stesura di un \PdP{}.
	\end{itemize}
\end{itemize}
\myparagraph{Metriche}
\begin{table} [H]
	\begin{center}
		\begin{tabular}{|c| p{12cm}|}
			\rowcolor{darkblue}
			\multicolumn{2}{|c|}{\textcolor{white}{\textbf{MPR04: Budget at Completion}}}\\ \hline
			Descrizione & L'indice di varianza del costo al completamento del progetto è definito nelle \NdPv{1.0.0} alla sezione \S{D1.1}.\\ \hline
			Risultato & Numero intero calcolato alla fine di ogni \glo{fase}.\\ \hline
			Valore ottimale & Corrispondente al preventivo.\\ \hline
			Valore minimale & Totale del preventivo $\pm$ 10\%.\\ \hline
		\end{tabular}
	\end{center}
	\caption{\label{tab:MPR04}Metrica relativa al costo al completamento del progetto.}
\end{table}
\begin{table} [H]
	\begin{center}
		\begin{tabular}{|c| p{12cm}|}
			\rowcolor{darkblue}
			\multicolumn{2}{|c|}{\textcolor{white}{\textbf{MPR05: Actual Cost of Work Performed}}}\\ \hline
			Descrizione & L'indice della somma delle spese sostenute dal gruppo è definito nelle \NdPv{1.0.0} alla sezione \S{D1.1}.\\ \hline
			Risultato & Numero con virgola calcolato ad intervalli regolari.\\ \hline
			Valore ottimale & Corrispondente al preventivo.\\ \hline
			Valore minimale & Corrispondente al preventivo.\\ \hline
		\end{tabular}
	\end{center}
	\caption{\label{tab:MPR05}Metrica relativa al costo attuale del lavoro svolto.}
\end{table}
\begin{table} [H]
	\begin{center}
		\begin{tabular}{|c| p{12cm}|}
			\rowcolor{darkblue}
			\multicolumn{2}{|c|}{\textcolor{white}{\textbf{MPR06: Budget Cost of Work Performed}}}\\ \hline
			Descrizione & Il costo sostenuto per il lavoro svolto è definito nelle \NdPv{1.0.0} alla sezione \S{D1.1}.\\ \hline
			Risultato & Numero con virgola calcolato ad intervalli regolari.\\ \hline
			Valore ottimale & Corrispondente al preventivo.\\ \hline
			Valore minimale & Corrispondente al preventivo.\\ \hline
		\end{tabular}
	\end{center}
	\caption{\label{tab:MPR06}Metrica relativa al costo sostenuto per il lavoro svolto.}
\end{table}
\begin{table} [H]
	\begin{center}
		\begin{tabular}{|c| p{12cm}|}
			\rowcolor{darkblue}
			\multicolumn{2}{|c|}{\textcolor{white}{\textbf{MPR07: Budget Cost of Work Scheduled}}}\\ \hline
			Descrizione & L'indice del costo corrispondente alla spesa pianificata per il progetto è definito nelle \NdPv{1.0.0} alla sezione \S{D1.1}.\\ \hline
			Risultato & Numero con virgola calcolato ad intervalli regolari.\\ \hline
			Valore ottimale & Corrispondente al preventivo.\\ \hline
			Valore minimale & Corrispondente al preventivo.\\ \hline
		\end{tabular}
	\end{center}
	\caption{\label{tab:MPR07}Metrica relativa al costo pianificato per il lavoro svolto.}
\end{table}
\begin{table} [H]
	\begin{center}
		\begin{tabular}{|c| p{12cm}|}
			\rowcolor{darkblue}
			\multicolumn{2}{|c|}{\textcolor{white}{\textbf{MPR08: Schedule Variance}}}\\ \hline
			Descrizione & L'indice di varianza rispetto allo schedule è definito nelle \NdPv{1.0.0} alla sezione \S{D1.5}.\\ \hline
			Risultato & Numero che indica se si è in anticipo o in ritardo nella schedulazione rispetto a quanto pianificato.\\ \hline
			Valore ottimale & 0 giorni.\\ \hline
			Valore minimale & $<$ 6 giorni.\\ \hline
		\end{tabular}
	\end{center}
	\caption{\label{tab:MPR08}Metrica relativa allo scostamento rispetto allo schedule.}
\end{table}
\begin{table} [H]
	\begin{center}
		\begin{tabular}{|c| p{12cm}|}
			\rowcolor{darkblue}
			\multicolumn{2}{|c|}{\textcolor{white}{\textbf{MPR09: Budget Variance}}}\\ \hline
			Descrizione & L'indice di varianza dei costi è definito nelle \NdPv{1.0.0} alla sezione \S{D1.6}.\\ \hline
			Risultato & Numero intero che rappresenta l'efficienza e la produttività.\\ \hline
			Valore ottimale & 0\%.\\ \hline
			Valore minimale & $<$ 10\%.\\ \hline
		\end{tabular}
	\end{center}
	\caption{\label{tab:MPR09}Metrica relativa ai costi del progetto.}
\end{table}
\subsubsection{Processo di miglioramento continuo}
\myparagraph{Obiettivi}
\begin{itemize}
	\item \textbf{OPR04: Risoluzione dei problemi}\\
	Lo scopo del processo di risoluzione dei problemi è di tracciare delle problematiche riscontrate durante lo svolgimento del progetto e di offrire soluzioni.
	\item \textbf{Esiti del processo}\\
	I risultati di un processo di risoluzione dei problemi di qualità sono:
	\begin{itemize}
		\item L'identificazione e l'analisi di problemi riscontrati durante lo svolgimento del progetto;
		\item La definizione di strategie e procedure per risolverli.
	\end{itemize}
\end{itemize}
\myparagraph{Metriche}
\begin{table} [H]
	\begin{center}
		\begin{tabular}{|c| p{12cm}|}
			\rowcolor{darkblue}
			\multicolumn{2}{|c|}{\textcolor{white}{\textbf{MPR10: Indice di risoluzione dei problemi}}}\\ \hline
			Descrizione & L'indice di velocità di risoluzione dei problemi è definito nelle \NdPv{1.0.0} alla sezione \S{D1.10}.\\ \hline
			Risultato & Numero intero che rappresenta la quantità di giorni trascorsi tra l'apertura della \glo{issue} e la sua chiusura.\\ \hline
			Valore ottimale & 1.\\ \hline
			Valore minimale & 8.\\ \hline
		\end{tabular}
	\end{center}
	\caption{\label{tab:MPR10}Metrica relativa all'indice di risoluzione dei problemi.}
\end{table}