\appendix
\section{Valutazioni per il miglioramento}
\label{miglioramento}
Di seguito viene riportata la valutazione fatta dal gruppo {\Gruppo} riguardo il lavoro svolto durante l'\glo{attività} appena conclusa. Lo scopo di questa scelta è far emergere tutte le problematiche sorte fino ad ora e poter procedere ad una loro risoluzione \glo{efficiente} così da evitare che si verifichino nuovamente in futuro.
I problemi analizzati riguardano:
\begin{itemize}
	\item \textbf{Organizzazione:} criticità relative all'organizzazione e alla comunicazione interna del gruppo;
	\item \textbf{Ruoli:} criticità riguardanti il corretto svolgimento di un ruolo;
	\item \textbf{Strumenti:} criticità relative all'uso degli strumenti scelti.
\end{itemize}
Per trattare ogni punto sopra descritto, è fondamentale l'autovalutazione di ciascun membro del gruppo, non essendo presente una figura esterna che possa valutare in modo oggettivo le criticità. Questo metodo ha permesso al gruppo di migliorare progressivamente la \glo{qualità} del lavoro. La sezione è al momento incompleta, verrà aggiornata con l'avanzamento del lavoro riportando nuove problematiche, ogni qual volta esse dovessero verificarsi. Per fornire valutazioni facilmente leggibili e consultabili esse sono organizzate mediante tabelle la cui struttura è stabilita in \NdPv{1.0.0}.
\subsection{Revisione dei requisiti}
\subsubsection{Valutazioni sull'organizzazione}
\begin{itemize}
	\item \textbf{Problema:} la difficoltà maggiore è stata quella di entrare nell'ottica del progetto e del lavoro di gruppo abituandosi ai cambi di ruolo e ai compiti da svolgere coordinandosi con gli altri membri del gruppo.\\
	\textbf{Soluzione proposta:} i membri del gruppo si sono incontrati e si è organizzato il lavoro dopo aver studiato il \glo{capitolato} e i documenti indicati nei riferimenti normativi. Ciascun componente del team ha avuto modo di ricoprire ogni ruolo attivo fino ad ora, perciò risulterà meno impegnativo in futuro rispettare la rotazione dei ruoli imposta nel \PdPv{1.0.0}.
	\item \textbf{Problema:} le \glo{issues} create richiedevano dei compiti che si sovrapponevano tra loro, rischiando di effettuare più volte un lavoro inutilmente.\\
	\textbf{Soluzione proposta:} ogni membro si impegnerà a creare issue più precise e circoscritte assegnandola a sè stesso e, nel caso fosse necessario, al membro del gruppo che lo aiuta nel lavoro.
	\item \textbf{Problema:} visti i numerosi impegni di ogni componente, durante le prime settimane si è notata una certa difficoltà nell'organizzare gli incontri in modo che fossero presenti tutti.\\
	\textbf{Soluzione proposta:} si è deciso di utilizzare un calendario condiviso per evidenziare gli impegni di ciascun membro del team. Alla fine di ogni incontro, si decide la data della possibile riunione successiva.
\end{itemize}
\subsubsection{Valutazioni sui ruoli}
\myparagraph{Responsabile di progetto}
\begin{itemize}
	\item \textbf{Problema:} la principale difficoltà di chi ha ricoperto questo ruolo è stata la suddivisione del lavoro in maniera equa ed omogenea provocando una serie di ridistribuzioni degli incarichi tra i vari membri.\\
	\textbf{Soluzione proposta:} per ridurre il possibile ritardo che si sarebbe potuto presentare, il gruppo ha deciso di dedicare del tempo ad analizzare la quantità di lavoro richiesta così da assegnare con più attenzione i compiti ai vari membri del gruppo. Questo è stato reso possibile usando lo strumento di \glo{ITS} offerto da \glo{GitHub}.
\end{itemize}
\myparagraph{Analista}
\begin{itemize}
	\item \textbf{Problema:} data l'inesperienza da parte degli \textit{Analisti}, il tracciamento dei requisiti e l'individuazione dei casi d'uso hanno richiesto più ore del previsto, come indicato anche all'interno del \PdPv{1.0.0}.\\
	\textbf{Soluzione proposta:} dopo aver studiato in maniera autonoma l'argomento, il gruppo ha deciso di dedicare del tempo aggiuntivo per analizzare meglio i casi d'uso da modellare e le tecnologie da impiegare.
\end{itemize}
\myparagraph{Verificatore}
\begin{itemize}
	\item \textbf{Problema:} la verifica è stata svolta in maniera non costante all'inizio e questo ha provocato una mole di documenti da verificare più ampia del previsto incontrando anche contraddizioni di contenuto e struttura tra i diversi documenti.\\
	\textbf{Soluzione proposta:} una pianificazione migliore del lavoro da svolgere ha aiutato in corso d'opera a evitare che questo succedesse nuovamente. Il gruppo ha poi deciso di dedicare del tempo per analizzare meglio i contenuti di tutti i singoli documenti per evitare l'insorgere di contraddizioni tra gli stessi.
\end{itemize}
\subsubsection{Valutazioni sugli strumenti}
\myparagraph{GitHub}
\begin{itemize}
	\item \textbf{Problema:} si sono riscontrati diverse difficoltà, specialmente nel periodo iniziale, dovute alla scarsa esperienza di alcuni membri del gruppo nell'utilizzo di tale strumento.\\
	\textbf{Soluzione proposta:} i membri carenti nell'uso di GitHub hanno effettuato un approfondito lavoro individuale per sanare le proprie mancanze.
\end{itemize}
\myparagraph{\LaTeX}
\begin{itemize}
	\item \textbf{Problema:} per l'inesperienza della quasi totalità dei membri del gruppo all'utilizzo di questo strumento, si sono riscontrate diverse difficoltà specialmente nell'inserimento di figure e nello redarre un template comune per tutti i documenti.\\
	\textbf{Soluzione proposta:} per risolvere in breve tempo questa problematica, si è deciso di dedicare parte delle prime settimane per apprendere e approfondire l'utilizzo di \LaTeX.
\end{itemize}
\subsection{Revisione di progettazione} \label{MiglioramentoRP}
\subsubsection{Valutazioni sull'organizzazione}
\begin{itemize}
	\item \textbf{Problema:} le attività relative al progetto si sono sovrapposte con gli impegni degli esami universitari. Diversi membri del gruppo hanno tralasciato quasi del tutto le attività di progetto; questo ha portato ad un pesante rallentamento dello sviluppo del prodotto.\\
	\textbf{Soluzione proposta:} i membri meno occupati dalle prove universitarie hanno cercato di far progredire lo sviluppo occupandosi delle parti che non richiedevano il coinvolgimento del gruppo nel suo complesso. Tutti si sono impegnati a dare il proprio contributo nonostante la sessione d'esami; i membri momentaneamente inattivi erano costantemente aggiornati mediante i canali di comunicazione adottati dal gruppo.
\end{itemize}
\subsubsection{Valutazioni sui ruoli}
\myparagraph{Analista}
\begin{itemize}
	\item \textbf{Problema:} durante lo sviluppo del \glo{PoC} sono emersi nuovi requisiti e si è vista la necessità di dettagliare maggiormente quanto già descritto all'interno dell'{\AdR}. In seguito ad un incontro sostenuto con il {\CR}, sono risultate evidenti altre imprecisioni in requisiti già individuati precedentemente portando ad una sostanziale ristrutturazione del documento stesso.\\
	\textbf{Soluzione proposta:} gli \textit{Analisti} hanno dovuto rivedere con maggiore attenzione il lavoro svolto nel precedente periodo e correggere gli errori. Dopo un'analisi più approfondita e consapevole; sono stati di grande aiuto il confronto avuto con {\Proponente} e lo studio più approfondito delle tecnologie coinvolte nello sviluppo del progetto.
\end{itemize}
\myparagraph{Progettista}
\begin{itemize}
	\item \textbf{Problema:} a causa della totale inesperienza nel ruolo, i \textit{Progettisti} hanno trovato molte avversità nell'individuare l'architettura ottimale per il prodotto richiesto.\\
	\textbf{Soluzione proposta:} vista l'intenzione di costruire un prodotto in modo incrementale e quindi di mantenere il più invariato possibile quanto sviluppato al termine di questo periodo, i \textit{Progettisti} hanno dovuto approfondire maggiormente le proprie conoscenze sul settore.
\end{itemize}
\myparagraph{Programmatore}
\begin{itemize}
	\item \textbf{Problema:} data la totale inesperienza del gruppo nel lavorare con le tecnologie coinvolte, i \textit{Programmatori} hanno incontrato diverse difficoltà nell'impiego dei linguaggi e delle tecnologie richieste; è risultato inoltre complicato anche il dover integrare i vari microservizi sviluppati.\\
	\textbf{Soluzione proposta:} i \textit{Programmatori} hanno colmato le mancanze rilevate tramite un ulteriore periodo di formazione personale e l'esercizio pratico con le tecnologie in uso.
\end{itemize}