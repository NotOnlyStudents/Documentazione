\section{Specifica dei test}
\label{specificatest}
Il gruppo ha deciso di utilizzare il modello a V, come modello di sviluppo del software per assicurare la \glo{qualità} del prodotto. Questo modello illustra le relazioni tra ogni \glo{fase} del ciclo di vita di sviluppo e la fase di test ad essa associata.
\subsection{Test di sistema}
Lo scopo dei test di sistema è quello di verificare che i requisiti espressi nel documento \AdR{} vengano rispettati. Di seguito l'elenco di questi test con la loro descrizione.
\begin{longtable}{|c| C{10cm} |C{3cm}|}
	\rowcolor{darkblue}
	\textcolor{white}{\textbf{ID Test}}&
	\textcolor{white}{\textbf{Descrizione}}&
	\textcolor{white}{\textbf{Implementato}}\label{tab:TestSistema1}\\
	TRFO1 & Viene verificato che l'utente possa registrarsi ed eseguire l'accesso con e-mail e password & Implementato\\ \hline
	TRFO2 & Viene verificato che l'utente non autenticato possa essere in grado di fare il reset della password & Non ancora in sviluppo\\ \hline
	TRFO3 & Viene verificato che l'utente autenticato sia in grado di eseguire il logout dalla piattaforma & Non ancora in sviluppo\\ \hline
	TRFO4 & Viene verificato che l'utente sia in grado di visualizzare la propria schermata principale con: descrizione dell'azienda e prodotti in evidenza & Non ancora in sviluppo\\ \hline
	TRFO5 & Viene verificato che l'utente non autenticato o l'acquirente possa cercare i prodotti utilizzando delle parole chiave, e deve essere avvisato se la ricerca non fornisce risultati & Non ancora in sviluppo\\ \hline
	TRFO6 & Viene verificato che l'utente non autenticato o l'acquirente sia in grado di filtrare i prodotti nella \glo{PLP} per categoria, prezzo e disponibilità in magazzino & Non ancora in sviluppo\\ \hline
	TRFO7 & Viene verificato che l'utente non autenticato o l'acquirente sia in grado di cambiare l'ordinamento & Non ancora in sviluppo\\ \hline
	TRFO8 & Viene verificato che l'utente non autenticato o l'acquirente possa aprire la \glo{PDP} di un prodotto partendo dalla PLP, dalla schermata dei prodotti in evidenza e dal riepilogo ordine & Non ancora in sviluppo\\ \hline
	TRFO9 & Viene verificato che l'utente non autenticato o l'acquirente possa accedere alla PDP di un prodotto dalla pagina del carrello & Non ancora in sviluppo\\ \hline
  	TRFO10 & Viene verificato che l'utente non autenticato o l'acquirente possa visualizzare le informazioni, il prezzo totale, gli sconti dei prodotti presenti nel carrello & Non ancora in sviluppo\\ \hline
  	TRFO11 & Viene verificato che l'utente non autenticato o l'acquirente possa visualizzare, aggiungere, modificare la quantità ed eliminare i prodotti nel carrello & Non ancora in sviluppo\\ \hline
  	TRFO12 & Viene verificato che l'acquirente possa procedere all'acquisto del carrello & Non ancora in sviluppo\\ \hline
  	TRFO13 & Viene verificato che l'utente non autenticato debba autenticarsi prima di procedere con il checkout & Non ancora in sviluppo\\ \hline
	TRFO14 & Viene verificato che l'acquirente possa inserire l'indirizzo di spedizione e possa procedere al pagamento attraverso il servizio fornito da \glo{Stripe} & Non ancora in sviluppo\\ \hline
	TRFO15 & Viene verificato che l'acquirente o il venditore possa visualizzare i prodotti contenuti in un ordine, le loro quantità, l'indirizzo di spedizione e il prezzo totale di pagamento & Non ancora in sviluppo\\ \hline
	TRFO16 & Viene verificato che l'acquirente o il venditore possa modificare le informazioni personali, l'indirizzo email e la password & Non ancora in sviluppo\\ \hline
	TRFO17 & Viene verificato che l'acquirente possa eliminare il proprio account & Non ancora in sviluppo\\ \hline
	TRFO18 & Viene verificato che il venditore possa filtrare l'elenco degli ordini a suo carico & Non ancora in sviluppo\\ \hline
	TRFO19 & Viene verificato che il venditore possa aggiungere un prodotto alla piattaforma, specificando: descrizione, categoria, prezzo, sconti applicati, quantità e se deve essere presente tra i prodotti in evidenza & Non ancora in sviluppo\\ \hline
	TRFO20 & Viene verificato che il venditore possa modificare, eliminare e rifornire i prodotti & Non ancora in sviluppo\\ \hline
	TRFO21 & Viene verificato che il venditore possa visualizzare una lista con tutti gli ordini effettuati & Non ancora in sviluppo\\ \hline
	TRFO22 & Viene verificato che il carrello dell'acquirente possa essere salvato in remoto e sincronizzato su più browser/dispositivi a cui accedere & Non ancora in sviluppo\\ \hline
	TRFO23 & Viene verificato che l'utente non autenticato possa aggiungere prodotti nel carrello come ospite e, appena si autentica, vengano aggiunti al suo carrello personale & Non ancora in sviluppo\\ \hline
  \caption{Descrizione dei test di sistema.}
\end{longtable}
\begin{longtable}{|c| C{13cm}|}
	\rowcolor{white}
	\rowcolor{darkblue}
	\textcolor{white}{\textbf{ID Test}}&
	\textcolor{white}{\textbf{ID Requisito}}\label{tab:TestSistema2}\\
	TRFO1 & RFO1\_1\\ \hline
	TRFO2 & RFO9\_4\\ \hline
	TRFO3 & RFO11\_5\\ \hline
	TRFO4 & RFO68\\ \hline
	TRFO5 & RFO12\_6\\\hline
	TRFO6 & RFO13\_7, RFO14\_7.1, RFO15\_7.2, RFO16\_7.3\\ \hline
	TRFO7 & RFO20\_8, RFO21\_9, RFO22\_10\\ \hline
	TRFO8 & RFO24, RFO25, RFO27\\ \hline
	TRFO9 & RFO26\\ \hline
	TRFO10 & RFO33\_14\\ \hline
	TRFO11 & RFO29\_12, RFO30\_13, RFO34\_15, RFO35\_16\\ \hline
	TRFO12 & RFO38\_17\\ \hline
	TRFO13 & RFO38\_17\\ \hline
	TRFO14 & RFO39\_17.1, RFO41\_17.2\\ \hline
	TRFO16 & RFO70\_26, da RFO70\_26.1 a RFO70\_26.7, RFO62\_24, da RFO62\_24.1 a RFO62\_24.4   \\ \hline
	TRFO17 & RFO67\_25\\ \hline
	TRFO18 & RFO126\_43, da RFO126\_43.1 a RFO126\_43.2.2\\ \hline
	TRFO19 & RFO85\_27, da RFO85\_27.1 a RFO85\_27.7, RFO102\_29\\ \hline
	TRFO20 & RFO94\_28, da RFO94\_28.1 a RFO94\_28.7, RFO104\_31, RFO105\_32\\ \hline
	\rowcolor{white}
	\caption{Relazione tra test di sistema e requisiti.}\\
\end{longtable}
\newpage
\subsection{Test di unità}
Lo scopo dei test di unità è quello di isolare ciascuna parte di un programma e mostrarne correttezza e completezza nell'\glo{implementazione}, facendo emergere eventuali difetti, in modo che essi possano essere corretti prima dell'integrazione. Questi test sono prematuri e verranno eseguiti in parallelo con la stesura del codice.
\subsection{Test di integrazione}
Lo scopo dei test di integrazione è quello di verificare che l'unione di più parti di codice porti al risultato atteso. Ad oggi, è troppo prematuro eseguire questi test in quanto vengono effettuati dopo i test di unità sulle singole parti da integrare.
\subsection{Test di accettazione}
Servono a controllare che il software creato soddisfi i requisiti concordati e le esigenze del proponente. I test di accettazione vengono eseguiti assieme al proponente prima del rilascio del software, dopo aver testato l’intero sistema e dopo la correzione dei difetti riscontrati.