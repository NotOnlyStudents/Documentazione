\section{Esiti delle revisioni}
\label{esito}
Il gruppo \Gruppo\ ha deciso di dedicare alcuni incontri per discutere sulla valutazione critica del lavoro svolto fino a quel momento in seguito alla pubblicazione delle osservazioni da parte dei committenti e di \Proponente\, oltre alle relative correzioni da apportare ai documenti. È importante riportare e tener traccia delle modifiche svolte per ogni osservazione ricevuta. In questo modo si ha sempre sotto controllo il lavoro svolto dai membri del gruppo e definire una possibile soluzione in maniera univoca. Per dare una struttura coerente con lo stile del documento che faciliti l'individuazione della possibile soluzione abbiamo adottato il formato tabellare.
\subsection{Revisione dei requisiti}
	\begin{longtable}{|C{4cm}|C{9cm}|c|}
		\rowcolor{darkblue}
		\textcolor{white}{\textbf{Osservazione}}&
		\textcolor{white}{\textbf{Soluzione adottata}}&
		\textcolor{white}{\textbf{ID}}\label{tab:CriticitàRR}\\ \hline
		\rowcolor{darkblue}
		\multicolumn{3}{|c|}{\textcolor{white}{\textbf{Criticità generali}}}\\ \hline
		Insufficiente attenzione nella stesura e nella verifica dei documenti. & Il team ha cercato di migliorare il procedimento di verifica continua eseguendo le operazioni di verifica con maggior cura. & CAVE01\\ \hline %1-2-3-4
		"Scatto" di versione in un prodotto soggetto a manutenzione dovrebbe essere associato solo a modifiche verificate e validate. & Ad ogni cambiamento inserito nel registro delle modifiche viene svolta un'\glo{attività} di verifica. I membri del gruppo hanno inoltre deciso di eliminare il terzo valore presente nella versione in modo da indicare con maggiore attenzione che il documento è effettivamente stato verificato e validato. & CADO01\\ \hline %5
		\rowcolor{darkblue}
		\multicolumn{3}{|c|}{\textcolor{white}{\textbf{Piano di progetto}}}\\ \hline
		Adottato un modello di sviluppo ibrido difficile da governare e decifrare. & La stesura dei documenti deve rispecchiare maggiormente la scelta di adottare il modello di sviluppo incrementale inserendo all'interno del \PdP\ i rischi associati a tale modello e le possibili soluzioni. & CPD01\\ \hline %2
		Il preventivo a finire è un esercizio contabile che non mostra cambiamenti alla pianificazione iniziale. & È possibile inserire una colonna con gli sforamenti avvenuti durante il procedimento di sviluppo. Il gruppo ha deciso poi di scrivere nella sezione della pianificazione \textbf{come} e \textbf{cosa} è cambiato rispetto alla pianificazione iniziale col passare del tempo. & CAPN01\\ \hline \newpage %1 
		\rowcolor{darkblue}
		\multicolumn{3}{|c|}{\textcolor{white}{\textbf{Analisi dei requisiti}}}\\ \hline
		Imprecisioni tecniche nel disegno dei diagrammi dei casi d'uso. & A seguito di uno studio più approfondito e ad una discussione con \CR{}, i diagrammi in questione sono stati corretti. & CPD02\\ \hline %4-5-6-7-8
		Da approfondire i requisiti funzionali. & Il team ha eseguito un'analisi più profonda del \glo{capitolato} e si sono fissate le metriche per verificare la soddisfazioni di questi. & CAVE02\\ \hline %11
		Insufficiente attenzione nella stesura del documento. & Rivisto il documento in modo più approfondito. Sistemata la numerazione dei sotto-casi d'uso per renderla più coerente con la gerarchia individuata. & CPD03\\ \hline %2-12
		Amministratore come ruolo del sistema. & L'amministratore è stato rimosso dagli attori dal momento che tutte le operazioni che questo può compiere non vengono gestite dalla piattaforma \NomeProgetto, ma dalla \glo{dashboard} di \glo{AWS} & CPAN01\\ \hline %1
		\rowcolor{darkblue}
		\multicolumn{3}{|c|}{\textcolor{white}{\textbf{Norme di progetto}}}\\ \hline
		Presenza di circolarità di riferimento all'inizio delle \NdP\. & Rimossi i riferimenti che causavano il ciclo con \PdP\ e  \PdQ{}. & CPD04\\ \hline %1
		\rowcolor{darkblue}
		\multicolumn{3}{|c|}{\textcolor{white}{\textbf{Piano di qualifica}}}\\ \hline
		Insufficiente allineamento tra le \NdP\ e il \PdQ{} rispetto agli obiettivi di \glo{qualità}. & Suddivisi i vari processi individuati all'interno del \PdQ\ in \textbf{processi primari}, \textbf{processi di supporto} e \textbf{processi organizzativi}. Corretta la definizione degli obiettivi di qualità e il metodo con cui calcolare le varie metriche all'interno delle \NdP{}. Nel \PdQ\ è stato aggiunto un "cruscotto" che restituisce una valutazione aggiornata sul raggiungimento degli obiettivi prefissati. & CPD05\\ \hline %1
		Le metriche che si è deciso di adottare non vanno relegate in una appendice dedicata. & Riorganizzata la struttura del \PdQ{}. & CPD06\\ \hline %2
		\rowcolor{white}
		\caption{Miglioramenti apportati in seguito alla RR.}\\
	\end{longtable}
\subsection{Revisione di progettazione}
\begin{longtable}{|C{4cm}|C{9cm}|c|}
	\rowcolor{darkblue}
	\textcolor{white}{\textbf{Osservazione}}&
	\textcolor{white}{\textbf{Soluzione adottata}}&
	\textcolor{white}{\textbf{ID}}\label{tab:CriticitàRP}\\ \hline
	\rowcolor{darkblue}
	\multicolumn{3}{|c|}{\textcolor{white}{\textbf{Piano di progetto}}}\\ \hline
	Immagine del modello incrementale concettualmente errata. & Si è analizzata l'immagine in questione e si è sostituita con una più coerente, rileggendo il paragrafo in cui è presente per assicurarsi la consistenza con quanto scritto precedentemente. & CPD07  \\ \hline
	Associazione di impegno orario ad obiettivi. & Si introducono delle tabelle di tracciamento obiettivo-quantità ore previste per il suo raggiungimento. & CPD08 \\ \hline
	Sforamento del preventivo. & Per poter rientrare nel preventivo concordato l'impegno orario del prossimo periodo di validazione e collaudo è stato rimodulato. & CPD09 \\ \hline
	Rendere proattivi i contenuti dell'attualizzazione dei rischi. & I rischi vengono individuati anche per i periodi successivi e non solo per il periodo in cui ci si trova. & CAQU01 \\ \hline
	\rowcolor{darkblue}
	\multicolumn{3}{|c|}{\textcolor{white}{\textbf{Analisi dei requisiti}}}\\ \hline
	Diagrammi dei casi d'uso generici. & I diagrammi generici inseriti sono stati rimossi. & CPD10  \\ \hline
	UC in cui si suppone il ruolo dell'attore prima della sua autenticazione. & I casi d'uso sono stati corretti gestendo gli attori esclusivamente come utente autenticato e non. & CPD11 \\ \hline
	UC modifica password distinti per i diversi attori. & Vengono riuniti i casi d'uso di modifica password venditore/acquirente riferendosi esclusivamente a utente autenticato. & CPD12 \\ \hline
	\rowcolor{darkblue}
	\multicolumn{3}{|c|}{\textcolor{white}{\textbf{Norme di progetto}}}\\ \hline
	Errata interpretazione del ciclo \glo{PDCA}. & Il gruppo ha approfondito lo studio dello standard e ha corretto la sezione modificandone la definizione. & CAQU02 \\ \hline
	\caption{Miglioramenti apportati in seguito alla RP.}\\
\end{longtable}
\subsection{Revisione di qualifica}
\begin{longtable}{|C{4cm}|C{9cm}|c|}
	\rowcolor{darkblue}
	\textcolor{white}{\textbf{Osservazione}}&
	\textcolor{white}{\textbf{Soluzione adottata}}&
	\textcolor{white}{\textbf{ID}}\label{tab:CriticitàRQ}\\ \hline
	\rowcolor{darkblue}
	\multicolumn{3}{|c|}{\textcolor{white}{\textbf{Criticità generali}}}\\ \hline
	Simbolo \S\ sostituisce la parola "sezione". & Si corregge all'interno dei documenti la ripetizione. & CPD13 \\ \hline
	\rowcolor{darkblue}
	\multicolumn{3}{|c|}{\textcolor{white}{\textbf{Piano di progetto}}}\\ \hline
	Non ancora sanato eccesso del preventivo. & Si è descritto meglio quali spese sono state ridotte e il motivo. & CPD14 \\ \hline
	\rowcolor{darkblue}
	\multicolumn{3}{|c|}{\textcolor{white}{\textbf{Manuale sviluppatore}}}\\ \hline
	Interfacce non devono avere attributi. & Vengono eliminati gli attributi presenti nelle interfacce & CPD15 \\ \hline
	Diagrammi architetturali devono essere descritti maggiormente. & Si introduce una descrizione più dettagliata nei diagrammi presenti. & CPD16 \\ \hline
	Descrizione front end insufficiente. & Si approfondisce maggiormente la sezione del front end nel documento. & CPD17 \\ \hline
	\rowcolor{darkblue}
	\multicolumn{3}{|c|}{\textcolor{white}{\textbf{Manuale utente}}}\\ \hline
	Opportuno inserire specifica delle funzionalità. & Il gruppo inserisce all'interno del manuale le funzionalità previste dal prodotto. & CPD18 \\ \hline
	Opportuno inserire come segnalare malfunzionamenti. & Viene inserita all'interno del manuale una sezione con le indicazioni per segnalare al gruppo i malfunzionamenti. & CPD19 \\ \hline
	\rowcolor{darkblue}
	\multicolumn{3}{|c|}{\textcolor{white}{\textbf{Verbali}}}\\ \hline
	Le decisioni prese non sono precise. & Quanto stabilito nelle nuove riunioni vengono descritte in modo più specifico. & CPD20 \\ \hline
	\rowcolor{white}
	\caption{Miglioramenti apportati in seguito alla RQ.}\\
\end{longtable}