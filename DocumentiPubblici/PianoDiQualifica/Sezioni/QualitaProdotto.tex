\section{Qualità di Prodotto}
\label{qualità_prodotto}
Per valutare la \glo{qualità} di \glo{prodotto}, il gruppo {\Gruppo} ha deciso di far riferimento allo \glo{standard} \glo{ISO/IEC 9126}, descritto in maniera esaustiva all'interno delle \textit{\NdPv{1.0.0}} alla sezione \S{C}. \\
Il modello di qualità stabilito nello standard, ISO/IEC 9126, è definito da sei caratteristiche generali esposte in modo approfondito nel proseguo del documento e varie sottocaratteristiche misurabili attraverso delle \glo{metriche}. Il \glo{team} di lavoro ha deciso di selezionare solamente alcune di queste sottocaratteristiche, omettendo i parametri ritenuti meno rilevanti ai fini del progetto.

\subsection{Funzionalità}
\subsubsection{Obiettivi}
\begin{itemize}
	\item \textbf{OPDS01: Funzionalità} \\
	É la capacità di un software di fornire funzioni dedicate a soddisfare i bisogni evidenziati nell'\textit{\AdR}, e che permettano di operare sotto  condizioni specifiche.
	\item \textbf{Caratteristiche da rispettare} \\
	Le caratteristiche di qualità da rispettare sono:
	\begin{itemize}
		\item \textbf{Appropriatezza:} capacità del prodotto software di fornire un insieme di funzioni per i compiti specificati ed obiettivi prefissati all'utente;
		\item \textbf{Accuratezza:} capacità del prodotto software di fornire i risultati corretti con la precisione richiesta;
		\item \textbf{Sicurezza:} capacità del prodotto software di proteggere le informazioni e dati. Non deve essere permesso a persone e/o sistemi non autorizzati di accedervi o modificarli;
		\item \textbf{Interoperabilità:} capacità del prodotto software di interagire ed operare con uno o diversi sistemi specificati.
	\end{itemize}
\end{itemize}

\subsubsection{Metriche}
\begin{table} [H]
	\begin{center}
		\begin{tabular}{|c| p{12cm}|}
			\rowcolor{darkblue}
			\multicolumn{2}{|c|}{\textcolor{white}{\textbf{MPDS01: Completezza dell'implementazione}}} \\ \hline
			Descrizione & Il metodo di misura è riportato nelle \textit{\NdPv{1.0.0}} alla sezione \S{D2.1}. \\ \hline
			Risultato & Indice compreso tra 0 e 100 che stabilisce la completezza del prodotto software. \\ \hline
			Valore ottimale & 100\%. \\ \hline
			Valore minimale & 100\%. \\ \hline
		\end{tabular}
	\end{center}
	\caption{\label{tab:MPDS01}Metrica relativa alla completezza dell'implementazione.}
\end{table} 

\subsection{Affidabilità}
\subsubsection{Obiettivi}
\begin{itemize}
	\item \textbf{OPDS02: Affidabilità} \\
	É la capacità del prodotto software di mantenere un certo livello di prestazioni quando viene usato in specifiche condizioni per un dato periodo.
	\item \textbf{Caratteristiche da rispettare} \\
	Le caratteristiche di qualità da rispettare sono:
	\begin{itemize}
		\item \textbf{Maturità:} capacità di un prodotto software di evitare errori e malfunzionamenti durante la sua esecuzione;
		\item \textbf{Robustezza:} capacità di mantenere specifici livelli di prestazioni anche in presenza di malfunzionamenti e usi inappropriati del prodotto;
		\item \textbf{Recuperabilità:} capacità di ripristinare prestazioni e dati in caso di errori o malfunzionamenti.
	\end{itemize}
\end{itemize}

\subsubsection{Metriche}
\begin{table} [H]
	\begin{center}
		\begin{tabular}{|c| p{12cm}|}
			\rowcolor{darkblue}
			\multicolumn{2}{|c|}{\textcolor{white}{\textbf{MPDS02: Densità errori}}} \\ \hline
			Descrizione & Consiste nella capacità di un prodotto software di resistere ai malfunzionamenti ed è riportato nelle \textit{\NdPv{1.0.0}} alla sezione \S{D2.2}. \\ \hline
			Risultato & Indice compreso tra 0 e 100. \\ \hline
			Valore ottimale & Scelto in seguito. \\ \hline
			Valore minimale & Scelto in seguito. \\ \hline
		\end{tabular}
	\end{center}
	\caption{\label{tab:MPDS02}Metrica relativa alla densità degli errori.}
\end{table}

\subsection{Efficienza}
\subsubsection{Obiettivi}
\begin{itemize}
	\item \textbf{OPDS03: \glo{Efficienza}} \\
Capacità di fornire determinate prestazioni relative alla quantità di risorse utilizzate.
	\item \textbf{Caratteristiche da rispettare} \\
	Le caratteristiche di qualità da rispettare sono:
	\begin{itemize}
		\item \textbf{Comportamento rispetto al tempo:} capacità di fornire sotto specifiche condizioni adeguati tempi di risposta e di elaborazione;
		\item \textbf{Utilizzo delle risorse:} capacità di impiegare quantità e tipo di risorse in maniera adeguata;
		\item \textbf{Conformità:} capacità di aderire a standard e date convenzioni sull'efficienza.
	\end{itemize}
\end{itemize}

\subsubsection{Metriche}
Il \glo{proponente} {\Proponente} non ha ancora espresso requisiti in termini di efficienza. Si è cosi deciso di non presentare metriche di efficienza in questa sessione. In seguito, questa sezione potrà essere aggiornata.

\subsection{Usabilità}
\subsubsection{Obiettivi}
\begin{itemize}
	\item \textbf{OPDS04: Usabilità} \\
	É la capacità del prodotto software di essere compreso, appreso e utilizzato dal suo \glo{target} di utenza.
	\item \textbf{Caratteristiche da rispettare} \\
	Le caratteristiche di qualità da rispettare sono:
	\begin{itemize}
		\item \textbf{Comprensibilità:} capacità di essere inequivocabilmente chiaro rispetto alle funzionalità offerte e le modalità di utilizzo;
		\item \textbf{Apprendibilità:} capacità di diminuire l'impegno richiesto agli utenti per imparare ad usare il software;
		\item \textbf{Operabilità:} capacità di mettere gli utenti in condizione di fare uso del software per i propri scopi e controllarne l'utilizzo;
		\item \textbf{Attrattiva:} capacità del software di essere piacevole per l'utente che ne fa uso;
		\item \textbf{Conformità:} capacità del software di aderire a standard o convenzioni relativi all'usabilità.
	\end{itemize}
\end{itemize}

\subsubsection{Metriche}
\begin{table} [H]
	\begin{center}
		\begin{tabular}{|c| p{12cm}|}
			\rowcolor{darkblue}
			\multicolumn{2}{|c|}{\textcolor{white}{\textbf{MPDS03: Facilità di utilizzo}}} \\ \hline
			Descrizione & Indica la facilità con cui l'utente raggiunge ciò che vuole ed è rappresentata tramite il numero di click necessari per arrivare al contenuto desiderato. È riportato nelle \textit{\NdPv{1.0.0}} alla sezione \S{D2.3}. \\ \hline
			Risultato & Numero intero che rappresenta i click necessari per raggiungere le schermata di cui si necessita. \\ \hline
			Valore ottimale & $\leq$ 10. \\ \hline
			Valore minimale & $\leq$ 15. \\ \hline
		\end{tabular}
	\end{center}
	\caption{\label{tab:MPDS03}Metrica relativa alla facilità di utilizzo.}
\end{table}

\begin{table} [H]
	\begin{center}
		\begin{tabular}{|c| p{12cm}|}
			\rowcolor{darkblue}
			\multicolumn{2}{|c|}{\textcolor{white}{\textbf{MPDS04: Facilità di apprendimento}}} \\ \hline
			Descrizione & Indica la facilità con cui l'utente riesce ad imparare ad usare le funzionalità del prodotto e viene rappresentata tramite il tempo medio che serve per comprenderle. È riportato nelle \textit{\NdPv{1.0.0}} alla sezione \S{D2.4} \\ \hline
			Risultato & Numero intero che rappresenta i minuti per raggiungere la pagina desiderata. \\ \hline
			Valore ottimale & $\leq$ 3. \\ \hline
			Valore minimale & $\leq$ 5. \\ \hline
		\end{tabular}
	\end{center}
	\caption{\label{tab:MPDS04}Metrica relativa alla facilità di apprendimento.}
\end{table}

\begin{table} [H]
	\begin{center}
		\begin{tabular}{|c| p{12cm}|}
			\rowcolor{darkblue}
			\multicolumn{2}{|c|}{\textcolor{white}{\textbf{MPDS05: Profondità della gerarchia}}} \\ \hline
			Descrizione & Indica la profondità del sito. Un sito per essere facile da utilizzare non deve essere troppo profondo. È riportato nelle \textit{\NdPv{1.0.0}} alla sezione \S{D2.5} \\ \hline
			Risultato & Numero intero che rappresenta la profondità delle pagine. \\ \hline
			Valore ottimale & $\leq$ 4. \\ \hline
			Valore minimale & $\leq$ 7. \\ \hline
		\end{tabular}
	\end{center}
	\caption{\label{tab:MPDS05}Metrica relativa alla profondità della gerarchia.}
\end{table}

\subsection{Manutenibilità}
\subsubsection{Obiettivi}
\begin{itemize}
	\item \textbf{OPDS05: Manutenibilità} \\
	É la capacità del software di essere modificato da correzioni, miglioramenti o adattamenti.
	\item \textbf{Caratteristiche da rispettare} \\
	Le caratteristiche di qualità da rispettare sono:
	\begin{itemize}
		\item \textbf{Analizzabilità:} facilità con la quale è possibile analizzare il codice per localizzare un errore;
		\item \textbf{Modificabilità:} capacità del prodotto software di essere modificato agevolmente a livello di codice, progettazione o documentazione;
		\item \textbf{Stabilità:} capacità del software di evitare effetti indesiderati in seguito ad una modifica;
		\item \textbf{\glo{Testabilità}:} capacità del software di essere facilmente sottoposto a \glo{test} per verificare e validare le modifiche apportate.
	\end{itemize}
\end{itemize}

\subsubsection{Metriche}
\begin{table} [H]
	\begin{center}
		\begin{tabular}{|c| p{12cm}|}
			\rowcolor{darkblue}
			\multicolumn{2}{|c|}{\textcolor{white}{\textbf{MPDS06: Facilità di comprensione}}} \\ \hline
			Descrizione & Indica la facilità con cui è possibile comprendere cosa fa il codice ed è rappresentata dal numero di linee di commento nel codice. È riportato nelle \textit{\NdPv{1.0.0}} alla sezione \S{D2.6}. \\ \hline
			Risultato & Indice compreso tra 0 e 100 che stabilisce la facilità di comprensione del codice. \\ \hline
			Valore ottimale & Scelto in seguito. \\ \hline
			Valore minimale & Scelto in seguito. \\ \hline
		\end{tabular}
	\end{center}
	\caption{\label{tab:MPDS06}Metrica relativa alla facilità di comprensione.}
\end{table}

\begin{table} [H]
	\begin{center}
		\begin{tabular}{|c| p{12cm}|}
			\rowcolor{darkblue}
			\multicolumn{2}{|c|}{\textcolor{white}{\textbf{MPDS07: Semplicità delle funzioni}}} \\ \hline
			Descrizione & La semplicità di un metodo può essere rappresentata dal numero di parametri per metodo; meno parametri ha una funzione, più è semplice e intuitiva. È riportato nelle \textit{\NdPv{1.0.0}} alla sezione \S{D2.7} \\ \hline
			Risultato & Numero intero che rappresenta il numero di parametri di un metodo. \\ \hline
			Valore ottimale & $\leq$ 3. \\ \hline
			Valore minimale & $\leq$ 6. \\ \hline
		\end{tabular}
	\end{center}
	\caption{\label{tab:MPDS07}Metrica relativa alla semplicità delle funzioni.}
\end{table}

\begin{table} [H]
	\begin{center}
		\begin{tabular}{|c| p{12cm}|}
			\rowcolor{darkblue}
			\multicolumn{2}{|c|}{\textcolor{white}{\textbf{MPDS08: Semplicità delle classi}}} \\ \hline
			Descrizione & La semplicità di una classe è rappresentata dal numero di metodi che essa contiene; meno metodi ha al suo interno più la classe ha uno scopo ben specifico e comprensibile. È riportato nelle \textit{\NdPv{1.0.0}} alla sezione \S{D2.8} \\ \hline
			Risultato & Numero intero che rappresenta il numero di metodi all'interno di una classe. \\ \hline
			Valore ottimale & $\leq$ 10. \\ \hline
			Valore minimale & $\leq$ 16. \\ \hline
		\end{tabular}
	\end{center}
	\caption{\label{tab:MPDS08}Metrica relativa alla facilità di comprensione.}
\end{table}
