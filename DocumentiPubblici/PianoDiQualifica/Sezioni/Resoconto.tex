\section{Resoconto delle \glo{attività} di verifica}
\label{resoconto}

\subsection{Periodo di Analisi}
Durante il periodo di Analisi, i documenti redatti da presentare in ingresso alla Revisione dei Requisiti e i \glo{processi} eseguiti vengono verificati. I documenti sono verificati dai \textit{Verificatori} secondo i criteri per l'analisi statica definiti nelle \textit{\NdPv{1.0.0}}, seguendo le metodologie di \glo{\textit{Walkthrough}} ed \glo{\textit{Inspection}}.

\subsection{Strategia adottata per l'analisi statica}
Per ciascun documento si è creato inizialmente una struttura di base comune così da evitare possibili conflitti e sprechi di tempo futuri. A questo punto si è applicato il metodo dell’\textit{Inspection} tramite l'uso della Lista di Controllo. Si è potuto infatti svolgere un’ulteriore esame nei confronti del documento sottoposto a verifica per scoprire gli errori non visti nelle verifiche precedenti. Il \textit{Verificatore} valuta la correttezza del documento nella sua interezza, cercando di individuare eventuali errori e trattandoli nel modo seguente:
	\begin{itemize}
		\item Correzione degli errori ortografici e sintattici o di eventuali violazioni delle norme tipografiche stabilite nelle \textit{\NdPv{1.0.0}};
		\item Inserimento degli errori più ricorrenti nella Lista di Controllo, redatta durante la \glo{fase} di verifica dei documenti;
		\item Applicazione del ciclo \glo{\textit{PDCA}} per migliorare e velocizzare le verifiche future.
	\end{itemize}
\subsection{Esiti verifica}
Per ciascun documento redatto si è calcolato l’\glo{\textit{indice di Gulpease}}. Per evitare risultati errati nel calcolo di tali misurazioni, si è deciso di non prendere in considerazione:
\begin{itemize}
	\item Il frontespizio di ogni documento;
	\item Le eventuali tabelle presenti all'interno dei documenti;
	\item I registri delle modifiche di ogni documento.
\end{itemize}