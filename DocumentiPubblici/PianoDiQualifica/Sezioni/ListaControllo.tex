\section{Lista di controllo}
\label{lista_controllo}
Gli errori più frequenti riscontrati durante la verifica della documentazione si suddividono in:
\begin{itemize}
	\item \textbf{Errori ortografici:}
	\begin{itemize}
		\item Errori di battitura e/o distrazione;
		\item Assenza o uso improprio della punteggiatura;
		\item Mancata concordanza tra le parti variabili del discorso.
	\end{itemize}
	\item \textbf{Errori grammaticali:}
	\begin{itemize}
		\item Errata coniugazione dei verbi;
		\item Discordanza tra le persone utilizzate, si passa da una discorso impersonale alla prima persona plurale e viceversa.
	\end{itemize}
	\item \textbf{Errori visuali:}
	\begin{itemize}
		\item Mancanza di grafici;
		\item Presenza di grafici non aggiornati con la versione corrente del documento.
	\end{itemize}
	\item \textbf{Errori rispetto a quanto definito nelle norme di progetto:}
	\begin{itemize}
		\item Mancanza dei caratteri ";" e "." alla fine di ogni voce di un elenco puntato;
		\item Mancato uso del grassetto per le voci di un elenco puntato;
		\item Uso della prima lettera minuscola per indicare i nomi propri dei documenti e delle figure professionali;
		\item Mancato uso del corsivo nel momento in cui ci si riferisce a nomi propri di documenti, figure professionali o termini tecnici.
	\end{itemize}
	\item \textbf{Errori relativi al glossario:}
	\begin{itemize}
		\item Mancata individuazione di termini che risultano ambigui;
		\item Presenza dell'apice $^G$ in parole che non ne hanno bisogno o già segnate nella medesima sezione.
	\end{itemize}
	\item \textbf{Template e comandi:}
	\begin{itemize}
		\item Assenza dei comandi appositamente definiti per semplificare, velocizzare e standardizzare la stesura dei documenti.
	\end{itemize}
\end{itemize}