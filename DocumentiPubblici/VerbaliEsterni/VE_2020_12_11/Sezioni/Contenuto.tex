\section{Informazioni Generali}
\begin{itemize}
\item \textbf{Luogo:} Incontro \glo{\textit{Slack}};
\item \textbf{Data:} \Data;
\item \textbf{Ora:} 18:00 - 19:00;
\item \textbf{Partecipanti del gruppo:}
	\begin{itemize}
		\item \BL{}; 
		\item \FF{};
		\item \MM{}; 
		\item \PC{};
		\item \TG{};
		\item \TL{};
		\item \VD{}.
	\end{itemize} 
\item \textbf{Segretario:} \PC{}.
\end{itemize}

\section{Ordine del Giorno}
\begin{enumerate}
	\item presentazioni;
	\item motivazioni che hanno portato il gruppo {\Gruppo} alla scelta del \glo{capitolato} \textit{C2};
	\item consigli sulla comunicazione tra gruppo e proponenti;
	\item domande varie del gruppo ai proponenti.
\end{enumerate}

\section{Resoconto}
\subsection{Presentazioni}
I membri del gruppo {\Gruppo} si sono presentati ad uno ad uno ponendo un accento sulla loro situazione accademica. A loro volta i due responsabili dell'azienda proponente, Milo Ertola e Alessandro Maccagnan, hanno fatto una breve presentazione riguardante {\Proponente} e gli altri progetti che seguono al momento.

\subsection{Motivazioni della scelta}
Il gruppo {\Gruppo}, in seguito alla domanda posta dai proponenti, ha spiegato le ragioni che hanno fatto ricadere la scelta del \glo{capitolato} su {\NomeProgetto}, ovvero:
\begin{itemize}
	\item La compatibilità con le preferenze degli altri gruppi, che ha reso possibile evitare di entrare in conflitto con questi ultimi;
	\item L'interesse dei membri del gruppo verso le tecnologie che verranno impiegate per il compimento del progetto, in particolare l'architettura a \glo{microservizi} e i servizi offerti da \glo{\textit{AWS}};
	\item La disponibilità dimostrata fin dall'inizio dai committenti ad offrire il loro supporto qualora il gruppo ne necessitasse per chiarimenti sulle tecnologie da utilizzare e consigli riguardi le scelte più critiche.
\end{itemize}

\subsection{Consigli sulla comunicazione tra gruppo e proponenti}
I proponenti hanno sottolineato fin da subito l'importanza di una efficace comunicazione con il gruppo. Il loro tipico approccio è quello di tenersi in contatto con i gruppi in maniera costante sui vari canali \glo{\textit{Slack}} e di organizzare quando necessario un meeting.
Non vogliono che si approfitti del loro tempo in maniera inappropriata, ovvero ponendo domande la cui risposta potrebbe essere facilmente reperita utilizzando libri di testo o mediante una rapida ricerca su \textit{Google}.
Viene vivamente sconsigliato al gruppo di interrompere la comunicazione con i proponenti per un lungo lasso di tempo perché ciò porta quasi inevitabilmente al fallimento del progetto.

\subsection{Domande varie}
Alla domanda del gruppo riguardante il pagamento del piano di abbonamento per usufruire degli \textit{AWS} viene risposto che i limiti del piano gratuito sono più che sufficienti a soddisfare il fabbisogno di risorse necessarie al sistema da realizzare.

\section{Registro delle decisioni}
\begin{itemize}
  \item \textbf{VE\_\Data.1} Presentazione tra membri e proponenti;
  \item \textbf{VE\_\Data.2} Spiegate le motivazioni della scelta del \glo{capitolato} \textit{C2};
  \item \textbf{VE\_\Data.3} Una frequente comunicazione con il proponente riduce il rischio di fallimento se fatta in modo oculato;
  \item \textbf{VE\_\Data.4} Vista la natura del progetto e le risorse che questo richiede, non sarà necessario pagare per usare i servizi offerti da \glo{\textit{AWS}}.
\end{itemize}




