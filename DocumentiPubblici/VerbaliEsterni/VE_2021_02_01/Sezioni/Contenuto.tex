\section{Informazioni Generali}
\begin{itemize}
\item \textbf{Luogo:} Incontro Zoom;
\item \textbf{Data:} \Data;
\item \textbf{Ora:} 12:00-13:00;
\item \textbf{Partecipanti:}
	\begin{itemize}
		\item \BL{}; 
		\item \FF{};
		\item \MM{};
		\item \TG{};
		\item \TL{};
		\item \VD{};
		\item \VT{}.
	\end{itemize} 
\item \textbf{Segretario:} \MM{}.
\end{itemize}

\section{Ordine del Giorno}
\begin{enumerate}
	\item Chiarimenti dubbi riguardo alla correzione della \textbf{RR}:
	\begin{itemize}
        \item Cambio di versione dei file;
        \item Scelta di adottare modello incrementale;
        \item Utilizzo preventivo a finire e consuntivo;
        \item Considerazione sull'\AdR\;
        \item Insufficiente allineamento norme e obiettivi di qualità nel \PdQ{}.
    \end{itemize}
	\item Consigli da parte del professore per migliorare.
\end{enumerate}

\section{Resoconto}
\subsection{Chiarimenti dubbi riguardo alla correzione della RR:}
\subsubsection{Cambio di versione dei file}
Come detto dal professore, abbiamo adottato un sistema di \glo{versionamento} ibrido creando così confusione nella \glo{repository}. La repository deve contenere solo materiale usufruibile all'esterno o utile per l'avanzamento del progetto da parte del gruppo. Il gruppo deve quindi adottare un metodo univoco per decidere cosa caricare nella repository così da poter risolvere il problema del sistema di versionamento.
2 Esempi sono \textbf{Chromium} e \textbf{LibreOffice}. Il primo carica sempre l'ultima versione stabile e scatta di versione ogni qual volta il prodotto è funzionante avendo così sempre una latest version affidabile, mentre il secondo prevede le modifiche che si andranno a fare e, promettendole al pubblico, andrà a cambiare di versione ogni volta che queste promesse saranno rispettate e rese disponibili all'utenza.
\subsubsection{Scelta di adottare il modello incrementale}
Il professore ci fa notare la difficoltà di applicare un modello incrementale in questo inizio di progetto data anche l'inesperienza del gruppo. Ci invita quindi a rivedere la pianificazione e il modo di lavorare così da poter raggiungere quel modello di sviluppo. Suggerisce in particolare di concentrarsi maggiormente sul breve termine rispetto al lungo termine.
\subsubsection{Utilizzo preventivo a finire e consuntivo}
Il professore sottolinea come l'utilizzo effettuato dal gruppo del preventivo a finire e del consuntivo è un mero calcolo matematico, mentre questo dovrebbe aiutarci a valutare come sia andata la pianificazione e agire in caso qualcosa non abbia funzionato come pensato. Sottolinea nuovamente l'importanza di una buona pianificazione a breve termine.
\subsubsection{Considerazione sull'Analisi dei requisiti}
Il professore descrive come i requisiti di processo non dovrebbe essere scritti all'interno dell'\AdR{} (che è un documento che si può definire "statico"), ma di definirli in un documento più flessibile come le \NdP{}. Evidenzia poi il concetto di "App Web Single Page" che descrive in maniera ideale il capitolato: con questa struttura è molto importante che siano presenti le versione dei browser dove il progetto viene testato.
\subsubsection{Insufficiente allineamento norme e obiettivi di qualità nel Piano di qualifica}
Il professore evidenzia lo scopo degli obiettivi di qualità: distinguere i vari fornitori, in quanto un proponente decide a chi affidare il capitolato non riguardo all'\AdR\, ma secondo gli obiettivi di qualità che vengono dichiarati. Il \textbf{way of working} (\NdP\) deve poi supportare quanto scritto nel \PdQ\, altrimenti questo risulta poco credibile.
\subsection{Consigli da parte del professore per migliorare}
Ginnastica intellettuale: mettere in discussione quanto fatto, capire cosa si è sbagliato per poter correggere gli errori e migliorare il modo di lavoro del gruppo grazie a questo sforzo collaborativo.

\section{Registro delle decisioni}
\begin{itemize}
	\item \textbf{VE\_\Data.1} Cambiare la numerazione relativa alle versioni;
	\item \textbf{VE\_\Data.2} Mantenere il modello incrementale aggiungendo la difesa nel \PdP\;
	\item \textbf{VE\_\Data.3} Concentrarsi sul breve termine per la pianificazione ed il preventivo all'interno del \PdP\;
	\item \textbf{VE\_\Data.4} Spostare i requisiti di processo all'interno delle \NdP\;
	\item \textbf{VE\_\Data.5} Allineare \NdP\ e \PdQ\;
	\item \textbf{VE\_\Data.6} Mettere in discussione quanto fatto fino ad ora per decidere cosa cambiare e cosa mantenere invariato.
\end{itemize}