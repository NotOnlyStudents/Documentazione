\section{Informazioni Generali}
\begin{itemize}
\item \textbf{Luogo:} Incontro \glo{\textit{Slack}};
\item \textbf{Data:} \Data;
\item \textbf{Ora:} 18:00 - 19:00;
\item \textbf{Partecipanti del gruppo:}
	\begin{itemize}
		\item \BL{}; 
		\item \FF{};
		\item \MM{}; 
		\item \PC{};
		\item \TG{};
		\item \TL{};
		\item \VD{}.
	\end{itemize} 
\item \textbf{Segretario:} \PC{}.
\end{itemize}

\section{Ordine del Giorno}
\begin{enumerate}
	\item Discussione sugli attori e sui requisiti individuati;
	\item Discussione sulla strumentazione da usare.
\end{enumerate}

\section{Resoconto}
\subsection{Discussione sugli attori e sui requisiti individuati}
Il gruppo ha esposto ai proponenti i principali attori individuati nel sistema che si andrà a produrre. Il ruolo dell'\glo{Amministratore} non dovrà avere un account all'interno della piattaforma ma interagisce con essa mediante \glo{\textit{AWS}}.\\
Modificati alcuni requisiti individuati come l'inserimento di elementi nella \glo{PLP} e il calcolo di un eventuale sconto. Eliminati alcuni requisiti non ritenuti fondamentali da parte dei proponenti come la presenza del buy-now.

\subsection{Discussione sulla strumentazione da usare}
Il proponente ci ha consigliato \textit{commercetools} come riferimento per l'utilizzo di \glo{\textit{API}} all'interno del progetto. L'idea generale consigliata dai proponenti è \textit{Keep It Simple, Stupid}, strategia da adottare durante tutto lo sviluppo del progetto.

\section{Registro delle decisioni}
\begin{itemize}
  \item \textbf{VE\_\Data.1} Cambiati ed eliminati alcuni casi d'uso e ruoli all'interno dell'\textit{\AdR};
  \item \textbf{VE\_\Data.2} Mantenere il più semplice possibile lo sviluppo del progetto.
\end{itemize}




