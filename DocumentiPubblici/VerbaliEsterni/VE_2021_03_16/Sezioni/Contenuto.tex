\section{Informazioni generali}
\begin{itemize}
\item \textbf{Luogo:} Incontro \glo{Google Meet};
\item \textbf{Data:} \Data;
\item \textbf{Ora:} 17:45 - 18:15;
\item \textbf{Partecipanti:}
	\begin{itemize}
		\item \BL{}; 
		\item \FF{};
		\item \MM{};
		\item \TG{};
		\item \PC{};
		\item \TL{};
		\item \Proponente{}.
	\end{itemize}
\item \textbf{Segretario:} \MM{}.
\end{itemize}

\section{Ordine del giorno}
\begin{enumerate}
	\item Discutere riguardo la fase di progettazione della product baseline.
\end{enumerate}

\section{Resoconto}
\subsection{Discutere riguardo la fase di progettazione della product Baseline}
Il gruppo ha discusso insieme ai proponenti varie proposte per come sviluppare la fase di progettazione. In elenco i principali punti:
\begin{enumerate}
	\item Accorgimenti riguardo l'utlizzo delle \textit{A.P.I.};
	\item Attenzione alla parametrizzazione su enviroment di alcune variabili che vengono utilizzate più volte come i nomi delle tabelle utilizzate da dynamoDB;
	\item Attenzione alla \textit{separation of concerns} durante la progettazione cercando quindi di suddividere i servizi per domini e rendendo possibile la distribuzione di ognuno separatamente;
	\item Ci è stato consigliato di studiare il funzionamento del \glo{\textit{Domain-driven design}}.
\end{enumerate}

\section{Registro delle decisioni}
\begin{itemize}
	\item \textbf{VE\_\Data.1} Correggere quanto segnalato riguardo alle \textit{A.P.I.}; 
	\item \textbf{VE\_\Data.2} Studio dettagliato del \textit{Domain-driven design}.
	\item \textbf{VE\_\Data.3} Basare la progettazione sulla \textit{separation of concerns} e prestare attenzione a suddividere i domini
\end{itemize}
