\section{Informazioni Generali}
\begin{itemize}
\item \textbf{Luogo:} Incontro \textit{Zoom};
\item \textbf{Data:} \Data;
\item \textbf{Ora:} 12:00-13:00;
\item \textbf{Partecipanti:}
	\begin{itemize}
		\item \BL{}; 
		\item \FF{};
		\item \MM{}; 
		\item \PC{};
		\item \TG{};
		\item \TL{};
		\item \VD{}.
		\item Tullio Vardanega;
	\end{itemize} 
\item \textbf{Segretario:} \MM{}.
\end{itemize}

\section{Ordine del Giorno}
\begin{enumerate}
	\item Chiarimenti dubbi riguardo alla correzione della RR:
	\begin{itemize}
        \item Cambio di versione dei file
        \item Scelta di adottare modello incrementale
        \item Utilizzo preventivo a finire e consuntivo
        \item Considerazione sull'Analisi dei Requisiti
        \item Insufficiente allineamento norme e obiettivi di qualità nel PdQ
    \end{itemize}
	\item Consigli da parte del professore per migliorare;
\end{enumerate}

\section{Resoconto}
\subsection{Chiarimenti dubbi riguardo alla correzione della RR:}
\subsubsection{Cambio di versione dei file}
Come detto dal professore, abbiamo adottato un sistema di versionamento ibridio creando così confusione nella repository. La repository deve contenere solo materiale usufruibile all'esterno o utili per l'avanzamento del progetto da parte del gruppo. Il gruppo quindi deve decidere con che criterio decidere cosa caricare nella repository così da poter sistemare il sistema di versionamento.
2 Esempi sono Chromium e LibreOffice. Il primo carica sempre l'ultima versione stabile e scatta di versione ogni qual volta il prodotto è funzionante avendo così sempre una latest version affidabile, mentre il secondo prevede le modifiche che andrà a fare e, promettendole al pubblico, andrà a cambiare di versione ogni volta che queste promesse saranno rispettate e pubblicate.

\subsubsection{Scelta di adottare il modello incrementale}
Il professore ci fa notare la difficoltà di applicare un modello incrementale in questo inizio di progetto data anche l'inesperienza del gruppo. Ci invita quindi a rivedere la pianificazione e il modo di lavorare così da poter raggiungere quel modello di sviluppo. Suggerisce in particolare di concentrarsi maggiormente sul breve termine rispetto al lungo termine.

\subsubsection{Utilizzo preventivo a finire e consuntivo}
Il professore sottolinea come l'utilizzo effettuato dal gruppo del preventivo a finire e del consuntivo è un mero calcolo matematico, mentre questo dovrebbe aiutarci a valutare come sia andata la pianificazione e agire in caso qualcosa non abbia funzionato come pensato. Sottolinea nuovamente l'importanza di una buona pianificazione a breve termine.

\subsubsection{Considerazione sull'Analisi dei Requisiti}
Il professore descrive come i requisiti di processo non dovrebbe essere scritti nell'analisi dei requisiti (che è un documento che si può definire "statico"), ma di definirli in un documento più flessibile come le norme di progetto. Evidenzia poi il concetto di "App Web Single Page" che descrive in maniera ideale il capitolato: con questa struttura è molto importante che siano presenti le versione dei browser dove il progetto viene testato.

\subsubsection{Insufficiente allineamento norme e obiettivi di qualità nel PdQ}
Il professore evidenzia lo scopo degli obiettivi di qualità: distinguere i vari fornitori, in quanto un proponente decide a chi affidare il capitolato non riguardo all'analisi dei requisti, ma secondo gli obiettivi di qualità che vengono dichiarati.Il Way of working (Norme di Progetto) deve poi supportare quanto scritto nel Piano di Qualifica, altrimenti questo risulta poco credibile.

\subsection{Consigli da parte del professore per migliorare}
Ginnastica intellettuale: mettere in discussione quanto fatto, capire cosa si è sbagliato per poter correggere gli errori e migliorare il modo di lavoro del gruppo grazie a questo






