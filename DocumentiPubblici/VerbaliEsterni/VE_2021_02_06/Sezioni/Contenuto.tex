\section{Informazioni generali}
\begin{itemize}
\item \textbf{Luogo:} Incontro \glo{Google Meet};
\item \textbf{Data:} \Data;
\item \textbf{Ora:} 14:00 - 14:30;
\item \textbf{Partecipanti:}
	\begin{itemize}
		\item \BL{}; 
		\item \FF{};
		\item \TG{};
		\item \TL{};
		\item \Proponente{}.
	\end{itemize}
\item \textbf{Segretario:} \TG{}.
\end{itemize}

\section{Ordine del giorno}
\begin{enumerate}
	\item Discussione su come realizzare il Proof of concept;
	\item Consigli per il \glo{deploy} della piattaforma;
	\item Domande su \glo{typescript-eslint};
	\item Resoconto andamento del progetto.
\end{enumerate}

\section{Resoconto}
\subsection{Discussione su come realizzare il Proof of concept}
A seguito di quanto discusso con il \CR{}, il gruppo ha suggerito un'idea per la corretta realizzazione del Proof of concept. La proposta è stata ritenuta valida prestando però attenzione a non entrare troppo nel dettaglio, necessità della futura progettazione di dettaglio e non dell'attuale progettazione architetturale.
\subsection{Consigli per il deploy della piattaforma}
Il gruppo ha chiesto consigli sul metodo di deploy della piattaforma, i proponenti hanno consigliato:
\begin{itemize}
	\item \textbf{Vercel}, complicato per l'autenticazione machine to machine da implementare;
	\item \textbf{Serverless AWS} la soluzione meno complicata;
	\item \textbf{AWS Amplify}, command line interface per gestire da AWS il deploy di un'applicazione full stack.
\end{itemize}
\subsection{Domande su typescript-eslint}
Il gruppo aveva alcuni dubbi sul corretto utilizzo di typescript-eslint, in particolare è stato chiesto quali regole applicare e se associare altri strumenti esterni come Preetier per ottenere migliori risultati. I proponenti hanno specificato di utilizzare regole già fatte come Airbnb o quelle di Microsoft mentre l'utilizzo di strumenti esterni è una possibilità ma non una priorità.
\subsection{Resoconto andamento del progetto}
Il gruppo ha riportato come intende proseguire con lo svolgimento delle varie attività e ha brevemente riassunto la valutazione ottenuta in sede di Revisione dei requisiti. 

\section{Registro delle decisioni}
\begin{itemize}
	\item \textbf{VE\_\Data.1} Il gruppo mantiene come base di partenza la proposta fatta a \Proponente.
   \item \textbf{VE\_\Data.2} Lo strumento per il \glo{deploy} della piattaforma sarà deciso nelle prossime settimane.
   \item \textbf{VE\_\Data.3} Le regole per l'utilizzo di \glo{typescript-eslint} saranno tra quelle consigliateci.
\end{itemize}