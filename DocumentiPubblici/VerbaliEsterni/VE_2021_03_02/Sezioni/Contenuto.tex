\section{Informazioni generali}
\begin{itemize}
\item \textbf{Luogo:} Incontro \glo{Google Meet};
\item \textbf{Data:} \Data;
\item \textbf{Ora:} 17:15 - 18:00;
\item \textbf{Partecipanti:}
	\begin{itemize}
		\item \BL{}; 
		\item \FF{};
		\item \MM{};
		\item \TG{};
		\item \Proponente{}.
	\end{itemize}
\item \textbf{Segretario:} \TG{}.
\end{itemize}

\section{Ordine del giorno}
\begin{enumerate}
	\item Presentazione del \glo{PoC}.
\end{enumerate}

\section{Resoconto}
\subsection{Presentazione del PoC}
Il gruppo ha presentato il \glo{PoC} realizzato fino a quel momento, le principali critiche riportateci sono:
\begin{enumerate}
	\item Serve un maggiore controllo sul flusso del pagamento, ad oggi infatti un client esperto può facilmente modificare il risultato riportato dal gestore dei pagamenti \glo{Stripe};
	\item Attenzione all'utilizzo di await e della programmazione asincrona;
	\item Utilizzare di più \glo{Typescript} assegnando i tipi corretti alle variabili e non any;
	\item Attenzione al funzionamento del linter;
	\item Ci è stato consigliato di cercare i design \glo{pattern} e \glo{framework} più famosi dedicati allo unit testing delle \glo{AWS lambda}.
\end{enumerate}

\section{Registro delle decisioni}
\begin{itemize}
	\item \textbf{VE\_\Data.1} Rivedere flusso del pagamento.
	\item \textbf{VE\_\Data.2} Correggere gli errori nell'utilizzo della programmazione asincrona.
	\item \textbf{VE\_\Data.3} Sostituzione dei tipi alle variabili da any al tipo corretto.
	\item \textbf{VE\_\Data.4} Controllare l'utilizzo di \glo{typescript-eslint}.
	\item \textbf{VE\_\Data.5} Per lo unit testing verranno testati i \glo{framework} \glo{mocha.js} e \glo{jest.js}.
\end{itemize}
