\section{T}
\TermineGlossario{Tag RFID}
\DefinizioneGlossario{
	Si riferisce ad una tecnologia che permette di identificare un oggetto utilizzando onde radio attraverso un segnale elettromagnetico. Consiste solitamente di un microchip, chiamato tag RFID, applicato sull'oggetto da identificare e collegato a un’antenna che invia i segnali.
}
\TermineGlossario{Technology Baseline}
\DefinizioneGlossario{
	Consiste nella \glo{baseline} tecnologica, in cui si definiscono le varie tecnologie, librerie e \glo{framework} che verranno utilizzate.
}
\TermineGlossario{Telegram}
\DefinizioneGlossario{
	Servizio di messaggistica istantanea basato su \glo{cloud}.
}
\TermineGlossario{Teamwork}
\DefinizioneGlossario{
	Lavoro collaborativo tra più membri che punta a raggiungere un obiettivo comune in modo efficace ed efficiente.
}
\TermineGlossario{Tempo di Slack}
\DefinizioneGlossario{
	Il periodo di tempo durante il quale un'\glo{attività} può essere ritardata senza ritardare l'intero progetto di cui fa parte.
}
\TermineGlossario{Testing}
\DefinizioneGlossario{
	Ambiente di sviluppo in cui vengono eseguiti i vari test per controllare che il prodotto soddisfi i requisiti previsti e per individuare eventuali problemi o malfunzionamenti.
}
%\TermineGlossario{Testabilità}
%\DefinizioneGlossario{
%	La testabilità del software è il grado in cui un prodotto software supporta i \glo{test} in un dato contesto di test. Se la testabilità del manufatto software è elevata, è più facile trovare i guasti nel sistema (se ne ha) mediante test.
%}
%\TermineGlossario{Test d'integrazione}
%\DefinizioneGlossario{
%	Processo di verifica dell'interazione tra componenti software.
%}
%\TermineGlossario{Test d'unità}
%\DefinizioneGlossario{
%	\glo{Attività} svolte su singole unità di software.
%}
\TermineGlossario{Test end to end}
\DefinizioneGlossario{
	Metodologia utilizzata per verificare se il flusso di un'applicazione sta funzionando come progettato dall'inizio alla fine.
}
\TermineGlossario{Thread}
\DefinizioneGlossario{
	Suddivisione di un processo in due o più sottoprocessi che vengono eseguiti concorrentemente da un sistema di elaborazione.
}
\TermineGlossario{Tomcat}
\DefinizioneGlossario{
	Server web \glo{open source} per implementare le specifiche JavaServer Pages e Java Servlet, fornendo quindi una piattaforma software per l'esecuzione di applicazioni web sviluppate in linguaggio \glo{Java}.
}
\TermineGlossario{Typescript}
\DefinizioneGlossario{
	Il linguaggio di programmazione \glo{open source} sviluppato da Microsoft. Estende la sintassi di \glo{JavaScript} in modo che qualunque programma scritto in JavaScript sia anche in grado di funzionare con \glo{TypeScript} senza nessuna modifica.
}
\TermineGlossario{Typescript-eslint}
\DefinizioneGlossario{
	Strumento di analisi del codice statico per identificare i modelli problematici trovati nel codice \glo{TypeScript}.
}