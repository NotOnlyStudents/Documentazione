\section{I}
\TermineGlossario{IaaS}
\DefinizioneGlossario{
	Acronimo di \textit{Infrastructure as a service}, offerta di \glo{cloud computing} in cui un venditore fornisce agli utenti l'accesso alle risorse di calcolo, ad esempio server, storage e connessione di rete. Le organizzazioni utilizzano le loro piattaforme e applicazioni all'interno dell'infrastruttura di un provider di servizi.
}
\TermineGlossario{IaaS Kubernetes}
\DefinizioneGlossario{
	\glo{Iaas} su cui \glo{\textit{Kubernetes}} può essere distribuito come servizio che fornisce tale infrastruttura.
}
\TermineGlossario{IDE}
\DefinizioneGlossario{
	Un IDE o ambiente di sviluppo integrato è un software che, in fase di programmazione, supporta i programmatori nello sviluppo del codice sorgente di un programma.
}
\TermineGlossario{Identity Manager}
\DefinizioneGlossario{
	Gestori ddelle identità digitali per l'accesso a servizi web.
}
\TermineGlossario{Identity Provider}
\DefinizioneGlossario{
	Fornitori di identità digitale per l'accesso a servizi web.
}
\TermineGlossario{IEEE 830-1998}
\DefinizioneGlossario{
	Definisce una generica struttura per un documento dei requisiti che deve essere istanziata per ogni specifico sistema.
}
\TermineGlossario{Implementation}
\DefinizioneGlossario{
	Realizzazione concreta di una procedura a partire dalla sua definizione logica.
}
\TermineGlossario{Inspection}
\DefinizioneGlossario{
	E' un processo che ha l'obiettivo di rilevare la presenza di difetti, eseguendo una lettura mirata dell'oggetto di verifica.
}
\TermineGlossario{ISO 8601}
\DefinizioneGlossario{
	\glo{Standard} internazionale per la rappresentazione di date e orari. I valori delle date e degli orari sono organizzati dal più significativo a quello di minor importanza. Ogni valore deve avere un numero fisso di cifre che può essere raggiunto aggiungendo zeri.
}
\TermineGlossario{ISO/IEC 9126}
\DefinizioneGlossario{Con la sigla ISO/IEC 9126 si individua una serie di normative e linee guida preposte a descrivere un \glo{modello} di \glo{qualità} del software. Il modello propone un approccio alla qualità in modo tale che le società di software possano migliorare l'organizzazione, i processi e la qualità del prodotto sviluppato.}
\TermineGlossario{ISO/IEC 12207:1995}
\DefinizioneGlossario{
	Lo \glo{standard} ISO/IEC 12207 stabilisce un processo di ciclo di vita del software, compreso processi ed \glo{attività} relative alle specifiche ed alla configurazione di un sistema. Ha come obiettivo principale quello di fornire una struttura comune che permetta a clienti, fornitori, sviluppatori, tecnici, manager di usare gli stessi termini e lo stesso linguaggio per definire gli stessi processi.
}
\TermineGlossario{\hypertarget{spice}{ISO/IEC 15504}}
\DefinizioneGlossario{
	Insieme di documenti tecnici che permettono di valutare oggettivamente la \glo{qualità} di un processo software a fini migliorativi. Fornisce delle valutazioni sui processi in maniera ripetibile, oggettiva e comparabile.
}
\TermineGlossario{Issue}
\DefinizioneGlossario{
	Funzionalità messa a disposizione da \glo{\textit{GitHub}} con cui si intende una qualsiasi idea, miglioramento, compito o errore su cui lavorare. É possibile assegnare issues a più membri del gruppo di lavoro ed etichettarle per cercarle più rapidamente.
}
\TermineGlossario{Issue Tracking System}
\DefinizioneGlossario{
	Un Issue Tracking System o ITS, indica un sistema informatico il cui scopo è tracciare richieste di assistenza, miglioramenti, problemi e non solo.
}