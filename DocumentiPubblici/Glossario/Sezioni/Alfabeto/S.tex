\section{S}
\TermineGlossario{Scalabilità orizzontale}
\DefinizioneGlossario{
	 Capacità di un sistema di aumentare o diminuire in modo dinamico le risorse a sua disposizione in funzione delle necessità e disponibilità. Ai fini di questo progetto si parla di scalabilità di carico orizzontale che consiste nell'aggiunta di più istanze dell'applicazione in modo da funzionare come una singola unità logica.
}
\TermineGlossario{SEO}
\DefinizioneGlossario{
	Acronimo di Search Engine Optimization, indica la pratica di adattare e modificare i siti perché compaiano tra i primi risultati di ricerca.
}
\TermineGlossario{Serverless}
\DefinizioneGlossario{
	Network la cui gestione non viene incentrata su dei server, ma viene dislocata fra i vari utenti che utilizzano il network stesso, quindi il lavoro necessario di gestione del network viene eseguito dagli stessi utilizzatori.
} 
\TermineGlossario{Sistema Qualità}
\DefinizioneGlossario{
	Struttura organizzativa, responsabilità, procedure, risorse messe in atto per il perseguimento della qualità.
}
\TermineGlossario{Slack}
\DefinizioneGlossario{
	Software che rientra nella categoria degli strumenti di collaborazione aziendale. Utilizzato per inviare messaggi in modo istantaneo tra i membri del gruppo e il proponente.
}
\TermineGlossario{Socket}
\DefinizioneGlossario{
	Indica un'astrazione software progettata per utilizzare delle \glo{API} \glo{standard} e condivise per la trasmissione e la ricezione di dati attraverso una rete oppure come meccanismo di IPC.
}
%\TermineGlossario{SPICE}
%\DefinizioneGlossario{
%	Si rimanda alla voce \glo{\hyperlink{spice}{ISO/IEC 15504}} del \textit{\Glossario}.
%}
\TermineGlossario{SQL}
\DefinizioneGlossario{
	Linguaggio standardizzato per database basati sul modello relazionale (RDBMS).
}
\TermineGlossario{SSG - Server Static Generation}
\DefinizioneGlossario{
	Descrive il processo di compilazione e rendering di un sito web o app a tempo di compilazione.
}
\TermineGlossario{SSR - Server Side Rendering}
\DefinizioneGlossario{
	Descrive il processo di compilazione e rendering di un sito web o app a tempo di esecuzione.
}
\TermineGlossario{Stadi temporali}
\DefinizioneGlossario{
	Arco di tempo in cui verranno svolte determinate \glo{attività}.
}
\TermineGlossario{Staging}
\DefinizioneGlossario{
	Arco di tempo successivo a quello di sviluppo del software che consiste nell'assemblare tutte le componenti e testarle in un server. Se il software possiede il comportamento desiderato allora può passare alla fase di produzione.
}
\TermineGlossario{Stakeholder}
\DefinizioneGlossario{
	All'interno di un progetto rappresenta un "portatore di interessi" (investitore, fornitore, cliente tipo...) o in generale chiunque possa avere un'opinione rilevante ai fini realizzativi del prodotto.
}
\TermineGlossario{Standard}
\DefinizioneGlossario{
	Insieme di norme, raccomandazioni o specifiche puramente convenzionali, prestabilite da un'autorità e riconosciute tali con lo scopo di rappresentare una base di riferimento per la realizzazione di tecnologie fra loro compatibili e interoperabili.
}
\TermineGlossario{Stripe}
\DefinizioneGlossario{
	Piattaforma globale per la gestione dei pagamenti completamente integrata e basata su \glo{cloud}, consente di integrare diversi metodi di pagamento in un sito web di \glo{e-commerce}, in un marketplace o in un'app.
}
\TermineGlossario{Subscriber}
\DefinizioneGlossario{
	Destinatario di un messaggio.
}
\TermineGlossario{Sviluppo locale}
\DefinizioneGlossario{
	\glo{Fase} iniziale che consiste nel creare le diverse componenti scrivendo codice. Queste poi verranno testate e assemblate per comporre il prodotto.
}
\TermineGlossario{Swift}
\DefinizioneGlossario{
	 Linguaggio di programmazione orientato agli oggetti per sistemi macOS, iOS, watchOS, tvOS e Linux, presentato da Apple.
}