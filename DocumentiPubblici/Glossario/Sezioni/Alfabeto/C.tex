\section{C}
\TermineGlossario{Capitolato}
\DefinizioneGlossario{
	Documento tecnico redatto dal committente che specifica le caratteristiche del prodotto richiesto e i suoi vincoli.
}
\TermineGlossario{Checkout}
\DefinizioneGlossario{
	Processo che porta l'utente al pagamento dei prodotti inseriti nel carrello.
}
\TermineGlossario{\hypertarget{cicloDeming}{Ciclo di Deming}}
\DefinizioneGlossario{
	Il ciclo di Deming (\glo{\textit{PDCA}}: Plan, Do, Check, Act) è un metodo di gestione iterativo composto da quattro fasi. Viene utilizzato per il controllo e il miglioramento continuo dei processi e dei prodotti.
}
\TermineGlossario{Cloud} 
\DefinizioneGlossario{
 Con il termine inglese cloud, in alternativa anche cloud computing, si indica un paradigma di erogazione di servizi offerti \glo{on demand} da un fornitore ad un cliente finale attraverso la rete Internet. 
}
\TermineGlossario{Cluster} 
\DefinizioneGlossario{
	Insieme degli elementi connessi tra loro tramite una rete telematica.
}
\TermineGlossario{CMS}
\DefinizioneGlossario{
	Sistema di gestione dei contenuti, è uno strumento software, installato su un server web, il cui compito è facilitare la gestione dei contenuti di siti web, svincolando il webmaster da conoscenze tecniche specifiche di programmazione Web.
}
\TermineGlossario{Configuration item}
\DefinizioneGlossario{
	Identifica tutti gli elementi che devono essere sottoposti alla gestione di configurazione.
}
\TermineGlossario{Consuntivo}
\DefinizioneGlossario{
	Rendiconto finale dei risultati di un dato periodo di lavoro.
}
\TermineGlossario{Contentful}
\DefinizioneGlossario{
	\glo{CMS} di riferimento per il progetto.
}
\TermineGlossario{Cross-platform}
\DefinizioneGlossario{
	Può essere riferito ad un linguaggio di programmazione, ad un'applicazione software o ad un dispositivo hardware che funziona su più di un sistema o appunto, piattaforma.
}
\TermineGlossario{CSS}
\DefinizioneGlossario{
	Abbreviazione di \textit{Cascading Style Sheets}, in italiano fogli di stile a cascata) è un linguaggio usato per definire la formattazione di documenti \glo{\textit{HTML}}, \textit{XHTML} e \textit{XML}.
}
\TermineGlossario{CSV}
\DefinizioneGlossario{
	Formato di file utilizzato per rappresentare dati in forma tabellare all'interno di un file di testo.
}

