\section{D}
\TermineGlossario{D3.js}
\DefinizioneGlossario{
	Libreria \glo{JavaScript} per la creazione di grafici per la visualizzazione dinamica dei dati.
}

\TermineGlossario{Dashboard}
\DefinizioneGlossario{
	Postazione o pagina web basata su una tecnologia che visualizza informazioni relative a un business, raccolte in \glo{real-time} da varie fonti nel settore. I dati vengono visualizzati in tempo reale nella pagina usando grafici, riepiloghi e liste.
}

\TermineGlossario{Deployment/Deploy}
\DefinizioneGlossario{
	Il significato più comune del termine deploy o deployment in informatica è la consegna o rilascio al cliente, con relativa installazione e messa in funzione, di una applicazione o di un sistema software.
}

\TermineGlossario{\hypertarget{designPattern}{Design Pattern}}
\DefinizioneGlossario{
	Soluzione progettuale generale ad un problema ricorrente. È la descrizione di un modello da applicare per risolvere un problema, il quale può presentarsi in diverse situazioni durante la progettazione e lo sviluppo del software.
}

\TermineGlossario{Diagrammi di Gantt}
\DefinizioneGlossario{
	Strumento di supporto alla gestione dei progetti che rappresenta graficamente la pianificazione e l'avanzamento delle attività in funzione del tempo. Tutte le parti interessate sono a conoscenza dei compiti e delle rispettive scadenze.
}

\TermineGlossario{Diagrammi UML}
\DefinizioneGlossario{
	Permettono la modellazione della struttura statica e del comportamento dinamico di un sistema. Il sistema è rappresentato come un insieme di oggetti (moduli software) che collaborano e reagiscono a eventi esterni per eseguire attività a beneficio dei clienti.
}

%\TermineGlossario{Distanza Euclidea}
%\DefinizioneGlossario{
%	Misura la lunghezza del segmento avente per estremi i due punti considerati.
%}
\TermineGlossario{Docker}
\DefinizioneGlossario{
	Progetto \glo{open source} per automatizzare la distribuzione di applicativi, come contenitori portabili e autosufficienti, che possono essere eseguiti nel \glo{cloud} o in locale.
}

\TermineGlossario{Domain-Driven Design}
\DefinizioneGlossario{
	Approccio dello sviluppo del software che risolve problemi complessi connettendo l'implementazione ad un modello in evoluzione.
}
