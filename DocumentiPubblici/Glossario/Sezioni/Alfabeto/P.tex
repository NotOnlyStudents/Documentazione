\section{P}
\TermineGlossario{PaaS}
\DefinizioneGlossario{
	Acronimo di Platform as a Service, è un ambiente di sviluppo e distribuzione completo nel \glo{cloud}. Come \glo{IasS} include l'infrastruttura ma anche strumenti di sviluppo, sistemi di gestione dei database e molto altro. 
}
\TermineGlossario{Pattern}
\DefinizioneGlossario{
	Si rimanda alla voce \glo{\hyperlink{designPattern}{Design Pattern}} del \textit{\Glossario}.
}
%\TermineGlossario{PDCA}
%\DefinizioneGlossario{
%	Si rimanda alla voce \glo{\hyperlink{cicloDeming}{ciclo di Deming}} del \textit{\Glossario}.
%}
\TermineGlossario{PDP}
\DefinizioneGlossario{
	Acronimo di Product Detail Pages, si tratta di un insieme di pagine dell'applicazione {\NomeProgetto}, ognuna di esse contiene informazioni
	riguardo un singolo prodotto/servizio in vendita nell'\glo{e-commerce}.
}
\TermineGlossario{PLP}
\DefinizioneGlossario{
	Acronimo di Product Listing Page, si riferisce ad una pagina del sito che contiene l'elenco di tutti i prodotti/servizi venduti nell'\glo{e-commerce}, derivato da un'istanza di {\NomeProgetto}.
}
\TermineGlossario{Powerup}
\DefinizioneGlossario{
	Nel campo dei videogiochi, è un oggetto mostrato a video che conferisce una particolare abilità temporanea al giocatore o ne incrementa le statistiche quando raccolto.
}
%\TermineGlossario{Precondizione}
%\DefinizioneGlossario{
%	Condizione o predicato che deve essere sempre vero prima di accedere allo scenario di un caso d'uso specifico o dell'esecuzione di una sezione di codice.
%}
%\TermineGlossario{Preventivo}
%\DefinizioneGlossario{
%	Indicazione dei costi preventivati dal gruppo per la realizzazione del progetto.
%}
\TermineGlossario{Processo}
\DefinizioneGlossario{
	Insieme di attività che lavorano insieme, correlate e coese. Queste attività trasformano ingressi(bisogni) in uscite(prodotti) secondo regole date(vincoli) consumando risorse nel farlo.
}
%\TermineGlossario{Prodotto}
%\DefinizioneGlossario{
%	Entità software progettata per essere rilasciata all'utente.
%}
\TermineGlossario{Product Baseline}
\DefinizioneGlossario{
	\glo{Baseline} del prodotto successiva alla \glo{\textit{Technology Baseline}} in cui viene mostrato il design definitivo del prodotto finale e illustra l'architettura del progetto.
}
\TermineGlossario{\hypertarget{produzione}{Production}}
\DefinizioneGlossario{
	Ambiente in cui il prodotto viene creato, gestito e modificato con l'obiettivo di renderlo pronto alla produzione.
}
\TermineGlossario{Production-ready}
\DefinizioneGlossario{
	Il prodotto deve trovarsi potenzialmente in uno stato di possibile messa in produzione nel mercato reale.
}
%\TermineGlossario{Produzione}
%\DefinizioneGlossario{
%	Si rimanda alla voce \glo{\hyperlink{produzione}{Production}} del \textit{\Glossario}.
%}
\TermineGlossario{Proof of Concept}
\DefinizioneGlossario{
	Realizzazione incompleta o abbozzata di un determinato progetto o metodo, allo scopo di provarne la fattibilità o dimostrare la fondatezza di alcuni principi o concetti costituenti.
}
%\TermineGlossario{Proponente}
%\DefinizioneGlossario{
%	Ente o azienda che compie l'atto di proporre il capitolato d'appalto per un progetto.
%}
%\TermineGlossario{Postcondizione}
%\DefinizioneGlossario{
%	Condizione o predicato che deve essere sempre vero dopo l'esecuzione di una sezione di codice o di un'operazione in una specifica formale. 
%}
%\TermineGlossario{Prototipo}
%\DefinizioneGlossario{
%	Modello approssimato o parziale idea del prodotto implementata in fase di sviluppo del software che soddisfa parte delle esigenze degli \glo{stakeholder}.
%}
\TermineGlossario{Publisher}
\DefinizioneGlossario{
	Mittente di un messaggio.
}
%\TermineGlossario{Punti di interesse}
%\DefinizioneGlossario{
%	Luoghi che potrebbero risultare utili o interessanti alle entità presenti nel sistema.
%}
\TermineGlossario{Python}  
\DefinizioneGlossario{
	Linguaggio di programmazione dinamico ad alto livello orientato agli oggetti utilizzabile per molti tipi di sviluppo software. Offre un forte supporto all'integrazione con altri linguaggi e programmi.
}