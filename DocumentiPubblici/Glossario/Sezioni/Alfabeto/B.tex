\section{B}
\TermineGlossario{Back end}
\DefinizioneGlossario{
	Una applicazione back end è un programma con il quale l'utente interagisce indirettamente, in generale attraverso l'utilizzo di una applicazione \glo{front end}. In una struttura client/server il back end è il server. 
}

\TermineGlossario{Back Office}
\DefinizioneGlossario{
	Parte di un'azienda che comprende tutte le \glo{attività} proprie dell'azienda che contribuiscono alla sua gestione operativa.
}

\TermineGlossario{Baseline}
\DefinizioneGlossario{
	Nell'ingegneria del software, una baseline costituisce un'istanza di configurazione che consolida lo stato di avanzamento del prodotto ad un dato istante, ovvero l'avanzamento incrementale che è parte del prodotto finale.
}

\TermineGlossario{BigData}
\DefinizioneGlossario{
	Indica genericamente una raccolta di dati informativi così estesa in termini di volume, velocità e varietà da richiedere tecnologie e metodi analitici specifici per l'estrazione di valore o conoscenza.
}

\TermineGlossario{Blockchain}
\DefinizioneGlossario{
	Struttura dati condivisa e "immutabile", definita come un registro digitale le cui voci sono raggruppate in "blocchi" la cui integrità è garantita dall'uso della crittografia. È immutabile in quanto di norma, il suo contenuto una volta scritto non è più né modificabile né eliminabile, a meno di non invalidare l'intera struttura.
}

\TermineGlossario{Business Logic}
\DefinizioneGlossario{
	Si riferisce a tutta quella logica di elaborazione (sotto forma di codice sorgente) che rende operativa un'applicazione. 
}
