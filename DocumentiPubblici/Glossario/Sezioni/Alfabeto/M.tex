\section{M}
\TermineGlossario{Media-type}
\DefinizioneGlossario{
	Formato di un file o di contenuti trasmessi in internet.
}
\TermineGlossario{Message Queue Telemetry Transport (MQTT)}
\DefinizioneGlossario{
	Protocollo ISO \glo{standard} (ISO/IEC PRF 20922) di messaggistica posizionato in cima a TCP/IP. È stato progettato per le situazioni in cui è richiesto un basso impatto e dove la banda è limitata.
}
\TermineGlossario{Metriche}
\DefinizioneGlossario{
	Ogni tipo di misurazione oggettiva di un sistema software, processo o documentazione.
}
\TermineGlossario{Microservizi}
\DefinizioneGlossario{
	 Approccio architetturale alla realizzazione di applicazioni. Quello che distingue l'architettura basata su microservizi dagli approcci monolitici tradizionali è la suddivisione del prodotto nelle sue funzioni di base. Ciascuna funzione, denominata servizio, può essere compilata e implementata in modo indipendente. In questo modo i singoli servizi possono funzionare, o meno, senza compromettere gli altri.
}
\TermineGlossario{Milestone}
\DefinizioneGlossario{
	Indica una tappa nel tempo che garantisce il raggiungimento di un obiettivo con importanza strategica nei tempi preventivati.
}
\TermineGlossario{Multiplayer}
\DefinizioneGlossario{
	Nell'ambito dei videogiochi è la modalità di utilizzo in cui più persone partecipano al gioco nello stesso tempo, per mezzo di un solo apparecchio con più periferiche oppure usando diversi apparecchi in connessione.
}
\TermineGlossario{MVC}
\DefinizioneGlossario{ 
	Acronimo di Model-View-Controller, si tratta di un pattern architetturale molto diffuso nello sviluppo di sistemi software, in grado di separare la logica di presentazione dei dati dalla logica di business.
}