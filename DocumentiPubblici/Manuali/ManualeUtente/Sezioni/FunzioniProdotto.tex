\section{Funzionalità della piattaforma}
\NomeProgetto\ permette all'utente di usufruire delle funzionalità principali che un \glo{e-commerce} deve presentare. Di seguito viene approfondito ciò che l'utente, acquirente o venditore, può fare all'interno della piattaforma.
\subsection{Funzionalità acquirente}\label{FunzionalitàAcq}
\subsubsection*{Registrazione}
All'interno di \NomeProgetto\ l'utente può innanzitutto registrarsi sulla piattaforma con le proprie credenziali. La registrazione permette all'utente di accedere al proprio account personale in un momento successivo in cui poter visualizzare tutti gli ordini effettuati, oltre a poter modificare le proprie informazioni personali.
\subsubsection*{Ricerca di un prodotto}
Effettuato l'accesso, l'acquirente potrà visualizzare tutti i prodotti presenti all'interno della piattaforma, al momento disponibili o meno. L'elenco dei prodotti può essere ordinato in base a:
\begin{itemize}
	\item \textbf{Disponibilità:} i prodotti disponibili saranno visualizzati per primi;
	\item \textbf{Prezzo crescente:} i prodotti con prezzo più basso saranno visualizzati per primi;
	\item \textbf{Prezzo decrescente:} i prodotti con prezzo più alto saranno visualizzati per primi.
\end{itemize}
Sull'elenco di prodotti presenti l'acquirente può effettuare una serie di ricerche per poter visualizzare esclusivamente i prodotti che:
\begin{itemize}
	\item Appartengano a specifiche \textbf{categorie};
	\item Siano al momento \textbf{disponibili};
	\item Il cui prezzo sia inferiore e/o superiore a un valore.
\end{itemize}
Per ogni prodotto è ovviamente possibile accedere alla schermata descrittiva dello stesso in cui visualizzare le informazioni di interesse.
\subsubsection*{Acquisto di prodotti}
I prodotti di interesse per l'utente possono essere inseriti all'interno del carrello e da qui è possibile procedere con l'acquisto degli stessi. All'interno del carrello l'acquirente visualizza un riepilogo dei prodotti inseriti e può inserire un nuovo indirizzo di spedizione o selezionare uno degli indirizzi di spedizione salvati precedentemente, oltre a poter modificare o eliminare gli stessi.
\subsubsection*{Pagamento}
Per il pagamento dei prodotti del carrello, l'acquirente non deve inserire all'interno della piattaforma nessun dato sensibile relativo ai metodi di pagamento messi a disposizione. L'effettivo processo di pagamento viene completamente gestito tramite un servizio di terze parti specializzato nei pagamenti online che garantisce all'utente la sicurezza degli stessi.
\subsection{Funzionalità venditore}\label{FunzionalitàVend}
\subsubsection*{Gestione dei prodotti}
Il venditore registrato alla piattaforma può inserire tutti i suoi prodotti con le loro caratteristiche, con la possibilità di scegliere quali prodotti mettere in evidenza nella schermata principale dell'acquirente. I prodotti inseriti possono essere gestiti dall'utente, sarà infatti possibile modificare o eliminare prodotti inseriti precedentemente. Nella piattaforma \NomeProgetto\ il venditore ha la possibilità di gestire le categorie assegnabili ai vari prodotti, creando nuove categorie, modificando o eliminando categorie esistenti. Anche il venditore può visualizzare l'elenco di tutti i prodotti presenti con la possibilità di visualizzare solo i prodotti inseriti in evidenza.
\subsubsection*{Gestione degli ordini}
Tutti gli ordini effettuati dagli acquirenti vengono visualizzati nella schermata principale del venditore; quando un acquirente effettua un ordine, quest'ultimo verrà inserito nell'elenco. Quando l'ordine sarà stato evaso il venditore potrà indicarlo come tale, in questo modo avrà sempre un controllo su quali ordini sono ancora da gestire e quali sono stati completati.