\section{Introduzione}
\label{Introduzione}
\subsection{Scopo del documento}
Questo documento ha lo scopo di illustrare all'utente della piattaforma \NomeProgetto\ le funzionalità usufruibili, mettendo a sua disposizione tutte le indicazioni per il corretto utilizzo. Il documento è da considerarsi incompleto in quanto i contenuti verranno aggiornati e modificati avanzando con lo sviluppo del prodotto.
\subsection{Scopo del prodotto} 
Il progetto {\NomeProgetto} ha come scopo quello di rendere disponibile un servizio \glo{e-commerce} agli utenti della stesso, mettendo a disposizione tutte le principali funzionalità che una piattaforma di shopping online dovrebbe avere. 
Per \Gruppo\ è importante che tutti gli utenti possano utilizzare correttamente la piattaforma e allo stesso tempo avere una user experience soddisfacente, per questo motivo si è prestata particolare attenzione all'accessibilità di \NomeProgetto\ da parte di qualunque tipologia di utente.
\subsection{Glossario}
All'interno del documento sono presenti termini che possono presentare significato ambiguo a seconda del contesto, viene fornito un glossario consultabile nell'appendice. I termini che possono assumere un significato ambiguo sono indicati da 'G' ad apice. 
\subsection{Riferimenti}
\begin{itemize}
	\item \textbf{Capitolato d'appalto C2:}\\
	\url{https://www.math.unipd.it/~tullio/IS-1/2020/Progetto/C2.pdf}
\end{itemize}