\section{Introduzione}
\label{introduzione}
\subsection{Scopo del documento}
Questo documento ha lo scopo di illustrare all'utente della piattaforma tutte le funzionalità usufruibili, mettendo a sua disposizione tutte le indicazioni per il corretto utilizzo. Il documento è da considerarsi incompleto in quanto i contenuti verranno aggiornati e modificati avanzando con lo sviluppo del prodotto.
\subsection{Scopo del prodotto} 
Il progetto {\NomeProgetto} ha come scopo quello di rendere disponibile un servizio \glo{e-commerce} sfruttando tutti i vantaggi di un'architettura \glo{Serverless}:
\begin{itemize}
	\item Gli sviluppatori potranno concentrare la propria attenzione sullo sviluppo del prodotto finale invece di focalizzarsi sulla gestione e sul funzionamento di server e di runtime, che siano nel \glo{cloud} o in locale;
	\item Comodità nel costruire un insieme di chiamate \glo{asincrone} che rispondono a diversi clienti contemporaneamente;
	\item Minori costi di sviluppo e di produzione;
	\item Semplicità nel suddividere il progetto in un insieme di \glo{microservizi}.
\end{itemize}
\subsection{Glossario}
All'interno del documento sono presenti termini che possono presentare significato ambiguo a seconda del contesto, viene fornito un glossario consultabile nell'appendice. I termini che possono assumere un significato ambiguo sono indicati da 'G' ad apice. 
