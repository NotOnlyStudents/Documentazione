\section{Backend}

\subsection{Descrizione generale}
Le funzionalità previste dalla piattaforma hanno permesso di individuare i seguenti domini:
\begin{itemize}
	\item \textbf{Products;}
	\item \textbf{Categories;}
	\item \textbf{Payments;}
	\item \textbf{Orders;}
	\item \textbf{Cart;}
	\item \textbf{Addresses;}
	\item \textbf{Users.}
\end{itemize}
I domini sono stati uniti riflettendo sulle loro interdipendenze al fine di individuare dei \glo{microservizi} utili per procedere con lo sviluppo, in particolare sono stati identificati i microservizi:
\begin{itemize}
	\item \textbf{Products-categories service;}
	\item \textbf{Payments-orders service;}
	\item \textbf{Carts service;}
	\item \textbf{Addresses service;}
	\item \textbf{Users service.}
\end{itemize}
Lo sviluppo di singoli servizi permette una maggiore possibilità di sviluppo in parallelo dei singoli domini grazie alla loro struttura e all'agilità degli aggiornamenti e la possibilità di gestire i dati in modo decentralizzato.

\subsection{Struttura dei microservizi}
\begin{figure}[H]
	\centering
	\includegraphics[scale=0.4]{Immagini/Backend/layer.png}
	\caption{layered architecture}
	\label{fig:layer}
\end{figure}
La struttura di ogni singolo servizio si basa su una Layered Architecture dove i componenti sono organizzati in vari strati che comunicano tra loro, il principale vantaggio è che risulta molto più semplice procedere ai test attraverso mock. \\
Andando ad analizzare il singolo microservizio abbiamo:
\begin{center}
	\begin{minipage}{0.3\textwidth}
		\centering
		\includegraphics[scale=0.78]{Immagini/Backend/Amazon-DynamoDB.png}
		\captionof{figure}{DynamoDB}
	\end{minipage}
	\begin{minipage}{0.3\textwidth}
		\centering
		\includegraphics[scale=0.19]{Immagini/Backend/AWSLambda.png}
		\captionof{figure}{Lambda}
	\end{minipage}
	\begin{minipage}{0.3\textwidth}
		\centering
		\includegraphics[scale=0.26]{Immagini/Backend/Gateway.png}
		\captionof{figure}{API Gateway}
	\end{minipage}
\end{center}
\begin{itemize}
	\item \textbf{DynamoDB:} per gestire i vari dati;
	\item \textbf{Lambda:} le funzioni che lavorano sui dati del database;
	\item \textbf{API Gateway:} riceve e gestisce le richieste del client richiamando le lambda e restituendone i risultati.
\end{itemize}

\subsubsection{Products-categories service}
\begin{figure}[H]
	\centering
	\includegraphics[scale=0.4]{Immagini/Backend/AWSProductsCategories.png}
	\caption{Products-categories service}
	\label{fig:ProductCategories}
\end{figure}

\subsubsection{Payments-orders service}
\begin{figure}[H]
	\centering
	\includegraphics[scale=0.4]{Immagini/Backend/AWSPaymentOrders.png}
	\caption{Payments-orders service}
	\label{fig:Payment-orders}
\end{figure}

\subsubsection{Carts service}
\begin{figure}[H]
	\centering
	\includegraphics[scale=0.4]{Immagini/Backend/AWSCart.png}
	\caption{Carts service}
	\label{fig:Cart}
\end{figure}

\subsubsection{Addresses service}
\begin{figure}[H]
	\centering
	\includegraphics[scale=0.4]{Immagini/Backend/AWSAddresses.png}
	\caption{Addresses service}
	\label{fig:Adresses}
\end{figure}
\subsubsection{Users service}
\begin{figure}[H]
	\centering
	\includegraphics[scale=0.7]{Immagini/Backend/AWSUserService.png}
	\caption{Users service}
	\label{fig:Users}
\end{figure} 

\subsection{Struttura del backend}
\begin{figure}[H]
	\centering
	\includegraphics[scale=0.4]{Immagini/Backend/AWSArchitecture.png}
	\caption{Architettura backend}
	\label{fig:backend}
\end{figure}
Per slegare i \glo{microservizi} tra di loro vengono utilizzate due strategie. La prima consiste nel gestire le chiamate asincrone attraverso SNS il servizio che fa da event broker su AWS, eliminando così lo sviluppo di integrazioni tra microservizi e limitandosi a quelle per il message broker.
La seconda invece è pensata per eliminare le chiamate sincrone tra microservizi, come quelle di validazione. Per far ciò si utilizza la firma digitale legata a un microservizio per verificare l'integrità delle richieste del client, sostituendo così le varie integrazioni necessarie a un semplice check attraverso la chiave pubblica del microservizio.

\subsubsection{Diagrammi di sequenza}
Di seguito sono riportati due diagrammi di sequenza per due principali operazioni:
\begin{itemize}
	\item l'acquisto di un carrello;
	\item la ricezione di un evento tramite Stripe Hooks.
\end{itemize}
\begin{figure}[H]
	\centering
	\includegraphics[scale=0.5]{Immagini/Backend/Diagrammiseq.png}
	\caption{Diagrammi di sequenza}
	\label{fig:Diagrammiseq}
\end{figure}