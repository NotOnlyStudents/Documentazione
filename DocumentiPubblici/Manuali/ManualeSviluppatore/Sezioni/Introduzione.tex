\section{Introduzione}
\label{introduzione}
\subsection{Scopo del documento}
Questo documento ha lo scopo di mostrare approfonditamente requisiti di sistema, tecnologie, architettura e struttura in generale della piattaforma \NomeProgetto. Il documento è da considerarsi incompleto in quanto i contenuti verranno aggiornati e modificati avanzando con lo sviluppo del prodotto.
\subsection{Scopo del prodotto} 
Il progetto {\NomeProgetto} ha come scopo quello di rendere disponibile un servizio \glo{e-commerce} sfruttando tutti i vantaggi di un'architettura \glo{Serverless}:
\begin{itemize}
	\item gli sviluppatori potranno concentrare la propria attenzione sullo sviluppo del prodotto finale invece di focalizzarsi sulla gestione e sul funzionamento di server e di runtime, che siano nel \glo{cloud} o in locale;
	\item comodità nel costruire un insieme di chiamate \glo{asincrone} che rispondono a diversi clienti contemporaneamente;
	\item minori costi di sviluppo e di produzione;
	\item semplicità nel suddividere il progetto in un insieme di \glo{microservizi}.
\end{itemize}
\subsection{Glossario}
All'interno del documento sono presenti termini che possono presentare significato ambiguo a seconda del contesto, viene fornito un glossario consultabile nell'appendice. I termini che possono assumere un significato ambiguo sono indicati da 'G' ad apice. 

\subsection{Riferimenti}
\subsubsection{Normativi}
\begin{itemize}
	\item \textbf{Capitolato d'appalto C2:}\\
	\url{https://www.math.unipd.it/~tullio/IS-1/2020/Progetto/C2.pdf}
\end{itemize}
\subsubsection{Informativi} \label{riferimenti_info}
\begin{itemize}
	\item \textbf{Slide \CR\  "Principi di programmazione SOLID":} \\ 
	\url{
	https://www.math.unipd.it/%7Ercardin/swea/2021/SOLID%20Principles%20of%20Object-Oriented%20Design_4x4.pdf}
	\item \textbf{Slide \CR\  "Stili Architetturali: Monolite e Microservizi":} \\ 
	\url{https://www.math.unipd.it/~rcardin/sweb/2021/L03.pdf}
\end{itemize}