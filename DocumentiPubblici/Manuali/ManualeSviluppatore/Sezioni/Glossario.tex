\newpage
\section{Glossario}\label{Glossario}
\appendix
\subsection*{A}
\TermineGlossario{API}
\DefinizioneGlossario{	
	Con Application Programming Interface (API) si indica un insieme di procedure per lo svolgimento di un dato compito. Tale termine spesso designa le librerie software disponibili in un linguaggio di programmazione.
}
\TermineGlossario{Asincrone}
\DefinizioneGlossario{
	I protocolli asincroni consentono la trasmissione per singolo carattere, nella trasmissione asincrona l'intervallo temporale tra il bit finale di un carattere e il bit iniziale del carattere successivo è indefinito.
}
\subsection*{B}
\TermineGlossario{Back end}
\DefinizioneGlossario{
	Una applicazione back end è un programma con il quale l'utente interagisce indirettamente, in generale attraverso l'utilizzo di una applicazione \glo{front end}. In una struttura client/server il back end è il server. 
}
\subsection*{C}
\TermineGlossario{CLI}
\DefinizioneGlossario{
	CLI (command line interface) indica una interfaccia a riga di comando o anche console. A volte detta semplicemente riga di comando o prompt dei comandi, è un tipo di interfaccia utente caratterizzata da un'interazione testuale tra utente ed elaboratore.
}
\TermineGlossario{Cloud}
\DefinizioneGlossario{
	Con il termine inglese cloud, in alternativa anche cloud computing, si indica un paradigma di erogazione di servizi offerti on demand da un fornitore ad un cliente finale attraverso la rete internet. 
}
\TermineGlossario{Cluster} 
\DefinizioneGlossario{
	Insieme degli elementi connessi tra loro tramite una rete telematica.
}
\subsection*{E}
\TermineGlossario{E-commerce}
\DefinizioneGlossario{
	Piattaforma web attraverso la quale commercianti e acquirenti vengono in contatto.
}
\TermineGlossario{Event Broker}
\DefinizioneGlossario{
	È un software middleware, un'appliance o SaaS utilizzato per trasmettere eventi tra produttori di eventi e consumatori in un modello di pubblicazione-sottoscrizione.
	Supportano lo sviluppo di app e microservizi cloud nativi e event-driven.
}
\subsection*{F}
\TermineGlossario{Framework}
\DefinizioneGlossario{
	Sistema che consente di estendere le funzionalità del linguaggio di programmazione su cui è basato, fornendo allo sviluppatore una struttura coerente al fine di effettuare azioni e comandi in modo semplice e veloce.
}
\subsection*{H}
\TermineGlossario{HTML/HTML5}
\DefinizioneGlossario{
	Linguaggio di markup per la strutturazione delle pagine web.
}
\subsection*{J}
\TermineGlossario{JSON}
\DefinizioneGlossario{
	È una rappresentazione testuale e priva di schemi di dati strutturati basata su elenchi ordinati. Sebbene JSON sia derivato da JavaScript, è supportato in modo nativo o tramite librerie nella maggior parte dei linguaggi di programmazione. JSON è comunemente, ma non esclusivamente, utilizzato per scambiare informazioni tra client web e server web.
}
\subsection*{M}
\TermineGlossario{Message Broker}
\DefinizioneGlossario{
	È un software che consente ad applicazioni, sistemi e servizi di comunicare tra loro e di scambiare informazioni. Il broker di messaggi esegue questa operazione traducendo i messaggi tra protocolli di messaggistica formali. Ciò consente a servizi interdipendenti di "dialogare" direttamente tra loro, anche se sono stati scritti in lingue diverse o implementati su piattaforme diverse.
}
\TermineGlossario{Microservizio}
\DefinizioneGlossario{
	Approccio architetturale alla realizzazione di applicazioni. Quello che distingue l'architettura basata su microservizi dagli approcci monolitici tradizionali è la suddivisione del prodotto nelle sue funzioni di base. Ciascuna funzione, denominata servizio, può essere compilata e implementata in modo indipendente. In questo modo i singoli servizi possono funzionare, o meno, senza compromettere gli altri.
}
\TermineGlossario{Mock}
\DefinizioneGlossario{
	Sono degli oggetti simulati che riproducono il comportamento degli oggetti reali in modo controllato. Un programmatore crea un oggetto mock per testare il comportamento di altri oggetti, reali, ma legati ad un oggetto inaccessibile o non implementato. Allora quest'ultimo verrà sostituito da un mock.
}
\subsection*{N}
\TermineGlossario{NoSQL}
\DefinizioneGlossario{
	Database caratterizzati dal fatto di non utilizzare il modello relazionale, di solito usato dalle basi di dati tradizionali. 
}
\subsection*{O}
\TermineGlossario{Open source}
\DefinizioneGlossario{
	Software non protetto da copyright, segue la filosofia della produzione collaborativa e dell'accesso pubblico al codice sorgente.
}
\subsection*{R}
\TermineGlossario{RESTful}
\DefinizioneGlossario{
	È un'interfaccia di programmazione delle applicazioni conforme ai vincoli dell'architettura REST, abbreviazione per Representational State Transfer, è un insieme di principi architetturali atti alla creazione di servizi web. Si espongono operazioni per manipolare i servizi, appoggiandosi a protocolli web esistenti, tipicamente HTTP.
}
\subsection*{S}
\TermineGlossario{Serverless}
\DefinizioneGlossario{
	Network la cui gestione non viene incentrata su dei server, ma viene dislocata fra i vari utenti che utilizzano il network stesso, quindi il lavoro necessario di gestione del network viene eseguito dagli stessi utilizzatori.
}