\section{Setup della piattaforma}\label{Setup}
\subsection{Requisiti minimi}
Per il deploy della piattaforma \NomeProgetto\ è necessario disporre di un account AWS valido per ottenere le credenziali da includere nella configurazione del profilo del \glo{framework} serverless.
Il sistema dove verrà effettuato il deploy deve avere installato:
\begin{itemize}
	\item Npm;
	\item NodeJS;
	\item Serverless Framework.
\end{itemize}
Di seguito sono riportati i browser che supportano la piattaforma.
\begin{itemize}
	\item \textbf{Mozilla Firefox}: versione per desktop 86.0.0 e successive, versione mobile 86.1.1 e successive;
	\item \textbf{Microsoft Edge}: versione per desktop 88.0.705.68 e successive, versione mobile 46.01.4.5140 e successive;
	\item \textbf{Google Chrome}: versione per desktop 88.0.4324.182 e successive, versione mobile 88.0.4324.181 e successive;
	\item \textbf{Safari}: versione per desktop 14.0.2 e successive, versione mobile 14.4 e successive.
\end{itemize}

\subsection{Deploy}
Per il deploy della piattaforma, seguire i seguenti passi:
\begin{itemize}
	\item Aprire una \glo{CLI} e posizionarsi nella cartella dove sono contenuti i file;
	\item Installare le dipendenze usando il comando \textbf{npm install};
	\item Creare una nuova applicazione serverless lanciando il comando \textbf{serverless};
	\item Iniziare il deploy della piattaforma \NomeProgetto\ lanciando il comando \textbf{serverless deploy}.
\end{itemize}

\subsection{Test}
L'analisi statica del codice redatto viene effettuata dal linter ESLint.\\
Per i test di unità è stato impiegato il framework Jest.js.