\section{Setup della piattaforma}\label{Setup}
\subsection{Requisiti minimi}
Per il deploy della piattaforma \NomeProgetto\ è necessario disporre di un account AWS valido per ottenere le credenziali da includere nella configurazione del profilo del framework serverless.
Il sistema dove verrà effettuato il deploy deve avere installato:
\begin{itemize}
	\item Npm;
	\item NodeJS;
	\item Serverless framework.
\end{itemize}
Di seguito sono riportati i sistemi operativi e i browser che supportano la piattaforma.
\begin{itemize}
	\item \textbf{Sistemi operativi}:
	\item \textbf{Browser}:
\end{itemize}

\subsection{Deploy}
Per il deploy della piattaforma, seguire i seguenti passi:
\begin{itemize}
	\item Aprire una \glo{CLI} e posizionarsi nella cartella dove sono contenuti i file;
	\item Installare le dipendenze usando il comando \textbf{npm install};
	\item Creare una nuova applicazione serverless lanciando il comando \textbf{serverless};
	\item Iniziare il deploy della piattaforma \NomeProgetto\ lanciando il comando \textbf{serverless deploy}.
\end{itemize}

\subsection{Test}
L'analisi statica del codice redatto viene effettuata dal linter EsLint.\\
Per i test di unità è stato impiegato il framework Jest.js.