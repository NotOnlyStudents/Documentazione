% Estensioni email
\subsection{UC24 - Visualizzazione messaggio di errore in caso di email già registrata nella piattaforma}
\label{UC24}
Visualizzazione messaggio di errore che segnala all'attore il tentativo di registrazione con un'email già utilizzata.
\begin{itemize}
    \item \textbf{Attori Primari:} Utente autenticato; Utente non autenticato.
    \item \textbf{Precondizione:} L'utente non autenticato sta cercando di registrarsi con un'email già registrata.
    \item \textbf{Postcondizione:} All'utente non autenticato viene impedita la registrazione e segnalata la causa.
    \item \textbf{Scenario Principale:} Visualizzazione messaggio di errore che segnala la già esistente della e-mail durante la registrazione e la impedisce.
\end{itemize}

\subsection{UC25 - Visualizzazione messaggio di errore in caso di email non registrata}
\label{UC25}
Visualizzazione messaggio di errore che segnala all'utente il tentativo di accesso alla piattaforma con una email non registrata.
\begin{itemize}
    \item \textbf{Attori Primari:} Utente autenticato; Utente non autenticato.
    \item \textbf{Precondizione:} L'utente non autenticato tenta di accedere alla piattaforma con un email non registrata.
    \item \textbf{Postcondizione:} All'utente non autenticato viene impedito l'accesso e segnalata la causa.
    \item \textbf{Scenario Principale:} Visualizzazione messaggio di errore che segnala il tentativo di accesso con un'email non presente nella piattaforma e lo impedisce.
\end{itemize}

\subsection{UC26 - Visualizzazione messaggio di errore in caso di email o password errata}
\label{UC26}
Visualizzazione messaggio di errore che segnala all'utente il tentativo di accesso alla piattaforma con un email o una password errata.
\begin{itemize}
    \item \textbf{Attori Primari:} Utente autenticato; Utente non autenticato.
    \item \textbf{Precondizione:} L'utente non autenticato tenta di accedere alla piattaforma con un email o una password errata.
    \item \textbf{Postcondizione:} All'utente non autenticato viene impedito l'accesso e segnalata la causa.
    \item \textbf{Scenario Principale:} Visualizzazione messaggio di errore che segnala il tentativo di accesso con un'email o una password errata e lo impedisce.
\end{itemize}

\subsection{UC27 - Visualizzazione messaggio di errore in caso di email non valida}
\label{UC27}
Visualizzazione messaggio di errore che segnala all'utente che l'email inserita non rispetta il formato di un indirizzo email.
\begin{itemize}
    \item \textbf{Attori Primari:} Utente autenticato; Utente non autenticato.
    \item \textbf{Precondizione:} L'utente generico ha inserito un'email nel formato sbagliato.
    \item \textbf{Postcondizione:} All'utente non autenticato viene segnalato che l'email non è nel formato esatto.
    \item \textbf{Scenario Principale:} Visualizzazione messaggio di errore che segnala l'inserimento di un'email non nel formato corretto.
\end{itemize}

% Estensioni password
\subsection{UC28 - Visualizzazione messaggio di errore in caso di password troppo debole}
\label{UC28}
Visualizzazione messaggio di errore che segnala all'utente il tentativo di registrazione della password troppo debole che non rispetta le condizioni minime.
\begin{itemize}
    \item \textbf{Attori Primari:} Utente autenticato; Utente non autenticato.
    \item \textbf{Precondizione:} L'utente non autenticato sta cercando di impostare una password troppo debole che non rispetta le condizioni minime.
    \item \textbf{Postcondizione:} All'utente non autenticato viene impedita la registrazione e segnalata la causa.
    \item \textbf{Scenario Principale:} Visualizzazione messaggio di errore che segnala il non rispetto delle condizioni minime della password e ne blocca la registrazione.
\end{itemize}

\subsection{UC29 - Visualizzazione messaggio di errore in caso di password e password di conferma diverse}
\label{UC29}
Visualizzazione messaggio di errore che segnala all'utente la non corrispondenza della password e della password di conferma inserite.
\begin{itemize}
    \item \textbf{Attori Primari:} Utente autenticato; Utente non autenticato.
    \item \textbf{Precondizione:} L'utente non autenticato inserisce una password di conferma che non corrisponde con quella inserita come password da registrare.
    \item \textbf{Postcondizione:} All'utente non autenticato viene impedita la registrazione della password e viene segnalata la causa.
    \item \textbf{Scenario Principale:} Visualizzazione messaggio di errore che segnala la non corrispondenza tra la password di conferma e quella inserita come password e ne blocca la registrazione.
\end{itemize}

\subsection{UC30 - Visualizzazione messaggio campo dati obbligatorio non inserito}
\label{UC30}
Visualizzazione messaggio che segnala all'utente il mancato inserimento di un campo dati obbligatorio.
\begin{itemize}
    \item \textbf{Attori Primari:} Utente autenticato; Utente non autenticato.
    \item \textbf{Precondizione:} L'attore non ha inserito un campo dati obbligatorio.
    \item \textbf{Postcondizione:} All'attore viene impedita la continuazione dell'azione e segnalata la causa.
    \item \textbf{Scenario Principale:} Visualizzazione messaggio che segnala all'utente il mancato inserimento di un campo dati obbligatorio e blocca la continuazione dell'azione.
\end{itemize}

% Estensioni carrello
\subsection{UC31 - Visualizzazione messaggio di errore prodotto non disponibile}
\label{UC31}
L'utente autenticato o non autenticato richiede un prodotto o la quantità di prodotto che non è disponibile.
\begin{itemize}
    \item \textbf{Attori Primari:} Utente autenticato; Utente non autenticato.
    \item \textbf{Precondizione:} L'utente richiede di aggiungere al carrello un prodotto o una quantità di prodotto non disponibile. 
    \item \textbf{Postcondizione:} All'utente viene impedita l'aggiunta al carrello e segnalata la causa.
    \item \textbf{Scenario Principale:}
        \begin{itemize}
            \item L'utente richiede di aggiungere al carrello un prodotto o una quantità di prodotto non disponibile
            \item viene scartata la modifica, il carrello rimane invariato
            \item viene visualizzato un visualizzato un errore che indica la non disponibilità del prodotto
        \end{itemize}
\end{itemize}

% Estensioni info personali
\subsection{UC32 - Visualizzazione messaggio di errore in caso che il file non sia del tipo \glo{media-type}}
\label{UC32}
Visualizzazione messaggio di errore che segnala al venditore che il file non è di tipo media-type.
\begin{itemize}
    \item \textbf{Attori Primari:} Venditore.
    \item \textbf{Precondizione:} L'attore inserisce un file di tipo non media-type.
    \item \textbf{Postcondizione:} All'attore viene impedito il caricamento del file e viene segnalata la causa.
    \item \textbf{Scenario Principale:} Visualizzazione messaggio di errore che segnala che il file non è del tipo corretto e ne blocca il caricamento.
\end{itemize}

% Estensioni dashboard
\subsection{UC33 - Visualizzazione messaggio di errore in caso di quantità minore o uguale a zero}
\label{UC33}
Visualizzazione messaggio di errore che segnala all'utente autenticato o non autenticato che la quantità di prodotto inserita è minore o uguale a zero.
\begin{itemize}
    \item \textbf{Attori Primari:} Utente autenticato; Utente non autenticato.
    \item \textbf{Precondizione:} L'utente autenticato o non autenticato inserisce una quantità minore o uguale a zero.
    \item \textbf{Postcondizione:} All'utente autenticato o non autenticato viene impedito l'operazione e viene segnalata la causa.
    \item \textbf{Scenario Principale:} Visualizzazione messaggio di errore che segnala che la quantità inserita è minore o uguale a zero e blocca l'operazione che stava eseguendo.
\end{itemize}

\subsection{UC34 - Visualizzazione avviso in caso di prodotto selezionato per l'eliminazione ma non ancora evaso in alcuni ordini}
\label{UC34}
Visualizzazione avviso che segnala al venditore che il prodotto che vuole eliminare è presente all'interno di un ordine che non è ancora stato evaso.
\begin{itemize}
    \item \textbf{Attori Primari:} Venditore.
    \item \textbf{Precondizione:} Il venditore elimina un prodotto ordinato ma non ancora evaso.
    \item \textbf{Postcondizione:} Al venditore viene segnalata la causa dell'avviso.
    \item \textbf{Scenario Principale:} Visualizzazione avviso che segnala che il prodotto eliminato è presente in un ordine che non è ancora stato evaso.
\end{itemize}
