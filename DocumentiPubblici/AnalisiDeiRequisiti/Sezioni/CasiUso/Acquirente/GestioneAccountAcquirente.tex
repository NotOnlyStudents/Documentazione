%%%%%%%%%%%%%%%%%%%%%%%%%%%%%%%%%%%%%%%%%%%%%%%%%%%%%%%%%%%%%%%%%%%%%%%%%%%%%%%%%%%%%%%%%%%%%%%%%%%%%%%%%%%%%%%%%

\UC{Modifica informazioni acquirente}
L'acquirente può modificare le sue informazioni personali.
\begin{itemize}
    \item \textbf{Attori primari:} acquirente;
    \item \textbf{Precondizione:} l'acquirente si trova nella vista di modifica informazioni personali;
    \item \textbf{Postcondizione:} l'acquirente ha aggiornato le sue informazioni personali;
    \item \textbf{Scenario principale:} l'acquirente ha selezionato l'azione per la modifica delle informazioni personali e, dopo aver compiuto la modifica, potrà confermarle attraverso l'azione di salvataggio. Le informazioni che si possono modificare sono:
    \begin{itemize}
        \item (\actualUC.1) - Modifica del nome;
        \item (\actualUC.1) - Modifica del cognome;
        \item (\actualUC.3) - Modifica dell'indirizzo email.
        \item (\actualUC.4) - Inserimento nuova password.
        \item (\actualUC.5) - Inserimento conferma nuova password.
    \end{itemize}
    In seguito ci sarà la conferma delle modifiche compiute e le informazioni personali verranno modificate.
    \item \textbf{Scenari alternativi:}
    \begin{enumerate}[label=\lett]
        \item L'acquirente non conferma le modifiche effettuate, perciò non verranno modificate le informazioni personali.
    \end{enumerate}
    \item \textbf{Estensioni:} 
    \begin{enumerate}[label=\lett] 
        \item L'acquirente inserisce una password di conferma diversa da quella inserita nel campo di inserimento nuova password. In questo caso:
        \begin{itemize}
            \item (UC) - Verrà visualizzato il messaggio di errore di password e password di conferma diverse;
            \item Verrà impedita la modifica delle informazioni.
        \end{itemize}
    \end{enumerate}
\end{itemize}

\resetSubUC

\subUC{Modifica del nome}
L'acquirente modifica il proprio nome.
\begin{itemize}
    \item \textbf{Attori primari:} acquirente;
    \item \textbf{Precondizione:} l'acquirente si trova nella schermata di modifica informazioni personali e vuole modificare il proprio nome;
    \item \textbf{Postcondizione:} l'acquirente ha aggiornato il proprio nome sulla piattaforma;
    \item \textbf{Scenario principale:} l'acquirente si trova nella schermata di modifica informazioni personali e vuole modificare il proprio nome attraverso le seguenti azioni:
        \begin{itemize}
            \item Si posiziona nel campo di inserimento del nome dove è presente quello attualmente utilizzata;
            \item Inserisce quello nuova o modifica quello attuale.
        \end{itemize}
    \item \textbf{Scenari alternativi:} 
    \begin{enumerate}[label=\lett]
        \item L'acquirente non modifica il proprio nome attuale e, perciò, non cambierà.
    \end{enumerate}
    \item \textbf{Estensioni:} 
    \begin{enumerate}[label=\lett]
        \item L'acquirente elimina il proprio nome attuale e non ne inserisce uno nuovo. In questo caso:
        \begin{itemize}
            \item (UC) - Verrà visualizzato il messaggio di errore campo dati obbligatorio non inserito;
            \item Verrà impedita la modifica delle informazioni.
        \end{itemize}
        \item L'acquirente modifica il proprio nome inserendogli dei caratteri non alfabetici. In questo caso:
        \begin{itemize}
            \item (UC) - Verrà visualizzato il messaggio di errore caratteri non alfabetici non permessi;
            \item Verrà impedita la modifica delle informazioni.
        \end{itemize}
    \end{enumerate}
\end{itemize}

\subUC{Modifica del cognome}
L'acquirente modifica il proprio cognome.
\begin{itemize}
    \item \textbf{Attori primari:} acquirente;
    \item \textbf{Precondizione:} l'acquirente si trova nella schermata di modifica informazioni personali e vuole modificare il proprio cognome;
    \item \textbf{Postcondizione:} l'acquirente ha aggiornato il proprio cognome sulla piattaforma;
    \item \textbf{Scenario principale:} l'acquirente si trova nella schermata di modifica informazioni personali e vuole modificare il proprio cognome attraverso le seguenti azioni:
        \begin{itemize}
            \item Si posiziona nel campo di inserimento del cognome dove è presente quello attualmente utilizzata;
            \item Inserisce quello nuova o modifica quello attuale.
        \end{itemize}
    \item \textbf{Scenari alternativi:} 
    \begin{enumerate}[label=\lett]
        \item L'acquirente non modifica il proprio cognome attuale e, perciò, non cambierà.
    \end{enumerate}
    \item \textbf{Estensioni:} 
    \begin{enumerate}[label=\lett]
        \item L'acquirente elimina il proprio cognome attuale e non ne inserisce uno nuovo. In questo caso:
        \begin{itemize}
            \item (UC) - Verrà visualizzato il messaggio di errore campo dati obbligatorio non inserito;
            \item Verrà impedita la modifica delle informazioni.
        \end{itemize}
        \item L'acquirente modifica il proprio cognome inserendogli dei caratteri non alfabetici. In questo caso:
        \begin{itemize}
            \item (UC) - Verrà visualizzato il messaggio di errore caratteri non alfabetici non permessi;
            \item Verrà impedita la modifica delle informazioni.
        \end{itemize}
    \end{enumerate}
\end{itemize}

\subUC{Modifica dell'indirizzo email}
L'acquirente vuole modificare la propria email.
\begin{itemize}
    \item \textbf{Attori primari:} acquirente;
    \item \textbf{Precondizione:} l'acquirente si trova nella schermata di modifica informazioni personali e vuole modificare il proprio indirizzo email;
    \item \textbf{Postcondizione:} l'acquirente ha aggiornato la sua email;
    \item \textbf{Scenario principale:} l'acquirente si trova nella schermata di modifica informazioni personali e vuole modificare la propria email attraverso le seguenti azioni:
        \begin{itemize}
            \item Si posiziona nel campo di inserimento dell'email dove è presente quella attualmente utilizzata;
            \item Inserisce quella nuova o modifica quella attuale.
        \end{itemize}
    \item \textbf{Scenari alternativi:} 
    \begin{enumerate}[label=\lett]
        \item L'acquirente non modifica l'attuale indirizzo email utilizzato e, perciò, non cambierà.
    \end{enumerate}
    \item \textbf{Estensioni:} 
    \begin{enumerate}[label=\lett]
        \item L'acquirente modifica la propria email con una già utilizzata nella piattaforma. In questo caso:
        \begin{itemize}
            \item (UC) - Verrà visualizzato il messaggio di errore in caso di cambio email con una già utilizzata nella piattaforma;
            \item Verrà impedita la modifica delle informazioni.
        \end{itemize}
        \item L'acquirente modifica la propria email con una non valida. In questo caso:
        \begin{itemize}
            \item (UC) - Verrà visualizzato un messaggio di errore in caso di email non valida;
            \item Verrà impedita la modifica delle informazioni.
        \end{itemize}
        \item L'acquirente elimina la propria email attuale e non ne inserisce una nuova. In questo caso:
        \begin{itemize}
            \item (UC) - Verrà visualizzato il messaggio di errore campo dati obbligatorio non inserito;
            \item Verrà impedita la modifica delle informazioni.
        \end{itemize}
    \end{enumerate}
\end{itemize}

\subUC{Inserimento nuova password}
L'acquirente vuole modificare la propria password.
\begin{itemize}
    \item \textbf{Attori primari:} acquirente;
    \item \textbf{Precondizione:} l'acquirente si trova nella schermata di modifica informazioni personali e vuole inserire una nuova password;
    \item \textbf{Postcondizione:} l'acquirente ha inserito la nuova password;
    \item \textbf{Scenario principale:} l'acquirente si trova nella schermata di modifica informazioni personali e vuole modificare la propria password attraverso le seguenti azioni:
    \begin{itemize}
        \item Si posiziona nel campo di inserimento vuoto della nuova password;
        \item Inserisce la nuova password;
    \end{itemize}
    \item \textbf{Scenari alternativi:}
    \begin{enumerate}[label=\lett]
        \item L'acquirente non inserisce alcuna password e lascia il campo di inserimento nuova password vuoto. In questo caso non verrà modificata la password.
    \end{enumerate}
    \item \textbf{Estensioni:}
    \begin{enumerate}[label=\lett]
        \item L'acquirente inserisce una password troppo debole. In questo caso:
        \begin{itemize}
            \item (UC) - Verrà visualizzato il messaggio di errore di password troppo debole;
            \item Verrà impedita la modifica delle informazioni.
        \end{itemize}
    \end{enumerate}
\end{itemize}

\subUC{Inserimento conferma nuova password}
L'acquirente vuole modificare la propria password e deve confermare quella nuova da inserire.
\begin{itemize}
    \item \textbf{Attori primari:} acquirente;
    \item \textbf{Precondizione:} l'acquirente si trova nella schermata di modifica informazioni personali e vuole inserire una nuova password;
    \item \textbf{Postcondizione:} l'acquirente ha inserito la conferma della nuova password;
    \item \textbf{Scenario principale:} l'acquirente si trova nella schermata di modifica informazioni personali, ha già inserito la nuova password e la deve confermare attraverso le seguenti azioni:
    \begin{itemize}
        \item Si posiziona nel campo di inserimento vuoto della conferma nuova password;
        \item Inserisce la stessa password che è stata inserita nel campo conferma password.
    \end{itemize}
    \item \textbf{Scenari alternativi:}
    \begin{enumerate}
        \item L'acquirente non inserisce alcuna password e lascia il campo di inserimento conferma nuova password vuoto. In questo caso non verrà modificata la password.
    \end{enumerate}
\end{itemize}

%%%%%%%%%%%%%%%%%%%%%%%%%%%%%%%%%%%%%%%%%%%%%%%%%%%%%%%%%%%%%%%%%%%%%%%%%%%%%%%%%%%%%%%%%%%%%%%%%%%%%%%%%%%%

\UC{Eliminazione account}
L'acquirente può eliminare il proprio account.
\begin{itemize}
    \item \textbf{Attori primari:} acquirente;
    \item \textbf{Precondizione:} l'acquirente si trova nella schermata personale e ha selezionato l'azione di eliminazione del proprio account;
    \item \textbf{Postcondizione:} l'account dell'acquirente non è più presente nella piattaforma;
    \item \textbf{Scenario principale:} l'acquirente si trova nella schermata personale e ha selezionato l'azione di eliminazione del proprio account. In seguito verrà visualizzato un messaggio di conferma dell'operazione e, se l'acquirente conferma, viene eliminato l'account;
    \item \textbf{Scenari alternativi:}
    \begin{enumerate}[label=\lett]
        \item Se non conferma viene riportato alla propria pagina personale e l'account non viene eliminato.
    \end{enumerate}.
\end{itemize}
