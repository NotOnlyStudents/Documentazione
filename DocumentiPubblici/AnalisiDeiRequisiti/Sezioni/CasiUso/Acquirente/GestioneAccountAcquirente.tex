%%%%%%%%%%%%%%%%%%%%%%%%%%%%%%%%%%%%%%%%%%%%%%%%%%%%%%%%%%%%%%%%%%%%%%%%%%%%%%%%%%%%%%%%%%%%%%%%%%%%%%%%%%%%%%%%%

\UC{Modifica informazioni}
L'acquirente vuole modificare le sue informazioni personali.
\begin{itemize}
    \item \textbf{Attori primari:} acquirente;
    \item \textbf{Precondizione:} l'acquirente si trova nella vista di modifica informazioni personali;
    \item \textbf{Postcondizione:} l'acquirente ha aggiornato le sue informazioni personali;
    \item \textbf{Scenario principale:} l'acquirente ha premuto sull'azione per la modifica delle informazioni personali e, dopo aver compiuto la modifica, potrà confermarle attraverso l'azione di salvataggio. Le informazioni che si possono modificare sono:
    \begin{itemize}
        \item (\actualUC.1) - Inserimento nuovo indirizzo email.
        \item (\actualUC.2) - Inserimento nuova password.
        \item (\actualUC.3) - Inserimento conferma nuova password.
    \end{itemize}
    In seguito ci sarà la conferma delle modifiche compiute e le informazioni personali verrano modificate.
    \item \textbf{Scenario alternativo:} l'acquirente non conferma le modifiche effettuate, perciò non verranno modificate le informazioni personali.
\end{itemize}
\resetSubUC

\subUC{Inserimento nuovo indirizzo email}
L'acquirente vuole modificare la propria email.
\begin{itemize}
    \item \textbf{Attori primari:} acquirente;
    \item \textbf{Precondizione:} l'acquirente si trova nella schermata di modifica informazioni personali e vuole inserire un nuovo indirizzo email;
    \item \textbf{Postcondizione:} l'acquirente ha aggiornato la sua email;
    \item \textbf{Scenario principale:} l'acquirente si trova nella schermata di modifica informazioni personali e vuole modificare la propria email attraverso le seguenti azioni:
        \begin{itemize}
            \item Si posiziona nel campo di inserimento della nuova email dove è presente quella attualmente utilizzata;
            \item Inserisce quella nuova o modifica quella attuale.
        \end{itemize}
    \item \textbf{Scenario alternativo:} l'acquirente non modifica l'attuale indirizzo email utilizzato, perciò non cambierà;
    \item \textbf{Estensioni:} 
     \begin{enumerate}[label=\lett]
    \item l'acquirente inserisce un'email già utilizzata nella piattaforma. In questo caso:
    \begin{itemize}
        \item (UC) - Verrà visualizzato il messaggio di errore in caso di email già registrata nella piattaforma;
        \item Verrà impedita la modifica delle informazioni.
    \end{itemize}
    \item l'acquirente inserisce un'email non valida. In questo caso:
    \begin{itemize}
        \item (UC) - Verrà visualizzato un messaggio di errore in caso di email non valida;
        \item Verrà impedita la modifica delle informazioni.
    \end{itemize}
    \item l'acquirente non inserisce l'email e lascia il campo di inserimento vuoto. In questo caso:
    \begin{itemize}
        \item (UC) - Verrà visualizzato il messaggio di errore campo dati obbligatorio non inserito;
        \item Verrà impedita la modifica delle informazioni.
    \end{itemize}
\end{enumerate}
\end{itemize}

\subUC{Inserimento nuova password}
L'acquirente vuole modificare la propria password.
\begin{itemize}
    \item \textbf{Attori primari:} acquirente;
    \item \textbf{Precondizione:} l'acquirente si trova nella schermata di modifica informazioni personali e vuole inserire una nuova password;
    \item \textbf{Postcondizione:} l'acquirente ha inserito la nuova password;
    \item \textbf{Scenario principale:} l'acquirente si trova nella schermata di modifica informazioni personali e vuole modificare la propria password attraverso le seguenti azioni:
        \begin{itemize}
            \item Si posiziona nel campo di inserimento vuoto della nuova password;
            \item La scrive;
            \item Inserisce la stessa password nel campo di conferma nuova password.
        \end{itemize}
    \item \textbf{Scenario alternativo:} L'acquirente non inserisce alcuna password e lascia il campo di inserimento nuova password vuoto. In questo caso non verrà modificata la password;
    \item \textbf{Estensioni:} L'acquirente inserisce una password troppo debole. In questo caso:
    \begin{enumerate}[label=\lett]
        \item (UC) - Verrà visualizzato il messaggio di errore di password troppo debole;
        \item Verrà impedita la modifica delle informazioni.
    \end{enumerate}
\end{itemize}

\subUC{Inserimento conferma nuova password}
L'acquirente vuole modificare la propria password e deve confermare quella nuova da inserire.
\begin{itemize}
    \item \textbf{Attori primari:} acquirente;
    \item \textbf{Precondizione:} l'acquirente si trova nella schermata di modifica informazioni personali e vuole inserire una nuova password;
    \item \textbf{Postcondizione:} l'acquirente ha inserito la conferma della nuova password;
    \item \textbf{Scenario principale:} l'acquirente si trova nella schermata di modifica informazioni personali, ha già inserito la nuova password e la deve confermare attraverso le seguenti azioni:
        \begin{itemize}
            \item Si posiziona nel campo di inserimento vuoto della conferma nuova password;
            \item Inserisce la stessa password che è stata inserita nel campo conferma password.
        \end{itemize}
    \item \textbf{Scenario alternativo:} l'acquirente non inserisce alcuna password e lascia il campo di inserimento conferma nuova password vuoto. In questo caso non verrà modificata la password.
    \item \textbf{Estensioni:} \begin{enumerate}[label=\lett] 
    \item l'acquirente inserisce una password di conferma diversa da quella inserita nel campo di inserimento nuova password. In questo caso:
    \begin{itemize}
        \item (UC) - Verrà visualizzato il messaggio di errore di password e password di conferma diverse;
        \item Verrà impedita la modifica delle informazioni.
    \end{itemize}
\end{enumerate}
\end{itemize}

%%%%%%%%%%%%%%%%%%%%%%%%%%%%%%%%%%%%%%%%%%%%%%%%%%%%%%%%%%%%%%%%%%%%%%%%%%%%%%%%%%%%%%%%%%%%%%%%%%%%%%%%%%%%

\UC{Eliminazione account}
L'acquirente può eliminare il proprio account.
\begin{itemize}
    \item \textbf{Attori Primari:} Acquirente.
    \item \textbf{Precondizione:} L'acquirente si trova nella propria pagina personale (UC3.2.1) e ha selezionato l'azione di cancellazione del proprio account.
    \item \textbf{Postcondizione:} L'account dell'acquirente non è più presente nella piattaforma
    \item \textbf{Scenario principale:} L'acquirente vuole eliminare il proprio account e compie le seguenti operazioni:
    \begin{itemize}
        \item Preme sull'azione di cancellazione del proprio account.
        \item Viene visualizzato un messaggio di conferma dell'operazione.
        \item Se conferma viene eliminato l'account.
    \end{itemize}
    \item \textbf{Scenario alternativo:} Se non conferma viene riportato alla propria pagina personale e l'account non viene eliminato.
\end{itemize}
