%%%%%%%%%%%%%%%%%%%%%%%%%%%%%%%%%%%%%%%%%%%%%%%%%%%%%%%%%%%%%%%%%%%%%%%%%%%%%%%%%%%%%%%%%%%%%%%%%%%%%%%%%%%%%%%%%

\UC{Modifica informazioni acquirente}
L'acquirente vuole modificare le sue informazioni personali.
\begin{itemize}
    \item \textbf{Attori primari:} Acquirente.
    \item \textbf{Precondizione:} L'acquirente si trova nella vista di modifica informazioni personali.
    \item \textbf{Postcondizione:} L'acquirente ha aggiornato le sue informazioni personali.
    \item \textbf{Scenario Principale:} L'acquirente ha premuto sull'azione per la modifica delle informazioni personali e, dopo aver compiuto la modifica, potrà confermarle attraverso l'azione di salvataggio. Le informazioni che si possono modificare sono:
    \begin{itemize}
        \item (\actualUC.1) - Inserimento nuovo indirizzo email.
        \item (\actualUC.2) - Inserimento nuova password.
        \item (\actualUC.3) - Inserimento conferma nuova password.
    \end{itemize}
    In seguito ci sarà la conferma delle modifiche compiute e le informazioni personali verrano modificate.
    \item \textbf{Scenario Alternativo:} L'acquirente non conferma le modifiche effettuato e, perciò, non verranno modificate le informazioni personali.
\end{itemize}
\resetSubUC

\subUC{Inserimento nuovo indirizzo email}
L'acquirente vuole modificare la propria email.
\begin{itemize}
    \item \textbf{Attori primari:} Acquirente.
    \item \textbf{Precondizione:} L'acquirente si trova nella vista di modifica informazioni personali e vuole inserire un nuovo indirizzo email.
    \item \textbf{Postcondizione:} L'acquirente ha aggiornato la sua email.
    \item \textbf{Scenario Principale:} L'acquirente si trova nella vista di modifica informazioni personali e vuole modificare la propria email attraverso le seguenti azioni:
        \begin{itemize}
            \item Si posiziona nel campo di inserimento della nuova email dove è presente quella attualmente utilizzata.
            \item Inserisce quella nuova o modifica quella attuale.
        \end{itemize}
    \item \textbf{Scenario Alternativo 1:} L'acquirente non modifica l'attuale indirizzo email utilizzato e, perciò, non cambierà.
    \item \textbf{Scenario Alternativo 2:} L'acquirente inserisce un'email già utilizzata nella piattaforma. In questo caso:
    \begin{itemize}
        \item (UC) - Verrà visualizzato il messaggio di errore in caso di email già registrata nella piattaforma.
        \item Verrà impedita la modifica delle informazioni.
    \end{itemize}
    \item \textbf{Scenario Alternativo 3:} L'acquirente inserisce un'email non valida. In questo caso:
    \begin{itemize}
        \item (UC) - Verrà visualizzato un messaggio di errore in caso di email non valida.
        \item Verrà impedita la modifica delle informazioni.
    \end{itemize}
    \item \textbf{Scenario Alternativo 4:} L'acquirente non inserisce l'email e lascia il campo di inserimento vuoto. In questo caso:
    \begin{itemize}
        \item (UC) - Verrà visualizzato il messaggio di errore campo dati obbligatorio non inserito.
        \item Verrà impedita la modifica delle informazioni.
    \end{itemize}
    \item \textbf{Estensioni:}
    \begin{itemize}
        \item (UC) - Visualizzazione messaggio di errore in caso di email già utilizzata nella piattaforma.
        \item (UC) - Visualizzazione messaggio di errore in caso di email non valida.
        \item (UC) - Visualizzazione messaggio campo dati obbligatorio non inserito.
    \end{itemize}
\end{itemize}

\subUC{Inserimento nuova password}
L'acquirente vuole modificare la propria password.
\begin{itemize}
    \item \textbf{Attori primari:} Acquirente.
    \item \textbf{Precondizione:} L'acquirente si trova nella vista di modifica informazioni personali e vuole inserire una nuova password.
    \item \textbf{Postcondizione:} L'acquirente ha inserito la nuova password.
    \item \textbf{Scenario Principale:} L'acquirente si trova nella vista di modifica informazioni personali e vuole modificare la propria password attraverso le seguenti azioni:
        \begin{itemize}
            \item Si posiziona nel campo di inserimento vuoto della nuova password.
            \item La scrive.
            \item Inserisce la stessa password nel campo di conferma nuova password.
        \end{itemize}
    \item \textbf{Scenario Alternativo 1:} L'acquirente non inserisce alcuna password e lascia il campo di inserimento nuova password vuoto. In questo caso non verrà modificata la password.
    \item \textbf{Scenario Alternativo 2:} L'acquirente inserisce una password troppo debole. In questo caso:
    \begin{itemize}
        \item (UC) - Verrà visualizzato il messaggio di errore di password troppo debole.
        \item Verrà impedita la modifica delle informazioni.
    \end{itemize}
    \item \textbf{Estensioni:}
    \begin{itemize}
        \item (UC) - Visualizzazione messaggio di errore in caso di password troppo debole.
    \end{itemize}
\end{itemize}

\subUC{Inserimento conferma nuova password}
L'acquirente vuole modificare la propria password e deve confermare quella nuova da inserire.
\begin{itemize}
    \item \textbf{Attori primari:} Acquirente.
    \item \textbf{Precondizione:} L'acquirente si trova nella vista di modifica informazioni personali e vuole inserire una nuova password.
    \item \textbf{Postcondizione:} L'acquirente ha inserito la conferma della nuova password.
    \item \textbf{Scenario Principale:} L'acquirente si trova nella vista di modifica informazioni personali, ha già inserito la nuova password e la deve confermare attraverso le seguenti azioni:
        \begin{itemize}
            \item Si posiziona nel campo di inserimento vuoto della conferma nuova password.
            \item Inserisce la stessa password che è stata inserita nel campo conferma password.
        \end{itemize}
    \item \textbf{Scenario Alternativo 1:} L'acquirente non inserisce alcuna password e lascia il campo di inserimento conferma nuova password vuoto. In questo caso non verrà modificata la password.
    \item \textbf{Scenario Alternativo 2:} L'acquirente inserisce una password di conferma diversa da quella inserita nel campo di inserimento nuova password. In questo caso:
    \begin{itemize}
        \item (UC) - Verrà visualizzato il messaggio di errore di password e password di conferma diverse.
        \item Verrà impedita la modifica delle informazioni.
    \end{itemize}
    \item \textbf{Estensioni:}
    \begin{itemize}
        \item (UC) - Visualizzazione messaggio di errore in caso di password e password di conferma diverse
    \end{itemize}
\end{itemize}



\UC{Modifica indirizzo di spedizione}
L'acquirente vuole modificare l'indirizzo di spedizione.
\begin{itemize}
    \item \textbf{Attori primari:} Acquirente.
    \item \textbf{Precondizione:} L'acquirente si trova nella vista di modifica indirizzo di spedizione e vuole modificare l'indirizzo di spedizione. E' possibile modificare anche solo parzialmente l'indirizzo già salvato.
    \item \textbf{Postcondizione:} L'acquirente ha inserito il nuovo indirizzo di spedizione da confermare.
    \item \textbf{Scenario Principale:} L'acquirente si trova nella vista di modifica indirizzo di spedizione, per modificare l'indirizzo deve eseguire le seguenti azioni:
        \begin{itemize}
            \item Si posiziona nel campo di inserimento della via da modificare.
            \item Inserisce la via del nuovo indirizzo.
            \item Si posiziona nel campo di inserimento del numero civico da modificare.
            \item Inserisce il nuovo civico.
            \item Si posiziona nel campo di inserimento del CAP.
            \item Inserisce il nuovo CAP.
        \end{itemize}
    \item \textbf{Scenario Alternativo 1:} L'acquirente non inserisce alcuna via di spedizione. In questo caso la via non verrà modificata.
    \item \textbf{Scenario Alternativo 2:} L'acquirente non inserisce alcun numero civico. In questo caso il civico non verrà modificato.
    \item \textbf{Scenario Alternativo 3:} L'acquirente non inserisce il CAP. In questo caso il CAP non verrà modificato.
    \item \textbf{Scenario Alternativo 4:} L'acquirente inserisce dei caratteri non numerici o in quantità diversa da 5 rendendo il CAP invalido. In questo caso:
    \begin{itemize}
        \item (UC) - Verrà mostrato un messaggio di errore con la segnalazione della causa.
        \item Verrà impedito il proseguimento dell'operazione.
    \end{itemize}
    \item \textbf{Estensioni:}
    \begin{itemize}
        \item (UC) - Visualizzazione messaggio di errore  indirizzo non valido.
    \end{itemize}
\end{itemize}

\UC{Inserimento indirizzo di spedizione}
L'acquirente vuole inserire l'indirizzo di spedizione.
\begin{itemize}
    \item \textbf{Attori primari:} Acquirente.
    \item \textbf{Precondizione:} L'acquirente si trova nella vista di modifica indirizzo di spedizione e vuole inserire l'indirizzo.
    \item \textbf{Postcondizione:} L'acquirente ha inserito il nuovo indirizzo di spedizione da confermare.
    \item \textbf{Scenario Principale:} L'acquirente si trova nella vista di modifica indirizzo di spedizione, per inserire l'indirizzo deve eseguire le seguenti azioni:
        \begin{itemize}
            \item Si posiziona nel campo vuoto di inserimento della via da modificare.
            \item Inserisce la via del nuovo indirizzo.
            \item Si posiziona nel campo vuoto di inserimento del numero civico da modificare.
            \item Inserisce il numero civico.
            \item Si posiziona nel campo vuoto di inserimento del CAP.
            \item Inserisce il nuovo CAP.
        \end{itemize}
    \item \textbf{Scenario Alternativo 1:} L'acquirente non inserisce alcuna via di spedizione. In questo caso l'indirizzo non verrà salvato.
    \item \textbf{Scenario Alternativo 2:} L'acquirente non inserisce alcun numero civico. In questo caso l'indirizzo non verrà salvato.
    \item \textbf{Scenario Alternativo 3:} L'acquirente non inserisce il CAP. In questo caso l'indirizzo non verrà salvato.
    \item \textbf{Scenario Alternativo 4:} L'acquirente inserisce dei caratteri non numerici o in quantità diversa da 5 rendendo il CAP invalido. In questo caso:
    \begin{itemize}
        \item (UC) - Verrà mostrato un messaggio di errore con la segnalazione della causa.
        \item Verrà impedito il proseguimento dell'operazione.
    \end{itemize}
    \item \textbf{Estensioni:}
    \begin{itemize}
        \item (UC) - Visualizzazione messaggio di errore  indirizzo non valido.
    \end{itemize}
\end{itemize}

\UC{Eliminazione indirizzo di spedizione}
L'acquirente vuole eliminare l'indirizzo di spedizione attuale.
\begin{itemize}
    \item \textbf{Attori primari:} Acquirente.
    \item \textbf{Precondizione:}  L'acquirente si trova nella vista di modifica indirizzo di spedizione e vuole eliminare l'indirizzo di spedizione.
    \item \textbf{Postcondizione:} L'acquirente ha confermato l'eliminazione dell'indirizzo di spedizoione.
    \item \textbf{Scenario Principale:} L'acquirente si trova nella vista di modifica indirizzo di spedizione, per eliminare l'indirizzo deve eseguire le seguenti azioni:
        \begin{itemize}
            \item Selezionare l'indirizzo da eliminare.
            \item Confermare la scelta di eliminare l'indirizzo.
        \end{itemize}
    \item \textbf{Scenario Alternativo 1:} L'acquirente non seleziona l'indirizzo da eliminare e l'operazione non viene eseguita.
    \item \textbf{Scenario Alternativo 2:} L'acquirente seleziona l'indirizzo da eliminare ma non conferma l'operazione. L'indirizzo non viene eliminato.
\end{itemize}

%%%%%%%%%%%%%%%%%%%%%%%%%%%%%%%%%%%%%%%%%%%%%%%%%%%%%%%%%%%%%%%%%%%%%%%%%%%%%%%%%%%%%%%%%%%%%%%%%%%%%%%%%%%%

\UC{Eliminazione account}
L'acquirente può eliminare il proprio account.
\begin{itemize}
    \item \textbf{Attori Primari:} Acquirente.
    \item \textbf{Precondizione:} L'acquirente si trova nella propria pagina personale (UC3.2.1) e ha selezionato l'azione di cancellazione del proprio account.
    \item \textbf{Postcondizione:} L'account dell'acquirente non è più presente nella piattaforma
    \item \textbf{Scenario Principale:} L'acquirente vuole eliminare il proprio account e compie le seguenti operazioni:
    \begin{itemize}
        \item Preme sull'azione di cancellazione del proprio account.
        \item Viene visualizzato un messaggio di conferma dell'operazione.
        \item Se conferma viene eliminato l'account.
    \end{itemize}
    \item \textbf{Scenario Alternativo:} Se non conferma viene riportato alla propria pagina personale e l'account non viene eliminato.
\end{itemize}
