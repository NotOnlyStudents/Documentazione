\subsection{Attori} \label{Attori}

\begin{figure}[ht]
    \centering
    \includegraphics[width=\textwidth]{Immagini/DiagrammiUC/Attori.png}
    \caption{Gerarchia degli utenti} 
    \label{fig:Registrazione}
\end{figure}

\subsubsection{Attori primari}
\begin{itemize}
    \item \textbf{Utente non autenticato:} utente che può consultare la parte pubblica del sito cercando prodotti e aggiungendoli al carrello, oppure fare il login o registrarsi come acquirente al sito. Non può fare però acquisti sulla piattaforma;
    \item \textbf{Utente autenticato:}
    \begin{itemize}
        \item \textbf{Acquirente:} utente che può fare tutto ciò che fa l'utente non autenticato dopo aver effettuato la login o la registrazione. Può effettuare acquisti comprando i prodotti che ha nel carrello, consultare gli ordini che ha fatto, modificare il suo profilo e eseguire il logout;
        \item \textbf{Venditore:} utente che può aggiungere, modificare ed eliminare i prodotti dalla piattaforma, oltre a poter consultare gli ordini fatti dagli acquirenti.
    \end{itemize}
    % \item \textbf{Amministratore:} l'amministratore è in grado di fare il \glo{Deploy} dell'applicazione sul \glo{cloud}, configurare le integrazioni di componenti di terze parti nella piattaforma e gestire gli utenti di tipo venditore. Questo utente non fa parte di quelli autenticati perché non svolgerà le azioni all'interno della piattaforma ma dove questa è eseguita.
\end{itemize}
\subsubsection{Attori secondari}
\begin{itemize}
\item \textbf{Gestore dei pagamenti:} servizio per la gestione di transazioni online, che verrà utilizzato per i pagamenti all'interno della piattaforma.% Stripe è aderente alle normative per il pagamento online del nord America, Europa e Oceania.
\item \textbf{Identity manager:} sistema integrato di procedure e criteri che è in grado di facilitare e controllare la gestione degli accessi alla piattaforma. 
\end{itemize}