% Estensioni campi dati
\UC{Visualizzazione messaggio di errore caratteri non alfabetici non permessi}
Visualizzazione messaggio di errore che segnala all'attore l'inserimento di caratteri non alfabetici.
\begin{itemize}
    \item \textbf{Attori primari:} acquirente;
    \item \textbf{Precondizione:} l'attore ha inserito dei caratteri non alfabetici;
    \item \textbf{Postcondizione:} all'attore verrà mostrato un messaggio il quale lo informa della presenza di carattere non alfabetici, dove non è permessa la loro presenza;
    \item \textbf{Scenario principale:} l'attore ha inserito dei caratteri non alfabetici dove non è permessa la loro presenza e verrà avvisato della presenza di caratteri non alfabetici.
\end{itemize}

% Estensioni email
\UC{Visualizzazione messaggio di errore in caso di registrazione con un'email già utilizzata nella piattaforma}
Visualizzazione messaggio di errore che segnala all'attore il tentativo di registrazione con un'email già utilizzata nella piattaforma.
\begin{itemize}
    \item \textbf{Attori primari:} utente non autenticato;
    \item \textbf{Precondizione:} l'utente non autenticato sta cercando di registrarsi con un'email già utilizzata nella piattaforma;
    \item \textbf{Postcondizione:} all'attore viene mostrato un messaggio il quale indica l'impossibilità di registrarsi con un'email già utilizzata nella piattaforma;
    \item \textbf{Scenario principale:} l'attore sta cercando di registrarsi con un'email già utilizzata nella piattaforma e viene avvisato dell'impossibilità di registrarsi con un'email già utilizzata nella piattaforma.
\end{itemize}

\UC{Visualizzazione messaggio di errore in caso di cambio email con una già utilizzata nella piattaforma}
Visualizzazione messaggio di errore che segnala all'attore il tentativo di cambio dell'email attuale con un'email già utilizzata nella piattaforma.
\begin{itemize}
    \item \textbf{Attori primari:} acquirente o venditore;
    \item \textbf{Precondizione:} l'attore sta cercando di cambiare la propria email con una già utilizzata nella piattaforma;
    \item \textbf{Postcondizione:} all'attore viene mostrato un messaggio il quale indica l'impossibilità di cambiare la propria email attuale con un'email già utilizzata nella piattaforma;
    \item \textbf{Scenario principale:} l'attore sta cercando di cambiare la propria email attuale con un'email già utilizzata nella piattaforma e viene avvisato dell'impossibilità di cambiare la propria email con una già utilizzata nella piattaforma.
\end{itemize}

\UC{Visualizzazione messaggio di errore in caso di email non registrata}
Visualizzazione messaggio di errore che segnala all'attore il tentativo di accesso alla piattaforma con una email non registrata.
\begin{itemize}
    \item \textbf{Attori primari:} utente non autenticato;
    \item \textbf{Precondizione:} l'utente non autenticato ha inserito un'email non registrata nella piattaforma;
    \item \textbf{Postcondizione:} all'utente non autenticato viene mostrato un messaggio il quale indica l'impossibilità di inserire un'email non registrata nella piattaforma;
    \item \textbf{Scenario principale:} l'utente non autenticato ha inserito un'email non registrata nella piattaforma e viene informato dell'impossibilità di inserire un'email non registrata nella piattaforma.
\end{itemize}

\UC{Visualizzazione messaggio di errore in caso di credenziali non presenti nella piattaforma}
Visualizzazione messaggio di errore che segnala all'utente il tentativo di accesso alla piattaforma attraverso delle credenziali non presenti nella piattaforma.
\begin{itemize}
    \item \textbf{Attori primari:} utente non autenticato;
    \item \textbf{Precondizione:} l'utente non autenticato tenta di accedere alla piattaforma attraverso delle credenziali non presenti in essa;
    \item \textbf{Postcondizione:} all'utente non autenticato viene mostrato un messaggio il quale lo avvisa dell'utilizzo di credenziali non presenti nella piattaforma;
    \item \textbf{Scenario principale:} l'utente non autenticato tenta di accedere alla piattaforma attraverso delle credenziali non presenti in essa e viene informato dell'utilizzo di credenziali non presenti nella piattaforma.
\end{itemize}

\UC{Visualizzazione messaggio di errore in caso di email non valida}
Visualizzazione messaggio di errore che segnala all'utente che l'email inserita non rispetta il formato di un indirizzo email.
\begin{itemize}
    \item \textbf{Attori primari:} utente autenticato o utente non autenticato;
    \item \textbf{Precondizione:} l'attore ha inserito un'email nel formato sbagliato;
    \item \textbf{Postcondizione:} all'utente non autenticato viene mostrato un messaggio il quale lo avvisa dell'inserimento di un'email in un formato errato;
    \item \textbf{Scenario principale:} l'attore ha inserito un'email nel formato sbagliato e viene informato dell'inserimento di un'email nel formato sbagliato.
\end{itemize}

% Estensioni password
\UC{Visualizzazione messaggio di errore in caso di password troppo debole}
Visualizzazione messaggio di errore che segnala all'utente l'inserimento di una password troppo debole che non rispetta le condizioni minime per essere accettabile.
\begin{itemize}
    \item \textbf{Attori primari:} utente autenticato o utente non autenticato;
    \item \textbf{Precondizione:} l'attore ha inserito una password troppo debole;
    \item \textbf{Postcondizione:} all'attore viene mostrato un messaggio il quale lo avvisa dell'inserimento di una password che non rispetta le condizioni minime per essere accettabile;
    \item \textbf{Scenario principale:} l'attore ha inserito una password troppo debole che non rispetta le condizioni minime per essere accettabile e viene informato del non rispetto delle condizioni minime.
\end{itemize}

\UC{Visualizzazione messaggio di errore in caso di password e password di conferma diverse}
Visualizzazione messaggio di errore che segnala all'utente la non corrispondenza della password e della password di conferma inserite.
\begin{itemize}
    \item \textbf{Attori primari:} utente autenticato o utente non autenticato;
    \item \textbf{Precondizione:} l'attore inserisce una password di conferma che non corrisponde a quella inserita come password da registrare;
    \item \textbf{Postcondizione:} all'attore viene mostrato un messaggio il quale lo avvisa della non corrispondenza tra la password e la password di conferma;
    \item \textbf{Scenario principale:} l'attore inserisce una password di conferma che non corrisponde a quella inserita come password da registrare e viene informato della non corrispondenza.
\end{itemize}

\UC{Visualizzazione messaggio campo dati obbligatorio non inserito}
Visualizzazione messaggio che segnala all'utente il mancato riempimento di un campo dati obbligatorio.
\begin{itemize}
    \item \textbf{Attori primari:} utente autenticato o utente non autenticato;
    \item \textbf{Precondizione:} l'attore non ha inserito, oppure ha inserito solo caratteri vuoti come spazi o tab, in un campo dati obbligatorio;
    \item \textbf{Postcondizione:} all'attore viene mostrato un messaggio che lo avvisa del mancato riempimento di quel campo dati obbligatorio;
    \item \textbf{Scenario principale:} l'attore non ha inserito, oppure ha inserito solo caratteri vuoti come spazi o tab, in un campo dati obbligatorio e viene informato del mancato riempimento di esso.
\end{itemize}

% Estensioni ordini
\UC{Visualizzazione messaggio d'errore data non valida}
Visualizzazione messaggio di errore che segnala all'attore l'inserimento di una data non valida.
\begin{itemize}
    \item \textbf{Attori primari:} acquirente o venditore;
    \item \textbf{Precondizione:} l'attore ha inserito una data che non è nel formato corretto;
    \item \textbf{Postcondizione:} all'attore verrà mostrato un messaggio il quale lo informa dell'inserimento di una data non valida;
    \item \textbf{Scenario principale:} l'attore ha inserito una data non nel formato giorno/mese/anno e viene informato del formato non valido della data.
\end{itemize}

\UC{Visualizzazione messaggio d'errore codice dell'ordine non valido}
Visualizzazione messaggio di errore che segnala all'attore l'inserimento di caratteri non numerici nel codice per cui cercare.
\begin{itemize}
    \item \textbf{Attori primari:} acquirente o venditore;
    \item \textbf{Precondizione:} l'attore ha inserito dei caratteri non numerici nel codice per cui cercare l'ordine;
    \item \textbf{Postcondizione:} all'attore verrà mostrato un messaggio il quale lo informa dell'inserimento di un codice non valido;
    \item \textbf{Scenario principale:} l'attore ha inserito dei caratteri non numerici nel codice per cui cercare e viene informato dell'errore.
\end{itemize}

\UC{Visualizzazione messaggio d'errore data iniziale maggiore di quella finale}
Visualizzazione messaggio di errore che segnala all'attore l'inserimento di una data iniziale maggiore di quella finale.
\begin{itemize}
    \item \textbf{Attori primari:} utente autenticato o utente non autenticato;
    \item \textbf{Precondizione:} l'attore ha inserito una data iniziale maggiore di quella finale;
    \item \textbf{Postcondizione:} all'attore verrà mostrato un messaggio il quale lo informa dell'inserimento di una data iniziale maggiore di quella finale;
    \item \textbf{Scenario principale:} l'attore ha inserito una data iniziale maggiore di quella finale e viene informato dell'impossibilità di inserire una data iniziale maggiore di quella finale.
\end{itemize}

\UC{Visualizzazione messaggio nessun ordine effettuato nell'arco temporale impostato}
Visualizzazione di un messaggio il quale segnala all'acquirente che non è stato effettuato alcun ordine nell'arco temporale impostato.
\begin{itemize}
    \item \textbf{Attori primari:} acquirente;
    \item \textbf{Precondizione:} l'acquirente ha inserito un intervallo temporale nel quale non è stato effettuato alcun ordine;
    \item \textbf{Postcondizione:} all'acquirente verrà mostrato un messaggio il quale lo informa della non effettuazione di alcun ordine nell'arco temporale impostato;
    \item \textbf{Scenario principale:} l'acquirente ha inserito un intervallo temporale nel quale non è stato effettuato alcun ordine e viene informato della non effettuazione di alcun ordine nell'arco temporale impostato.
\end{itemize}

\UC{Visualizzazione messaggio nessun ordine effettuato a cui corrisponde il codice inserito}
Visualizzazione di un messaggio il quale segnala all'acquirente che non è stato effettuato alcun ordine a cui corrisponde il codice che ha inserito.
\begin{itemize}
    \item \textbf{Attori primari:} acquirente;
    \item \textbf{Precondizione:} l'acquirente ha inserito un codice a cui non corrisponde alcun ordine;
    \item \textbf{Postcondizione:} all'acquirente verrà mostrato un messaggio il quale lo informa della non effettuazione di alcun ordine a cui corrisponde il codice che ha inserito;
    \item \textbf{Scenario principale:} l'acquirente ha inserito un codice a cui non corrisponde alcun ordine e viene informato della non effettuazione di alcun ordine a cui corrisponde il codice.
\end{itemize}

% Estensioni PDP
\UC{Visualizzazione messaggio d'errore quantità del prodotto non valida}
Visualizzazione messaggio di errore che segnala all'acquirente o all'utente non autenticato che la quantità di prodotto inserita è minore o uguale a zero.
\begin{itemize}
    \item \textbf{Attori primari:} acquirente o utente non autenticato;
    \item \textbf{Precondizione:} l'attore inserisce una quantità minore o uguale a zero di un prodotto da aggiungere al carrello;
    \item \textbf{Postcondizione:} all'attore viene visualizzato un messaggio il quale lo informa che la quantità inserita non è valida;
    \item \textbf{Scenario principale:} l'attore inserisce una quantità minore o uguale a zero di un prodotto da aggiungere al carrello e viene informato della non validità della quantità.
\end{itemize}

% Estensioni info personali
\UC{Visualizzazione messaggio di errore nel caso in cui il file selezionato non sia del tipo immagine}
Visualizzazione messaggio di errore che segnala al venditore che il file inserito non è di tipo immagine.
\begin{itemize}
    \item \textbf{Attori primari:} venditore;
    \item \textbf{Precondizione:} il venditore inserisce un file di tipo non immagine;
    \item \textbf{Postcondizione:} all'attore viene mostrato un messaggio il quale lo avvisa che si possono selezionare solo file di tipo immagine;
    \item \textbf{Scenario principale:} il venditore inserisce un file di tipo non immagine e viene informato del tipo errato del file.
\end{itemize}

% Estensioni prodotto
\UC{Visualizzazione messaggio di errore in caso di prezzo minore o uguale a zero}
Visualizzazione messaggio di errore il quale segnala al venditore che il prezzo del prodotto inserito è minore o uguale a zero.
\begin{itemize}
    \item \textbf{Attori primari:} venditore;
    \item \textbf{Precondizione:} il venditore inserisce un prezzo minore o uguale a zero;
    \item \textbf{Postcondizione:} al venditore viene mostrato un messaggio il quale lo avvisa che il prezzo del prodotto inserito è minore o uguale a zero;
    \item \textbf{Scenario principale:} il venditore inserisce un prezzo minore o uguale a zero e viene informato dell'inserimento di un prezzo minore o uguale a zero.
\end{itemize}

\UC{Visualizzazione messaggio di errore in caso di quantità minore o uguale a zero}
Visualizzazione messaggio di errore che segnala al venditore che la quantità del prodotto inserita è minore o uguale a zero.
\begin{itemize}
    \item \textbf{Attori primari:} venditore;
    \item \textbf{Precondizione:} il venditore inserisce una quantità del prodotto minore o uguale a zero;
    \item \textbf{Postcondizione:} al venditore viene mostrato un messaggio il quale lo avvisa che la quantità del prodotto inserita è minore o uguale a zero;
    \item \textbf{Scenario principale:} il venditore inserisce una quantità del prodotto minore o uguale a zero e viene informato dell'inserimento di una quantità minore o uguale a zero.
\end{itemize}

\UC{Visualizzazione messaggio di errore in caso di sconto minore di zero}
Visualizzazione messaggio di errore il quale segnala al venditore che lo sconto inserito è minore di zero.
\begin{itemize}
    \item \textbf{Attori primari:} venditore;
    \item \textbf{Precondizione:} il venditore inserisce uno sconto minore di zero;
    \item \textbf{Postcondizione:} al venditore viene mostrato un messaggio il quale lo avvisa che lo sconto appena inserito è minore di zero;
    \item \textbf{Scenario principale:} il venditore inserisce uno sconto minore di zero e viene informato dell'inserimento di uno sconto minore di zero.
\end{itemize}

\UC{Visualizzazione messaggio di errore in caso di sconto maggiore di 100\%}
Visualizzazione messaggio di errore il quale segnala al venditore che lo sconto inserito è maggiore di 100\%.
\begin{itemize}
    \item \textbf{Attori primari:} venditore;
    \item \textbf{Precondizione:} il venditore inserisce uno sconto maggiore di 100\%;
    \item \textbf{Postcondizione:} al venditore viene mostrato un messaggio il quale lo avvisa che lo sconto appena inserito è maggiore di 100\%;
    \item \textbf{Scenario principale:} il venditore inserisce uno sconto maggiore di 100\% e viene informato dell'inserimento di uno sconto maggiore di 100\%.
\end{itemize}

\UC{Visualizzazione messaggio di errore tentivo di aggiunta di più di quattro foto relative ad un prodotto}
Visualizzazione messaggio di errore il quale segnala al venditore il tentativo di aggiunta di una nuova foto relativa ad un prodotto, quando si è già raggiunto il limite massimo di quattro.
\begin{itemize}
    \item \textbf{Attori primari:} venditore;
    \item \textbf{Precondizione:} il venditore cerca di aggiungere una nuova foto relativa ad un prodotto, anche quando si è già raggiunto il limite massimo di quattro;
    \item \textbf{Postcondizione:} al venditore viene mostrato un messaggio il quale lo avvisa che non possono essere inserite più di 4 quattro foto per prodotto;
    \item \textbf{Scenario principale:} il venditore cerca di aggiungere una nuova foto relativa ad un prodotto, anche quando si è già raggiunto il limite massimo di quattro e viene informato del superamento del limite di quattro foto per prodotto.
\end{itemize}

\UC{Visualizzazione messaggio di errore nessuna foto inserita}
Visualizzazione messaggio di errore il quale segnala al venditore che non è stata inserita alcuna foto relativa al prodotto.
\begin{itemize}
    \item \textbf{Attori primari:} venditore;
    \item \textbf{Precondizione:} il venditore non inserisce alcuna foto relativa ad un prodotto;
    \item \textbf{Postcondizione:} al venditore viene mostrato un messaggio il quale lo avvisa che bisogna inserire almeno una foto relativa ad un prodotto;
    \item \textbf{Scenario principale:} il venditore non inserisce alcuna foto relativa ad un prodotto e viene informato del mancato inserimento di almeno una foto relativa ad un prodotto.
\end{itemize}

% Estensioni Categoria
\UC{Visualizzazione messaggio nome per la categoria già utilizzato}
Visualizzazione messaggio di errore il quale segnala al venditore che sta cercando di assegnare ad una categoria un nome già utilizzato da un'altra.
\begin{itemize}
    \item \textbf{Attori primari:} venditore;
    \item \textbf{Precondizione:} il venditore vuole assegnare ad una categoria un nome già utilizzato da un'altra categoria;
    \item \textbf{Postcondizione:} al venditore viene mostrato un messaggio il quale lo avvisa che il nome che sta cercando di assegnare ad una categoria è già in uso;
    \item \textbf{Scenario principale:} il venditore vuole assegnare ad una categoria un nome già utilizzato da un'altra categoria e viene informato dell'impossibilità di assegnare ad una categoria un nome già utilizzato da un'altra categoria.
\end{itemize}

% Estensioni dashboard

\UC{Visualizzazione messaggio di ricerca senza alcun risultato}
Nel caso in cui la ricerca del prodotto non dia risultati verrà visualizzato un opportuno messaggio.
\begin{itemize}
	\item \textbf{Attori primari:} acquirente o utente non autenticato;
	\item \textbf{Precondizione:} l'attore ha svolto una ricerca di un prodotto che non ha dato alcun risultato;
	\item \textbf{Postcondizione:} all'attore verrà mostrato un messaggio il quale lo informa che la ricerca appena effettuata non ha prodotto alcun risultato;
	\item \textbf{Scenario principale:} l'attore effettua una ricerca che non ha prodotto alcun risultato e viene informato del mancato risultato della ricerca appena effettuata.
\end{itemize}

\UC{Visualizzazione messaggio in caso di prodotto selezionato per l'eliminazione ma non ancora evaso in alcuni ordini}
Visualizzazione messaggio il quale segnala al venditore che il prodotto che vuole eliminare è presente all'interno di un ordine che non è ancora stato evaso.
\begin{itemize}
    \item \textbf{Attori primari:} venditore;
    \item \textbf{Precondizione:} il venditore sta cercando di eliminare un prodotto all'interno in un ordine che non è stato ancora evaso;
    \item \textbf{Postcondizione:} al venditore viene mostrato un messaggio il quale lo informa che il prodotto non può essere eliminato perché è presente in un ordine che non è stato ancora evaso;
    \item \textbf{Scenario principale:} il venditore sta cercando di eliminare un prodotto presente in un ordine che non è stato ancora evaso, viene quindi informato che per questo motivo non è possibile procedere con l'eliminazione.
\end{itemize}

% Estensioni carta per il pagamento
\UC{Visualizzazione messaggio d'errore numero della carta non valido}
Nel caso in cui venga inserito un numero di una carta non valido, verrà segnalato attraverso un messaggio di errore opportuno.
\begin{itemize}
	\item \textbf{Attori primari:} acquirente;
	\item \textbf{Precondizione:} l'acquirente ha inserito un numero di una carta non valido;
	\item \textbf{Postcondizione:} all'acquirente verrà mostrato un messaggio il quale lo informa della non validità del numero della carta;
	\item \textbf{Scenario principale:} l'attore non ha inserito esattamente 16 numeri compresi tra 0 e 9 come numero della carta e verrà avvisato della non validità del numero della carta.
\end{itemize}

\UC{Visualizzazione messaggio d'errore CVV della carta non valido}
Nel caso in cui venga inserito un CVV di una carta non valido, verrà segnalato attraverso un messaggio di errore opportuno.
\begin{itemize}
	\item \textbf{Attori primari:} acquirente;
	\item \textbf{Precondizione:} l'acquirente ha inserito un CVV di una carta non valido;
	\item \textbf{Postcondizione:} all'acquirente verrà mostrato un messaggio il quale lo informa della non validità del CVV della carta;
	\item \textbf{Scenario principale:} l'attore non ha inserito esattamente 3 numeri compresi tra 0 e 9 come CVV della carta e verrà avvisato della non validità del CVV della carta.
\end{itemize}

\UC{Visualizzazione messaggio d'errore data di scandenza non valida}
Nel caso in cui venga inserita una data di scadenza di una carta non valida, verrà segnalato attraverso un messaggio di errore opportuno.
\begin{itemize}
	\item \textbf{Attori primari:} acquirente;
	\item \textbf{Precondizione:} l'acquirente ha inserito una data di scadenza di una carta non valida;
	\item \textbf{Postcondizione:} all'acquirente verrà mostrato un messaggio il quale lo informa della non validità della data di scadenza della carta;
	\item \textbf{Scenario principale:} l'attore non ha inserito una data di scadenza valida nel seguente formato "mese/anno", dove l'anno sarà composto solo dalle ultime due cifre dell'anno di scadenza, e verrà avvisato della non validità della data di scadenza.
\end{itemize}

% Estensioni indirizzo di consegna
\UC{Visualizzazione messaggio d'errore CAP non valido}
Nel caso in cui venga inserito un CAP per un indirizzo di consegna non valido, verrà segnalato attraverso un messaggio di errore opportuno.
\begin{itemize}
	\item \textbf{Attori primari:} acquirente;
	\item \textbf{Precondizione:} l'acquirente ha inserito un CAP per un indirizzo di consegna non valido;
	\item \textbf{Postcondizione:} all'acquirente verrà mostrato un messaggio il quale lo informa della non validità del CAP inserito;
	\item \textbf{Scenario principale:} l'attore non ha inserito un CAP composto da esattamente 5 numeri compresi tra 0 e 9 e verrà avvisato della non validità del CAP.
\end{itemize}

% Estensioni stripe
\UC{Visualizzazione messaggio d'errore pagamento non andato a buon fine}
Nel caso in cui avvenga un errore durante lo svolgimento del pagamento da parte di stripe, questo verrà visualizzato all'acquirente.
\begin{itemize}
	\item \textbf{Attori primari:} acquirente;
	\item \textbf{Attori secondari:} stripe;
	\item \textbf{Precondizione:} stripe sta svolgendo il pagamento dell'acquirente e si verifica un errore;
	\item \textbf{Postcondizione:} all'acquirente verrà mostrato un messaggio il quale lo informa della non riuscita del pagamento con il relativo messaggio;
	\item \textbf{Scenario principale:} stripe sta svolgendo il pagamento dell'acquirente, si verifica un errore e questo verrà visualizzato all'acquirente informandolo dell'errore specifico.
\end{itemize}
