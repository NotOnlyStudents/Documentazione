% Estensioni campi dati
\UC{Visualizzazione messaggio di errore caratteri non alfabetici non permessi}
% \label{estensione:caratteri-non-alfabetici-non-permessi}

Visualizzazione messaggio di errore che segnala all'attore l'inserimento di caratteri non alfabetici.
\begin{itemize}
    \item \textbf{Attori primari:} acquirente;
    \item \textbf{Precondizione:} l'attore ha inserito dei caratteri non alfabetici;
    \item \textbf{Postcondizione:} all'attore verrà mostrato un messaggio il quale lo informa della presenza di carattere non alfabetici, dove non è permessa la loro presenza;
    \item \textbf{Scenario principale:} l'attore ha inserito dei caratteri non alfabetici dove non è permessa la loro presenza e verrà avvisato dell'errore commesso.
\end{itemize}

% Estensioni e-mail
\UC{Visualizzazione messaggio di errore in caso di registrazione con un'e-mail già utilizzata nella piattaforma}
\label{estensione:registrazione-con-email-non-esistente}

Visualizzazione messaggio di errore che segnala all'attore il tentativo di registrazione con un'e-mail già utilizzata nella piattaforma.
\begin{itemize}
    \item \textbf{Attori primari:} utente non autenticato;
    \item \textbf{Precondizione:} l'utente non autenticato sta cercando di registrarsi con un'e-mail già utilizzata nella piattaforma;
    \item \textbf{Postcondizione:} all'attore viene mostrato un messaggio il quale indica l'impossibilità di registrarsi con un'e-mail già utilizzata nella piattaforma;
    \item \textbf{Scenario principale:} l'attore sta cercando di registrarsi con un'e-mail già utilizzata nella piattaforma e viene avvisato dell'impossibilità di proseguire con la registrazione alla piattaforma.
\end{itemize}

\UC{Visualizzazione messaggio di errore in caso di cambio e-mail con una già utilizzata nella piattaforma}
% \label{estensione:cambio-con-email-esistente}

Visualizzazione messaggio di errore che segnala all'attore il tentativo di cambio dell'indirizzo e-mail attuale con uno già utilizzato nella piattaforma.
\begin{itemize}
    \item \textbf{Attori primari:} acquirente o venditore;
    \item \textbf{Precondizione:} l'attore sta cercando di cambiare la propria e-mail con una già utilizzata nella piattaforma;
    \item \textbf{Postcondizione:} all'attore viene mostrato un messaggio il quale indica l'impossibilità di cambiare la propria e-mail attuale con un'e-mail già utilizzata nella piattaforma;
    \item \textbf{Scenario principale:} l'attore sta cercando di cambiare la propria e-mail attuale con un'e-mail già utilizzata nella piattaforma e viene avvisato dell'impossibilità di proseguire con il cambio della propria e-mail.
\end{itemize}

\UC{Visualizzazione messaggio di errore in caso di e-mail non registrata}
% \label{estensione:email-non-esistente}

Visualizzazione messaggio di errore che segnala all'attore il tentativo di accesso alla piattaforma con una e-mail non registrata.
\begin{itemize}
    \item \textbf{Attori primari:} utente non autenticato;
    \item \textbf{Precondizione:} l'utente non autenticato ha inserito un'e-mail non registrata nella piattaforma;
    \item \textbf{Postcondizione:} all'utente non autenticato viene mostrato un messaggio il quale indica l'impossibilità di inserire un'e-mail non registrata nella piattaforma;
    \item \textbf{Scenario principale:} l'utente non autenticato ha inserito un'e-mail non registrata nella piattaforma e viene informato dell'errore commesso.
\end{itemize}

\UC{Visualizzazione messaggio di errore in caso di credenziali non presenti nella piattaforma}
% \label{estensione:credenziali-non-presenti}

Visualizzazione messaggio di errore che segnala all'utente il tentativo di accesso alla piattaforma attraverso delle credenziali non presenti nella piattaforma.
\begin{itemize}
    \item \textbf{Attori primari:} utente non autenticato;
    \item \textbf{Precondizione:} l'utente non autenticato tenta di accedere alla piattaforma attraverso delle credenziali non presenti in essa;
    \item \textbf{Postcondizione:} all'utente non autenticato viene mostrato un messaggio il quale lo avvisa dell'utilizzo di credenziali non presenti nella piattaforma;
    \item \textbf{Scenario principale:} l'utente non autenticato tenta di accedere alla piattaforma inserendo delle credenziali e viene informato dell'utilizzo di credenziali non presenti nella piattaforma.
\end{itemize}

\UC{Visualizzazione messaggio di errore in caso di e-mail non valida}
% \label{estensione:email-non-valida}

Visualizzazione messaggio di errore che segnala all'utente che l'e-mail inserita non rispetta il formato di un indirizzo e-mail.
\begin{itemize}
    \item \textbf{Attori primari:} utente autenticato o utente non autenticato;
    \item \textbf{Precondizione:} l'attore ha inserito un'e-mail nel formato sbagliato;
    \item \textbf{Postcondizione:} all'utente non autenticato viene mostrato un messaggio il quale lo avvisa dell'inserimento di un'e-mail in un formato errato;
    \item \textbf{Scenario principale:} l'attore ha inserito un'e-mail e viene informato dell'inserimento di un'e-mail nel formato sbagliato.
\end{itemize}

% Estensioni password
\UC{Visualizzazione messaggio di errore password non rispetta i requisiti di complessità}
% \label{estensione:password-non-valida}

Visualizzazione messaggio di errore che segnala all'utente l'inserimento di una password che non rispetta le condizioni minime per essere accettabile.
\begin{itemize}
    \item \textbf{Attori primari:} utente autenticato o utente non autenticato;
    \item \textbf{Precondizione:} l'attore ha inserito una password che non rispetta i requisiti di complessità;
    \item \textbf{Postcondizione:} all'attore viene mostrato un messaggio il quale lo avvisa dell'inserimento di una password che non rispetta le condizioni minime per essere accettabile;
    \item \textbf{Scenario principale:} l'attore ha inserito una password troppo debole che non rispetta le condizioni minime per essere accettabile e viene informato dell'errore commesso.
\end{itemize}

\UC{Visualizzazione messaggio di errore in caso di password e password di conferma diverse}
% \label{estensione:password-conferma-diverse}

Visualizzazione messaggio di errore che segnala all'utente la non corrispondenza della password e della password di conferma inserite.
\begin{itemize}
    \item \textbf{Attori primari:} utente autenticato o utente non autenticato;
    \item \textbf{Precondizione:} l'attore inserisce una password di conferma che non corrisponde a quella inserita come password da registrare;
    \item \textbf{Postcondizione:} all'attore viene mostrato un messaggio il quale lo avvisa della non corrispondenza tra la password e la password di conferma;
    \item \textbf{Scenario principale:} l'attore inserisce una password di conferma che non corrisponde a quella inserita come password da registrare e viene informato della non corrispondenza.
\end{itemize}

\UC{Visualizzazione messaggio campo dati obbligatorio non inserito}
% \label{estensione:campo-obbligatorio-non-inserito}

Visualizzazione messaggio che segnala all'utente il mancato riempimento di un campo dati obbligatorio.
\begin{itemize}
    \item \textbf{Attori primari:} utente autenticato o utente non autenticato;
    \item \textbf{Precondizione:} l'attore non ha inserito, oppure ha inserito solo caratteri vuoti come spazi o tab, in un campo dati obbligatorio;
    \item \textbf{Postcondizione:} all'attore viene mostrato un messaggio che lo avvisa del mancato riempimento di quel campo dati obbligatorio;
    \item \textbf{Scenario principale:} l'attore non ha inserito, oppure ha inserito solo caratteri vuoti come spazi o tab, in un campo dati obbligatorio e viene informato del mancato riempimento di esso.
\end{itemize}

% Estensioni ordini
\UC{Visualizzazione messaggio d'errore data non valida}
% \label{estensione:data-non-valida}

Visualizzazione messaggio di errore che segnala all'attore l'inserimento di una data non valida.
\begin{itemize}
    \item \textbf{Attori primari:} acquirente o venditore;
    \item \textbf{Precondizione:} l'attore ha inserito una data che non è nel formato corretto;
    \item \textbf{Postcondizione:} all'attore verrà mostrato un messaggio il quale lo informa dell'inserimento di una data non valida;
    \item \textbf{Scenario principale:} l'attore ha inserito una data non nel formato giorno/mese/anno e viene avvisato del formato non valido della data.
\end{itemize}

\UC{Visualizzazione messaggio d'errore codice dell'ordine non valido}
% \label{estensione:codice-ordine-non-valido}

Visualizzazione messaggio di errore che segnala all'attore l'inserimento di caratteri non numerici nel codice per cui cercare.
\begin{itemize}
    \item \textbf{Attori primari:} acquirente o venditore;
    \item \textbf{Precondizione:} l'attore ha inserito dei caratteri non numerici nel codice per cui cercare l'ordine;
    \item \textbf{Postcondizione:} all'attore verrà mostrato un messaggio il quale lo informa dell'inserimento di un codice non valido;
    \item \textbf{Scenario principale:} l'attore ha inserito dei caratteri non numerici nel codice per cui cercare e viene informato dell'errore commesso.
\end{itemize}

\UC{Visualizzazione messaggio d'errore data iniziale maggiore di quella finale}
% \label{estensione:data-iniziale-maggiore-data-finale}

Visualizzazione messaggio di errore che segnala all'attore l'inserimento di una data iniziale maggiore di quella finale.
\begin{itemize}
    \item \textbf{Attori primari:} utente autenticato o utente non autenticato;
    \item \textbf{Precondizione:} l'attore ha inserito una data iniziale maggiore di quella finale;
    \item \textbf{Postcondizione:} all'attore verrà mostrato un messaggio il quale lo informa dell'inserimento di una data iniziale maggiore di quella finale;
    \item \textbf{Scenario principale:} l'attore ha inserito una data iniziale maggiore di quella finale e viene informato dell'impossibilità di avere una data iniziale maggiore di quella finale.
\end{itemize}

% Estensioni PDP
\UC{Visualizzazione messaggio d'errore quantità da aggiungere al carrello non valida}
% \label{estensione:quantita-non-valida}

Visualizzazione messaggio di errore che segnala all'acquirente o all'utente non autenticato che la quantità di prodotto inserita è minore o uguale a zero.
\begin{itemize}
    \item \textbf{Attori primari:} acquirente o utente non autenticato;
    \item \textbf{Precondizione:} l'attore inserisce una quantità minore o uguale a zero di un prodotto da aggiungere al carrello;
    \item \textbf{Postcondizione:} all'attore viene visualizzato un messaggio il quale lo informa che la quantità inserita non è valida;
    \item \textbf{Scenario principale:} l'attore inserisce una quantità minore o uguale a zero di un prodotto da aggiungere al carrello e viene informato della non validità della quantità.
\end{itemize}

\UC{Visualizzazione messaggio di errore in caso di inserimento di un numero eccessivo di caratteri}
% \label{estensione:numero-eccessivo-caratteri}

Visualizzazione messaggio di errore in caso di inserimento di un numero eccessivo di caratteri.
\begin{itemize}
	\item \textbf{Attori primari:} venditore;
	\item \textbf{Precondizione:} il venditore inserisce un testo composto da un numero eccessivo di caratteri;
	\item \textbf{Postcondizione:} all'attore viene mostrato un messaggio il quale lo avvisa che si il testo inserito è troppo lungo;
	\item \textbf{Scenario principale:} il venditore inserisce un testo troppo lungo e viene informato dell'eccessivo numero di caratteri inseriti nel campo dati.
\end{itemize}

% Estensioni info personali
\UC{Visualizzazione messaggio di errore nel caso in cui il file selezionato non sia del tipo immagine}
% \label{estensione:file-no-tipo-immagine}

Visualizzazione messaggio di errore che segnala al venditore che il file inserito non è di tipo immagine.
\begin{itemize}
    \item \textbf{Attori primari:} venditore;
    \item \textbf{Precondizione:} il venditore inserisce un file di tipo non immagine;
    \item \textbf{Postcondizione:} all'attore viene mostrato un messaggio il quale lo avvisa che si possono selezionare solo file di tipo immagine;
    \item \textbf{Scenario principale:} il venditore inserisce un file di tipo non immagine e viene informato del tipo errato del file.
\end{itemize}

% Estensioni prodotto
\UC{Visualizzazione messaggio di errore in caso di prezzo minore o uguale a zero}
% \label{estensione:prezzo-minore-o-uguale-zero}

Visualizzazione messaggio di errore il quale segnala al venditore che il prezzo del prodotto inserito è minore o uguale a zero.
\begin{itemize}
    \item \textbf{Attori primari:} venditore;
    \item \textbf{Precondizione:} il venditore inserisce un prezzo minore o uguale a zero;
    \item \textbf{Postcondizione:} al venditore viene mostrato un messaggio il quale lo avvisa che il prezzo del prodotto inserito è minore o uguale a zero;
    \item \textbf{Scenario principale:} il venditore inserisce un prezzo minore o uguale a zero e viene informato dell'errore commesso.
\end{itemize}

\UC{Visualizzazione messaggio di errore in caso di quantità minore a zero}
% \label{estensione:quantita-minore-a-zero}

Visualizzazione messaggio di errore che segnala al venditore che la quantità del prodotto inserita è minore a zero.
\begin{itemize}
    \item \textbf{Attori primari:} venditore;
    \item \textbf{Precondizione:} il venditore inserisce una quantità del prodotto minore a zero;
    \item \textbf{Postcondizione:} al venditore viene mostrato un messaggio il quale lo avvisa che la quantità del prodotto inserita è minore a zero;
    \item \textbf{Scenario principale:} il venditore inserisce una quantità del prodotto minore a zero e viene informato dell'errore commesso.
\end{itemize}

\UC{Visualizzazione messaggio di errore in caso di sconto minore di 0\%}
% \label{estensione:sconto-minore-zero}

Visualizzazione messaggio di errore il quale segnala al venditore che lo sconto inserito è minore di 0\%.
\begin{itemize}
    \item \textbf{Attori primari:} venditore;
    \item \textbf{Precondizione:} il venditore inserisce uno sconto minore di 0\%;
    \item \textbf{Postcondizione:} al venditore viene mostrato un messaggio il quale lo avvisa che lo sconto appena inserito è minore di 0\%;
    \item \textbf{Scenario principale:} il venditore inserisce uno sconto minore di 0\% e viene informato dell'impossibilità di applicare tale sconto.
\end{itemize}

\UC{Visualizzazione messaggio di errore in caso di sconto maggiore di 100\%}
% \label{estensione:sconto-maggiore-cento}

Visualizzazione messaggio di errore il quale segnala al venditore che lo sconto inserito è maggiore di 100\%.
\begin{itemize}
    \item \textbf{Attori primari:} venditore;
    \item \textbf{Precondizione:} il venditore inserisce uno sconto maggiore di 100\%;
    \item \textbf{Postcondizione:} al venditore viene mostrato un messaggio il quale lo avvisa che lo sconto appena inserito è maggiore di 100\%;
    \item \textbf{Scenario principale:} il venditore inserisce uno sconto maggiore di 100\% e viene informato dell'impossibilità di applicare tale sconto.
\end{itemize}

\UC{Visualizzazione messaggio di errore tentativo di aggiunta di più del numero massimo di foto consentite relative ad un prodotto}
% \label{estensione:limite-foto-raggiunto}

Visualizzazione messaggio di errore il quale segnala al venditore il tentativo di aggiunta di una nuova foto relativa ad un prodotto, quando si è già raggiunto il limite massimo di foto consentite.
\begin{itemize}
    \item \textbf{Attori primari:} venditore;
    \item \textbf{Precondizione:} il venditore cerca di aggiungere una nuova foto relativa ad un prodotto, anche quando si è già raggiunto il limite massimo di foto consentite per un prodotto;
    \item \textbf{Postcondizione:} al venditore viene mostrato un messaggio il quale lo avvisa che non possono essere inserite ulteriori foto relative al prodotto;
    \item \textbf{Scenario principale:} il venditore cerca di aggiungere una nuova foto relativa ad un prodotto, anche quando si è già raggiunto il limite massimo di foto consentite e viene informato del superamento di tale limite.
\end{itemize}

\UC{Visualizzazione messaggio di errore nessuna foto inserita}
% \label{estensione:nessuna-foto-inserita}

Visualizzazione messaggio di errore il quale segnala al venditore che non è stata inserita alcuna foto relativa al prodotto.
\begin{itemize}
    \item \textbf{Attori primari:} venditore;
    \item \textbf{Precondizione:} il venditore non inserisce alcuna foto relativa ad un prodotto;
    \item \textbf{Postcondizione:} al venditore viene mostrato un messaggio il quale lo avvisa che bisogna inserire almeno una foto relativa ad un prodotto;
    \item \textbf{Scenario principale:} il venditore non inserisce alcuna foto relativa ad un prodotto e viene informato del mancato inserimento di almeno una foto relativa ad un prodotto.
\end{itemize}

% Estensioni Categoria
\UC{Visualizzazione messaggio di errore nome per la categoria già utilizzato}
% \label{estensione:categoria-esistente}

Visualizzazione messaggio di errore il quale segnala al venditore che sta cercando di assegnare ad una categoria un nome già utilizzato da un'altra.
\begin{itemize}
    \item \textbf{Attori primari:} venditore;
    \item \textbf{Precondizione:} il venditore vuole assegnare ad una categoria un nome già utilizzato da un'altra categoria;
    \item \textbf{Postcondizione:} al venditore viene mostrato un messaggio il quale lo avvisa che il nome che sta cercando di assegnare ad una categoria è già in uso;
    \item \textbf{Scenario principale:} il venditore vuole assegnare ad una categoria un nome già utilizzato da un'altra categoria e viene informato della conflittualità del nome inserito per la nuova categoria.
\end{itemize}

% Estensioni indirizzo di consegna
\UC{Visualizzazione messaggio d'errore CAP non valido}
% \label{estensione:cap-non-valido}

Nel caso in cui venga inserito un CAP per un indirizzo di consegna non valido, verrà segnalato attraverso un messaggio di errore opportuno.
\begin{itemize}
	\item \textbf{Attori primari:} acquirente;
	\item \textbf{Precondizione:} l'acquirente ha inserito un CAP per un indirizzo di consegna non valido;
	\item \textbf{Postcondizione:} all'acquirente verrà mostrato un messaggio il quale lo informa della non validità del CAP inserito;
	\item \textbf{Scenario principale:} l'attore non ha inserito un CAP composto da esattamente 5 numeri e verrà avvisato della non validità di quanto scritto.
\end{itemize}

% Estensioni gestore dei pagamenti
\UC{Visualizzazione messaggio d'errore pagamento non andato a buon fine}
% \label{estensione:pagamento-fallito}

Nel caso in cui avvenga un errore durante lo svolgimento del pagamento fornito dal gestore dei pagamenti, questo verrà visualizzato all'acquirente.
\begin{itemize}
	\item \textbf{Attori primari:} acquirente;
	\item \textbf{Attori secondari:} gestore dei pagamenti;
	\item \textbf{Precondizione:} il gestore dei pagamenti sta processando il pagamento dell'acquirente e si verifica un errore;
	\item \textbf{Postcondizione:} all'acquirente verrà mostrato un messaggio il quale lo informa della non riuscita del pagamento con il relativo messaggio;
	\item \textbf{Scenario principale:} il gestore dei pagamenti sta processando la transazione dell'acquirente, si verifica un errore e questo verrà visualizzato all'acquirente informandolo dell'errore specifico.
\end{itemize}