%%%%%%%%%%%%%%%%%%%%%%%%%%%%%%%%%%%%%%%%%%%%%%%%%%%%%%%%%%%%%%%%%%%%%%%%%%%%%%%%%%%%%%%%%%%%%%%%%%%%%%%%%%%%%%%%%%%%%%%%%%%%%%%%%%%%%%%%%%%%%%%%%%%%%%%%%%%%%%%%%%%%%%%%%%%%%%%%%%%%%%%%%%%%%%

\UC{Aggiunta prodotto}
% \begin{figure}[H]
%     \centering
%     \includegraphics[scale=0.1]{Immagini/DiagrammiUC/UC18AggiuntaProdotto.png}
%     \caption{Diagramma di \actualUC: Aggiunta di un prodotto nella piattaforma da parte del venditore} 
%     \label{fig:AggiuntaProdotto}
% \end{figure}

Il venditore aggiunge un prodotto alla piattaforma, così da poter essere venduto.
\begin{itemize}
    \item \textbf{Attori Primari:} Venditore.
    \item \textbf{Precondizione:} Il venditore si trova nella pagina della \glo{dashboard} e seleziona l'azione per aggiungere un prodotto.
    \item \textbf{Postcondizione:} Il venditore ha aggiunto il prodotto nella piattaforma.
    \item \textbf{Scenario Principale:} Il venditore preme sull'azione che porta alla pagina di aggiunta di un prodotto e compie i seguenti passi:
    \begin{itemize}
        \item (\actualUC.1) - Inserimento del nome del prodotto.
        \item (\actualUC.2) - Inserimento della descrizione del prodotto.
        \item (\actualUC.3) - Inserimento delle categorie del prodotto.
        \item (\actualUC.4) - Inserimento del prezzo del prodotto.
        \item (\actualUC.5) - Inserimento dello sconto percentuale al prezzo del prodotto.
        \item (\actualUC.6) - Inserimento della quantità del prodotto disponibile in magazzino.
        \item (\actualUC.7) - Inserimento delle foto del prodotto.
        \item (\actualUC.8) - Scelta di aggiunta del prodotto tra quelli in evidenza.
        \item Conferma gli inserimenti e aggiunge il prodotto alla piattaforma.
    \end{itemize}
\end{itemize}

\resetSubUC

\subUC{Inserimento del nome del prodotto}
Il venditore inserisce il nome del prodotto da aggiungere.
\begin{itemize}
    \item \textbf{Attori Primari:} Venditore.
    \item \textbf{Precondizione:} Il venditore si trova nella vista di aggiunta di un nuovo prodotto.
    \item \textbf{Postcondizione:} Il venditore ha inserito il nome del prodotto .
    \item \textbf{Scenario Principale:} Il venditore si trova nella vista di aggiunta di un nuovo prodotto ed inserisce il nome del prodotto.
    \item \textbf{Scenario Alternativo:} Il venditore non inserisce il nome del prodotto. In questo caso:
    \begin{itemize}
        \item (UC) - Verrà mostrato un messaggio di errore con la segnalazione della causa.
        \item Verrà impedito il proseguimento dell'operazione.
    \end{itemize}
    \item \textbf{Estensioni:}
    \begin{itemize}
        \item (UC30) - Visualizzazione messaggio campo dati obbligatorio non inserito.
    \end{itemize}
\end{itemize}

\subUC{Inserimento della descrizione del prodotto}
Il venditore inserisce la descrizione del prodotto da aggiungere.
\begin{itemize}
    \item \textbf{Attori Primari:} Venditore.
    \item \textbf{Precondizione:} Il venditore si trova nella vista di aggiunta di un nuovo prodotto.
    \item \textbf{Postcondizione:} Il venditore ha inserito la descrizione del prodotto.
    \item \textbf{Scenario Principale:} Il venditore si trova nella vista di aggiunta di un nuovo prodotto ed inserisce la descrizione del prodotto.
    \item \textbf{Scenario Alternativo:} Il venditore non inserisce la descrizione del prodotto. In questo caso:
    \begin{itemize}
        \item (UC) - Verrà mostrato un messaggio di errore con la segnalazione della causa.
        \item Verrà impedito il proseguimento dell'operazione.
    \end{itemize}
    \item \textbf{Estensioni:}
    \begin{itemize}
        \item (UC30) - Visualizzazione messaggio campo dati obbligatorio non inserito.
    \end{itemize}
\end{itemize}

\subUC{Inserimento delle categorie del prodotto}
Il venditore inserisce le categorie del prodotto da aggiungere.
\begin{itemize}
    \item \textbf{Attori Primari:} Venditore.
    \item \textbf{Precondizione:} Il venditore si trova nella vista di aggiunta di un nuovo prodotto.
    \item \textbf{Postcondizione:} Il venditore ha inserito le categorie del prodotto.
    \item \textbf{Scenario Principale:} Il venditore si trova nella vista di aggiunta di un nuovo prodotto ed inserisce le categorie a cui fa parte il prodotto, prendendole dalla lista di categorie disponibili.
    \item \textbf{Scenario Alternativo:} Il venditore non ha inserito alcuna categoria e il prodotto viene automaticamente categorizzato come senza categoria.
\end{itemize}

\subUC{Inserimento del prezzo del prodotto}
Il venditore inserisce il prezzo a cui vendere il prodotto da aggiungere.
\begin{itemize}
    \item \textbf{Attori Primari:} Venditore.
    \item \textbf{Precondizione:} Il venditore si trova nella vista di aggiunta di un nuovo prodotto.
    \item \textbf{Postcondizione:} Il venditore ha inserito il prezzo a cui vendere il prodotto.
    \item \textbf{Scenario Principale:} Il venditore si trova nella vista di aggiunta di un nuovo prodotto ed inserisce il prezzo a cui vendere il prodotto.
    \item \textbf{Scenario Alternativo 1:} Il venditore non inserisce il prezzo del prodotto. In questo caso:
    \begin{itemize}
        \item (UC) - Verrà mostrato un messaggio di errore con la segnalazione della causa.
        \item Verrà impedito il proseguimento dell'operazione.
    \end{itemize}
    \item \textbf{Scenario Alternativo 2:} Il venditore inserisce un prezzo minore o uguale a zero. In questo caso:
    \begin{itemize}
        \item (UC) - Verrà mostrato un messaggio di errore con la segnalazione della causa.
        \item Verrà impedito il proseguimento dell'operazione.
    \end{itemize}
    \item \textbf{Estensioni:}
    \begin{itemize}
        \item (UC) - Visualizzazione messaggio di errore in caso di prezzo minore o uguale a zero.
        \item (UC30) - Visualizzazione messaggio campo dati obbligatorio non inserito.
    \end{itemize}
\end{itemize}

\subUC{Inserimento dello sconto percentuale al prezzo del prodotto}
Il venditore inserisce lo sconto percentuale da applicare al prezzo del prodotto da aggiungere.
\begin{itemize}
    \item \textbf{Attori Primari:} Venditore.
    \item \textbf{Precondizione:} Il venditore si trova nella vista di aggiunta di un nuovo prodotto.
    \item \textbf{Postcondizione:} Il venditore ha inserito lo sconto percentuale da applicare al prezzo del prodotto.
    \item \textbf{Scenario Principale:} Il venditore si trova nella vista di aggiunta di un nuovo prodotto ed inserisce lo sconto percentuale da applicare al prezzo del prodotto. In seguito il prezzo a cui vendere quel prodotto sarà scontato della percentuale applicata.
    \item \textbf{Scenario Alternativo 1:} Nel caso in cui il venditore non inserica alcunché, allora non verrà applicato alcuno sconto. 
    \item \textbf{Scenario Alternativo 2:} Il venditore inserisce uno sconto minore di zero. In questo caso:
    \begin{itemize}
        \item (UC) - Verrà mostrato un messaggio di errore con la segnalazione della causa.
        \item Verrà impedito il proseguimento dell'operazione.
    \end{itemize}
    \item \textbf{Estensioni:}
    \begin{itemize}
        \item (UC) - Visualizzazione messaggio di errore in caso di sconto minore di zero.
    \end{itemize}
\end{itemize}

\subUC{Inserimento della quantità del prodotto disponibile in magazzino}
Il venditore inserisce la quantità disponibile attualmente in magazzino del prodotto da aggiungere.
\begin{itemize}
    \item \textbf{Attori Primari:} Venditore.
    \item \textbf{Precondizione:} Il venditore si trova nella vista di aggiunta di un nuovo prodotto.
    \item \textbf{Postcondizione:} Il venditore ha inserito la quantità disponibile attualmente in magazzino del prodotto.
    \item \textbf{Scenario Principale:} Il venditore si trova nella vista di aggiunta di un nuovo prodotto ed inserisce la quantità disponibile attualmente in magazzino del prodotto.
    \item \textbf{Scenario Alternativo 1:} Nel caso in cui il venditore inserisca 0 come disponibilità attuale, allora il prodotto verrà indicato come non disponibile.
    \item \textbf{Scenario Alternativo 2:} Il venditore inserisce una quantità minore o uguale a zero. In questo caso:
    \begin{itemize}
        \item (UC) - Verrà mostrato un messaggio di errore con la segnalazione della causa.
        \item Verrà impedito il proseguimento dell'operazione.
    \end{itemize}
    \item \textbf{Scenario Alternativo 3:} Il venditore non inserisce la quantità del prodotto. In questo caso:
    \begin{itemize}
        \item (UC) - Verrà mostrato un messaggio di errore con la segnalazione della causa.
        \item Verrà impedito il proseguimento dell'operazione.
    \end{itemize}
    \item \textbf{Estensioni:}
        \begin{itemize}
            \item (UC) - Visualizzazione messaggio di errore in caso di quantità minore o uguale a zero.
            \item (UC30) - Visualizzazione messaggio campo dati obbligatorio non inserito.
        \end{itemize}
\end{itemize}

\subUC{Inserimento foto del prodotto}
Il venditore inserisce le foto relative al prodotto da aggiungere.
\begin{itemize}
    \item \textbf{Attori Primari:} Venditore.
    \item \textbf{Precondizione:} Il venditore si trova nella vista di aggiunta di un nuovo prodotto.
    \item \textbf{Postcondizione:} Il venditore ha inserito le foto relative al prodotto.
    \item \textbf{Scenario Principale:} Il venditore si trova nella vista di aggiunta di un nuovo prodotto ed inserisce massimo quattro foto relative al prodotto.
    \item \textbf{Scenario Alternativo 1:} Il venditore non inserisce alcuna foto del prodotto. In questo caso:
    \begin{itemize}
        \item (UC) - Verrà mostrato un messaggio di errore con la segnalazione della causa.
        \item Verrà impedito il proseguimento dell'operazione.
    \end{itemize}
    \item \textbf{Scenario Alternativo 2:} Il venditore seleziona un file che non è del tipo immagine. In questo caso:
    \begin{itemize}
        \item (UC) - Verrà mostrato un messaggio di errore con la segnalazione della causa.
        \item Verrà impedito il proseguimento dell'operazione.
    \end{itemize}
    \item \textbf{Scenario Alternativo 3:} Il venditore cerca di inserire più di quattro foto relative ad un prodotto. In questo caso:
    \begin{itemize}
        \item (UC) - Verrà mostrato un messaggio di errore con la segnalazione della causa.
        \item Verrà impedito il proseguimento dell'operazione.
    \end{itemize}
    \item \textbf{Estensioni:}
    \begin{itemize}
        \item (UC) - Visualizzazione messaggio di errore tentivo di aggiunta di più di quattro foto relative ad un prodotto.
        \item (UC) - Visualizzazione messaggio di errore nel caso in cui il file selezionato non sia del tipo immagine.
        \item (UC) - Visualizzazione messaggio di errore nessuna foto inserita.
    \end{itemize}
\end{itemize}

\subUC{Scelta di aggiunta del prodotto tra quelli in evidenza}
Il venditore decide se il prodotto è da inserire tra quelli in evidenza da visualizzare nella home.
\begin{itemize}
    \item \textbf{Attori Primari:} Venditore.
    \item \textbf{Precondizione:} Il venditore si trova nella vista di aggiunta di un nuovo prodotto.
    \item \textbf{Postcondizione:} Il venditore ha deciso se il prodotto è da inserire tra quelli in evidenza oppure no ed agisce di conseguenza.
    \item \textbf{Scenario Principale:} Il venditore si trova nella vista di aggiunta di un nuovo prodotto e lo aggiunge tra quelli in evidenza da visualizzare nella home.
    \item \textbf{Scenario Alternativo:} Il venditore decide di non inserire il prodotto tra quelli in evidenza e, per questo motivo, non compie alcuna azione.
\end{itemize}

%%%%%%%%%%%%%%%%%%%%%%%%%%%%%%%%%%%%%%%%%%%%%%%%%%%%%%%%%%%%%%%%%%%%%%%%%%%%%%%%%%%%%%%%%%%%%%%%%%%%%%%%%%%%%%%%%%%%%%%%%%%%%%%%%%%%%%%%%%%%%%%%%%%%%%%%%%%%%%%%%%%%%%%%%%%%%%%%%%%%%%%%%%%%%%

\UC{Aggiunta o rimozione prodotto nella sezione dei prodotti in evidenza}
Il venditore aggiunge o rimuove un prodotto nella sezione dei prodotti in evidenza presente nella home.
\begin{itemize}
    \item \textbf{Attori Primari:} Venditore.
    \item \textbf{Precondizione:} Il venditore si trova nella pagina di aggiunta o modifica di un prodotto e ha premuto sull'azione di aggiunta o rimozione del prodotto nella sezione dei prodotti in evidenza.
    \item \textbf{Postcondizione:} Il prodotto viene aggiunto o rimosso, in base al suo stato attuale, nella sezione sezione dei prodotti in evidenza. 
    \item \textbf{Scenario Principale:} Il venditore vuole aggiungere o rimuovere un prodotto nella sezione dei prodotti in evidenza presente nella home e clicca sulla rispettiva azione.
\end{itemize}

\UC{Modifica prodotto}

\begin{figure}[H]
    \centering
    \includegraphics[scale=0.1]{Immagini/DiagrammiUC/UC19ModificaProdotto.png}
    \caption{Diagramma di \actualUC: Modifica di un prodotto nella piattaforma da parte del venditore}
    \label{fig:ModificaProdotto}
\end{figure}

Il venditore modifica uno o più campi di un prodotto della piattaforma.
\begin{itemize}
    \item \textbf{Attori Primari:} Venditore.
    \item \textbf{Precondizione:} Il venditore si trova nella pagina della dashboard e seleziona un prodotto già presente.
    \item \textbf{Postcondizione:} Il venditore ha modificato il prodotto con le modifiche selezionato.
    \item \textbf{Scenario Principale:} Il venditore preme sull'azione che porta alla pagina di modifica di un prodotto e compie zero, o più, dei seguenti passi:
    \begin{itemize}
        \item (UC23.4) - Inserisce il nuovo nome del prodotto obbligatorio.
        \item (UC23.4) - Inserisce la nuova descrizione del prodotto.
        \item (UC23.4) - Inserisce le nuove categorie del prodotto.
        \item (UC23.5) - Inserisce il nuovo prezzo del prodotto obbligatorio.
        \item (UC23.5) - Inserisce il nuovo sconto percentuale sul prezzo del prodotto.
        \item (UC23.5) - Inserisce la nuova quantità del prodotto disponibile obbligatoria.
        \item (UC23.6) - Aggiunge o rimuove le foto del prodotto, sempre lasciandone almeno una.
        \item (UC18.1) - Aggiunge o rimuove il prodotto, in base al suo stato attuale, nella sezione dei prodotti in evidenza presente nella home.
        \item conferma gli inserimenti e il prodotto viene modificato.
    \end{itemize}
    \item \textbf{Estensioni:}
    \begin{itemize}
        \item (UC30) - Visualizzazione messaggio nome del prodotto non inserito.
        \item (UC30) - Visualizzazione messaggio categoria del prodotto non inserita.
        \item (UC30) - Visualizzazione messaggio prezzo del prodotto non inserito.
        \item (UC30) - Visualizzazione messaggio quantità del prodotto non inserita.
        \item (UC30) - Visualizzazione messaggio foto del prodotto non inserita.
    \end{itemize}
\end{itemize}

%%%%%%%%%%%%%%%%%%%%%%%%%%%%%%%%%%%%%%%%%%%%%%%%%%%%%%%%%%%%%%%%%%%%%%%%%%%%%%%%%%%%%%%%%%%%%%%%%%%%%%%%%%%%%%%%%%%%%%%%%%%%%%%%%%%%%%%%%%%%%%%%%%%%%%%%%%%%%%%%%%%%%%%%%%%%%%%%%%%%%%%%%%%%%%

\UC{Eliminazione prodotto}
\begin{figure}[H]
    \centering
    \includegraphics[width=\textwidth]{Immagini/DiagrammiUC/UC20EliminazioneProdotto}
    \caption{Diagramma di \actualUC: Eliminazione di un prodotto da parte del venditore} 
    \label{fig:EliminazioneProdotto}
\end{figure}

Il venditore vuole eliminare un prodotto precedentemente inserito.
\begin{itemize}
    \item \textbf{Attori Primari:} Venditore.
    \item \textbf{Precondizione:} Il venditore si trova nella pagina della dashboard.
    \item \textbf{Postcondizione:} Il venditore ha eliminato il prodotto selezionato.
    \item \textbf{Scenario Principale:} Il venditore seleziona un prodotto dalla lista di quelli che ha inserito nella piattaforma e preme sul link per eliminarlo definitivamente.
    \begin{itemize}
    \item Il prodotto viene rimosso da PDP, PLP, home, carrelli e dashboard.
    \item Il prodotto rimane negli ordini già effettuati.
    \end{itemize}
    \item \textbf{Scenario Alternativo:} Il venditore seleziona per l'eliminazione, un prodotto che è stato ordinato ma non ancora pagato quindi verrà visualizzato un messaggio di errore e viene negata la possibilità di eliminare il prodotto selezionato.
    \item \textbf{Estensioni:}
    \begin{itemize}
        \item (UC34) - Visualizzazione avviso in caso di prodotto selezionato per l’eliminazione ma non ancora evaso in alcuni ordini.
    \end{itemize}
\end{itemize}

\UC{Rifornimento prodotto}
Il venditore vuole rifornire un prodotto precedentemente inserito che sta per esaurire o è esaurito.
\begin{itemize}
    \item \textbf{Attori Primari:} Venditore.
    \item \textbf{Precondizione:} Il venditore si trova nella pagina della dashboard.
    \item \textbf{Postcondizione:} Il venditore ha rifornito un prodotto.
    \item \textbf{Scenario Principale:} Il venditore seleziona un prodotto dalla lista di quelli che ha inserito nella piattaforma e preme sull'azione per rifornirlo. Per completare il rifornimento deve svolgere i seguenti passi:
    \begin{itemize}
        \item (UC23.5) - Inserisce la quantità.
        \item Preme sull'azione per il salvataggio della modifica.
    \end{itemize}
\end{itemize}

\UC{Lista riepilogo ordini}
Il venditore visualizza una lista con tutti gli ordini effettuati.
\begin{itemize}
    \item \textbf{Attori Primari:} Venditore.
    \item \textbf{Precondizione:} Il venditore accede alla dashboard.
    \item \textbf{Postcondizione:} Viene visualizzata una lista con tutti gli ordini effettuati non ancora evasi, specificandone anche l'email del richiedente.
    \item \textbf{Scenario Principale:} Il venditore accede alla dashboard e visualizza gli ordini effettuati dagli acquirenti che non sono ancora stati evasi. Le azioni che il venditore potrà compiere dalla lista sono: 
    \begin{itemize}
        \item Visualizzare solo i prodotti non evasi (opzione di default), visualizzare solo gli ordini evasi o visualizzarli entrambi.
        \item Evadere un ordine.
        \item Accedere alla pagina di riepilogo ordine premendo sull'azione opportuna (UC14).
    \end{itemize}
\end{itemize}
