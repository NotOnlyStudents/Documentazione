\UC{Visualizzazione riepilogo ordini in gestione}
Il venditore vuole vedere gli ordini chiusi o da gestire.
\begin{itemize}
    \item \textbf{Attori primari:} venditore;
    \item \textbf{Precondizione:} il venditore da qualsiasi schermata in cui si trovi vuole visualizzare gli ordini da gestire o già chiusi;
    \item \textbf{Postcondizione:} il venditore vede tutti gli ordini a suo carico;
    \item \textbf{Scenario principale:} il venditore vuole vedere tutti gli ordini a suo carico da gestire o già chiusi ed accede alla schermata di riepilogo ordini. Da qui può:
    \begin{itemize}
    	\item (UC) - Visualizzazione ordini cronologicamente;
        \item (UC) - Modificare lo stato di un ordine;
        \item (UC) - Visualizzare i dettagli di un ordine;
        \item (UC) - Ricerca di un ordine.
        %\item (UC) - Contattare il cliente che ha effettuato un determinato ordine.
    \end{itemize}
\end{itemize}

\resetSubUC
\subUC{Visualizzazione ordini cronologicamente}
Il venditore vuole visualizzare gli ordini a suo carico in ordine cronologico, dal più al meno recente o viceversa.
\begin{itemize}
	\item \textbf{Attori primari:} venditore;
	\item \textbf{Precondizione:} il venditore è nella schermata di riepilogo ordini ed ha intenzione di cambiare ordine con cui visualizzare gli ordini;
	\item \textbf{Postcondizione:} il venditore visualizzerà nella schermata di riepilogo ordini gli ordini a suo carico in base al metodo da lui selezionato;
	\item \textbf{Scenario principale:} il venditore vuole visualizzare gli ordini cronologicamente ed esegue l'operazione che li riordina dal più al meno recente o viceversa;
	\item \textbf{Scenari alternativi:}
	\begin{enumerate}[label=\lett]
		\item Se il venditore non esegue l'operazione di riordinamento, allora la schermata di riepilogo ordini non verrà aggiornata.
	\end{enumerate}
\end{itemize}

\subUC{Modifica stato ordine}
Il venditore vuole modificare lo stato di un determinato ordine.
\begin{itemize}
	\item \textbf{Attori primari:} venditore;
	\item \textbf{Precondizione:} il venditore ha intenzione di modificare lo stato di un ordine a suo carico;
	\item \textbf{Postcondizione:} il venditore ha modificato lo stato di un ordine;
	\item \textbf{Scenario principale:}
	\begin{itemize}
		\item Il venditore seleziona un ordine dalla lista;
		\item Il venditore modifica lo stato dell'ordine.
	\end{itemize}
\end{itemize}

\subUC{Visualizzazione dettagli ordine}
Il venditore vuole visualizzare i dettagli di un determinato ordine.
\begin{itemize}
	\item \textbf{Attori primari:} venditore;
	\item \textbf{Precondizione:} il venditore vuole vedere i dettagli di un determinato ordine a suo carico;
	\item \textbf{Postcondizione:} il venditore visualizza tutti i dettagli dell'ordine;
	\item \textbf{Scenario principale:}
	\begin{itemize}
		\item Il venditore seleziona un ordine dalla lista;
		\item Il venditore apre la schermata con i dettagli dell'ordine selezionato.
	\end{itemize}
\end{itemize}

\UC{Ricerca di un ordine per codice} 
Il venditore può cercare un ordine dalla propria schermata di riepilogo ordini.
\begin{itemize}
	\item \textbf{Attori primari:} venditore;
	\item \textbf{Precondizione:} il venditore ha selezionato la funzionalità per la ricerca e l'inserimento del codice per individuare un ordine;
	\item \textbf{Postcondizione:} il venditore visualizza l'ordine che corrisponde al codice per il quale si è svolta la ricerca;
	\item \textbf{Scenario principale:} il venditore ha selezionato la funzione prevista per la ricerca. Dopo aver inserito il codice per individuare l'ordine, conferma la ricerca e viene aggiornata la schermata di riepilogo ordini che mostra solamente l'ordine corrispondente al codice indicato;
	\item \textbf{Estensioni:}
	\begin{enumerate}[label=\lett]
		\item Il venditore ha svolto una ricerca che non ha prodotto alcun risultato. In questo caso:
		\begin{itemize}
			\item (UC) - Viene visualizzato il messaggio di ricerca senza alcun risultato.
		\end{itemize}
		\item Il venditore inserisce dei caratteri non numerici nel codice per cui cercare. In questo caso:
		\begin{itemize}
			\item (UC) - Verrà visualizzato il messaggio d'errore codice dell'ordine non valido;
			\item Non verrà eseguita la ricerca.
		\end{itemize}
	\end{enumerate}
\end{itemize}

\subUC{Ricerca di un ordine per cliente}
Il venditore può cercare gli ordini in base al cliente che lo ha effettuato.
\begin{itemize}
	\item \textbf{Attori primari:} venditore;
	\item \textbf{Precondizione:} il venditore è nella schermata di riepilogo ordine e ha inserito il nome di uno uno tra i clienti che hanno effettuato un ordine a suo carico;
	\item \textbf{Postcondizione:} il venditore visualizza la lista di ordini effettuati dal cliente per il quale si è svolta la ricerca;
	\item \textbf{Scenario principale:} il venditore ha selezionato la funzione prevista per la ricerca. Dopo aver inserito il cliente per individuare gli ordini, conferma la ricerca e viene aggiornata la schermata di riepilogo ordini che mostra solamente gli ordini effettuati dal cliente indicato;
	\item \textbf{Estensioni:}
	\begin{enumerate}[label=\lett]
		\item Il venditore ha svolto una ricerca che non ha prodotto alcun risultato. In questo caso:
		\begin{itemize}
			\item (UC) - Viene visualizzato il messaggio di ricerca senza alcun risultato.
		\end{itemize}
	\end{enumerate}
\end{itemize}

%\UC{Contatta cliente}
%Il venditore vuole contattare il cliente che ha effettuato un ordine.
%\begin{itemize}
%	\item \textbf{Attori primari:} venditore;
%	\item \textbf{Precondizione:} il venditore si trova nella schermata di riepilogo ordini;
%	\item \textbf{Postcondizione:} il venditore ha contattato il cliente con successo;
%	\item \textbf{Scenario principale:}
%	\begin{itemize}
%		\item Il venditore ha aperto il form di contatto;
%		\item Compila il modulo apposito;
%		\item Il venditore contatta con successo il cliente;
%	\end{itemize}
%\end{itemize}
% Se si decide di mantenere questo caso d'uso va esteso meglio

\UC{Filtraggio ordini nella schermata di riepilogo ordini}
Il venditore può filtrare gli ordini nella schermata di riepilogo ordini per lo stato dell'ordine secondo un intervallo temporale ed ordinarli in base alla loro data di accettazione.
\begin{itemize}
	\item \textbf{Attori primari:} venditore;
	\item \textbf{Precondizione:} il venditore è nella schermata di riepilogo ordini e ha impostato uno (o più) dei filtri disponibili per i quali cercare;
	\item \textbf{Postcondizione:} il venditore avrà a disposizione tutti gli ordini che soddisfano tutte le condizioni dei vari filtri impostati;
	\item \textbf{Scenario principale:} l'attore è nella schermata di riepilogo ordini e ha impostato uno (o più) dei seguenti filtri:
	\begin{itemize}
		\item (UC) - Impostazione del filtro per stato;
		\item (UC) - Impostazione del filtro per intervallo temporale.
	\end{itemize}
	Di seguito la schermata di riepilogo ordini verrà aggiornata con gli ordini che rispettano tutti i filtri applicati;
	%\item \textbf{Estensioni:}
	%\begin{enumerate}[label=\lett]
	%	\item Il venditore ha impostato i filtri in modo tale che la ricerca con la loro combinazione non dia alcun risultato. In questo caso:
	%	\begin{itemize}
	%		\item (UC) - Viene visualizzato il messaggio di ricerca senza alcun risultato.
	%	\end{itemize}
	%\end{enumerate}
\end{itemize}

\subUC{Filtro per stato}
Il venditore può cercare gli ordini in base al loro stato, selezionando quelle di interesse tra tutti quelli disponibili.
\begin{itemize}
	\item \textbf{Attori primari:} venditore;
	\item \textbf{Precondizione:} il venditore è nella schermata di riepilogo ordini e ha selezionato uno (o più) stati tra quelli disponibili per quali filtrare;
	\item \textbf{Postcondizione:} il venditore visualizzerà nella schermata di riepilogo ordini gli ordini a suo carico che si trovano in uno degli stati selezionati;
	\item \textbf{Scenario principale:} il venditore è nella schermata di riepilogo ordini e ha selezionato uno (o più) stati tra quelli disponibili per quali filtrare e di seguito verranno visualizzati gli ordini che appartengono ad almeno uno degli stati selezionati;
	\item \textbf{Scenari alternativi:}
	\begin{enumerate}[label=\lett]
		\item Se il venditore non imposta il seguente filtro, allora verranno visualizzati tutti i prodotti a prescindere della categoria.
	\end{enumerate}
\end{itemize}

\subUC{Filtro temporale per gli ordini effettuati}
Il venditore filtra temporalmente l'elenco degli ordini a suo carico sulla piattaforma.
\begin{itemize}
	\item \textbf{Attori primari:} venditore;
	\item \textbf{Precondizione:} il venditore si trova nella schermata di riepilogo ordini;
	\item \textbf{Postcondizione:} il venditore visualizza tutti gli ordini che sono stati fatti tra la data di inizio e quella di fine impostate;
	\item \textbf{Scenario principale:} il venditore si trova nella schermata di riepilogo ordini e vuole filtrare gli ordini a suo carico ricevuti durante un certo intervallo temporale. Per farlo dovrà:
	\begin{itemize}
		\item (UC) - Impostare la data iniziale dell'intervallo temporale;
		\item (UC) - Impostare la data finale dell'intervallo temporale.
	\end{itemize}
	\item \textbf{Estensioni:}
	\begin{enumerate}[label=\lett]
		\item Il venditore imposta un intervallo temporale nel quale non sono stati effettuati ordini. In questo caso:
		\begin{itemize}
			\item (UC) - Viene visualizzato il messaggio nessun ordine effettuato nell'arco temporale impostato.
		\end{itemize}
		\item Il venditore inserisce una data iniziale maggiore di quella finale. In questo caso:
		\begin{itemize}
			\item (UC) - Viene visualizzato il messaggio d'errore data iniziale maggiore di quella finale;
			\item Non verrà eseguita la ricerca.
		\end{itemize}
	\end{enumerate}
\end{itemize}

\resetSubUC
\subUC{Impostazione della data iniziale dell'intervallo temporale}
Il venditore imposta la data iniziale dell'intervallo per il quale filtrare l'elenco degli ordini a suo carico ricevuti sulla piattaforma.
\begin{itemize}
	\item \textbf{Attori primari:} venditore;
	\item \textbf{Precondizione:} il venditore si trova nella schermata di riepilogo ordini;
	\item \textbf{Postcondizione:} il venditore ha impostato la data iniziale dell'intervallo per la quale filtrare l'elenco degli ordini ricevuti sulla piattaforma;
	\item \textbf{Scenario principale:} il venditore inserisce una data valida e minore o uguale a quella finale, come data iniziale dell'intervallo per la quale filtrare l'elenco degli ordini ricevuti;
	\item \textbf{Estensioni:}
	\begin{enumerate}[label=\lett]
		\item Il venditore inserisce una data iniziale in un formato non corretto. In questo caso:
		\begin{itemize}
			\item (UC) - Viene visualizzato il messaggio d'errore data non valida;
			\item Non verrà eseguita la ricerca.
		\end{itemize} 
	\end{enumerate}
\end{itemize}

\subUC{Impostazione della data finale dell'intervallo temporale}
Il venditore imposta la data finale dell'intervallo per il quale filtrare l'elenco degli ordini a suo carico ricevuti sulla piattaforma.
\begin{itemize}
	\item \textbf{Attori primari:} venditore;
	\item \textbf{Precondizione:} il venditore si trova nella schermata di riepilogo ordini;
	\item \textbf{Postcondizione:} il venditore ha impostato la data finale dell'intervallo per la quale filtrare l'elenco degli ordini ricevuti sulla piattaforma;
	\item \textbf{Scenario principale:} il venditore inserisce una data valida e maggiore o uguale di quella iniziale, come data finale dell'intervallo per la quale filtrare l'elenco degli ordini ricevuti;
	\item \textbf{Estensioni:}
	\begin{enumerate}[label=\lett]
		\item Il venditore inserisce una data finale non nel formato corretto. In questo caso:
		\begin{itemize}
			\item (UC) - Viene visualizzato il messaggio d'errore data non valida;
			\item Non verrà eseguita la ricerca.
		\end{itemize}
	\end{enumerate}
\end{itemize}