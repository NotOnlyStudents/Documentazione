\UC{Modifica informazioni venditore}
\label{modifica-informazioni-venditore}

\begin{figure}[H]
    \centering
    \includegraphics[scale=0.4]{Immagini/DiagrammiUC/Venditore/ModificaInformazioniVenditore.png}
    \caption{Diagramma di \actualUC: Modifica informazioni venditore}
    \label{fig:modifica-informazioni-venditore}
\end{figure}

Il venditore può modificare le sue informazioni personali.
\begin{itemize}
    \item \textbf{Attori primari:} venditore;
    \item \textbf{Attori secondari:} identity manager;
    \item \textbf{Precondizione:} il venditore si trova nella schermata della propria area personale e ha selezionato l'azione di modifica delle proprie informazioni;
    \item \textbf{Postcondizione:} il venditore ha aggiornato le proprie informazioni personali;
    \item \textbf{Scenario principale:} il venditore si trova nella schermata della propria area personale e può compiere le seguenti azioni:
    \begin{itemize}
    	\item (UC\ref{modifica-informazioni-venditore.nome}) - Modifica del nome del venditore;
    	\item (UC\ref{modifica-informazioni-venditore.cognome}) - Modifica del cognome del venditore;
        \item (UC\ref{modifica-informazioni-venditore.email}) - Modifica dell'indirizzo e-mail del venditore;
        \item (UC\ref{modifica-password}) - Modifica della password;
        \item (UC\ref{modifica-informazioni-venditore.descrizione-azienda}) - Modifica della descrizione dell'azienda.
    \end{itemize}
    Il venditore conferma le modifiche compiute e le informazioni personali verranno modificate;
    \item \textbf{Scenari alternativi:}
    \begin{enumerate}[label=\lett]
    	\item Il venditore non conferma le modifiche effettuate e di conseguenza non verranno aggiornate le informazioni personali;
    \end{enumerate}
    \item \textbf{Estensioni:}
    \begin{enumerate}[label=\lett]
    	\item Il venditore compila i campi dati per la password e la conferma della stessa con due password diverse, in questo caso:
    	\begin{itemize}
    		\item L'identity manager segnala che le due password inserite non coincidono;
    		\item (UC\ref{estensione:password-conferma-diverse}) - Visualizzazione messaggio di errore in caso di password e password di conferma diverse;
    		\item Il venditore può modificare le password inserite.
    	\end{itemize}
    	\item Il venditore cambia il proprio indirizzo e-mail con un'e-mail già presente nella piattaforma e l'identity manager lo segnala. In questo caso:
    	\begin{itemize}
    		\item (UC\ref{estensione:cambio-con-email-esistente}) - Visualizzazione messaggio di errore in caso di cambio e-mail con una già utilizzata nella piattaforma;
    		\item Viene fornita al venditore la possibilità di modificare il proprio indirizzo e-mail inserito.
    	\end{itemize}
    \end{enumerate}
\end{itemize}

\subUC{Modifica del nome del venditore}
\label{modifica-informazioni-venditore.nome}

Il venditore vuole modificare il proprio nome.
\begin{itemize}
	\item \textbf{Attori primari:} venditore;
	\item \textbf{Attori secondari:} identity manager;
	\item \textbf{Precondizione:} il venditore si trova nella schermata di modifica delle proprie informazioni;
	\item \textbf{Postcondizione:} il venditore ha aggiornato il proprio nome;
	\item \textbf{Scenario principale:} il venditore modifica il proprio nome;
	\item \textbf{Scenari alternativi:}
	\begin{enumerate}[label=\lett]
		\item Il venditore non modifica l'attuale nome utilizzato e di conseguenza non cambierà.
	\end{enumerate}
	\item \textbf{Estensioni:}
	\begin{enumerate}[label=\lett]
		\item Il venditore non inserisce il nome e l'identity manager controlla il campo dati che risulta essere vuoto, in questo caso:
		\begin{itemize}
			\item (UC\ref{estensione:campo-obbligatorio-non-inserito}) - Viene mostrato un messaggio d'errore campo dati obbligatorio non inserito;
			\item Viene fornita al venditore la possibilità di inserire un nome.
		\end{itemize}
	\end{enumerate} 
\end{itemize}

\subUC{Modifica del cognome del venditore}
\label{modifica-informazioni-venditore.cognome}

Il venditore vuole modificare il proprio cognome.
\begin{itemize}
	\item \textbf{Attori primari:} venditore;
	\item \textbf{Attori secondari:} identity manager;
	\item \textbf{Precondizione:} il venditore si trova nella schermata di modifica delle proprie informazioni;
	\item \textbf{Postcondizione:} il venditore ha aggiornato il proprio cognome;
	\item \textbf{Scenario principale:} il venditore modifica il proprio cognome;
	\item \textbf{Scenari alternativi:}
	\begin{enumerate}[label=\lett]
		\item Il venditore non modifica l'attuale cognome utilizzato e di conseguenza non cambierà.
	\end{enumerate}
	\item \textbf{Estensioni:}
	\begin{enumerate}[label=\lett]
		\item Il venditore non inserisce il cognome e l'identity manager controlla il campo dati che risulta essere vuoto, in questo caso:
		\begin{itemize}
			\item (UC\ref{estensione:campo-obbligatorio-non-inserito}) - Viene mostrato un messaggio d'errore campo dati obbligatorio non inserito;
			\item Viene fornita al venditore la possibilità di inserire un cognome.
		\end{itemize}
	\end{enumerate}
\end{itemize}

\subUC{Modifica dell'indirizzo e-mail del venditore}
\label{modifica-informazioni-venditore.email}

Il venditore vuole modificare la propria e-mail.
\begin{itemize}
    \item \textbf{Attori primari:} venditore;
    \item \textbf{Attori secondari:} identity manager;
    \item \textbf{Precondizione:} il venditore si trova nella schermata di modifica delle proprie informazioni;
    \item \textbf{Postcondizione:} il venditore ha aggiornato il suo indirizzo e-mail;
    \item \textbf{Scenario principale:} il venditore modifica il proprio indirizzo e-mail;
    \item \textbf{Scenari alternativi:}
    \begin{enumerate}[label=\lett]
    	\item Il venditore non modifica l'attuale indirizzo e-mail utilizzato e di conseguenza non cambierà.
    \end{enumerate}
    \item \textbf{Estensioni:}
    \begin{enumerate}[label=\lett]
    	\item Il venditore ha inserito un indirizzo e-mail e l'identity manager segnala che non è nel formato non corretto, in questo caso:
    	\begin{itemize}
    		\item (UC\ref{estensione:email-non-valida}) - Viene mostrato un messaggio d'errore indirizzo e-mail non rispetta il formato;
    		\item Viene fornita al venditore la possibilità di modificare l'indirizzo e-mail inserito.
    	\end{itemize}
	    \item Il venditore non inserisce l'e-mail e l'identity manager controlla il campo dati che risulta essere vuoto, in questo caso:
	    \begin{itemize}
	    	\item (UC\ref{estensione:campo-obbligatorio-non-inserito}) - Viene mostrato un messaggio d'errore campo dati obbligatorio non inserito;
	    	\item Viene fornita al venditore la possibilità di inserire un indirizzo e-mail.
	    \end{itemize}
    \end{enumerate} 
\end{itemize}

\subUC{Modifica della descrizione dell'azienda}
\label{modifica-informazioni-venditore.descrizione-azienda}

Il venditore vuole modificare la descrizione dell'azienda che viene mostrata nella schermata principale.
\begin{itemize}
    \item \textbf{Attori primari:} venditore;
    \item \textbf{Precondizione:} il venditore si trova nella schermata di modifica delle proprie informazioni;
    \item \textbf{Postcondizione:} il venditore ha aggiornato la descrizione dell'azienda;
    \item \textbf{Scenario principale:} il venditore modifica la descrizione dell'azienda;
    \item \textbf{Scenari alternativi:}
    \begin{enumerate}[label=\lett]
    	\item Il venditore non modifica l'attuale descrizione dell'azienda e di conseguenza non cambierà.
    \end{enumerate}
\end{itemize}

%%%%%%%%%%%%%%%%%%%%%%%%%%%%%%%%%%%%%%%%%%%%%%%%%%%%%%%%%%%%%%%%%%%%%%%%%%%%%%%%%%%%%%%%%%%%%%%%%%%%%%%%%%%%%%%%%%%%%%%%%%%%%%%%%%%%%%%%%%%%%%%%%%%%%%%%%%%%%%%%

\UC{Modifica della password}
\label{modifica-password}

\begin{figure}[H]
    \centering
    \includegraphics[scale=0.6]{Immagini/DiagrammiUC/UtenteAutenticato/ModificaPassword.png}
    \caption{Diagramma di \actualUC: Modifica della password}
    \label{fig:modifica-password}
\end{figure}

L'utente autenticato vuole modificare la propria password.

\begin{itemize}
    \item \textbf{Attori primari:} utente autenticato;
    \item \textbf{Attori secondari:} identity manager;
    \item \textbf{Precondizione:} l'utente autenticato si trova nella schermata di modifica delle proprie informazioni;
    \item \textbf{Postcondizione:} l'utente autenticato ha modificato la sua password;
    \item \textbf{Scenario principale:} l'utente autenticato per poter cambiare la password deve compiere i seguenti passi:
    \begin{itemize}
        \item (UC\ref{modifica-password.nuova-password}) - Inserimento password per la modifica delle informazioni personali;
        \item (UC\ref{modifica-password.conferma-nuova-password}) - Inserimento conferma password per la modifica delle informazioni personali.
    \end{itemize}
    \item \textbf{Scenari alternativi:}
    \begin{enumerate}[label=\lett]
        \item L'utente autenticato non modifica la password attuale e per questo non cambierà. 
    \end{enumerate}
\end{itemize}

\subUC{Inserimento password per la modifica delle informazioni personali}
\label{modifica-password.nuova-password}

L'utente autenticato vuole modificare la propria password e inserisce una nuova password.
\begin{itemize}
    \item \textbf{Attori primari:} utente autenticato;
    \item \textbf{Attori secondari:} identity manager;
    \item \textbf{Precondizione:} l'utente autenticato si trova nella schermata di modifica informazioni personali e vuole inserire una nuova password;
    \item \textbf{Postcondizione:} l'utente autenticato ha inserito una nuova password valida;
    \item \textbf{Scenario principale:} l'utente autenticato si trova nella schermata di modifica informazioni personali e vuole modificare la propria password. In questo caso l'utente autenticato si posiziona nel campo di inserimento vuoto della nuova password e la scrive;
    \item \textbf{Estensioni:}
    \begin{enumerate}[label=\lett]
        \item L'utente autenticato inserisce una password che non rispetta i requisiti minimi di complessità e l'identity manager lo segnala. In questo caso:
        \begin{itemize}
            \item (UC\ref{estensione:password-non-valida}) - Verrà visualizzato il messaggio di errore password non rispetta i requisiti di complessità;
            \item Verrà impedita la modifica delle informazioni.
        \end{itemize}
        \item L'utente autenticato elimina la propria password attuale, non ne inserisce una nuova e l'identity manager lo segnala. In questo caso:
        \begin{itemize}
            \item (UC\ref{estensione:campo-obbligatorio-non-inserito}) - Verrà visualizzato il messaggio di errore campo dati obbligatorio non inserito;
            \item Verrà impedita la modifica delle informazioni.
        \end{itemize}
    \end{enumerate}
\end{itemize}

\subUC{Inserimento conferma password per la modifica delle informazioni personali}
\label{modifica-password.conferma-nuova-password}

L'utente autenticato vuole modificare la propria password e deve confermare quella nuova da inserire.
\begin{itemize}
    \item \textbf{Attori primari:} utente autenticato;
    \item \textbf{Attori secondari:} identity manager;
    \item \textbf{Precondizione:} l'utente autenticato si trova nella schermata di modifica informazioni personali e vuole inserire una nuova password;
    \item \textbf{Postcondizione:} l'utente autenticato ha inserito la conferma della nuova password;
    \item \textbf{Scenario principale:} l'utente autenticato si trova nella schermata di modifica informazioni personali, ha già inserito la nuova password e la deve confermare attraverso le seguenti azioni:
    \begin{itemize}
        \item Si posiziona nel campo di inserimento vuoto della conferma nuova password;
        \item Inserisce la stessa password che è stata inserita nel campo nuova password.
    \end{itemize}
    \item \textbf{Estensioni:}
    \begin{enumerate}[label=\lett]
        \item L'utente autenticato inserisce una password che non rispetta i requisiti minimi di complessità e l'identity manager lo segnala. In questo caso:
        \begin{itemize}
            \item (UC\ref{estensione:password-non-valida}) - Verrà visualizzato il messaggio di errore password non rispetta i requisiti di complessità;
            \item Verrà impedita la modifica delle informazioni.
        \end{itemize}
        \item L'utente autenticato elimina la conferma della propria password attuale, non ne inserisce una nuova e l'identity manager lo segnala. In questo caso:
        \begin{itemize}
            \item (UC\ref{estensione:campo-obbligatorio-non-inserito}) - Verrà visualizzato il messaggio di errore campo dati obbligatorio non inserito;
            \item Verrà impedita la modifica delle informazioni.
        \end{itemize}
    \end{enumerate}
\end{itemize}
