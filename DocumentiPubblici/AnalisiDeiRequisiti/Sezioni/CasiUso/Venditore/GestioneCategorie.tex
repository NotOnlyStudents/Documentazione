%%%%%%%%%%%%%%%%%%%%%%%%%%%%%%%%%%%%%%%%%%%%%%%%%%%%%%%%%%%%%%%%%%%%%%%%%%%%%%%%%%%%%%%%%%%%%%%%%%%%%%%%%%%%%%%%%%%%%%%%%%%%%%%%%%%%%%%%%%%%%%%%%%%%%%%

\UC{Aggiunta nuova categoria}
Il venditore aggiunge una nuova categoria.
\begin{itemize}
    \item \textbf{Attori Primari:} Venditore.
    \item \textbf{Precondizione:} Il venditore si trova nella vista di visualizzazione di tutte le categorie e seleziona l'azione di aggiunta di una nuova categoria.
    \item \textbf{Postcondizione:} La categoria inserita viene creata.
    \item \textbf{Scenario Principale:} Il venditore si trova nella vista di visualizzazione di tutte le categorie e seleziona l'azione di aggiunta di una nuova categoria. Per fare ciò dovrà (\actualUC.1) inserire il nome della nuova categoria e dare la conferma della creazione.
    \item \textbf{Scenario Alternativo:} Il venditore non da la conferma alla creazione della categoria e non verrà aggiunta.
\end{itemize}

\resetSubUC

\subUC{Inserimento nome della categoria}
Il venditore inserisce il nome della nuova categoria da aggiungere.
\begin{itemize}
    \item \textbf{Attori Primari:} Venditore.
    \item \textbf{Precondizione:} Il venditore sta eseguendo l'azione di aggiunta di una nuova categoria.
    \item \textbf{Postcondizione:} Il venditore ha inserito il nome della categoria.
    \item \textbf{Scenario Principale:} Il venditore sta eseguendo l'azione di aggiunta di una nuova categoria e inserisce un nome, per la categoria, che non è già utilizzato.
    \item \textbf{Scenario Alternativo 1:} Il venditore non inserisce alcun nome per la nuova categoria. In questo caso:
    \begin{itemize}
        \item (UC) - Verrà visualizzato il messaggio di errore campo dati obbligatorio non inserito.
        \item Verrà impedita l'aggiunta della nuova categoria.
    \end{itemize}
    \item \textbf{Scenario Alternativo 2:} Il venditore inserisce un nome per la nuova categoria, che è già assegnato ad un'altra categoria. In questo caso:
    \begin{itemize}
        \item (UC) - Verrà visualizzato il messaggio di errore nome per la categoria già utilizzato.
        \item Verrà impedita l'aggiunta della nuova categoria.
    \end{itemize}
    \item \textbf{Estensioni:}
    \begin{itemize}
        \item (UC) - Visualizzazione messaggio campo dati obbligatorio non inserito.
        \item (UC) - Visualizzazione messaggio nome per la categoria già utilizzato.
    \end{itemize}
\end{itemize}


%%%%%%%%%%%%%%%%%%%%%%%%%%%%%%%%%%%%%%%%%%%%%%%%%%%%%%%%%%%%%%%%%%%%%%%%%%%%%%%%%%%%%%%%%%%%%%%%%%%%%%%%%%%%%%%%%%%%%%%%%%%%%%%%%%%%%%%%%%%%%%%%%%%%%%%

\UC{Modifica di una categoria}
Il venditore modifica una categoria precedentemente inserita.
\begin{itemize}
    \item \textbf{Attori Primari:} Venditore.
    \item \textbf{Precondizione:} Il venditore si trova nella vista di visualizzazione di tutte le categorie e seleziona l'azione di modifica di una categoria precedentemente inserita.
    \item \textbf{Postcondizione:} La categoria viene modificata.
    \item \textbf{Scenario Principale:} Il venditore si trova nella vista di visualizzazione di tutte le categorie e seleziona l'azione di modifica di una categoria precedentemente inserita. Per fare ciò dovrà (\actualUC.1) modificare il nome della categoria e dare la conferma della modifica.
    \item \textbf{Scenario Alternativo:} Il venditore non da la conferma alle modifiche effettuate e la categoria non verrà modificata.
\end{itemize}

\resetSubUC

\subUC{Modifica nome della categoria}
Il venditore modifica il nome attuale di una categoria precedentemente inserita.
\begin{itemize}
    \item \textbf{Attori Primari:} Venditore.
    \item \textbf{Precondizione:} Il venditore sta eseguendo l'azione di modifica di una categoria.
    \item \textbf{Postcondizione:} Il venditore ha modificato il nome della categoria.
    \item \textbf{Scenario Principale:} Il venditore sta eseguendo l'azione di modifica di una categoria e modifica il nome della categoria attraverso le seguenti azioni:
    \begin{itemize}
        \item Si posiziona nel campo di inserimento del nome dove è presente quello attualmente utilizzato.
        \item Inserisce il nome nuovo o modifica quello attuale.
    \end{itemize}
    \item \textbf{Scenario Alternativo 1:} Il venditore elimina il nome utilizzato in precedenza non inserendone uno nuovo. In questo caso:
    \begin{itemize}
        \item (UC) - Verrà visualizzato il messaggio di errore campo dati obbligatorio non inserito.
        \item Verrà impedita la modifica della nuova categoria.
    \end{itemize}
    \item \textbf{Scenario Alternativo 2:} Il venditore modifica il nome attuale della categoria, con uno già assegnato ad un'altra categoria. In questo caso:
    \begin{itemize}
        \item (UC) - Verrà visualizzato il messaggio di errore nome per la categoria già utilizzato.
        \item Verrà impedita la modifica della nuova categoria.
    \end{itemize}
    \item \textbf{Estensioni:}
    \begin{itemize}
        \item (UC) - Visualizzazione messaggio campo dati obbligatorio non inserito.
        \item (UC) - Visualizzazione messaggio nome per la categoria già utilizzato.
    \end{itemize}
\end{itemize}

%%%%%%%%%%%%%%%%%%%%%%%%%%%%%%%%%%%%%%%%%%%%%%%%%%%%%%%%%%%%%%%%%%%%%%%%%%%%%%%%%%%%%%%%%%%%%%%%%%%%%%%%%%%%%%%%%%%%%%%%%%%%%%%%%%%%%%%%%%%%%%%%%%%%%%%

\UC{Eliminazione di una categoria}
Il venditore elimina una categoria precedentemente inserita.
\begin{itemize}
    \item \textbf{Attori Primari:} Venditore.
    \item \textbf{Precondizione:} Il venditore si trova nella vista di visualizzazione di tutte le categorie e seleziona l'azione di eliminazione di una categoria precedentemente inserita.
    \item \textbf{Postcondizione:} La categoria viene eliminata.
    \item \textbf{Scenario Principale:} Il venditore si trova nella vista di visualizzazione di tutte le categorie e seleziona l'azione di eliminazione di una categoria. In seguito verrà mostrato un messaggio di conferma con la segnalazione della conseguente eliminazione anche nei prodotti facenti parte di quella specifica categoria. Dopo aver dato la conferma, la categoria verrà eliminata e rimossa anche da tutti i prodotti facenti parte di quella categoria.
    \item \textbf{Scenario Alternativo:} Durante la visualizzazione del messaggio di conferma, il venditore non da il suo consenso per l'eliminazione e la categoria non verrà eliminata.
\end{itemize}
