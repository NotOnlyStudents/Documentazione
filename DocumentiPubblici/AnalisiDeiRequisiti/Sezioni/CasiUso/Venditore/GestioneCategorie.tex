\UC{Aggiunta nuova categoria}
Il venditore aggiunge una nuova categoria.
\begin{itemize}
    \item \textbf{Attori primari:} venditore;
    \item \textbf{Precondizione:} il venditore vuole aggiungere una nuova categoria di prodotti nel sistema;
    \item \textbf{Postcondizione:} la categoria inserita viene creata;
    \item \textbf{Scenario principale:}
    \begin{itemize}
    	\item Il venditore si trova nella schermata di visualizzazione di tutte le categorie;
    	\item Il venditore seleziona la funzionalità per aggiungere una nuova categoria di prodotti;
    	\item (UC) - Inserimento nome della nuova categoria;
    	\item Il venditore conferma la creazione della categoria di prodotti.
    \end{itemize} 
    \item \textbf{Scenari alternativi:}
    \begin{enumerate}[label=\lett]
    	\item Il venditore non conferma la creazione della categoria e di conseguenza questa non verrà aggiunta.
    \end{enumerate} 
\end{itemize}

\resetSubUC
\subUC{Inserimento nome della categoria}
Il venditore inserisce il nome della nuova categoria da aggiungere.
\begin{itemize}
    \item \textbf{Attori primari:} venditore;
    \item \textbf{Attori secondari:} identity manager;
    \item \textbf{Precondizione:} il venditore sta eseguendo l'azione di aggiunta di una nuova categoria;
    \item \textbf{Postcondizione:} il venditore ha inserito il nome della categoria;
    \item \textbf{Scenario principale:} il venditore compila il modulo per l'aggiunta della nuova categoria inserendo un nome;
    \item \textbf{Estensioni:}
    \begin{enumerate}[label=\lett]
    	\item Il venditore non inserisce alcun nome per la nuova categoria e l'identity manager controlla il campo dati per il nome che non risulta compilato. In questo caso:
    	\begin{itemize}
    		\item (UC) - Verrà visualizzato il messaggio di errore campo dati obbligatorio non inserito;
    		\item Viene fornita al venditore la possibilità di modificare il nome della nuova categoria di prodotti.
    	\end{itemize}
    	\item Il venditore inserisce un nome per la nuova categoria e l'identity manager segnala che è già assegnato ad un'altra categoria. In questo caso:
    	\begin{itemize}
    		\item (UC) - Verrà visualizzato il messaggio di errore nome per la categoria già utilizzato;
    		\item Viene fornita al venditore la possibilità di modificare il nome della nuova categoria di prodotti.
    	\end{itemize}
    \end{enumerate}
\end{itemize}

\UC{Modifica di una categoria}
Il venditore modifica una categoria precedentemente inserita.
\begin{itemize}
    \item \textbf{Attori primari:} venditore;
    \item \textbf{Precondizione:} il venditore vuole modificare una categoria precedentemente inserita;
    \item \textbf{Postcondizione:} la categoria viene aggiornata;
    \item \textbf{Scenario principale:}
    \begin{itemize}
    	\item Il venditore si trova nella schermata di visualizzazione di tutte le categorie;
    	\item Il venditore seleziona la funzionalità per modificare una categoria di prodotti;
    	\item (UC) - Modifica nome della categoria;
    	\item Il venditore conferma la modifica della categoria di prodotti.
    \end{itemize} 
    \item \textbf{Scenari alternativi:} 
    \begin{enumerate}[label=\lett]
    	\item Il venditore non da la conferma alle modifiche effettuate e di conseguenza la categoria non verrà modificata.
    \end{enumerate}
\end{itemize}

\resetSubUC
\subUC{Modifica nome della categoria}
Il venditore modifica il nome attuale di una categoria precedentemente inserita.
\begin{itemize}
    \item \textbf{Attori primari:} venditore;
    \item \textbf{Attori secondari:} identity manager;
    \item \textbf{Precondizione:} il venditore sta eseguendo l'azione di modifica di una categoria;
    \item \textbf{Postcondizione:} il venditore ha modificato il nome della categoria;
    \item \textbf{Scenario principale:} il venditore compila il modulo per l'aggiunta della nuova categoria inserendo un nuovo nome o modificando quello attuale;
    \item \textbf{Estensioni:} 
    \begin{enumerate}[label=\lett]
    	\item Il venditore elimina il nome utilizzato in precedenza non inserendone uno nuovo e l'identity manager controlla il campo dati per il nome che non risulta compilato. In questo caso:
    	\begin{itemize}
    		\item (UC) - Verrà visualizzato il messaggio di errore campo dati obbligatorio non inserito.
    		\item Viene fornita al venditore la possibilità di modificare il nome della nuova categoria di prodotti.
    	\end{itemize}
    	\item Il venditore inserisce un nome per la nuova categoria e l'identity manager segnala che è già assegnato ad un'altra categoria. In questo caso:
		\begin{itemize}
			\item (UC) - Verrà visualizzato il messaggio di errore nome per la categoria già utilizzato;
			\item Viene fornita al venditore la possibilità di modificare il nome della nuova categoria di prodotti.
		\end{itemize}
    \end{enumerate}
\end{itemize}

\UC{Eliminazione di una categoria}
Il venditore elimina una categoria precedentemente inserita.
\begin{itemize}
    \item \textbf{Attori primari:} venditore;
    \item \textbf{Precondizione:} il venditore vuole eliminare una categoria precedentemente inserita;
    \item \textbf{Postcondizione:} la categoria viene eliminata;
    \item \textbf{Scenario principale:}
    \begin{itemize}
    	\item Il venditore si trova nella schermata di visualizzazione di tutte le categorie;
    	\item Il venditore seleziona la funzionalità per eliminare una categoria di prodotti;
    	\item Il venditore conferma la rimozione della categoria di prodotti con la conseguente eliminazione anche nei prodotti appartenenti alla categoria indicata.
    \end{itemize}
    \item \textbf{Scenari alternativi:}
    \begin{enumerate}[label=\lett]
    	\item Durante la visualizzazione del messaggio di conferma, il venditore non da il suo consenso per l'eliminazione e di conseguenza la categoria non verrà eliminata.
    \end{enumerate}
\end{itemize}