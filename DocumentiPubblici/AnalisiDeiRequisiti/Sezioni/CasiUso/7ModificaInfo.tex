%% 5 Lorenzo
%5. My profile. Where the registered user can:
%    a. Update profile information (ie: picture, description, email address);
%    b. Have an overview of all the orders he made on the website;

%%%%%%%%%%%%%%%%%%%%%%%%%%%%%%%%%%%%%%%%%%%%%%%%%%%%%%%%%%%%%%%%%%%%%%%%%%%%%%%%%%%%%%%%%%%%%%%%%%%%%%%%%%%%%%%%%

\UC{Modifica informazioni acquirente}

L'acquirente vuole modificare le sue informazioni personali.
\begin{itemize}
    \item \textbf{Attori primari:} Acquirente.
    \item \textbf{Precondizione:} L'acquirente si trova nella vista di modifica informazioni personali.
    \item \textbf{Postcondizione:} L'acquirente ha aggiornato le sue informazioni personali.
    \item \textbf{Scenario Principale:} L'acquirente ha premuto sull'azione per la modifica delle informazioni personali e, dopo aver compiuto la modifica, potrà confermarle attraverso l'azione di salvataggio. Le informazioni che si possono modificare sono:
    \begin{itemize}
        \item (\actualUC.1) - Inserimento nuovo indirizzo email.
        \item (\actualUC.2) - Inserimento nuova password.
        \item (\actualUC.3) - Inserimento conferma nuova password.
    \end{itemize}
    \item \textbf{Scenario Alternativo:} L'acquirente non conferma le modifiche effettuato e, perciò, non verrano modificate le informazioni personali.
\end{itemize}

\resetSubUC

\subUC{Inserimento nuovo indirizzo email}
L'acquirente vuole modificare la propria email.
\begin{itemize}
    \item \textbf{Attori primari:} Acquirente.
    \item \textbf{Precondizione:} L'acquirente si trova nella vista di modifica informazioni personali e vuole inserire un nuovo indirizzo email.
    \item \textbf{Postcondizione:} L'acquirente ha aggiornato la sua email.
    \item \textbf{Scenario Principale:} L'acquirente si trova nella vista di modifica informazioni personali e vuole modificare la propria email attraverso le seguenti azioni:
        \begin{itemize}
            \item Si posiziona nel campo di inserimento della nuova email dove è presente quella attualmente utilizzata.
            \item Inserisce quella nuova o modifica quella attuale.
        \end{itemize}
    \item \textbf{Scenario Alternativo 1:} L'acquirente non modfica l'attuale indirizzo email utilizzato e, perciò, non cambierà.
    \item \textbf{Scenario Alternativo 2:} L'acquirente inserisce un'email già utilizzata nella piattaforma. In questo caso:
    \begin{itemize}
        \item (UC) - Verrà mostrato un messaggio di errore con la segnalazione della causa.
        \item Verrà impedito il proseguimento dell'operazione.
    \end{itemize}
    \item \textbf{Scenario Alternativo 3:} L'acquirente inserisce un'email non valida. In questo caso:
    \begin{itemize}
        \item (UC) - Verrà mostrato un messaggio di errore con la segnalazione della causa.
        \item Verrà impedito il proseguimento dell'operazione.
    \end{itemize}
    \item \textbf{Scenario Alternativo 4:} L'acquirente non inserisce l'email e lascia il campo di inserimento vuoto. In questo caso:
    \begin{itemize}
        \item (UC) - Verrà mostrato un messaggio di errore con la segnalazione della causa.
        \item Verrà impedito il proseguimento dell'operazione.
    \end{itemize}
    \item \textbf{Estensioni:}
    \begin{itemize}
        \item (UC) - Visualizzazione messaggio di errore in caso di email già utilizata nella piattaforma.
        \item (UC) - Visualizzazione messaggio di errore in caso di email non valida.
        \item (UC) - Visualizzazione messaggio campo dati obbligatorio non inserito.
    \end{itemize}
\end{itemize}

\subUC{Inserimento nuova password}
L'acquirente vuole modificare la propria password.
\begin{itemize}
    \item \textbf{Attori primari:} Acquirente.
    \item \textbf{Precondizione:} L'acquirente si trova nella vista di modifica informazioni personali e vuole inserire una nuova password.
    \item \textbf{Postcondizione:} L'acquirente ha inserito la nuova password.
    \item \textbf{Scenario Principale:} L'acquirente si trova nella vista di modifica informazioni personali e vuole modificare la propria password attraverso le seguenti azioni:
        \begin{itemize}
            \item Si posiziona nel campo di inserimento vuoto della nuova password.
            \item La scrive.
            \item Inserisce la stessa password nel campo di conferma nuova password.
        \end{itemize}
    \item \textbf{Scenario Alternativo 1:} L'acquirente non inserisce alcuna password e lascia il campo di inserimento nuova password vuoto. In questo caso non verrà modificata la password.
    \item \textbf{Scenario Alternativo 2:} L'acquirente inserisce una password troppo debole. In questo caso:
    \begin{itemize}
        \item (UC) - Verrà mostrato un messaggio di errore con la segnalazione della causa.
        \item Verrà impedito il proseguimento dell'operazione.
    \end{itemize}
    \item \textbf{Estensioni:}
    \begin{itemize}
        \item (UC) - Visualizzazione messaggio di errore in caso di password troppo debole.
    \end{itemize}
\end{itemize}

\subUC{Inserimento conferma nuova password}
L'acquirente vuole modificare la propria password e deve confermare quella nuova da inserire.
\begin{itemize}
    \item \textbf{Attori primari:} Acquirente.
    \item \textbf{Precondizione:} L'acquirente si trova nella vista di modifica informazioni personali e vuole inserire una nuova password.
    \item \textbf{Postcondizione:} L'acquirente ha inserito la conferma della nuova password.
    \item \textbf{Scenario Principale:} L'acquirente si trova nella vista di modifica informazioni personali, ha già inserito la nuova password e la deve confermare attraverso le seguenti azioni:
        \begin{itemize}
            \item Si posiziona nel campo di inserimento vuoto della conferma nuova password.
            \item Inserisce la stessa password che è stata inserita nel campo conferma password.
        \end{itemize}
    \item \textbf{Scenario Alternativo 1:} L'acquirente non inserisce alcuna password e lascia il campo di inserimento congerma nuova password vuoto. In questo caso non verrà modificata la password.
    \item \textbf{Scenario Alternativo 2:} L'acquirente inserisce una password di conferma diversa da quella inserita nel campo di inserimento nuova password. In questo caso:
    \begin{itemize}
        \item (UC) - Verrà mostrato un messaggio di errore con la segnalazione della causa.
        \item Verrà impedito il proseguimento dell'operazione.
    \end{itemize}
    \item \textbf{Estensioni:}
    \begin{itemize}
        \item (UC) - Visualizzazione messaggio di errore in caso di password e password di conferma diverse
    \end{itemize}
\end{itemize}

%%%%%%%%%%%%%%%%%%%%%%%%%%%%%%%%%%%%%%%%%%%%%%%%%%%%%%%%%%%%%%%%%%%%%%%%%%%%%%%%%%%%%%%%%%%%%%%%%%%%%%%%%%%%%%%%%

\UC{Modifica informazioni venditore}

Il venditore vuole modificare le sue informazioni personali.
\begin{itemize}
    \item \textbf{Attori primari:} Venditore.
    \item \textbf{Precondizione:} Il venditore si trova nella pagina personale e ha premuto sull'azione per la modifica delle informazioni.
    \item \textbf{Postcondizione:} Il venditore ha aggiornato le sue informazioni personali.
    \item \textbf{Scenario Principale:} Il venditore ha premuto sull'azione per la modifica delle informazioni personali e, dopo aver compiuto la modifica, potrà confermarle attraverso l'azione di salvataggio. Le informazioni che si possono modificare sono:
    \begin{itemize}
        \item (\actualUC.1) - Inserimento nuovo indirizzo e-mail.
        \item (\actualUC.2) - Inserimento nuova password.
        \item (\actualUC.3) - Inserimento conferma nuova password.
        \item (\actualUC.4) - Inserimento nuova foto profilo.
        \item (\actualUC.5) - Rimozione foto profilo.
        \item (\actualUC.6) - Modifica della descrizione dell'azienda.
    \end{itemize}
    \item \textbf{Scenario Alternativo:} Il venditore non conferma le modifiche effettuato e, perciò, non verrano modificate le informazioni personali.
\end{itemize}

\resetSubUC

\subUC{Inserimento nuovo indirizzo email}
Il venditore vuole modificare la propria email.
\begin{itemize}
    \item \textbf{Attori primari:} Venditore.
    \item \textbf{Precondizione:} Il venditore si trova nella vista di modifica informazioni personali e vuole inserire un nuovo indirizzo email.
    \item \textbf{Postcondizione:} Il venditore ha aggiornato la sua email.
    \item \textbf{Scenario Principale:} Il venditore si trova nella vista di modifica informazioni personali e vuole modificare la propria email attraverso le seguenti azioni:
    \begin{itemize}
        \item Si posiziona nel campo di inserimento della nuova email dove è presente quella attualmente utilizzata.
        \item Inserisce quella nuova o modifica quella attuale.
        \item Preme sull'azione di salvataggio delle modifiche
    \end{itemize}
    \item \textbf{Scenario Alternativo 1:} Il venditore non modfica l'attuale indirizzo email utilizzato e, perciò, non cambierà.
    \item \textbf{Scenario Alternativo 2:} Il venditore inserisce un'email già utilizzata nella piattaforma. In questo caso:
    \begin{itemize}
        \item (UC) - Verrà mostrato un messaggio di errore con la segnalazione della causa.
        \item Verrà impedito il proseguimento dell'operazione.
    \end{itemize}
    \item \textbf{Scenario Alternativo 3:} Il venditore inserisce un'email non valida. In questo caso:
    \begin{itemize}
        \item (UC) - Verrà mostrato un messaggio di errore con la segnalazione della causa.
        \item Verrà impedito il proseguimento dell'operazione.
    \end{itemize}
    \item \textbf{Scenario Alternativo 4:} Il venditore non inserisce l'email e lascia il campo di inserimento vuoto. In questo caso:
    \begin{itemize}
        \item (UC) - Verrà mostrato un messaggio di errore con la segnalazione della causa.
        \item Verrà impedito il proseguimento dell'operazione.
    \end{itemize}
    \item \textbf{Estensioni:}
    \begin{itemize}
        \item (UC) - Visualizzazione messaggio di errore in caso di email già registrata nella piattaforma.
        \item (UC) - Visualizzazione messaggio di errore in caso di email non valida.
        \item (UC) - Visualizzazione messaggio campo dati obbligatorio non inserito.
    \end{itemize}
\end{itemize}

\subUC{Inserimento nuova password}
Il venditore vuole modificare la propria password.
\begin{itemize}
    \item \textbf{Attori primari:} Acquirente.
    \item \textbf{Precondizione:} Il venditore si trova nella vista di modifica informazioni personali e vuole inserire una nuova password.
    \item \textbf{Postcondizione:} Il venditore ha inserito la nuova password.
    \item \textbf{Scenario Principale:} Il venditore si trova nella vista di modifica informazioni personali e vuole modificare la propria password attraverso le seguenti azioni:
        \begin{itemize}
            \item Si posiziona nel campo di inserimento vuoto della nuova password.
            \item La scrive.
            \item Inserisce la stessa password nel campo di conferma nuova password.
        \end{itemize}
    \item \textbf{Scenario Alternativo 1:} Il venditore non inserisce alcuna password e lascia il campo di inserimento nuova password vuoto. In questo caso non verrà modificata la password.
    \item \textbf{Scenario Alternativo 2:} Il venditore inserisce una password troppo debole. In questo caso:
    \begin{itemize}
        \item (UC) - Verrà mostrato un messaggio di errore con la segnalazione della causa.
        \item Verrà impedito il proseguimento dell'operazione.
    \end{itemize}
    \item \textbf{Estensioni:}
    \begin{itemize}
        \item (UC) - Visualizzazione messaggio di errore in caso di password troppo debole.
    \end{itemize}
\end{itemize}

\subUC{Inserimento conferma nuova password}
Il venditore vuole modificare la propria password e deve confermare quella nuova da inserire.
\begin{itemize}
    \item \textbf{Attori primari:} Acquirente.
    \item \textbf{Precondizione:} Il venditore si trova nella vista di modifica informazioni personali e vuole inserire una nuova password.
    \item \textbf{Postcondizione:} Il venditore ha inserito la conferma della nuova password.
    \item \textbf{Scenario Principale:} Il venditore si trova nella vista di modifica informazioni personali, ha già inserito la nuova password e la deve confermare attraverso le seguenti azioni:
        \begin{itemize}
            \item Si posiziona nel campo di inserimento vuoto della conferma nuova password.
            \item Inserisce la stessa password che è stata inserita nel campo conferma password.
        \end{itemize}
    \item \textbf{Scenario Alternativo 1:} Il venditore non inserisce alcuna password e lascia il campo di inserimento congerma nuova password vuoto. In questo caso non verrà modificata la password.
    \item \textbf{Scenario Alternativo 2:} Il venditore inserisce una password di conferma diversa da quella inserita nel campo di inserimento nuova password. In questo caso:
    \begin{itemize}
        \item (UC) - Verrà mostrato un messaggio di errore con la segnalazione della causa.
        \item Verrà impedito il proseguimento dell'operazione.
    \end{itemize}
    \item \textbf{Estensioni:}
    \begin{itemize}
        \item (UC) - Visualizzazione messaggio di errore in caso di password e password di conferma diverse
    \end{itemize}
\end{itemize}

\subUC{Inserimento nuova foto profilo}
Il venditore vuole modificare la propria foto profilo.
\begin{itemize}
    \item \textbf{Attori Primari:} Venditore.
    \item \textbf{Precondizione:} Il venditore si trova nella vista di modifica informazioni personali e vuole modificare la sua foto profilo.
    \item \textbf{Postcondizione:} Il venditore ha inserito una nuova foto profilo.
    \item \textbf{Scenario Principale:} Il venditore vuole inserire una nuova foto profilo attraverso le seguenti azioni:
    \begin{itemize}
        \item Si posiziona nel campo di inserimento della nuova foto profilo.
        \item Seleziona un'immagine tra quelle disponibili localmente nel proprio computer.
        \item Viene caricata.
        \item Preme sull'azione di salvataggio delle modifiche.
    \end{itemize}
    \item \textbf{Scenario Alternativo 1:} Il venditore non seleziona alcun file e la foto profilo non verrà cambiata.
    \item \textbf{Scenario Alternativo 2:} Il venditore seleziona un file del tipo non immagine. In questo caso:
    \begin{itemize}
        \item (UC) - Verrà mostrato un messaggio di errore con la segnalazione della causa.
        \item Verrà impedito il proseguimento dell'operazione.
    \end{itemize}
    \item \textbf{Estensioni:}
    \begin{itemize}
        \item (UC32) - Visualizzazione messaggio di errore nel caso in cui il file selezionato non sia del tipo immagine.
    \end{itemize}
\end{itemize}

\subUC{Rimozione foto profilo}
Il venditore vuole rimuovere la propria foto profilo.
\begin{itemize}
    \item \textbf{Attori Primari:} Venditore.
    \item \textbf{Precondizione:} Il venditore si trova nella vista di modifica informazioni personali e ha già inserito una foto profilo.
    \item \textbf{Postcondizione:} Il venditore ha rimosso la sua foto profilo.
    \item \textbf{Scenario Principale:} Il venditore rimuove rimuovere la sua foto profilo attraverso l'adeguata azione e come foto profilo verrà utilizzata un'immagine di riempimento.
\end{itemize}

\subUC{Modifica della descrizione dell'azienda}
Il venditore vuole modificare la descrizione dell'azienda che viene mostrata nella vista home.
\begin{itemize}
    \item \textbf{Attori Primari:} Venditore.
    \item \textbf{Precondizione:} Il venditore si trova nella vista di modifica informazioni personali e vuole modificare l'attuale descrizione dell'azienda.
    \item \textbf{Postcondizione:} Il venditore ha modificato l'attuale descrizione dell'azienda.
    \item \textbf{Scenario Principale:} Il venditore vuole modificare l'attuale descrizione dell'azienda attraverso le seguenti azioni:
    \begin{itemize}
        \item Si posiziona nel campo di modifica della descrizione dell'azienda dove è presente quella attualmente inserita.
        \item La modifica come meglio desidera.
        \item Preme sull'azione di salvataggio delle modifiche.
    \end{itemize}
    \item \textbf{Scenario Alternativo:} Il venditore non modfica l'attuale descrizione dell'azienda e, perciò, non cambierà.
\end{itemize}

%%%%%%%%%%%%%%%%%%%%%%%%%%%%%%%%%%%%%%%%%%%%%%%%%%%%%%%%%%%%%%%%%%%%%%%%%%%%%%%%%%%%%%%%%%%%%%%%%%%%%%%%%%%%


\UC{Eliminazione account}
L'acquirente può eliminare il proprio account.
\begin{itemize}
    \item \textbf{Attori Primari:} Acquirente.
    \item \textbf{Precondizione:} L'acquirente si trova nella propria pagina personale (UC3.2.1) e ha selezionato l'azione di cancellazione del proprio account. 
    \item \textbf{Postcondizione:} L'account dell'acquirente non è più presente nella piattaforma
    \item \textbf{Scenario Principale:} L'acquirente vuole eliminare il proprio account e compie le seguenti operazioni:
    \begin{itemize}
        \item Preme sull'azione di cancellazione del proprio account.
        \item Viene visualizzato un messaggio di conferma dell'operazione.
        \item Se conferma viene eliminato l'account.
    \end{itemize}
    \item \textbf{Scenario Alternativo:} Se non conferma viene riportato alla propria pagina personale.
\end{itemize}

\UC{Visualizzazione ordini effettuati}
L'acquirente vuole vedere l'elenco degli ordini effettuati sulla piattaforma.
\begin{itemize}
    \item \textbf{Attori primari:} Acquirente.
    \item \textbf{Precondizione:} L'acquirente si trova nella pagina del suo profilo.
    \item \textbf{PostCondizione:} L'acquirente visualizza l'elenco degli ordini.
    \item \textbf{Scenario Principale:} L'utente, all'interno della pagina del profilo, vede l'elenco degli ordini effettuati.
        Per ogni ordine effettuato sono indicati i prodotti inclusi nell'ordine, la loro quantità e il prezzo totale pagato.
    \item \textbf{Scenario Alternativo:} L'utente non ha ancora effettuato ordini, viene visualizzato il messaggio "Nessun ordine effettuato" e sarà data la possibilità all'attore di andare alla home per iniziare gli acquisti.
\end{itemize}
