\rfun{O}{\ref{visualizzazione-ordini-in-gestione}} & Il venditore può visualizzare l'elenco degli ordini chiusi o da gestire. Per ognuno di questi potrà visualizzare: il codice numerico dell'ordine, lo stato dell'ordine, il prezzo totale che è stato pagato, l'indirizzo e-mail dell'acquirente che ha effettuato l'ordine, l'indirizzo a cui è stato consegnato o verrà consegnato e una lista di tutti i prodotti acquistati. & UC\ref{visualizzazione-ordini-in-gestione} \\

\rfun{O}{\ref{visualizzazione-ordini-in-gestione}} & Il venditore può visualizzare i prodotti che sono stati acquistati in un ordine ricevuto. Per ognuno di questi verrà visualizzato: il nome del prodotto, la quantità acquistata e il prezzo totale a cui è stato acquistato. & UC\ref{visualizzazione-ordini-in-gestione} \\

\rfun{O}{\ref{modifica-stato-ordine}} & Il venditore può modificare lo stato di un determinato ordine. & UC\ref{modifica-stato-ordine} \\

\rfun{O}{} & Il venditore visualizzerà gli ordini ricevuti per ordine cronologico decrescente come ordinamento predefinito. & Interna \\

\rfun{O}{\ref{ricerca-codice-ordine-venditore}} & Il venditore può cercare un ordine sapendo il suo codice numerico. & UC\ref{ricerca-codice-ordine-venditore} \\
     
\rfun{O}{\ref{ricerca-cliente-ordine-venditore}} & Il venditore può cercare gli ordini in base al cliente che lo ha effettuato. & UC\ref{ricerca-cliente-ordine-venditore} \\
    
\rfun{O}{\ref{filtro-ordini-venditore}} & Il venditore può filtrare gli ordini nella schermata di riepilogo ordini. & UC\ref{filtro-ordini-venditore} \\
    
\rfun{O}{\ref{filtro-ordini-venditore.stato}} & Il venditore può filtrare gli ordini in base al loro stato. & UC\ref{filtro-ordini-venditore.stato} \\
    
\rfun{O}{\ref{filtro-ordini-venditore.temporale}} & Il venditore può filtrare per intervallo temporale gli ordini ricevuti. & UC\ref{filtro-ordini-venditore.temporale} \\
    
\rfun{O}{\ref{filtro-ordini-venditore.temporale.data-iniziale}} & Il venditore può impostare la data iniziale dell'intervallo per il quale filtrare l'elenco degli ordini ricevuti sulla piattaforma. & UC\ref{filtro-ordini-venditore.temporale.data-iniziale} \\
    
\rfun{O}{\ref{filtro-ordini-venditore.temporale.data-finale}} & Il venditore può impostare la data finale dell'intervallo per il quale filtrare l'elenco degli ordini ricevuti sulla piattaforma. & UC\ref{filtro-ordini-venditore.temporale.data-finale} \\
