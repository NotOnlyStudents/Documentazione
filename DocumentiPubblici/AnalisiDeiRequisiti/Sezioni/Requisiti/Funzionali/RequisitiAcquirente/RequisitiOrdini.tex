\rfun{O}{} & L'acquirente può accedere alla schermata con tutti gli ordini effettuati da qualsiasi altra schermata della piattaforma. & Interna \\

\rfun{O}{\ref{visualizzazione-ordini-effettuati}} & L'acquirente può visualizzare l'elenco degli ordini effettuati sulla piattaforma. Per ognuno di questi potrà visualizzare: il codice dell'ordine, lo stato dell'ordine, il prezzo totale che è stato pagato, l'indirizzo a cui è stato consegnato o verrà consegnato e una lista di tutti i prodotti acquistati.& UC\ref{visualizzazione-ordini-effettuati} \\

\rfun{O}{\ref{visualizzazione-ordini-effettuati}} & L'acquirente può visualizzare i prodotti che sono stati acquistati in un ordine effettuato. Per ognuno di questi verrà visualizzato: il nome del prodotto, la quantità acquistata e il prezzo totale a cui è stato acquistato. & UC\ref{visualizzazione-ordini-effettuati} \\

\rfun{O}{\ref{ricerca-codice-ordine-acquirente}} & L'acquirente può cercare un ordine sapendo il suo codice identificativo. & UC\ref{ricerca-codice-ordine-acquirente} \\

\rfun{O}{\ref{filtro-temporale-ordini-acquirente}} & L'acquirente può filtrare temporalmente l'elenco degli ordini effettuati sulla piattaforma. & UC\ref{filtro-temporale-ordini-acquirente} \\

\rfun{O}{\ref{filtro-temporale-ordini-acquirente.data-iniziale}} & L'acquirente può impostare la data iniziale dell'intervallo per il quale filtrare l'elenco degli ordini effettuati sulla piattaforma. & UC\ref{filtro-temporale-ordini-acquirente.data-iniziale} \\

\rfun{O}{\ref{filtro-temporale-ordini-acquirente.data-finale}} & L'acquirente impostare la data finale dell'intervallo per il quale filtrare l'elenco degli ordini effettuati sulla piattaforma. & UC\ref{filtro-temporale-ordini-acquirente.data-finale} \\

% \rfun{O}{\ref{}} &  & UC\ref{} \\
