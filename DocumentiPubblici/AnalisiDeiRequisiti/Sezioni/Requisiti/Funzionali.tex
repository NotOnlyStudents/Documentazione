\rowcolors{2}{white}{celeste} 
\renewcommand{\arraystretch}{1.5}

\addcontentsline{lot}{table}{Requisiti funzionali}

\begin{longtable}{c C{9.5cm} C{2.5cm}} 

	\rowcolor{darkblue}
	\textcolor{white}{\textbf{Codice Requisito}}&
	\textcolor{white}{\textbf{Descrizione}}&
    \textcolor{white}{\textbf{Fonte}} \\
    
    % Acquirente

    \rfun{O}{\ref{registrazione}} & L'utente non autenticato che non dispone di credenziali può registrarsi e accedere come acquirente alla piattaforma. & UC\ref{registrazione} \\ 

\rfun{O}{\ref{registrazione.modulo.nome}} & L'utente non autenticato inserisce il nome con il quale vuole registrarsi. & UC\ref{registrazione.modulo.nome} \\

\rfun{O}{\ref{registrazione.modulo.cognome}} & L'utente non autenticato inserisce il cognome con il quale vuole registrarsi. & UC\ref{registrazione.modulo.cognome} \\

\rfun{O}{\ref{registrazione.modulo.email}} & L'utente non autenticato inserisce l'indirizzo e-mail con il quale vuole registrarsi. & UC\ref{registrazione.modulo.email} \\

\rfun{O}{\ref{registrazione.modulo.password}} & L'utente non autenticato inserisce la password con la quale vuole registrarsi. & UC\ref{registrazione.modulo.password} \\

\rfun{O}{\ref{registrazione.modulo.conferma-password}} & L'utente non autenticato inserisce la conferma della password con la quale vuole registrarsi. & UC\ref{registrazione.modulo.conferma-password} \\

\rfun{O}{\ref{autenticazione-venditore}} & L'utente non autenticato che dispone di credenziali venditore può accedere alla piattaforma usando l'e-mail e la password & UC\ref{autenticazione-venditore} \\

\rfun{O}{\ref{autenticazione-acquirente}} & L'utente non autenticato che dispone di credenziali acquirente può accedere nella piattaforma usando l'e-mail e la password & UC\ref{autenticazione-acquirente} \\

\rfun{O}{\ref{password-dimenticata}} & L'utente non autenticato che dispone di credenziali acquirente o venditore e si è dimenticato la propria password, potrà cambiarla. & UC\ref{password-dimenticata} \\

\rfun{O}{} & L'utente non autenticato potrà accedere alla schermata per l'autenticazione da qualsiasi schermata della piattaforma. & Interna \\

\rfun{O}{\ref{logout}} & L'utente autenticato potrà scollegarsi dalla piattaforma da qualsiasi schermata della piattaforma. & UC\ref{logout} \\


    \rfun{O}{\ref{ricerca-prodotti-acquirente}} & L'utente non autenticato e l'acquirente può cercare i prodotti attraverso delle parole dalla schermata principale, oppure dalla PLP. & UC\ref{ricerca-prodotti-acquirente} \\

\rfun{O}{\ref{filtro-prodotti-acquirente}} & L'utente non autenticato o l'acquirente può filtrare i prodotti all'interno della \glo{PLP}. & UC\ref{filtro-prodotti-acquirente} \\

\rfun{O}{\ref{filtro-prodotti-acquirente.categoria}} & L'utente non autenticato o l'acquirente può cercare i prodotti in base alla loro categoria, selezionando quelle di interesse tra tutte le categorie disponibili. & UC\ref{filtro-prodotti-acquirente.categoria} \\

\rfun{O}{\ref{filtro-prodotti-acquirente.prezzo}} & L'utente non autenticato o l'acquirente può cercare i prodotti in base al loro prezzo. & UC\ref{filtro-prodotti-acquirente.prezzo} \\

\rfun{O}{\ref{filtro-prodotti-acquirente.magazzino}} & L'utente non autenticato o l'acquirente può cercare i prodotti in base alla loro disponibilità in magazzino. & UC\ref{filtro-prodotti-acquirente.magazzino} \\


    \rfun{O}{} & L'utente non autenticato o l'acquirente visualizzerà i prodotti nella PLP ordinati alfabeticamente come ordinamento predefinito & Capitolato \\

\rfun{O}{} & L'utente non autenticato o l'acquirente, nella PLP, visualizzerà una lista di tutti i prodotti corrispondenti alla ricerca, dove per ogni prodotto sarà visualizzabile: il nome del prodotto, la prima immagine disponibile di esso, il suo prezzo per unità e se è disponibile in magazzino o no & Capitolato \\

\rfun{O}{} & I prodotti nella PLP non disponibili devono essere distinti da quelli che lo sono. & Capitolato \\

\rfun{O}{\ref{ordinamento-alfabetico}} & L'utente non autenticato o l'acquirente può ordinare i prodotti risultanti dalla ricerca effettuata in precedenza per ordine alfabetico. & UC\ref{ordinamento-alfabetico} \\

\rfun{O}{\ref{ordinamento-prezzo-crescente}} & L'utente non autenticato o l'acquirente può ordinare i prodotti risultanti dalla ricerca effettuata in precedenza per prezzo crescente. & UC\ref{ordinamento-prezzo-crescente} \\

\rfun{O}{\ref{ordinamento-prezzo-decrescente}} & L'utente non autenticato o l'acquirente può ordinare i prodotti risultanti dalla ricerca effettuata in precedenza per prezzo decrescente. & UC\ref{ordinamento-prezzo-decrescente} \\

\rfun{O}{\ref{aggiunta-carrello-plp}} & L'utente non autenticato o l'acquirente può aggiungere al carrello un'unità di un prodotto direttamente dalla PLP, avendo anche la possibilità di modificare in precedenza la quantità da aggiungere. & UC\ref{aggiunta-carrello-plp} \\

% \rfun{O}{\ref{}} &  & UC\ref{} \\


    \rfun{O}{} & L'utente non autenticato o l'acquirente può accedere alla \glo{PDP} di un prodotto in evidenza dalla schermata principale. & Interna \\

\rfun{O}{} & L'utente non autenticato o l'acquirente può accedere alla PDP di un prodotto acquistato dalla schermata di riepilogo ordine. & Interna \\

\rfun{O}{} & L'utente non autenticato o l'acquirente può accedere alla PDP di un prodotto che è stato aggiunto al carrello. & \glo{Capitolato} \\

\rfun{O}{} & L'utente non autenticato o l'acquirente può accedere alla PDP di un prodotto dalla PLP. & Capitolato \\

\rfun{O}{} & L'utente non autenticato o l'acquirente dalla PDP potrà visualizzare: nome del prodotto, la sua descrizione, le categorie alle quali appartiene, foto relative ad esso, il prezzo e relativi sconti. & Interna \\

\rfun{O}{\ref{aggiunta-carrello-pdp}} & L'utente non autenticato o l'acquirente può aggiungere al carrello un'unità di un prodotto direttamente dalla PDP, avendo anche la possibilità di modificare in precedenza la quantità da aggiungere. & UC\ref{aggiunta-carrello-pdp} \\

\rfun{O}{\ref{modifica-quantita-da-aggiungere-al-carrello}} & L'acquirente o l'utente non autenticato modifica la quantità del prodotto che vuole aggiungere nel carrello. & UC\ref{modifica-quantita-da-aggiungere-al-carrello} \\

% \rfun{O}{\ref{}} &  & UC\ref{} \\


    \rfun{O}{} & L'utente non autenticato o l'acquirente può accedere alla schermata del carrello da qualsiasi altra schermata della piattaforma. & Capitolato \\

\rfun{O}{\ref{visualizzazione-carrello}} & L'utente non autenticato o l'acquirente potrà visualizzare il proprio carrello, dove potrà visualizzare il prezzo totale e l'elenco dei prodotti. & UC\ref{visualizzazione-carrello} \\

\rfun{O}{\ref{visualizzazione-carrello}} & L'utente non autenticato o l'acquirente potrà visualizzare il proprio carrello dove per ogni prodotto inserito sarà indicato: il nome del prodotto, la quantità inserita, il prezzo del prodotto in base alla quantità inserita e agli sconti disponibili e la prima foto disponibile del prodotto. & UC\ref{visualizzazione-carrello} \\

\rfun{O}{\ref{eliminazione-prodotto-dal-carrello}} & L'utente non autenticato o l'acquirente può eliminare un prodotto che ha inserito nel carrello. & UC\ref{eliminazione-prodotto-dal-carrello} \\

\rfun{O}{\ref{modifica-quantita-nel-carrello}} & L'utente non autenticato o l'acquirente modifica la quantità di un prodotto precedentemente inserito nel carrello. & UC\ref{modifica-quantita-nel-carrello} \\

% \rfun{O}{\ref{}} &  & UC\ref{} \\


    \rfun{O}{\ref{checkout}} & L'acquirente può procedere al checkout per effettuare l'ordine con i prodotti inseriti nel carrello. & UC\ref{checkout} \\

\rfun{O}{\ref{checkout.indirizzo}} & L'acquirente seleziona l'indirizzo della consegna, ovvero dove verrà recapitato l'acquisto, tra gli indirizzi di consegna precedentemente inseriti. & UC\ref{checkout.indirizzo} \\

\rfun{O}{\ref{checkout.pagamento}} & L'acquirente procede al pagamento attraverso il servizio fornito dal gestore dei pagamenti. & UC\ref{checkout.pagamento} \\

\rfun{O}{\ref{inserimento-indirizzo-consegna}} & L'acquirente può aggiungere un nuovo indirizzo di consegna. & UC\ref{inserimento-indirizzo-consegna} \\

\rfun{O}{\ref{inserimento-indirizzo-consegna.modulo.nazione}} & L'acquirente può selezionare la nazione dell'indirizzo di consegna. & UC\ref{inserimento-indirizzo-consegna.modulo.nazione} \\

\rfun{O}{\ref{inserimento-indirizzo-consegna.modulo.comune}} & L'acquirente può inserire il comune dell'indirizzo di consegna. & UC\ref{inserimento-indirizzo-consegna.modulo.comune} \\

\rfun{O}{\ref{inserimento-indirizzo-consegna.modulo.via}} & L'acquirente può inserire la via dell'indirizzo di consegna, includendo anche il numero civico e l'eventuale interno. & UC\ref{inserimento-indirizzo-consegna.modulo.via} \\

\rfun{O}{\ref{inserimento-indirizzo-consegna.modulo.cap}} & L'acquirente può inserire il CAP dell'indirizzo di consegna. & UC\ref{inserimento-indirizzo-consegna.modulo.cap} \\

\rfun{O}{\ref{modifica-indirizzo-consegna}} & L'acquirente può modificare un indirizzo di consegna precedentemente inserito. & UC\ref{modifica-indirizzo-consegna} \\

\rfun{O}{\ref{modifica-indirizzo-consegna.nazione}} & L'acquirente può modificare la nazione dell'indirizzo di consegna. & UC\ref{modifica-indirizzo-consegna.nazione} \\

\rfun{O}{\ref{modifica-indirizzo-consegna.comune}} & L'acquirente può modificare il comune dell'indirizzo di consegna. & UC\ref{modifica-indirizzo-consegna.comune} \\

\rfun{O}{\ref{modifica-indirizzo-consegna.via}} & L'acquirente può modificare la via dell'indirizzo di consegna. & UC\ref{modifica-indirizzo-consegna.via} \\

\rfun{O}{\ref{modifica-indirizzo-consegna.cap}} & L'acquirente può modificare il CAP dell'indirizzo di consegna. & UC\ref{modifica-indirizzo-consegna.cap} \\

\rfun{O}{\ref{eliminazione-indirizzo-consegna}} & L'acquirente può eliminare un indirizzo di consegna precedentemente inserito. & UC\ref{eliminazione-indirizzo-consegna} \\

% \rfun{O}{\ref{}} &  & UC\ref{} \\


    \rfun{O}{} & L'acquirente può accedere alla schermata con tutti gli ordini effettuati da qualsiasi altra schermata della piattaforma. & Interna \\

\rfun{O}{\ref{visualizzazione-ordini-effettuati}} & L'acquirente può visualizzare l'elenco degli ordini effettuati sulla piattaforma. Per ognuno di questi potrà visualizzare: il codice numerico dell'ordine, lo stato dell'ordine, il prezzo totale che è stato pagato, l'indirizzo a cui è stato consegnato o verrà consegnato e una lista di tutti i prodotti acquistati.& UC\ref{visualizzazione-ordini-effettuati} \\

\rfun{O}{\ref{visualizzazione-ordini-effettuati}} & L'acquirente può visualizzare i prodotti che sono stati acquistati in un ordine effettuato. Per ognuno di questi verrà visualizzato: il nome del prodotto, la quantità acquistata e il prezzo totale a cui è stato acquistato. & UC\ref{visualizzazione-ordini-effettuati} \\

\rfun{O}{\ref{ricerca-codice-ordine-acquirente}} & L'acquirente può cercare un ordine sapendo il suo codice numerico. & UC\ref{ricerca-codice-ordine-acquirente} \\

\rfun{O}{\ref{filtro-temporale-ordini-acquirente}} & L'acquirente può filtrare temporalmente l'elenco degli ordini effettuati sulla piattaforma. & UC\ref{filtro-temporale-ordini-acquirente} \\

\rfun{O}{\ref{filtro-temporale-ordini-acquirente.data-iniziale}} & L'acquirente può impostare la data iniziale dell'intervallo per il quale filtrare l'elenco degli ordini effettuati sulla piattaforma. & UC\ref{filtro-temporale-ordini-acquirente.data-iniziale} \\

\rfun{O}{\ref{filtro-temporale-ordini-acquirente.data-finale}} & L'acquirente impostare la data finale dell'intervallo per il quale filtrare l'elenco degli ordini effettuati sulla piattaforma. & UC\ref{filtro-temporale-ordini-acquirente.data-finale} \\

% \rfun{O}{\ref{}} &  & UC\ref{} \\


    \rfun{O}{} & L'acquirente può accedere alla schermata con le proprie informazioni da qualsiasi altra schermata della piattaforma. & Interna \\

\rfun{O}{} & L'acquirente, nella propria schermata personale, potrà visualizzare: il nome ed il cognome che ha inserito e l'indirizzo e-mail collegato al proprio account. & Interna \\

\rfun{O}{\ref{modifica-informazioni-acquirente}} & L'acquirente può modificare le sue informazioni personali. & UC\ref{modifica-informazioni-acquirente} \\

\rfun{O}{\ref{modifica-informazioni-acquirente.nome}} & L'acquirente può modificare il nome con il quale si è registrato. & UC\ref{modifica-informazioni-acquirente.nome} \\

\rfun{O}{\ref{modifica-informazioni-acquirente.cognome}} & L'acquirente può modificare il cognome con il quale si è registrato. & UC\ref{modifica-informazioni-acquirente.cognome} \\

\rfun{O}{\ref{modifica-informazioni-acquirente.email}} & L'acquirente può modificare l'indirizzo e-mail con il quale si autenticherà. & UC\ref{modifica-informazioni-acquirente.email} \\

\rfun{O}{\ref{modifica-informazioni-acquirente.password}} & L'acquirente può modificare la password con la quale si autenticherà. & UC\ref{modifica-informazioni-acquirente.password} \\

\rfun{O}{\ref{eliminazione-account-acquirente}} & L'acquirente può eliminare il proprio account. & UC\ref{eliminazione-account-acquirente} \\

% \rfun{O}{\ref{}} &  & UC\ref{} \\

% \rfun{O}{\ref{}} &  & UC\ref{} \\

% \rfun{O}{\ref{}} &  & UC\ref{} \\

% \rfun{O}{\ref{}} &  & UC\ref{} \\

% \rfun{O}{\ref{}} &  & UC\ref{} \\

% \rfun{O}{\ref{}} &  & UC\ref{} \\

% \rfun{O}{\ref{}} &  & UC\ref{} \\

% \rfun{O}{\ref{}} &  & UC\ref{} \\

% \rfun{O}{\ref{}} &  & UC\ref{} \\

% \rfun{O}{\ref{}} &  & UC\ref{} \\

% \rfun{O}{\ref{}} &  & UC\ref{} \\

% \rfun{O}{\ref{}} &  & UC\ref{} \\

% \rfun{O}{\ref{}} &  & UC\ref{} \\

% \rfun{O}{\ref{}} &  & UC\ref{} \\

% \rfun{O}{\ref{}} &  & UC\ref{} \\

% \rfun{O}{\ref{}} &  & UC\ref{} \\


    \rfun{O}{} & L'utente non autenticato o l'acquirente, nella schermata principale, potrà visualizzare i prodotti in evidenza e la descrizione dell'azienda. & Capitolato \\

\rfun{O}{} & L'utente non autenticato o l'acquirente può accedere alla schermata principale da qualsiasi altra schermata della piattaforma. & Interna \\


    % Venditore

    \rfun{O}{\ref{modifica-informazioni-venditore}} & Il venditore può modificare le sue informazioni personali. & UC\ref{modifica-informazioni-venditore} \\
	
\rfun{O}{\ref{modifica-informazioni-venditore.nome}} & Il venditore può modificare il proprio nome. & UC\ref{modifica-informazioni-venditore.nome} \\
	
\rfun{O}{\ref{modifica-informazioni-venditore.cognome}} & Il venditore può modificare il proprio cognome. & UC\ref{modifica-informazioni-venditore.cognome} \\
	
\rfun{O}{\ref{modifica-informazioni-venditore.email}} & Il venditore può modificare la propria e-mail. & UC\ref{modifica-informazioni-venditore.email} \\
	
\rfun{O}{\ref{modifica-informazioni-venditore.descrizione-azienda}} & Il venditore può modificare la descrizione dell'azienda. & UC\ref{modifica-informazioni-venditore.descrizione-azienda} \\

\rfun{O}{\ref{modifica-password}} & L'utente autenticato può modificare la password con la quale si autenticherà. & UC\ref{modifica-password} \\

\rfun{O}{} & Il venditore può accedere alla schermata con le proprie informazioni da qualsiasi altra schermata della piattaforma. & Interna \\
	
\rfun{O}{} & Il venditore, nella propria schermata personale, potrà visualizzare: il nome, il cognome, l'indirizzo e-mail, il logo, la descrizione e il nome dell'azienda. & Interna \\ 
    
    \rfun{O}{} & Il venditore potrà accedere alla propria PLP da qualsiasi schermata. & Interna \\

\rfun{O}{} & Il venditore visualizzerà i prodotti nella PLP ordinati alfabeticamente come ordinamento predefinito. & Interna \\

\rfun{O}{} & Il venditore, nella PLP, visualizzerà una lista di tutti i prodotti corrispondenti alla ricerca, dove per ogni prodotto sarà visualizzabile: il nome del prodotto, la prima immagine disponibile di esso, il suo prezzo per unità e se è disponibile in magazzino o no. & Interna \\

\rfun{O}{} & Il venditore può accedere alla PDP di un prodotto acquistato dalla schermata di riepilogo di un ordine. & Interna \\

\rfun{O}{} & Il venditore può accedere alla PDP di un prodotto dalla propria PLP. & Capitolato \\

\rfun{O}{\ref{aggiunta-prodotto}} & Il venditore può aggiungere un nuovo prodotto alla piattaforma. & UC\ref{aggiunta-prodotto} \\
	
\rfun{O}{\ref{aggiunta-prodotto.nome}} & Il venditore può inserire il nome del prodotto da aggiungere. & UC\ref{aggiunta-prodotto.nome} \\
	
\rfun{O}{\ref{aggiunta-prodotto.descrizione}} & Il venditore può inserire la descrizione del prodotto da aggiungere. & UC\ref{aggiunta-prodotto.descrizione} \\
	
\rfun{O}{\ref{aggiunta-prodotto.categorie}} & Il venditore può inserire le categorie del prodotto da aggiungere. & UC\ref{aggiunta-prodotto.categorie} \\
	
\rfun{O}{\ref{aggiunta-prodotto.prezzo}} & Il venditore può inserire il prezzo del prodotto da aggiungere. & UC\ref{aggiunta-prodotto.prezzo} \\
	
\rfun{O}{\ref{aggiunta-prodotto.sconto}} & Il venditore può inserire lo sconto del prodotto da aggiungere. & UC\ref{aggiunta-prodotto.sconto} \\
	
\rfun{O}{\ref{aggiunta-prodotto.quantita}} & Il venditore può inserire la quantità disponibile del prodotto da aggiungere. & UC\ref{aggiunta-prodotto.quantita} \\
	
\rfun{O}{\ref{aggiunta-prodotto.foto}} & Il venditore può inserire le foto del prodotto da aggiungere. & UC\ref{aggiunta-prodotto.foto} \\
	
\rfun{O}{} & Il venditore potrà visualizzare il nome, la descrizione, le categorie, il prezzo, lo sconto, le foto e la quantità disponibile di tutti i suoi prodotti. & Interna \\
    
\rfun{O}{\ref{modifica-prodotto}} & Il venditore può modificare un prodotto già presente nella piattaforma. & UC\ref{modifica-prodotto} \\
    
\rfun{O}{\ref{modifica-prodotto.nome}} & Il venditore può modificare il nome di un prodotto già inserito nella piattaforma. & UC\ref{modifica-prodotto.nome} \\
    
\rfun{O}{\ref{modifica-prodotto.descrizione}} & Il venditore può modificare la descrizione di un prodotto già inserito nella piattaforma. & UC\ref{modifica-prodotto.descrizione} \\
    
\rfun{O}{\ref{modifica-prodotto.categorie}} & Il venditore può modificare le categorie di un prodotto già inserito nella piattaforma. & UC\ref{modifica-prodotto.categorie} \\
    
\rfun{O}{\ref{modifica-prodotto.prezzo}} & Il venditore può modificare il prezzo di un prodotto già inserito nella piattaforma. & UC\ref{modifica-prodotto.prezzo} \\
    
\rfun{O}{\ref{modifica-prodotto.sconto}} & Il venditore può modificare lo sconto percentuale di un prodotto già inserito nella piattaforma. & UC\ref{modifica-prodotto.sconto} \\
    
\rfun{O}{\ref{modifica-prodotto.aggiunta-foto}} & Il venditore può aggiungere nuove foto ad un prodotto già inserito nella piattaforma. & UC\ref{modifica-prodotto.aggiunta-foto} \\
    
\rfun{O}{\ref{modifica-prodotto.rimozione-foto}} & Il venditore può rimuovere le foto di un prodotto già inserito nella piattaforma. & UC\ref{modifica-prodotto.rimozione-foto} \\
    
\rfun{O}{\ref{aggiunta-prodotto-evidenza}} & Il venditore può aggiungere un prodotto alla sezione "in evidenza" nella vista principale. & UC\ref{aggiunta-prodotto-evidenza} \\

\rfun{O}{\ref{rimozione-prodotto-evidenza}} & Il venditore può rimuovere un prodotto dalla sezione "in evidenza" nella vista principale. & UC\ref{rimozione-prodotto-evidenza} \\
    
\rfun{O}{\ref{eliminazione-prodotto}} & Il venditore può eliminare un prodotto precedentemente inserito. & UC\ref{eliminazione-prodotto} \\

\rfun{O}{\ref{rifornimento-prodotto}} & Il venditore può rifornire un prodotto che sta per esaurire o è esaurito, dalla PDP del prodotto o dalla PLP. & UC\ref{rifornimento-prodotto} \\

% \rfun{O}{\ref{}} &  & UC\ref{} \\

    
    \rfun{O}{\ref{ricerca-prodotti-venditore}} & Il venditore può cercare i prodotti dalla propria PLP attraverso delle parole. & UC\ref{ricerca-prodotti-venditore} \\

\rfun{O}{\ref{filtro-prodotti-venditore}} & Il venditore può filtrare i prodotti all'interno della PLP. & UC\ref{filtro-prodotti-venditore} \\

\rfun{O}{\ref{filtro-prodotti-venditore.categoria}} & Il venditore può cercare i prodotti in base alla loro categoria, selezionando quelle di interesse tra tutte le categorie disponibili. & UC\ref{filtro-prodotti-venditore.categoria} \\

\rfun{O}{\ref{filtro-prodotti-venditore.prezzo}} & Il venditore può cercare i prodotti in base al loro prezzo. & UC\ref{filtro-prodotti-venditore.prezzo} \\

\rfun{O}{\ref{filtro-prodotti-venditore.magazzino}} & Il venditore può cercare i prodotti in base alla loro disponibilità in magazzino. & UC\ref{filtro-prodotti-venditore.magazzino} \\

\rfun{O}{\ref{filtro-prodotti-venditore.evidenza}} & Il venditore può cercare i prodotti in base alla loro caratteristica di trovarsi in evidenza o meno nella piattaforma. & UC\ref{filtro-prodotti-venditore.evidenza} \\
    
    \rfun{O}{\ref{aggiunta-categoria}} & Il venditore può aggiungere una nuova categoria. & UC\ref{aggiunta-categoria} \\
    
\rfun{O}{\ref{aggiunta-categoria.nome}} & Il venditore può aggiungere il nome della nuova categoria da aggiungere. & UC\ref{aggiunta-categoria.nome} \\
    
\rfun{O}{\ref{modifica-categoria}} & Il venditore può modificare una categoria già inserita. & UC\ref{modifica-categoria} \\
    
\rfun{O}{\ref{modifica-categoria.nome}} & Il venditore può modificare il nome attuale di una categoria già inserita. & UC\ref{modifica-categoria.nome} \\
    
\rfun{O}{\ref{eliminazione-categoria}} & Il venditore può eliminare una categoria già inserita. & UC\ref{eliminazione-categoria} \\
    
\rfun{O}{\ref{ricerca-categoria}} & Il venditore può cercare una categoria dalla schermata di amministrazione delle categorie. & UC\ref{ricerca-categoria} \\

    
    \rfun{O}{} & Il venditore, nella schermata riepilogo ordini, potrà visualizzare una lista dei prodotti acquistati in un determinato ordine. & Interna \\

\rfun{O}{\ref{visualizzazione-ordini-in-gestione}} & Il venditore può visualizzare l'elenco degli ordini chiusi o da gestire. Per ognuno di questi potrà visualizzare: il codice numerico dell’ordine, lo stato dell’ordine, il prezzo totale che è stato pagato, l’indirizzo a cui è stato consegnato o verrà consegnato e una lista di tutti i prodotti acquistati. & UC\ref{visualizzazione-ordini-in-gestione} \\

\rfun{O}{\ref{visualizzazione-ordini-in-gestione}} & Il venditore può visualizzare i prodotti che sono stati acquistati in un ordine ricevuto. Per ognuno di questi verrà visualizzato: il nome del prodotto, la quantità acquistata e il prezzo totale a cui è stato acquistato. & UC\ref{visualizzazione-ordini-in-gestione} \\

\rfun{O}{\ref{modifica-stato-ordine}} & Il venditore può modificare lo stato di un determinato ordine. & UC\ref{modifica-stato-ordine} \\

\rfun{O}{} & Il venditore visualizzerà gli ordini ricevuti in disposizione temporale come ordinamento predefinito & Interna \\

\rfun{O}{\ref{ricerca-codice-ordine-venditore}} & Il venditore può cercare un ordine sapendo il suo codice numerico. & UC\ref{ricerca-codice-ordine-venditore} \\
     
\rfun{O}{\ref{ricerca-cliente-ordine-venditore}} & Il venditore può cercare gli ordini in base al cliente che lo ha effettuato. & UC\ref{ricerca-cliente-ordine-venditore} \\
    
\rfun{O}{\ref{filtro-ordini-venditore}} & Il venditore può filtrare gli ordini nella schermata di riepilogo ordini. & UC\ref{filtro-ordini-venditore} \\
    
\rfun{O}{\ref{filtro-ordini-venditore.stato}} & Il venditore può filtrare gli ordini in base al loro stato. & UC\ref{filtro-ordini-venditore.stato} \\
    
\rfun{O}{\ref{filtro-ordini-venditore.temporale}} & Il venditore può filtrare per intervallo temporale gli ordini ricevuti. & UC\ref{filtro-ordini-venditore.temporale} \\
    
\rfun{O}{\ref{filtro-ordini-venditore.temporale.data-iniziale}} & Il venditore può impostare la data iniziale dell'intervallo per il quale filtrare l'elenco degli ordini ricevuti sulla piattaforma. & UC\ref{filtro-ordini-venditore.temporale.data-iniziale} \\
    
\rfun{O}{\ref{filtro-ordini-venditore.temporale.data-finale}} & Il venditore può impostare la data finale dell'intervallo per il quale filtrare l'elenco degli ordini ricevuti sulla piattaforma. & UC\ref{filtro-ordini-venditore.temporale.data-finale} \\

\rfun{O}{} & Il venditore può visualizzare lo stato degli ordini ricevuti. & Interna \\ 

    \rfun{O}{\ref{}} &  & UC\ref{} \\

    \rfun{O}{\ref{}} &  & UC\ref{} \\

    \rfun{O}{15} & Il venditore può anche inserire una nuova immagine profilo. & UC15 \\
    
    \rfun{O}{15} & Il venditore può anche inserire una nuova descrizione azienda. & UC15 \\
    
    \rfun{O}{16} & L'acquirente può eliminare il proprio account. & UC16 \\
    
    \rfun{O}{17} & L'acquirente può vedere l'elenco degli ordini effettuati. & UC17 \\
    
    \rfun{O}{18} & Il venditore può aggiungere un prodotto alla piattaforma, specificando nome. & UC18 \\
    
    \rfun{O}{18} & Il venditore può aggiungere un prodotto alla piattaforma, specificando descrizione. & UC18 \\
    
    \rfun{O}{18} & Il venditore può aggiungere un prodotto alla piattaforma, specificando categorie. & UC18 \\
    
    \rfun{O}{18} & Il venditore può aggiungere un prodotto alla piattaforma, specificando prezzo. & UC18 \\
    
    \rfun{O}{18} & Il venditore può aggiungere un prodotto alla piattaforma, specificando sconti applicati. & UC18 \\
    
    \rfun{O}{18.1} & Il venditore può aggiungere un prodotto alla piattaforma, specificando se deve andare tra quelli in evidenza. & UC18.1 \\
    
    \rfun{O}{18} & Il venditore può aggiungere un prodotto alla piattaforma, specificandone la quantità. & UC18 \\
    
    \rfun{O}{19} & Il venditore può modificare uno o più campi di un prodotto. & UC19 \\
    
    \rfun{O}{20} & Il venditore può eliminare un prodotto precedentemente inserito. & UC20 \\
    
    \rfun{O}{21} & Il venditore può rifornire un prodotto, precedentemente inserito, che sta per esaurire o è esaurito. & UC21 \\
    
    \rfun{O}{22} & Il venditore può visualizzare una lista con tutti gli ordini effettuati. & UC22 \\
    
    \rfun{O}{22} & Il venditore può passare alla pagina di riepilogo degli ordini da gestire da qualunque schermata in cui si trovi. & UC22 \\
    
    \rfun{O}{} & Il carrello dell'acquirente deve essere salvato in remoto e sincronizzato su più Browser/dispositivi a cui accedere. & \glo{Capitolato} \\
    
    \rfun{O}{} & L'utente non autenticato può aggiungere prodotti nel carrello come ospite e, appena si autentica, vengono aggiunti al suo carrello personale. & Interna \\
    
    \rfun{Z}{} & L'amministratore può modificare dinamicamente la copia del sito Web o cambiare il layout di PDP o PLP senza ridistribuire l'intero codice ad ogni modifica utilizzando \glo{\textit{Contentful}} come \glo{CMS}. & Capitolato \\
    
    \rfun{O}{15} & Il venditore può anche inserire una nuova immagine profilo. & UC15 \\
    
    \rfun{O}{15} & Il venditore può anche inserire una nuova descrizione azienda. & UC15 \\
    
    \rfun{O}{16} & L'acquirente può eliminare il proprio account. & UC16 \\
    
    \rfun{O}{17} & L'acquirente può vedere l'elenco degli ordini effettuati. & UC17 \\
    
    \rfun{O}{18} & Il venditore può aggiungere un prodotto alla piattaforma, specificando nome. & UC18 \\
    
    \rfun{O}{18} & Il venditore può aggiungere un prodotto alla piattaforma, specificando descrizione. & UC18 \\
    
    \rfun{O}{18} & Il venditore può aggiungere un prodotto alla piattaforma, specificando categorie. & UC18 \\
    
    \rfun{O}{18} & Il venditore può aggiungere un prodotto alla piattaforma, specificando prezzo. & UC18 \\
    
    \rfun{O}{18} & Il venditore può aggiungere un prodotto alla piattaforma, specificando sconti applicati. & UC18 \\
    
    \rfun{O}{18.1} & Il venditore può aggiungere un prodotto alla piattaforma, specificando se deve andare tra quelli in evidenza. & UC18.1 \\
    
    \rfun{O}{18} & Il venditore può aggiungere un prodotto alla piattaforma, specificandone la quantità. & UC18 \\
    
    \rfun{O}{19} & Il venditore può modificare uno o più campi di un prodotto. & UC19 \\
    
    \rfun{O}{20} & Il venditore può eliminare un prodotto precedentemente inserito. & UC20 \\
    
    \rfun{O}{21} & Il venditore può rifornire un prodotto, precedentemente inserito, che sta per esaurire o è esaurito. & UC21 \\
    
    \rfun{O}{22} & Il venditore può visualizzare una lista con tutti gli ordini effettuati. & UC22 \\
    
    \rfun{O}{22} & Il venditore può passare alla pagina di riepilogo degli ordini da gestire da qualunque schermata in cui si trovi. & UC22 \\
    
    \rfun{O}{} & Il carrello dell'acquirente deve essere salvato in remoto e sincronizzato su più Browser/dispositivi a cui accedere. & \glo{Capitolato} \\
    
    \rfun{O}{} & L'utente non autenticato può aggiungere prodotti nel carrello come ospite e, appena si autentica, vengono aggiunti al suo carrello personale. & Interna \\
    
    \rfun{Z}{} & L'amministratore può modificare dinamicamente la copia del sito Web o cambiare il layout di PDP o PLP senza ridistribuire l'intero codice ad ogni modifica utilizzando \glo{\textit{Contentful}} come \glo{CMS}. & Capitolato \\
    
    \rfun{O}{6} & La ricerca di prodotti deve essere possibile dalla schermata principale e dalla \glo{PLP}. & 

\end{longtable}
