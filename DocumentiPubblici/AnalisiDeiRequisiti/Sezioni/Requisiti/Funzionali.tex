\rowcolors{2}{white}{celeste} 
\renewcommand{\arraystretch}{1.5}

\addcontentsline{lot}{table}{Requisiti funzionali}

\begin{longtable}{c C{9.5cm} C{2.5cm}} 

	\rowcolor{darkblue}
	\textcolor{white}{\textbf{Codice Requisito}}&
	\textcolor{white}{\textbf{Descrizione}}&
    \textcolor{white}{\textbf{Fonte}} \\

    
    
    % Venditore
    \rfun{O}{\ref{modifica-informazioni-venditore}} & Il venditore può modificare le sue informazioni personali. & UC\ref{modifica-informazioni-venditore} \\
	
\rfun{O}{\ref{modifica-informazioni-venditore.nome}} & Il venditore può modificare il proprio nome. & UC\ref{modifica-informazioni-venditore.nome} \\
	
\rfun{O}{\ref{modifica-informazioni-venditore.cognome}} & Il venditore può modificare il proprio cognome. & UC\ref{modifica-informazioni-venditore.cognome} \\
	
\rfun{O}{\ref{modifica-informazioni-venditore.email}} & Il venditore può modificare la propria e-mail. & UC\ref{modifica-informazioni-venditore.email} \\
	
\rfun{O}{\ref{modifica-informazioni-venditore.descrizione-azienda}} & Il venditore può modificare la descrizione dell'azienda. & UC\ref{modifica-informazioni-venditore.descrizione-azienda} \\

\rfun{O}{\ref{modifica-password}} & L'utente autenticato può modificare la password con la quale si autenticherà. & UC\ref{modifica-password} \\

\rfun{O}{} & Il venditore può accedere alla schermata con le proprie informazioni da qualsiasi altra schermata della piattaforma. & Interna \\
	
\rfun{O}{} & Il venditore, nella propria schermata personale, potrà visualizzare: il nome, il cognome, l'indirizzo e-mail, il logo, la descrizione e il nome dell'azienda. & Interna \\ 
    
    \rfun{O}{} & Il venditore potrà accedere alla propria PLP da qualsiasi schermata. & Interna \\

\rfun{O}{} & Il venditore visualizzerà i prodotti nella PLP ordinati alfabeticamente come ordinamento predefinito. & Interna \\

\rfun{O}{} & Il venditore, nella PLP, visualizzerà una lista di tutti i prodotti corrispondenti alla ricerca, dove per ogni prodotto sarà visualizzabile: il nome del prodotto, la prima immagine disponibile di esso, il suo prezzo per unità e se è disponibile in magazzino o no. & Interna \\

\rfun{O}{} & Il venditore può accedere alla PDP di un prodotto acquistato dalla schermata di riepilogo di un ordine. & Interna \\

\rfun{O}{} & Il venditore può accedere alla PDP di un prodotto dalla propria PLP. & Capitolato \\

\rfun{O}{\ref{aggiunta-prodotto}} & Il venditore può aggiungere un nuovo prodotto alla piattaforma. & UC\ref{aggiunta-prodotto} \\
	
\rfun{O}{\ref{aggiunta-prodotto.nome}} & Il venditore può inserire il nome del prodotto da aggiungere. & UC\ref{aggiunta-prodotto.nome} \\
	
\rfun{O}{\ref{aggiunta-prodotto.descrizione}} & Il venditore può inserire la descrizione del prodotto da aggiungere. & UC\ref{aggiunta-prodotto.descrizione} \\
	
\rfun{O}{\ref{aggiunta-prodotto.categorie}} & Il venditore può inserire le categorie del prodotto da aggiungere. & UC\ref{aggiunta-prodotto.categorie} \\
	
\rfun{O}{\ref{aggiunta-prodotto.prezzo}} & Il venditore può inserire il prezzo del prodotto da aggiungere. & UC\ref{aggiunta-prodotto.prezzo} \\
	
\rfun{O}{\ref{aggiunta-prodotto.sconto}} & Il venditore può inserire lo sconto del prodotto da aggiungere. & UC\ref{aggiunta-prodotto.sconto} \\
	
\rfun{O}{\ref{aggiunta-prodotto.quantita}} & Il venditore può inserire la quantità disponibile del prodotto da aggiungere. & UC\ref{aggiunta-prodotto.quantita} \\
	
\rfun{O}{\ref{aggiunta-prodotto.foto}} & Il venditore può inserire le foto del prodotto da aggiungere. & UC\ref{aggiunta-prodotto.foto} \\
	
\rfun{O}{} & Il venditore potrà visualizzare il nome, la descrizione, le categorie, il prezzo, lo sconto, le foto e la quantità disponibile di tutti i suoi prodotti. & Interna \\
    
\rfun{O}{\ref{modifica-prodotto}} & Il venditore può modificare un prodotto già presente nella piattaforma. & UC\ref{modifica-prodotto} \\
    
\rfun{O}{\ref{modifica-prodotto.nome}} & Il venditore può modificare il nome di un prodotto già inserito nella piattaforma. & UC\ref{modifica-prodotto.nome} \\
    
\rfun{O}{\ref{modifica-prodotto.descrizione}} & Il venditore può modificare la descrizione di un prodotto già inserito nella piattaforma. & UC\ref{modifica-prodotto.descrizione} \\
    
\rfun{O}{\ref{modifica-prodotto.categorie}} & Il venditore può modificare le categorie di un prodotto già inserito nella piattaforma. & UC\ref{modifica-prodotto.categorie} \\
    
\rfun{O}{\ref{modifica-prodotto.prezzo}} & Il venditore può modificare il prezzo di un prodotto già inserito nella piattaforma. & UC\ref{modifica-prodotto.prezzo} \\
    
\rfun{O}{\ref{modifica-prodotto.sconto}} & Il venditore può modificare lo sconto percentuale di un prodotto già inserito nella piattaforma. & UC\ref{modifica-prodotto.sconto} \\
    
\rfun{O}{\ref{modifica-prodotto.aggiunta-foto}} & Il venditore può aggiungere nuove foto ad un prodotto già inserito nella piattaforma. & UC\ref{modifica-prodotto.aggiunta-foto} \\
    
\rfun{O}{\ref{modifica-prodotto.rimozione-foto}} & Il venditore può rimuovere le foto di un prodotto già inserito nella piattaforma. & UC\ref{modifica-prodotto.rimozione-foto} \\
    
\rfun{O}{\ref{aggiunta-prodotto-evidenza}} & Il venditore può aggiungere un prodotto alla sezione "in evidenza" nella vista principale. & UC\ref{aggiunta-prodotto-evidenza} \\

\rfun{O}{\ref{rimozione-prodotto-evidenza}} & Il venditore può rimuovere un prodotto dalla sezione "in evidenza" nella vista principale. & UC\ref{rimozione-prodotto-evidenza} \\
    
\rfun{O}{\ref{eliminazione-prodotto}} & Il venditore può eliminare un prodotto precedentemente inserito. & UC\ref{eliminazione-prodotto} \\

\rfun{O}{\ref{rifornimento-prodotto}} & Il venditore può rifornire un prodotto che sta per esaurire o è esaurito, dalla PDP del prodotto o dalla PLP. & UC\ref{rifornimento-prodotto} \\

% \rfun{O}{\ref{}} &  & UC\ref{} \\

    
    \rfun{O}{\ref{ricerca-prodotti-venditore}} & Il venditore può cercare i prodotti dalla propria PLP attraverso delle parole. & UC\ref{ricerca-prodotti-venditore} \\

\rfun{O}{\ref{filtro-prodotti-venditore}} & Il venditore può filtrare i prodotti all'interno della PLP. & UC\ref{filtro-prodotti-venditore} \\

\rfun{O}{\ref{filtro-prodotti-venditore.categoria}} & Il venditore può cercare i prodotti in base alla loro categoria, selezionando quelle di interesse tra tutte le categorie disponibili. & UC\ref{filtro-prodotti-venditore.categoria} \\

\rfun{O}{\ref{filtro-prodotti-venditore.prezzo}} & Il venditore può cercare i prodotti in base al loro prezzo. & UC\ref{filtro-prodotti-venditore.prezzo} \\

\rfun{O}{\ref{filtro-prodotti-venditore.magazzino}} & Il venditore può cercare i prodotti in base alla loro disponibilità in magazzino. & UC\ref{filtro-prodotti-venditore.magazzino} \\

\rfun{O}{\ref{filtro-prodotti-venditore.evidenza}} & Il venditore può cercare i prodotti in base alla loro caratteristica di trovarsi in evidenza o meno nella piattaforma. & UC\ref{filtro-prodotti-venditore.evidenza} \\
    
    \rfun{O}{\ref{aggiunta-categoria}} & Il venditore può aggiungere una nuova categoria. & UC\ref{aggiunta-categoria} \\
    
\rfun{O}{\ref{aggiunta-categoria.nome}} & Il venditore può aggiungere il nome della nuova categoria da aggiungere. & UC\ref{aggiunta-categoria.nome} \\
    
\rfun{O}{\ref{modifica-categoria}} & Il venditore può modificare una categoria già inserita. & UC\ref{modifica-categoria} \\
    
\rfun{O}{\ref{modifica-categoria.nome}} & Il venditore può modificare il nome attuale di una categoria già inserita. & UC\ref{modifica-categoria.nome} \\
    
\rfun{O}{\ref{eliminazione-categoria}} & Il venditore può eliminare una categoria già inserita. & UC\ref{eliminazione-categoria} \\
    
\rfun{O}{\ref{ricerca-categoria}} & Il venditore può cercare una categoria dalla schermata di amministrazione delle categorie. & UC\ref{ricerca-categoria} \\

    
    \rfun{O}{} & Il venditore, nella schermata riepilogo ordini, potrà visualizzare una lista dei prodotti acquistati in un determinato ordine. & Interna \\

\rfun{O}{\ref{visualizzazione-ordini-in-gestione}} & Il venditore può visualizzare l'elenco degli ordini chiusi o da gestire. Per ognuno di questi potrà visualizzare: il codice numerico dell’ordine, lo stato dell’ordine, il prezzo totale che è stato pagato, l’indirizzo a cui è stato consegnato o verrà consegnato e una lista di tutti i prodotti acquistati. & UC\ref{visualizzazione-ordini-in-gestione} \\

\rfun{O}{\ref{visualizzazione-ordini-in-gestione}} & Il venditore può visualizzare i prodotti che sono stati acquistati in un ordine ricevuto. Per ognuno di questi verrà visualizzato: il nome del prodotto, la quantità acquistata e il prezzo totale a cui è stato acquistato. & UC\ref{visualizzazione-ordini-in-gestione} \\

\rfun{O}{\ref{modifica-stato-ordine}} & Il venditore può modificare lo stato di un determinato ordine. & UC\ref{modifica-stato-ordine} \\

\rfun{O}{} & Il venditore visualizzerà gli ordini ricevuti in disposizione temporale come ordinamento predefinito & Interna \\

\rfun{O}{\ref{ricerca-codice-ordine-venditore}} & Il venditore può cercare un ordine sapendo il suo codice numerico. & UC\ref{ricerca-codice-ordine-venditore} \\
     
\rfun{O}{\ref{ricerca-cliente-ordine-venditore}} & Il venditore può cercare gli ordini in base al cliente che lo ha effettuato. & UC\ref{ricerca-cliente-ordine-venditore} \\
    
\rfun{O}{\ref{filtro-ordini-venditore}} & Il venditore può filtrare gli ordini nella schermata di riepilogo ordini. & UC\ref{filtro-ordini-venditore} \\
    
\rfun{O}{\ref{filtro-ordini-venditore.stato}} & Il venditore può filtrare gli ordini in base al loro stato. & UC\ref{filtro-ordini-venditore.stato} \\
    
\rfun{O}{\ref{filtro-ordini-venditore.temporale}} & Il venditore può filtrare per intervallo temporale gli ordini ricevuti. & UC\ref{filtro-ordini-venditore.temporale} \\
    
\rfun{O}{\ref{filtro-ordini-venditore.temporale.data-iniziale}} & Il venditore può impostare la data iniziale dell'intervallo per il quale filtrare l'elenco degli ordini ricevuti sulla piattaforma. & UC\ref{filtro-ordini-venditore.temporale.data-iniziale} \\
    
\rfun{O}{\ref{filtro-ordini-venditore.temporale.data-finale}} & Il venditore può impostare la data finale dell'intervallo per il quale filtrare l'elenco degli ordini ricevuti sulla piattaforma. & UC\ref{filtro-ordini-venditore.temporale.data-finale} \\

\rfun{O}{} & Il venditore può visualizzare lo stato degli ordini ricevuti. & Interna \\ 

     \rfun{O}{\ref{}} &  & UC\ref{} \\

    \rfun{O}{\ref{}} &  & UC\ref{} \\
    
    
    % \rfun{O}{15} & Il venditore può anche inserire una nuova immagine profilo. & UC15 \\
    
    % \rfun{O}{15} & Il venditore può anche inserire una nuova descrizione azienda. & UC15 \\
    
    % \rfun{O}{16} & L'acquirente può eliminare il proprio account. & UC16 \\
    
    % \rfun{O}{17} & L'acquirente può vedere l'elenco degli ordini effettuati. & UC17 \\
    
    % \rfun{O}{18} & Il venditore può aggiungere un prodotto alla piattaforma, specificando nome. & UC18 \\
    
    % \rfun{O}{18} & Il venditore può aggiungere un prodotto alla piattaforma, specificando descrizione. & UC18 \\
    
    % \rfun{O}{18} & Il venditore può aggiungere un prodotto alla piattaforma, specificando categorie. & UC18 \\
    
    % \rfun{O}{18} & Il venditore può aggiungere un prodotto alla piattaforma, specificando prezzo. & UC18 \\
    
    % \rfun{O}{18} & Il venditore può aggiungere un prodotto alla piattaforma, specificando sconti applicati. & UC18 \\
    
    % \rfun{O}{18.1} & Il venditore può aggiungere un prodotto alla piattaforma, specificando se deve andare tra quelli in evidenza. & UC18.1 \\
    
    % \rfun{O}{18} & Il venditore può aggiungere un prodotto alla piattaforma, specificandone la quantità. & UC18 \\
    
    % \rfun{O}{19} & Il venditore può modificare uno o più campi di un prodotto. & UC19 \\
    
    % \rfun{O}{20} & Il venditore può eliminare un prodotto precedentemente inserito. & UC20 \\
    
    % \rfun{O}{21} & Il venditore può rifornire un prodotto, precedentemente inserito, che sta per esaurire o è esaurito. & UC21 \\
    
    % \rfun{O}{22} & Il venditore può visualizzare una lista con tutti gli ordini effettuati. & UC22 \\
    
    % \rfun{O}{22} & Il venditore può passare alla pagina di riepilogo degli ordini da gestire da qualunque schermata in cui si trovi. & UC22 \\
    
    % \rfun{O}{} & Il carrello dell'acquirente deve essere salvato in remoto e sincronizzato su più Browser/dispositivi a cui accedere. & \glo{Capitolato} \\
    
    % \rfun{O}{} & L'utente non autenticato può aggiungere prodotti nel carrello come ospite e, appena si autentica, vengono aggiunti al suo carrello personale. & Interna \\
    
    % \rfun{Z}{} & L'amministratore può modificare dinamicamente la copia del sito Web o cambiare il layout di PDP o PLP senza ridistribuire l'intero codice ad ogni modifica utilizzando \glo{\textit{Contentful}} come \glo{CMS}. & Capitolato \\
    
    % \rfun{O}{6} & La ricerca di prodotti deve essere possibile dalla schermata principale e dalla \glo{PLP}. & 

\end{longtable}

   
\begin{comment}
\rfun{O}{6} & La ricerca di prodotti deve essere possibile dalla schermata principale e dalla \glo{PLP}. & 
\end{comment}