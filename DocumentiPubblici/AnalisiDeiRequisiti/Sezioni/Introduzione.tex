\section{Introduzione}\label{Intro}
\subsection{Scopo del documento}
Questo documento ha lo scopo di fornire una descrizione completa e dettagliata di tutti i requisiti e dei casi d'uso individuati durante l'analisi del \glo{capitolato} \NomeProgetto{}.

\subsection{Scopo del Prodotto}
Il progetto {\NomeProgetto} ha come scopo quello di rendere disponibile un servizio \glo{e-commerce} sfruttando tutti i vantaggi di un'architettura \glo{Serverless}:
\begin{itemize}
  \item Gli sviluppatori potranno concentrare la propria attenzione sullo sviluppo del prodotto finale invece di focalizzarsi sulla gestione e sul funzionamento di server e di runtime, che siano nel \glo{cloud} o in locale;
  \item Comodità nel costruire un insieme di chiamate asincrone che rispondono a diversi clienti contemporaneamente;
  \item Minori costi di sviluppo e di produzione;
  \item Semplicità nel suddividere il progetto in un insieme di \glo{microservizi}.
\end{itemize}

\subsection{Glossario}
Al fine di rendere il documento più chiaro e leggibile si fornisce un \Glossario. I termini che possono assumere un significato ambiguo sono indicati da una 'G' ad apice e fanno riferimento al documento \Glossariov{2.0}.

\subsection{Riferimenti}
\subsubsection{Riferimenti normativi}
\begin{itemize}
	\item \NdPv{2.0}
\end{itemize}
\subsubsection{Riferimenti informativi}
\begin{itemize}
	\item \textbf{Capitolato d'appalto C2 - \NomeProgetto{}:} \\ \url{https://www.math.unipd.it/~tullio/IS-1/2020/Progetto/C2.pdf}
	\item \textbf{Slide Ingegneria del Software - Analisi dei requisiti:}\\ \url{https://www.math.unipd.it/~tullio/IS-1/2020/Dispense/L07.pdf}
	\item \textbf{Slide Ingegneria del Software - Diagrammi dei casi d'uso:}\\ \url{https://www.math.unipd.it/~rcardin/swea/2021/Diagrammi\%20Use\%20Case_4x4.pdf}
\end{itemize}