\appendix
\section{Attualizzazione dei rischi}
\label{attualizzazione_dei_rischi}
In questa sezione viene esposto come il gruppo \Gruppo\ ha affrontato le varie problematiche riscontrate durante lo svolgimento del progetto e quali rischi con relative soluzioni si aspetta di incontrare durante i vari periodi.
\subsection{Rischi legati alle tecnologie}
\begin{table} [H]
	\centering
    \begin{tabular}{|c|p{11.5cm}|}
    \rowcolor{darkblue} \hline
    \multicolumn{2}{|c|}{\textcolor{white}{\textbf{RT1 - Inesperienza tecnologica}}}\\ \hline
    Periodo & Analisi.\\ \hline
    Mitigazione & Alcuni membri del gruppo non sono ancora esperti nell'utilizzo di \glo{GitHub} e sono sorti dei leggeri problemi riguardanti la gestione del \glo{repository}. I membri più esperti nell'utilizzo dello strumento hanno dedicato un po' del loro tempo per aiutare i compagni ad imparare ad utilizzare al meglio il repository.\\ \hline
	Periodo & Progettazione architetturale.\\ \hline
	Mitigazione & Data la poca esperienza della maggior parte dei componenti del gruppo con le tecnologie da impiegare nello sviluppo del progetto, in particolar modo con \glo{Typescript} e \glo{AWS Lambda}, i membri più esperti nell'utilizzo di queste tecnologie hanno condiviso le loro esperienze con gli altri membri del gruppo. Abbiamo deciso inoltre di dedicare una settimana intera per l'approfondimento individuale.\\ \hline
	Periodo & Progettazione di dettaglio e codifica.\\ \hline
	Mitigazione & I componenti che si sono occupati maggiormente dello sviluppo del \glo{PoC} hanno dato supporto al resto dei membri che hanno presentato alcune difficoltà legate all'utilizzo delle tecnologie. Il gruppo ha riscontrato delle difficoltà nell'individuare l'architettura e i relativi \glo{design pattern}. Per risolvere i dubbi si sono effettuate diverse ricerche e i componenti si sono confrontati più volte tra loro. \\ \hline	
	\end{tabular}
	\caption{\label{tab:ART1}Attualizzazione dei rischi per inesperienza tecnologica.}
\end{table}
\subsection{Rischi legati ai membri del gruppo}
\begin{table}[H]
	\centering
    \begin{tabular}{|c|p{11.5cm}|}
    \rowcolor{darkblue} \hline
    \multicolumn{2}{|c|}{\textcolor{white}{\textbf{RG1 - Inesperienza gestionale}}}\\ \hline
    Periodo & Analisi.\\ \hline
    Mitigazione & Data l'inesperienza del gruppo nel coordinare il lavoro e alcuni lievi problemi di comunicazione si è verificato un ritardo riguardante la stesura o la verifica. Il gruppo ha cercato di recuperare eventuali mancanze il prima possibile. Essendo in un periodo iniziale del progetto i brevi ritardi verificatosi non hanno causato problemi allo svolgimento del progetto.\\ \hline
    Periodo & Progettazione architetturale.\\ \hline
    Mitigazione & Il gruppo ha dovuto procedere ad una serie di ripianificazioni delle attività per recuperare dei ritardi dati da una errata pianificazione delle risorse.\\ \hline
    Periodo & Progettazione di dettaglio e codifica.\\ \hline
    Mitigazione & Per poter evitare di avere dei ritardi come nel precedente periodo, il gruppo ha innanzitutto ricontrollato e corretto la pianificazione delle attività. Durante il periodo si sono inserite più milestone per poter valutare meglio l'andamento del lavoro.\\ \hline
    Periodo & Validazione e collaudo.\\ \hline
    Mitigazione & Il gruppo prima di iniziare con le attività controllerà attentamente quanto pianificato nella sezione \S\ref{validazione_e_collaudo}.\\ \hline
    \end{tabular}
    \caption{\label{tab:ARG3}Analisi dei rischi per inesperienza gestionale.}
\end{table}
\begin{table}[H]
	\centering
	\begin{tabular}{|c|p{11.5cm}|}
	\rowcolor{darkblue} \hline
	\multicolumn{2}{|c|}{\textcolor{white}{\textbf{RG2 - Disponibilità dei membri}}}\\ \hline
	Periodo & Progettazione architetturale.\\ \hline
	Mitigazione & La sessione d'esame si è svolta in contemporanea con il periodo di progettazione architetturale e questo ha impegnato notevolmente alcuni membri del gruppo. Per assicurare il procedere del progetto, le varie \glo{attività} sono state momentaneamente suddivise tra i membri con maggiori disponibilità.\\ \hline
	Periodo & Progettazione di dettaglio e codifica.\\ \hline
	Mitigazione & In questo periodo i membri del gruppo si sono preparati per poter sostenere il primo appello scritto dell'esame Ingegneria del software, per assicurare il corretto andamento del progetto si sono effettuati incontri più ravvicinati tra i membri che si occupavano dello stesso ambito.\\ \hline
	Periodo & Validazione e collaudo.\\ \hline
	Mitigazione & In questo periodo è prevista l'eventualità che alcuni dei membri inizino lo stage universitario, è importante seguire con attenzione le attività pianificate per questo ci sarà uno stretto controllo sull'andamento del lavoro assegnato a ciascun membro.\\ \hline
	\end{tabular}
	\caption{\label{tab:ARG2}Analisi dei rischi per disponibilità dei membri.}
\end{table}
\subsection{Rischi legati all'organizzazione del lavoro}
\begin{table}[H]
	\centering
	\begin{tabular}{|c|p{11.5cm}|}
		\rowcolor{darkblue} \hline
		\multicolumn{2}{|c|}{\textcolor{white}{\textbf{RO1 - Costi delle attività}}}\\ \hline
		Periodo & Progettazione architetturale.\\ \hline
		Mitigazione & Si ha avuto una visione ottimistica nell'apprendimento da parte di tutti i membri delle numerose tecnologie impiegate nello sviluppo del progetto e nella riorganizzazione dell'\AdR{} in seguito alla discussione sostenuta con \CR{}. Per cercare di ridurre al minimo il tempo perso nello svolgimento del progetto, le varie \glo{attività} sono state momentaneamente suddivise tra i membri con maggiori disponibilità e si è deciso di dedicare maggior attenzione alla comprensione delle tecnologie e alla ristrutturazione dei casi d'uso presenti all'interno dell'\AdR{}.\\ \hline
		Periodo & Progettazione di dettaglio e codifica.\\ \hline
		Mitigazione & Per evitare ritardi, il gruppo si è diviso per poter inizialmente lavorare parallelamente assegnando ad alcuni membri le attività di sviluppo del \glo{front end} e ai restanti le attività di sviluppo del \glo{back end}.\\ \hline
	\end{tabular}
	\caption{\label{tab:ARO1}Analisi dei rischi per costi delle attività.}
\end{table}

\subsection{Rischi legati ai requisiti}
\begin{table}[H]
	\centering
	\begin{tabular}{|c|p{11.5cm}|}
		\rowcolor{darkblue} \hline
		\multicolumn{2}{|c|}{\textcolor{white}{\textbf{RR1 - Analisi dei requisiti imperfetta}}}\\ \hline
		Periodo & Progettazione architetturale.\\ \hline
		Mitigazione & A seguito delle osservazioni fatte in seguito alla \textbf{Revisione dei requisiti}, il gruppo {\Gruppo} ha rivisto completamente l'\AdR\ risultata incompleta e con gravi errori di base. Si è provveduto a completare le mancanze segnalate con la massima priorità.\\ \hline
		Periodo & Progettazione di dettaglio e codifica.\\ \hline
		Mitigazione & A seguito delle osservazioni fatte con la \textbf{Revisione di progettazione}, è stato raggiunto un buon livello di dettaglio nel documento ma sono stati segnalati nuovi errori che il gruppo ha provveduto a correggere.\\ \hline
	\end{tabular}
	\caption{\label{tab:ARR1}Analisi dei rischi per analisi dei requisiti imperfetta.}
\end{table}