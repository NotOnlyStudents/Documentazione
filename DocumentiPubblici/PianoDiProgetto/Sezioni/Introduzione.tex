\section{Introduzione}
\label{introduzione}
\subsection{Scopo del Documento}
Lo scopo del documento è quello di specificare come {\Gruppo} intende organizzare le \glo{attività}, analizzandone i rischi, suddividendole e assegnandole ai vari membri del gruppo. Così facendo è possiblie raggiungere gli obiettivi prefissati nel modo più \glo{efficace} ed \glo{efficiente} possibile. All'interno del documento sono presenti le strategie per gestire i possibili rischi, le suddivisioni dei compiti e il modello di sviluppo secondo cui il team si propone di lavorare. Il documento contiene il \glo{preventivo} ed un \glo{consuntivo} delle risorse da utilizzare.

\subsection{Scopo del Prodotto} %Da rivedere insieme con gli altri membri del gruppo
Il progetto {\NomeProgetto} ha come scopo quello di rendere disponibile un servizio e-commerce sfruttando però tutti i vantaggi di un'architettura Serverless:
\begin{itemize}
    \item Gli sviluppatori potranno concentrare la propria attenzione sullo sviluppo del prodotto finale invece di focalizzarsi sulla gestione e sul funzionamento di server e di runtime, che siano nel cloud o in locale.
    \item Comodità nel costruire un insieme di chiamate asincrone che rispondono a diversi clienti contemporaneamente.
    \item Minori costi di sviluppo e di produzione.
    \item Semplicità nel suddividere il progetto in un insieme di microservizi.
\end{itemize}

\subsection{Glossario}
Al fine di rendere il documento più chiaro e leggibile si fornisce un \textit{Glossario}. I termini che possono assumere un significato ambiguo sono indicati da una 'G' ad apice.

\subsection{Riferimenti}
\subsubsection{Normativi}
\begin{itemize}
    \item \textit{\NdP};
    \item \textbf{Capitolato d'appalto C2:}\\
    \url{https://www.math.unipd.it/~tullio/IS-1/2020/Progetto/C2.pdf}
    \item \textbf{Regolamento Organigramma e Specifica Tecnico-Economica:}\\
    \url{https://www.math.unipd.it/~tullio/IS-1/2020/Progetto/RO.html}
    \item \textbf{Regolamento del Progetto Didattico:}\\
    \url{https://www.math.unipd.it/~tullio/IS-1/2020/Dispense/P1.pdf}
\end{itemize}

\subsubsection{Informativi}
\begin{itemize}
    \item \textbf{Il Ciclo di vita del software - Materiale didattico del corso di Ingegneria del Software:}\\
    \url{https://www.math.unipd.it/~tullio/IS-1/2020/Dispense/L05.pdf}
    \item \textbf{Gestione di Progetto - Materiale didattico del corso di Ingegneria del Software:}\\
    \url{https://www.math.unipd.it/~tullio/IS-1/2020/Dispense/L06.pdf}
    \item \textbf{Immagine presente in \S\ref{modello_di_sviluppo}:} \\
    \url{https://binaryterms.com/incremental-development-model.html}
\end{itemize}

\subsection{Ruoli e Costi}
Durante lo svolgimento del progetto i membri del gruppo {\Gruppo} saranno chiamati a rivestire diversi ruoli. Ciascun componente dovrà ricoprire almeno una volta ogni ruolo, senza che vi siano conflitti d'interesse, così da poter avere un'esperienza più completa possibile sul lavoro di gruppo. Ogni ruolo presenterà un diverso costo come indicato nella \textit{Offerta tecnico-economica} presente al seguente link: \url{https://www.math.unipd.it/~tullio/IS-1/2020/Progetto/RO.html}

\subsection{Scadenze}
\label{sub:scadenze_fissate}
Il gruppo {\Gruppo} si impegna a rispettare le seguenti scadenze per lo sviluppo del progetto \NomeProgetto:
\begin{itemize}
    \item \textbf{Revisione dei requisiti:} 2021-01-18
    \item \textbf{Revisione di progettazione:} 2021-03-08
    \item \textbf{Revisione di qualifica:} 2021-04-09
    \item \textbf{Revisione di accettazione:} 2021-05-10
\end{itemize}
