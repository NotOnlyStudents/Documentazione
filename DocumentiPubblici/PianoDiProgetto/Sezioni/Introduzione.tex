\section{Introduzione}
\subsection{Scopo del Documento}
Lo scopo del documento è fornire un prospetto della pianificazione e delle moda-lità attraverso le quali il gruppo \Gruppo vilupperà il progetto \NomeProgetto.
\\ Il documento tratta i seguenti punti:
\begin{itemize}
    \item Analisi dei rischi
    \item Breve Descrizione del modello di sviluppo adottato
    \item Pianificazione delle attività e divisione dei ruoli
    \item Stima dei costi e delle risorse necessarie
\end{itemize}

\subsection{Scopo del Prodotto}
Il progetto \NomeProgetto\ ha come scopo quello di rendere disponibile un servizio
E-commerce sfruttando però tutti i vantaggi di un'architettura serverless:
\begin{itemize}
    \item Gli sviluppatori potranno concentrare la propria attenzione sul prodotto invece che su gestione e funzionamento di server e di runtime, che siano nel cloud o in locale.
    \item Comodità nel costruire un insieme di chiamate asincrone che rispondono a diversi clienti contemporaneamente.
    \item Minori costi di sviluppo e di produzione.
    \item Semplicità nel suddividere il progetto in un insieme di microservizi.
\end{itemize}

\subsection{Glossario}
Al fine di evitare ambiguità fra i termini, e per avere le terminologie chiare fra tutti gli stakeholder, il gruppo \Gruppo{} ha redatto un documento denominato \Glossariov{1.0.0}.
In tale documento sono presenti tutti i termini tecnici, ambigui, specifici del progetto e scelti dai membri del gruppo con le loro relative definizioni.
Un termine presente nel \Glossariov{1.0.0} e utilizzato in questo documento viene indicato con un apice \ap{G} alla fine della parola.

\subsection{Riferimenti}
\subsubsection{Normativi}
\begin{itemize}
    \item \NdPv{1.0.0};
    \item \textbf{Regolamento Organigramma e Specifica Tecnico-Economica:}\\
    \url{https://www.math.unipd.it/~tullio/IS-1/2020/Progetto/RO.html}
    \item \textbf{Capitolato d'appalto C2:}\\
    \url{https://www.math.unipd.it/~tullio/IS-1/2020/Progetto/C4.pdf}
\end{itemize}

\subsubsection{Informativi}
\begin{itemize}
    \item \textbf{Il Ciclo di vita del software - Materiale didattico del corso di Ingegneria del Software:}\\
    \url{https://www.math.unipd.it/~tullio/IS-1/2020/Dispense/L05.pdf}
    \begin{itemize}
        \item Modello Incrementale, slide 18-19.
    \end{itemize}

    \item \textbf{Gestione di Progetto - Materiale didattico del corso di Ingegneria del Software:}\\
    \url{https://www.math.unipd.it/~tullio/IS-1/2020/Dispense/L06.pdf}
    \begin{itemize}
        \item Pianificazione di Progetto, slide 10-15;
        \item Allocazione risorse, slide 17-18;
        \item Stima dei costi di progetto, slide 19;
        \item Piano di progetto, slide 25-26;
        \item Rischi di progetto, slide 27;
        \item Fonti di rischio, slide 28;
        \item Gestione dei rischi, slide 29;
    \end{itemize}
\end{itemize}

\subsection{Ruoli e Costi}
Durante lo svolgimento del progetto i membri del gruppo \Gruppo saranno chiamati a rivestire diversi ruoli. Ciascun componente del gruppo dovrà ricoprire almeno una volta ogni ruolo, senza che vi siano conflitti d'interesse, così da poter avere un'esperienza più completa possibile sul lavoro di gruppo. Ogni ruolo presenterà inoltre un diverso costo come indicato nella \textit{Offerta tecnico-economica} presente al seguente link: \url{https://www.math.unipd.it/~tullio/IS-1/2020/Progetto/RO.html}

\subsection{Scadenze}
Il gruppo \Gruppo impegna a rispettare le seguenti scadenze per lo svolgimento del
progetto \NomeProgetto:

\begin{itemize}
    \item \textbf{Revisione dei requisiti:} 2021-01-18
    \item \textbf{Revisione di progettazione:} 2021-03-08
    \item \textbf{Revisione di qualifica:} 2021-04-09
    \item \textbf{Revisione di accettazione:} 2021-05-10
\end{itemize}

