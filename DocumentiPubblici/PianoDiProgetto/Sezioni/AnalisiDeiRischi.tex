\section{Analisi dei rischi}
\label{analisi_dei_rischi}
Durante lo sviluppo di un progetto così complesso e di grandi dimensioni, la possibilità di incontrare delle problematiche è alta. Per cercare di evitare il più possibile queste criticità si è svolta un'attenta analisi sui possibili rischi del progetto. La procedura utilizzata per l'analisi dei rischi si suddivide nelle seguenti parti:
\begin{itemize}
    \item \textbf{Individuazione:} \glo{Attività} iniziale di identificazione di possibili problematiche che potrebbero interferire e minare il corretto avanzamento del progetto;
    \item \textbf{Analisi:} Attività di analisi dei singoli rischi con l'obiettivo di evidenziare la probabilità di occorrenza, l'indice di gravità e le conseguenze che  questi potrebbero causare;
    \item \textbf{Pianificazione di Controllo:} Attività di pianificazione delle misure da adottare per cercare di impedire il verificarsi di tali rischi o per gestirli al meglio qualora si verifichino;
    \item \textbf{Monitoraggio:} Attività di controllo costante con lo scopo di evitare il verificarsi di problematiche e permettere di attuare le strategie di contenimento individuate dal \glo{team} per limitarne i danni.
\end{itemize}

Abbiamo suddiviso i principali fattori di rischio nelle seguenti categorie:
\begin{itemize}
    \item \textbf{Rischi legati alle tecnologie;}
    \item \textbf{Rischi legati ai membri del gruppo;}
    \item \textbf{Rischi legati agli strumenti;}
    \item \textbf{Rischi legati all'orgranizzazione del lavoro;}
    \item \textbf{Rischi legati ai requisiti.}
\end{itemize}

\rowcolors{2}{white}{celeste}
\renewcommand{\arraystretch}{1.5}
\section{Tecnologie, framework e librerie impiegate}
Di seguito vengono descritte tecnologie, framework e servizi di terze parti utilizzati per il progetto.
\subsection{Swagger}
Swagger è un linguaggio di descrizione dell'interfaccia per descrivere le API RESTful espresse utilizzando JSON. Swagger viene utilizzato insieme a una serie di strumenti software open source per progettare, creare, documentare e utilizzare i servizi Web RESTful.
\subsection{AWS Cognito}
AWS Cognito fornisce autenticazione, autorizzazione e gestione degli utenti per le applicazioni Web e mobili. Gli utenti possono accedere direttamente con un nome utente e una password, oppure tramite terze parti, ad esempio Facebook, Amazon, Google o Apple.
I due componenti principali di Amazon Cognito da utilizzare separatamente o insieme sono i pool di utenti, directory utente che forniscono opzioni di registrazione e di accesso, e i pool di identità che consentono di concedere agli utenti l'accesso ad altri servizi AWS.
\subsection{AWS DynamoDB} 
Amazon DynamoDB è un database NoSQL che offre prestazioni veloci e predicibili, in particolare si tratta di un database completamente gestito che consente di scaricare gli oneri amministrativi legati al funzionamento e al ridimensionamento di un database distribuito in modo da non doversi preoccupare del provisioning, dell'installazione e della configurazione dell'hardware, della replica, delle patch del software o del ridimensionamento del cluster eliminando il carico operativo e la complessità coinvolti nella protezione dei dati sensibili. 

\subsection{AWS API Gateway}
Amazon API Gateway semplifica per gli sviluppatori la creazione, la pubblicazione, la manutenzione, il monitoraggio e la protezione delle API RESTful su qualsiasi scala. Le API fungono da “porta di entrata” per consentire l’accesso delle applicazioni ai dati, alla logica aziendale o alle funzionalità dai servizi back-end. 
API Gateway gestisce tutte le attività di accettazione ed elaborazione relative a centinaia di migliaia di chiamate API simultanee, inclusi gestione del traffico, controllo di accessi e autorizzazioni, monitoraggio e gestione delle versioni delle API. 

\subsection{AWS Lambda}
AWS Lambda è un servizio di elaborazione serverless che permette di eseguire il codice senza effettuare il provisioning o gestire i server, creare una logica di dimensionamento dei cluster in funzione dei carichi di lavoro, mantenere integrazioni degli eventi o gestire i runtime. Con Lambda è possibile eseguire codice per qualsiasi tipo di applicazione o servizio di back-end, senza alcuna amministrazione.  È possibile scrivere le funzioni Lambda nel linguaggio preferiro (Node.js, Python, Go, Java e altri ancora) e utilizzare strumenti sia serverless sia di container, come AWS SAM o Docker CLI, per creare, testare e distribuire le funzioni.
\subsection{Amazon S3 Bucket}
È un servizio di storage di oggetti che offre scalabilità, disponibilità dei dati, sicurezza e prestazioni all'avanguardia nel settore offrendo caratteristiche di gestione semplici da utilizzare che consentono di organizzare i dati e di configurare controlli di accesso ottimizzati per soddisfare requisiti aziendali, di pianificazione e di conformità specifici.
\subsection{Amazon SNS}
Amazon Simple Notification Service (Amazon SNS) è un servizio di messaggistica completamente gestito per la comunicazione application-to-person (A2P) e application-to-application (A2A).
Le comunicazioni avvengono in modo asincrono, con un punto di accesso logico e un canale di comunicazione. 
\subsection{Amazon SQS}
Amazon Simple Queue Service (SQS) è un servizio di accodamento messaggi completamente gestito che consente la separazione e la scalabilità di microservizi, sistemi distribuiti e applicazioni serverless. Con SQS, è possibile inviare, memorizzare e ricevere qualsiasi volume di messaggi tra componenti software senza perdite e senza dover impiegare altri servizi per mantenere la disponibilità. 
SQS offre due tipi di code di messaggi: le code standard offrono throughput massimo, ordinamento semplificato e distribuzione di tipo at-least-once mentre le code FIFO sono progettate per garantire che i messaggi vengano elaborati esattamente una sola volta, nell'ordine in cui sono inviati.
\subsection{Stripe}
Stripe è una piattaforma esterna per i pagamenti online che consente all'e-commerce di accettare pagamenti con bancomat, carta ricaricabile o carta di credito. È una soluzione sicura e rapida, i pagamenti vengono elaborati utilizzando strumenti ad hoc sviluppati per la gestione dei flussi di pagamento con controlli anti-frode.
\subsection{NodeJs}
Node.js è un runtime system open source multipiattaforma orientato agli eventi per l'esecuzione di codice JavaScript, ha un'architettura orientata agli eventi che rende possibile l’I/O asincrono. Questo design punta ad ottimizzare il Throughput e la scalabilità nelle applicazioni web con molte operazioni di input/output. Il modello di networking su cui si basa Node.js è I/O event-driven: ciò vuol dire che Node richiede al sistema operativo di ricevere notifiche al verificarsi di determinati eventi, e rimane quindi in sleep fino alla notifica stessa: solo in tale momento torna attivo per eseguire le istruzioni previste nella funzione di callback, così chiamata perché da eseguire una volta ricevuta la notifica che il risultato dell'elaborazione del sistema operativo è disponibile.
\subsection{Npm}
Npm è un package manager per il linguaggio di programmazione JavaScript, il predefinito per l'ambiente di runtime JavaScript Node.js. Consiste in un client da linea di comando, chiamato anch'esso npm, e un database online di moduli pubblici e privati che offrono diverse funzionalità: dalla gestione dell’upload di file, ai database MySQL o a Redis, attraverso framework, sistemi di template e la gestione della comunicazione in tempo reale con i visitatori.
\subsection{TypeScript}
TypeScript è un linguaggio di programmazione open source sviluppato da Microsoft che estende la sintassi di JavaScript, aggiungendo o rendendo più flessibili e potenti varie sue caratteristiche, in modo che qualunque programma scritto in JavaScript sia anche in grado di funzionare con TypeScript senza nessuna modifica. 
\subsection{Next.js}
Next.js è un framework JavaScript back-end per applicazioni React che non richiede alcun setup e che consente il rendering automatico lato server (SSR, server side rendering).
Con Next.js si possono sviluppare applicazioni web, app mobile, desktop e web app progressive: è costruito secondo il principio di “Build once, run anywhere“.
Altre caratteristiche di Next.js sono suddivisione automatica del codice, routing automatico, hot code reloading (viene ricaricato solo il codice modificato) ed esportazione statica (con un solo comando può esportare un sito statico).
\subsection{Jest.js}
Jest.js è un framework di test JavaScript con particolare attenzione alla semplicità e al supporto per applicazioni web di grandi dimensioni.  Fornisce diverse funzionalità come la creazione automatizzata di mock, l'esecuzione di test in parallelo per aumentarne la velocità e la possibilità di testare il codice asincrono in modo sincrono. Jest trova automaticamente i test da eseguire nel codice sorgente, e funziona su progetti JS che includono React, Babel, TypeScript, Node, Angular, Vue.
\subsection{ESLint}
ESLint è uno strumento di analisi del codice statico per identificare pattern problematici o codice che non rispetta certe linee guida predefinite nel codice JavaScript senza eseguirlo. Le regole in ESLint sono configurabili e le regole personalizzate possono essere definite e caricate. ESLint è scritto usando Node.js per fornire un ambiente a runtime veloce e di facile installazione attraverso npm.
\subsection{Material-UI}
Material-UI è una libreria di componenti React per sviluppare il design del proprio progetto o per poter utilizzare tutta una serie di componenti stabili predefiniti.
\subsection{Serverless Framework}
Serverless Framework è un framework Web gratuito e open source scritto utilizzando Node.js, sviluppato per la creazione di applicazioni su AWS Lambda. Il Serverless Framework è costituito da una CLI open source e da una dashboard che insieme, forniscono una gestione completa del ciclo di vita delle applicazioni serverless.

\newpage

\subsection{Rischi legati ai membri del gruppo}


    \begin{table}[H]
        \begin{tabular}{|c|p{10cm}|}
        \hline
        \rowcolor{darkblue}
        \multicolumn{2}{|c|}{\textcolor{white}{\textbf{RG1 - Contrasti tra i Componenti}}} \\
        \hline
         Descrizione & Il gruppo deve cooperare con professionalità nonostante la poca esperienza e non si conoscessero in precedenza, questo può causare tensioni o contrasti.\\ 
         \hline
         Conseguenze & Tensioni o contrasti rallentano e danneggiano il corretto svolgimento del progetto.\\
         \hline
         Probabilità di Occorrenza & Bassa.\\
         \hline
         Pericolosità & Alta.\\
         \hline
         Precauzioni & Ogni elemento del gruppo cercherà di limitare eventuali tensioni a favore del collettivo.\\
         \hline
         Piano di Contingenza & Il {\Responsabile} riassegnerà i compiti per limitare la vicinanza delle parti interessate, eventualmente insieme al resto del \glo{team} cercherà di sanare le incomprensioni.  In casi estremi verrà chiamato in causa il \VT.\\ 
         \hline
        \end{tabular}
        \caption{\label{tab:RG1}Analisi dei rischi per contrasti tra i componenti.}
    \end{table}


    \begin{table}[H]
        \begin{tabular}{|c|p{10cm}|}
        \hline
        \rowcolor{darkblue}
        \multicolumn{2}{|c|}{\textcolor{white}{\textbf{RG2 - Disponibilità dei Membri}}} \\
        \hline
         Descrizione & Ogni membro ha impegni universitari e personali oltre all’attività di progetto; possono inoltre insorgere problemi di salute e familiari che potrebbero rendere i membri improduttivi per alcuni periodi.\\ 
         \hline
         Conseguenze & Possibilità di ritardi su attività individuali o collettive.\\
         \hline
         Probabilità di Occorrenza & Media.\\
         \hline
         Pericolosità & Medio-Alta.\\
         \hline
         Precauzioni & Ogni membro del gruppo è tenuto a compilare un calendario condiviso e deve far presente agli altri componenti del team di eventuali periodi di improduttività. Il {\Responsabile} è così in grado di organizzare al meglio il lavoro.\\
         \hline
         Piano di Contingenza & In caso di mancanze prolungate che provocherebbero pesanti ritardi, il {\Responsabile} si occuperà di ridistribuire i compiti da svolgere ai restanti membri del gruppo di lavoro.\\ 
         \hline
        \end{tabular}
        \caption{\label{tab:RG2}Analisi dei rischi per disponibilità dei membri.}
    \end{table}


    \begin{table}[H]
        \begin{tabular}{|c|p{10cm}|}
        \hline
        \rowcolor{darkblue}
        \multicolumn{2}{|c|}{\textcolor{white}{\textbf{RG3 - Inesperienza Gestionale}}} \\
        \hline
         Descrizione & Il team non ha mai affrontato un progetto di tali dimensioni, nè a livello di carico di lavoro nè per grandezza del gruppo di lavoro. Inoltre, ciascun componente del gruppo non ha esperienza di lavoro che richieda il coordinamento di un gruppo di sette o più persone.\\ 
         \hline
         Conseguenze & Ogni singolo membro del team è disabituato a relazionarsi con un gruppo di persone. I membri del gruppo non hanno familiarità con i ruoli che devono impersonare e con i compiti che devono svolgere portando così possibili ritardi nello sviluppo del prodotto.\\
         \hline
         Probabilità di Occorrenza & Alta.\\
         \hline
         Pericolosità & Alta.\\
         \hline
         Precauzioni & Una frequente comunicazione interna di eventuali difficoltà, specialmente rivolta al {\Responsabile}, aiuterà a mitigare l'insorgere di problemi.\\
         \hline
         Piano di Contingenza & Qualunque difficoltà sarà notificata al {\Responsabile} e affrontata con la collaborazione di tutti. Il {\Responsabile} riassegnerà ai membri più in difficoltà dei compiti  più adatti alle loro competenze. Il componente in questione dovrà eseguire un’autoanalisi per comprendere le motivazioni dei suoi problemi e come migliorare.\\ 
         \hline
        \end{tabular}
        \caption{\label{tab:RG3}Analisi dei rischi per inesperienza gestionale.}
    \end{table}

    \begin{table}[H]
        \begin{tabular}{|c|p{10cm}|}
        \hline
        \rowcolor{darkblue}
        \multicolumn{2}{|c|}{\textcolor{white}{\textbf{RG4 - Scarsa Comunicazione}}} \\
        \hline
         Descrizione & Ciascun membro del gruppo non ha molta esperienza nel \glo{teamwork}, questo può portare a una cattiva comunicazione tra i membri del gruppo.\\ 
         \hline
         Conseguenze & Un'insufficiente comunicazione tra i membri del gruppo può portare a ritardi nello sviluppo del progetto o addirittura a contrasti tra i componeneti stessi.\\
         \hline
         Probabilità di Occorrenza & Bassa.\\
         \hline
         Pericolosità & Media.\\
         \hline
         Precauzioni & Il {\Responsabile} provvederà a monitorare e promuovere un adeguato livello di comunicazione attiva all'interno del team. Non dovranno insorgere situazioni dove, per mancata o scarsa comunicazione, ci siano ritardi nelle decisioni progettuali importanti  con una conseguente insorgenza di situazioni di tensione.\\
         \hline
         Piano di Contingenza & Nel caso si rilevi una scarsa comunicazione da parte di un membro del gruppo sarà compito del {\Responsabile} ricercarne il motivo, se questo si dimostra non lecito provvederà prima ad un richiamo informale (a voce o con gli strumenti di comunicazione interni). Se il comportamento venisse reiterato il {\Responsabile} programmerà una riunione interna al gruppo per discutere la situazione.\\ 
         \hline
        \end{tabular}
        \caption{\label{tab:RG4}Analisi dei rischi per scarsa comunicazione.}
    \end{table}
\newpage

\subsection{Rischi legati agli strumenti}
\newpage

\subsection{Rischi legati all'orgranizzazione}

\begin{table}[H]
    \begin{tabular}{|c | p{10cm}|}
    \hline
    \rowcolor{darkblue}
    \multicolumn{2}{|c|}{\textcolor{white}{\textbf{RO1 - Costi delle Attività}}} \\
    \hline
    Descrizione & Data l'inesperienza del team nella pianificazione di progetto e la messa in pratica della stessa su un arco di tempo medio-lungo, può verificarsi una sottostima o una sovrastima dei costi e dei tempi necessari alla realizzazione del progetto.\\ 
    \hline
    Conseguenze & Una sottostima provocherebbe ritardi nella pianificazione; una sovrastima porterebbe ad uno spreco di tempo. Entrambi i casi richiederebbero poi una nuova pianificazione delle attività.\\
    \hline
    Probabilità di Occorrenza & Medio-Alta.\\
    \hline
    Pericolosità & Alta.\\
    \hline
    Precauzioni & Ogni componente del gruppo controllerà periodicamente lo stato della propria attività rispetto alla tabella di marcia imposta nel documento \textit{\PdP}.\\ 
    \hline
    Piano di Contingenza & Per ogni attività è previsto un tempo di \glo{slack}, in modo da arginare il più possibile il ritardo nell'avanzamento del progetto.\\ 
    \hline
    \end{tabular}
    \caption{\label{tab:RO1}Analisi dei rischi per i costi delle attività.}
    
\end{table}
\newpage

\section{Requisiti}

\subsection{Introduzione}
In questa parte, vengono riportati i requisiti del progetto, classificati per tipologia. Ciascun requisito possiede un codice identificativo, il cui formalismo viene riportato all'interno del documento \NdPv{1.0.0}.

\subsection{Requisiti funzionali} \label{ReqFunz}
\rowcolors{2}{white}{celeste} 
\renewcommand{\arraystretch}{1.5}

\addcontentsline{lot}{table}{Requisiti funzionali}

\begin{longtable}{c C{9.5cm} C{2.5cm}} 

	\rowcolor{darkblue}
	\textcolor{white}{\textbf{Codice Requisito}}&
	\textcolor{white}{\textbf{Descrizione}}&
    \textcolor{white}{\textbf{Fonte}} \\
    
    % Acquirente
    \rfun{O}{\ref{registrazione}} & L'utente non autenticato che non dispone di credenziali può registrarsi e accedere come acquirente alla piattaforma. & UC\ref{registrazione} \\ 

\rfun{O}{\ref{registrazione.modulo.nome}} & L'utente non autenticato inserisce il nome con il quale vuole registrarsi. & UC\ref{registrazione.modulo.nome} \\

\rfun{O}{\ref{registrazione.modulo.cognome}} & L'utente non autenticato inserisce il cognome con il quale vuole registrarsi. & UC\ref{registrazione.modulo.cognome} \\

\rfun{O}{\ref{registrazione.modulo.email}} & L'utente non autenticato inserisce l'indirizzo e-mail con il quale vuole registrarsi. & UC\ref{registrazione.modulo.email} \\

\rfun{O}{\ref{registrazione.modulo.password}} & L'utente non autenticato inserisce la password con la quale vuole registrarsi. & UC\ref{registrazione.modulo.password} \\

\rfun{O}{\ref{registrazione.modulo.conferma-password}} & L'utente non autenticato inserisce la conferma della password con la quale vuole registrarsi. & UC\ref{registrazione.modulo.conferma-password} \\

\rfun{O}{\ref{autenticazione-venditore}} & L'utente non autenticato che dispone di credenziali venditore può accedere alla piattaforma usando l'e-mail e la password & UC\ref{autenticazione-venditore} \\

\rfun{O}{\ref{autenticazione-acquirente}} & L'utente non autenticato che dispone di credenziali acquirente può accedere nella piattaforma usando l'e-mail e la password & UC\ref{autenticazione-acquirente} \\

\rfun{O}{\ref{password-dimenticata}} & L'utente non autenticato che dispone di credenziali acquirente o venditore e si è dimenticato la propria password, potrà cambiarla. & UC\ref{password-dimenticata} \\

\rfun{O}{} & L'utente non autenticato potrà accedere alla schermata per l'autenticazione da qualsiasi schermata della piattaforma. & Interna \\

\rfun{O}{\ref{logout}} & L'utente autenticato potrà scollegarsi dalla piattaforma da qualsiasi schermata della piattaforma. & UC\ref{logout} \\


    \rfun{O}{\ref{ricerca-prodotti-acquirente}} & L'utente non autenticato e l'acquirente può cercare i prodotti attraverso delle parole dalla schermata principale, oppure dalla PLP. & UC\ref{ricerca-prodotti-acquirente} \\

\rfun{O}{\ref{filtro-prodotti-acquirente}} & L'utente non autenticato o l'acquirente può filtrare i prodotti all'interno della \glo{PLP}. & UC\ref{filtro-prodotti-acquirente} \\

\rfun{O}{\ref{filtro-prodotti-acquirente.categoria}} & L'utente non autenticato o l'acquirente può cercare i prodotti in base alla loro categoria, selezionando quelle di interesse tra tutte le categorie disponibili. & UC\ref{filtro-prodotti-acquirente.categoria} \\

\rfun{O}{\ref{filtro-prodotti-acquirente.prezzo}} & L'utente non autenticato o l'acquirente può cercare i prodotti in base al loro prezzo. & UC\ref{filtro-prodotti-acquirente.prezzo} \\

\rfun{O}{\ref{filtro-prodotti-acquirente.magazzino}} & L'utente non autenticato o l'acquirente può cercare i prodotti in base alla loro disponibilità in magazzino. & UC\ref{filtro-prodotti-acquirente.magazzino} \\


    \rfun{O}{} & L'utente non autenticato o l'acquirente visualizzerà i prodotti nella PLP ordinati alfabeticamente come ordinamento predefinito & Capitolato \\

\rfun{O}{} & L'utente non autenticato o l'acquirente, nella PLP, visualizzerà una lista di tutti i prodotti corrispondenti alla ricerca, dove per ogni prodotto sarà visualizzabile: il nome del prodotto, la prima immagine disponibile di esso, il suo prezzo per unità e se è disponibile in magazzino o no & Capitolato \\

\rfun{O}{} & I prodotti nella PLP non disponibili devono essere distinti da quelli che lo sono. & Capitolato \\

\rfun{O}{\ref{ordinamento-alfabetico}} & L'utente non autenticato o l'acquirente può ordinare i prodotti risultanti dalla ricerca effettuata in precedenza per ordine alfabetico. & UC\ref{ordinamento-alfabetico} \\

\rfun{O}{\ref{ordinamento-prezzo-crescente}} & L'utente non autenticato o l'acquirente può ordinare i prodotti risultanti dalla ricerca effettuata in precedenza per prezzo crescente. & UC\ref{ordinamento-prezzo-crescente} \\

\rfun{O}{\ref{ordinamento-prezzo-decrescente}} & L'utente non autenticato o l'acquirente può ordinare i prodotti risultanti dalla ricerca effettuata in precedenza per prezzo decrescente. & UC\ref{ordinamento-prezzo-decrescente} \\

\rfun{O}{\ref{aggiunta-carrello-plp}} & L'utente non autenticato o l'acquirente può aggiungere al carrello un'unità di un prodotto direttamente dalla PLP, avendo anche la possibilità di modificare in precedenza la quantità da aggiungere. & UC\ref{aggiunta-carrello-plp} \\

% \rfun{O}{\ref{}} &  & UC\ref{} \\


    \rfun{O}{} & L'utente non autenticato o l'acquirente può accedere alla \glo{PDP} di un prodotto in evidenza dalla schermata principale. & Interna \\

\rfun{O}{} & L'utente non autenticato o l'acquirente può accedere alla PDP di un prodotto acquistato dalla schermata di riepilogo ordine. & Interna \\

\rfun{O}{} & L'utente non autenticato o l'acquirente può accedere alla PDP di un prodotto che è stato aggiunto al carrello. & \glo{Capitolato} \\

\rfun{O}{} & L'utente non autenticato o l'acquirente può accedere alla PDP di un prodotto dalla PLP. & Capitolato \\

\rfun{O}{} & L'utente non autenticato o l'acquirente dalla PDP potrà visualizzare: nome del prodotto, la sua descrizione, le categorie alle quali appartiene, foto relative ad esso, il prezzo e relativi sconti. & Interna \\

\rfun{O}{\ref{aggiunta-carrello-pdp}} & L'utente non autenticato o l'acquirente può aggiungere al carrello un'unità di un prodotto direttamente dalla PDP, avendo anche la possibilità di modificare in precedenza la quantità da aggiungere. & UC\ref{aggiunta-carrello-pdp} \\

\rfun{O}{\ref{modifica-quantita-da-aggiungere-al-carrello}} & L'acquirente o l'utente non autenticato modifica la quantità del prodotto che vuole aggiungere nel carrello. & UC\ref{modifica-quantita-da-aggiungere-al-carrello} \\

% \rfun{O}{\ref{}} &  & UC\ref{} \\


    \rfun{O}{} & L'utente non autenticato o l'acquirente può accedere alla schermata del carrello da qualsiasi altra schermata della piattaforma. & Capitolato \\

\rfun{O}{\ref{visualizzazione-carrello}} & L'utente non autenticato o l'acquirente potrà visualizzare il proprio carrello, dove potrà visualizzare il prezzo totale e l'elenco dei prodotti. & UC\ref{visualizzazione-carrello} \\

\rfun{O}{\ref{visualizzazione-carrello}} & L'utente non autenticato o l'acquirente potrà visualizzare il proprio carrello dove per ogni prodotto inserito sarà indicato: il nome del prodotto, la quantità inserita, il prezzo del prodotto in base alla quantità inserita e agli sconti disponibili e la prima foto disponibile del prodotto. & UC\ref{visualizzazione-carrello} \\

\rfun{O}{\ref{eliminazione-prodotto-dal-carrello}} & L'utente non autenticato o l'acquirente può eliminare un prodotto che ha inserito nel carrello. & UC\ref{eliminazione-prodotto-dal-carrello} \\

\rfun{O}{\ref{modifica-quantita-nel-carrello}} & L'utente non autenticato o l'acquirente modifica la quantità di un prodotto precedentemente inserito nel carrello. & UC\ref{modifica-quantita-nel-carrello} \\

% \rfun{O}{\ref{}} &  & UC\ref{} \\


    \rfun{O}{\ref{checkout}} & L'acquirente può procedere al checkout per effettuare l'ordine con i prodotti inseriti nel carrello. & UC\ref{checkout} \\

\rfun{O}{\ref{checkout.indirizzo}} & L'acquirente seleziona l'indirizzo della consegna, ovvero dove verrà recapitato l'acquisto, tra gli indirizzi di consegna precedentemente inseriti. & UC\ref{checkout.indirizzo} \\

\rfun{O}{\ref{checkout.pagamento}} & L'acquirente procede al pagamento attraverso il servizio fornito dal gestore dei pagamenti. & UC\ref{checkout.pagamento} \\

\rfun{O}{\ref{inserimento-indirizzo-consegna}} & L'acquirente può aggiungere un nuovo indirizzo di consegna. & UC\ref{inserimento-indirizzo-consegna} \\

\rfun{O}{\ref{inserimento-indirizzo-consegna.modulo.nazione}} & L'acquirente può selezionare la nazione dell'indirizzo di consegna. & UC\ref{inserimento-indirizzo-consegna.modulo.nazione} \\

\rfun{O}{\ref{inserimento-indirizzo-consegna.modulo.comune}} & L'acquirente può inserire il comune dell'indirizzo di consegna. & UC\ref{inserimento-indirizzo-consegna.modulo.comune} \\

\rfun{O}{\ref{inserimento-indirizzo-consegna.modulo.via}} & L'acquirente può inserire la via dell'indirizzo di consegna, includendo anche il numero civico e l'eventuale interno. & UC\ref{inserimento-indirizzo-consegna.modulo.via} \\

\rfun{O}{\ref{inserimento-indirizzo-consegna.modulo.cap}} & L'acquirente può inserire il CAP dell'indirizzo di consegna. & UC\ref{inserimento-indirizzo-consegna.modulo.cap} \\

\rfun{O}{\ref{modifica-indirizzo-consegna}} & L'acquirente può modificare un indirizzo di consegna precedentemente inserito. & UC\ref{modifica-indirizzo-consegna} \\

\rfun{O}{\ref{modifica-indirizzo-consegna.nazione}} & L'acquirente può modificare la nazione dell'indirizzo di consegna. & UC\ref{modifica-indirizzo-consegna.nazione} \\

\rfun{O}{\ref{modifica-indirizzo-consegna.comune}} & L'acquirente può modificare il comune dell'indirizzo di consegna. & UC\ref{modifica-indirizzo-consegna.comune} \\

\rfun{O}{\ref{modifica-indirizzo-consegna.via}} & L'acquirente può modificare la via dell'indirizzo di consegna. & UC\ref{modifica-indirizzo-consegna.via} \\

\rfun{O}{\ref{modifica-indirizzo-consegna.cap}} & L'acquirente può modificare il CAP dell'indirizzo di consegna. & UC\ref{modifica-indirizzo-consegna.cap} \\

\rfun{O}{\ref{eliminazione-indirizzo-consegna}} & L'acquirente può eliminare un indirizzo di consegna precedentemente inserito. & UC\ref{eliminazione-indirizzo-consegna} \\

% \rfun{O}{\ref{}} &  & UC\ref{} \\


    \rfun{O}{} & L'acquirente può accedere alla schermata con tutti gli ordini effettuati da qualsiasi altra schermata della piattaforma. & Interna \\

\rfun{O}{\ref{visualizzazione-ordini-effettuati}} & L'acquirente può visualizzare l'elenco degli ordini effettuati sulla piattaforma. Per ognuno di questi potrà visualizzare: il codice numerico dell'ordine, lo stato dell'ordine, il prezzo totale che è stato pagato, l'indirizzo a cui è stato consegnato o verrà consegnato e una lista di tutti i prodotti acquistati.& UC\ref{visualizzazione-ordini-effettuati} \\

\rfun{O}{\ref{visualizzazione-ordini-effettuati}} & L'acquirente può visualizzare i prodotti che sono stati acquistati in un ordine effettuato. Per ognuno di questi verrà visualizzato: il nome del prodotto, la quantità acquistata e il prezzo totale a cui è stato acquistato. & UC\ref{visualizzazione-ordini-effettuati} \\

\rfun{O}{\ref{ricerca-codice-ordine-acquirente}} & L'acquirente può cercare un ordine sapendo il suo codice numerico. & UC\ref{ricerca-codice-ordine-acquirente} \\

\rfun{O}{\ref{filtro-temporale-ordini-acquirente}} & L'acquirente può filtrare temporalmente l'elenco degli ordini effettuati sulla piattaforma. & UC\ref{filtro-temporale-ordini-acquirente} \\

\rfun{O}{\ref{filtro-temporale-ordini-acquirente.data-iniziale}} & L'acquirente può impostare la data iniziale dell'intervallo per il quale filtrare l'elenco degli ordini effettuati sulla piattaforma. & UC\ref{filtro-temporale-ordini-acquirente.data-iniziale} \\

\rfun{O}{\ref{filtro-temporale-ordini-acquirente.data-finale}} & L'acquirente impostare la data finale dell'intervallo per il quale filtrare l'elenco degli ordini effettuati sulla piattaforma. & UC\ref{filtro-temporale-ordini-acquirente.data-finale} \\

% \rfun{O}{\ref{}} &  & UC\ref{} \\


    \rfun{O}{} & L'acquirente può accedere alla schermata con le proprie informazioni da qualsiasi altra schermata della piattaforma. & Interna \\

\rfun{O}{} & L'acquirente, nella propria schermata personale, potrà visualizzare: il nome ed il cognome che ha inserito e l'indirizzo e-mail collegato al proprio account. & Interna \\

\rfun{O}{\ref{modifica-informazioni-acquirente}} & L'acquirente può modificare le sue informazioni personali. & UC\ref{modifica-informazioni-acquirente} \\

\rfun{O}{\ref{modifica-informazioni-acquirente.nome}} & L'acquirente può modificare il nome con il quale si è registrato. & UC\ref{modifica-informazioni-acquirente.nome} \\

\rfun{O}{\ref{modifica-informazioni-acquirente.cognome}} & L'acquirente può modificare il cognome con il quale si è registrato. & UC\ref{modifica-informazioni-acquirente.cognome} \\

\rfun{O}{\ref{modifica-informazioni-acquirente.email}} & L'acquirente può modificare l'indirizzo e-mail con il quale si autenticherà. & UC\ref{modifica-informazioni-acquirente.email} \\

\rfun{O}{\ref{modifica-informazioni-acquirente.password}} & L'acquirente può modificare la password con la quale si autenticherà. & UC\ref{modifica-informazioni-acquirente.password} \\

\rfun{O}{\ref{eliminazione-account-acquirente}} & L'acquirente può eliminare il proprio account. & UC\ref{eliminazione-account-acquirente} \\

% \rfun{O}{\ref{}} &  & UC\ref{} \\

% \rfun{O}{\ref{}} &  & UC\ref{} \\

% \rfun{O}{\ref{}} &  & UC\ref{} \\

% \rfun{O}{\ref{}} &  & UC\ref{} \\

% \rfun{O}{\ref{}} &  & UC\ref{} \\

% \rfun{O}{\ref{}} &  & UC\ref{} \\

% \rfun{O}{\ref{}} &  & UC\ref{} \\

% \rfun{O}{\ref{}} &  & UC\ref{} \\

% \rfun{O}{\ref{}} &  & UC\ref{} \\

% \rfun{O}{\ref{}} &  & UC\ref{} \\

% \rfun{O}{\ref{}} &  & UC\ref{} \\

% \rfun{O}{\ref{}} &  & UC\ref{} \\

% \rfun{O}{\ref{}} &  & UC\ref{} \\

% \rfun{O}{\ref{}} &  & UC\ref{} \\

% \rfun{O}{\ref{}} &  & UC\ref{} \\

% \rfun{O}{\ref{}} &  & UC\ref{} \\


    \rfun{O}{} & L'utente non autenticato o l'acquirente, nella schermata principale, potrà visualizzare i prodotti in evidenza e la descrizione dell'azienda. & Capitolato \\

\rfun{O}{} & L'utente non autenticato o l'acquirente può accedere alla schermata principale da qualsiasi altra schermata della piattaforma. & Interna \\


    % Venditore
    \rfun{O}{\ref{modifica-informazioni-venditore}} & Il venditore può modificare le sue informazioni personali. & UC\ref{modifica-informazioni-venditore} \\
	
\rfun{O}{\ref{modifica-informazioni-venditore.nome}} & Il venditore può modificare il proprio nome. & UC\ref{modifica-informazioni-venditore.nome} \\
	
\rfun{O}{\ref{modifica-informazioni-venditore.cognome}} & Il venditore può modificare il proprio cognome. & UC\ref{modifica-informazioni-venditore.cognome} \\
	
\rfun{O}{\ref{modifica-informazioni-venditore.email}} & Il venditore può modificare la propria e-mail. & UC\ref{modifica-informazioni-venditore.email} \\
	
\rfun{O}{\ref{modifica-informazioni-venditore.descrizione-azienda}} & Il venditore può modificare la descrizione dell'azienda. & UC\ref{modifica-informazioni-venditore.descrizione-azienda} \\

\rfun{O}{\ref{modifica-password}} & L'utente autenticato può modificare la password con la quale si autenticherà. & UC\ref{modifica-password} \\

\rfun{O}{} & Il venditore può accedere alla schermata con le proprie informazioni da qualsiasi altra schermata della piattaforma. & Interna \\
	
\rfun{O}{} & Il venditore, nella propria schermata personale, potrà visualizzare: il nome, il cognome, l'indirizzo e-mail, il logo, la descrizione e il nome dell'azienda. & Interna \\ 
    
    \rfun{O}{} & Il venditore potrà accedere alla propria PLP da qualsiasi schermata. & Interna \\

\rfun{O}{} & Il venditore visualizzerà i prodotti nella PLP ordinati alfabeticamente come ordinamento predefinito. & Interna \\

\rfun{O}{} & Il venditore, nella PLP, visualizzerà una lista di tutti i prodotti corrispondenti alla ricerca, dove per ogni prodotto sarà visualizzabile: il nome del prodotto, la prima immagine disponibile di esso, il suo prezzo per unità e se è disponibile in magazzino o no. & Interna \\

\rfun{O}{} & Il venditore può accedere alla PDP di un prodotto acquistato dalla schermata di riepilogo di un ordine. & Interna \\

\rfun{O}{} & Il venditore può accedere alla PDP di un prodotto dalla propria PLP. & Capitolato \\

\rfun{O}{\ref{aggiunta-prodotto}} & Il venditore può aggiungere un nuovo prodotto alla piattaforma. & UC\ref{aggiunta-prodotto} \\
	
\rfun{O}{\ref{aggiunta-prodotto.nome}} & Il venditore può inserire il nome del prodotto da aggiungere. & UC\ref{aggiunta-prodotto.nome} \\
	
\rfun{O}{\ref{aggiunta-prodotto.descrizione}} & Il venditore può inserire la descrizione del prodotto da aggiungere. & UC\ref{aggiunta-prodotto.descrizione} \\
	
\rfun{O}{\ref{aggiunta-prodotto.categorie}} & Il venditore può inserire le categorie del prodotto da aggiungere. & UC\ref{aggiunta-prodotto.categorie} \\
	
\rfun{O}{\ref{aggiunta-prodotto.prezzo}} & Il venditore può inserire il prezzo del prodotto da aggiungere. & UC\ref{aggiunta-prodotto.prezzo} \\
	
\rfun{O}{\ref{aggiunta-prodotto.sconto}} & Il venditore può inserire lo sconto del prodotto da aggiungere. & UC\ref{aggiunta-prodotto.sconto} \\
	
\rfun{O}{\ref{aggiunta-prodotto.quantita}} & Il venditore può inserire la quantità disponibile del prodotto da aggiungere. & UC\ref{aggiunta-prodotto.quantita} \\
	
\rfun{O}{\ref{aggiunta-prodotto.foto}} & Il venditore può inserire le foto del prodotto da aggiungere. & UC\ref{aggiunta-prodotto.foto} \\
	
\rfun{O}{} & Il venditore potrà visualizzare il nome, la descrizione, le categorie, il prezzo, lo sconto, le foto e la quantità disponibile di tutti i suoi prodotti. & Interna \\
    
\rfun{O}{\ref{modifica-prodotto}} & Il venditore può modificare un prodotto già presente nella piattaforma. & UC\ref{modifica-prodotto} \\
    
\rfun{O}{\ref{modifica-prodotto.nome}} & Il venditore può modificare il nome di un prodotto già inserito nella piattaforma. & UC\ref{modifica-prodotto.nome} \\
    
\rfun{O}{\ref{modifica-prodotto.descrizione}} & Il venditore può modificare la descrizione di un prodotto già inserito nella piattaforma. & UC\ref{modifica-prodotto.descrizione} \\
    
\rfun{O}{\ref{modifica-prodotto.categorie}} & Il venditore può modificare le categorie di un prodotto già inserito nella piattaforma. & UC\ref{modifica-prodotto.categorie} \\
    
\rfun{O}{\ref{modifica-prodotto.prezzo}} & Il venditore può modificare il prezzo di un prodotto già inserito nella piattaforma. & UC\ref{modifica-prodotto.prezzo} \\
    
\rfun{O}{\ref{modifica-prodotto.sconto}} & Il venditore può modificare lo sconto percentuale di un prodotto già inserito nella piattaforma. & UC\ref{modifica-prodotto.sconto} \\
    
\rfun{O}{\ref{modifica-prodotto.aggiunta-foto}} & Il venditore può aggiungere nuove foto ad un prodotto già inserito nella piattaforma. & UC\ref{modifica-prodotto.aggiunta-foto} \\
    
\rfun{O}{\ref{modifica-prodotto.rimozione-foto}} & Il venditore può rimuovere le foto di un prodotto già inserito nella piattaforma. & UC\ref{modifica-prodotto.rimozione-foto} \\
    
\rfun{O}{\ref{aggiunta-prodotto-evidenza}} & Il venditore può aggiungere un prodotto alla sezione "in evidenza" nella vista principale. & UC\ref{aggiunta-prodotto-evidenza} \\

\rfun{O}{\ref{rimozione-prodotto-evidenza}} & Il venditore può rimuovere un prodotto dalla sezione "in evidenza" nella vista principale. & UC\ref{rimozione-prodotto-evidenza} \\
    
\rfun{O}{\ref{eliminazione-prodotto}} & Il venditore può eliminare un prodotto precedentemente inserito. & UC\ref{eliminazione-prodotto} \\

\rfun{O}{\ref{rifornimento-prodotto}} & Il venditore può rifornire un prodotto che sta per esaurire o è esaurito, dalla PDP del prodotto o dalla PLP. & UC\ref{rifornimento-prodotto} \\

% \rfun{O}{\ref{}} &  & UC\ref{} \\

    
    \rfun{O}{\ref{ricerca-prodotti-venditore}} & Il venditore può cercare i prodotti dalla propria PLP attraverso delle parole. & UC\ref{ricerca-prodotti-venditore} \\

\rfun{O}{\ref{filtro-prodotti-venditore}} & Il venditore può filtrare i prodotti all'interno della PLP. & UC\ref{filtro-prodotti-venditore} \\

\rfun{O}{\ref{filtro-prodotti-venditore.categoria}} & Il venditore può cercare i prodotti in base alla loro categoria, selezionando quelle di interesse tra tutte le categorie disponibili. & UC\ref{filtro-prodotti-venditore.categoria} \\

\rfun{O}{\ref{filtro-prodotti-venditore.prezzo}} & Il venditore può cercare i prodotti in base al loro prezzo. & UC\ref{filtro-prodotti-venditore.prezzo} \\

\rfun{O}{\ref{filtro-prodotti-venditore.magazzino}} & Il venditore può cercare i prodotti in base alla loro disponibilità in magazzino. & UC\ref{filtro-prodotti-venditore.magazzino} \\

\rfun{O}{\ref{filtro-prodotti-venditore.evidenza}} & Il venditore può cercare i prodotti in base alla loro caratteristica di trovarsi in evidenza o meno nella piattaforma. & UC\ref{filtro-prodotti-venditore.evidenza} \\
    
    \rfun{O}{\ref{aggiunta-categoria}} & Il venditore può aggiungere una nuova categoria. & UC\ref{aggiunta-categoria} \\
    
\rfun{O}{\ref{aggiunta-categoria.nome}} & Il venditore può aggiungere il nome della nuova categoria da aggiungere. & UC\ref{aggiunta-categoria.nome} \\
    
\rfun{O}{\ref{modifica-categoria}} & Il venditore può modificare una categoria già inserita. & UC\ref{modifica-categoria} \\
    
\rfun{O}{\ref{modifica-categoria.nome}} & Il venditore può modificare il nome attuale di una categoria già inserita. & UC\ref{modifica-categoria.nome} \\
    
\rfun{O}{\ref{eliminazione-categoria}} & Il venditore può eliminare una categoria già inserita. & UC\ref{eliminazione-categoria} \\
    
\rfun{O}{\ref{ricerca-categoria}} & Il venditore può cercare una categoria dalla schermata di amministrazione delle categorie. & UC\ref{ricerca-categoria} \\

    
    \rfun{O}{\ref{visualizzazione-ordini-in-gestione}} & Il venditore può visualizzare l'elenco degli ordini chiusi o da gestire. Per ognuno di questi potrà visualizzare: il codice dell'ordine, lo stato dell'ordine, il prezzo totale che è stato pagato, l'indirizzo e-mail dell'acquirente che ha effettuato l'ordine, l'indirizzo a cui è stato consegnato o verrà consegnato e una lista di tutti i prodotti acquistati. & UC\ref{visualizzazione-ordini-in-gestione} \\

\rfun{O}{\ref{visualizzazione-ordini-in-gestione}} & Il venditore può visualizzare i prodotti che sono stati acquistati in un ordine ricevuto. Per ognuno di questi verrà visualizzato: il nome del prodotto, la quantità acquistata e il prezzo totale a cui è stato acquistato. & UC\ref{visualizzazione-ordini-in-gestione} \\

\rfun{O}{\ref{modifica-stato-ordine}} & Il venditore può modificare lo stato di un determinato ordine. & UC\ref{modifica-stato-ordine} \\

\rfun{O}{} & Il venditore visualizzerà gli ordini ricevuti per ordine cronologico decrescente come ordinamento predefinito. & Interna \\

\rfun{O}{\ref{ricerca-codice-ordine-venditore}} & Il venditore può cercare un ordine sapendo il suo codice identificativo. & UC\ref{ricerca-codice-ordine-venditore} \\
     
\rfun{O}{\ref{ricerca-cliente-ordine-venditore}} & Il venditore può cercare gli ordini in base al cliente che lo ha effettuato. & UC\ref{ricerca-cliente-ordine-venditore} \\
    
\rfun{O}{\ref{filtro-ordini-venditore}} & Il venditore può filtrare gli ordini nella schermata di riepilogo ordini. & UC\ref{filtro-ordini-venditore} \\
    
\rfun{O}{\ref{filtro-ordini-venditore.stato}} & Il venditore può filtrare gli ordini in base al loro stato. & UC\ref{filtro-ordini-venditore.stato} \\
    
\rfun{O}{\ref{filtro-ordini-venditore.temporale}} & Il venditore può filtrare per intervallo temporale gli ordini ricevuti. & UC\ref{filtro-ordini-venditore.temporale} \\
    
\rfun{O}{\ref{filtro-ordini-venditore.temporale.data-iniziale}} & Il venditore può impostare la data iniziale dell'intervallo per il quale filtrare l'elenco degli ordini ricevuti sulla piattaforma. & UC\ref{filtro-ordini-venditore.temporale.data-iniziale} \\
    
\rfun{O}{\ref{filtro-ordini-venditore.temporale.data-finale}} & Il venditore può impostare la data finale dell'intervallo per il quale filtrare l'elenco degli ordini ricevuti sulla piattaforma. & UC\ref{filtro-ordini-venditore.temporale.data-finale} \\


\end{longtable}


\subsection{Requisiti di \glo{qualità}} \label{ReqQual}
\rowcolors{2}{white}{celeste} 
\renewcommand{\arraystretch}{1.5}

\addcontentsline{lot}{table}{Requisiti di qualità}

\begin{longtable}{c C{9.5cm} C{2.5cm}} 
	
	\rowcolor{darkblue}
	\textcolor{white}{\textbf{Codice Requisito}}&
	\textcolor{white}{\textbf{Descrizione}}&
	\textcolor{white}{\textbf{Fonte}}\\

	\rqua{O} & La piattaforma deve essere rilasciata sotto \glo{licenza MIT}. & Capitolato  \\

	\rqua{O} & La piattaforma dovrà rispettare la validazione \glo{W3C}. & Interna \\
	
	\rqua{O} & Deve essere realizzata e consegnata una documentazione delle \glo{API} realizzata automaticamente. & VE\_2020\_12\_28 \\
	
	\rqua{O} & La documentazione delle API deve essere scritta in lingua inglese. & VE\_2020\_12\_28 \\
	
	\rqua{D} & È desiderabile avere anche lo stadio di \glo{Production}. & Capitolato \\
	
\end{longtable}


\subsection{Requisiti di vincolo} \label{ReqVincolo}
\rowcolors{2}{white}{celeste} 
\renewcommand{\arraystretch}{1.5}

\addcontentsline{lot}{table}{Requisiti di vincolo}

\begin{longtable}{c C{9.5cm} C{2.5cm}} 
	
	\rowcolor{darkblue}
	\textcolor{white}{\textbf{Codice Requisito}}&
	\textcolor{white}{\textbf{Descrizione}}&
	\textcolor{white}{\textbf{Fonte}}\\

	\rvin{O} & Utilizzo di \glo{Typescript} come principale linguaggio di sviluppo. & \glo{Capitolato} \\
	
	\rvin{O} & Uso del \glo{framework} \glo{Serverless} per il controllo centrale dell'applicazione. & Capitolato \\
	
	\rvin{O} & Le componenti dell'applicazione devono fornire API basata su \glo{HTTP} o \glo{HTTPS}. & Capitolato \\
	
	\rvin{O} & La piattaforma dovrà essere distribuita su \glo{AWS} utilizzando \glo{AWS Lambda} come unità di calcolo. & Capitolato \\
	
	\rvin{O} & La piattaforma va sviluppata con un'architettura a \glo{microservizi}. & Capitolato \\
	
	\rvin{O} & Il \glo{front end} va sviluppato con \glo{Next.js}. & Capitolato \\
	
	\rvin{O} & Il \glo{front end} deve prevedere il pre-rendering della pagina \glo{HTML5} implementato attraverso uno dei due seguenti approcci: \glo{SSR} o \glo{SSG}. & Capitolato \\
	
	\rvin{O} & La piattaforma dovrà integrarsi con il servizio di pagamento Stripe. & Capitolato \\
	
	\rvin{O} & La parte di monitoraggio e controllo della piattaforma \glo{Amazon CloudWatch}. & Capitolato \\
	
	\rvin{Z} & E' necessario memorizzare le credenziali dell'utente su \glo{Auth0}, un \glo{identity provider} esterno di terze parti, per poter migrare il sito da un'infrastruttura all'altra. & Capitolato \\
	
	\rvin{O} & La piattaforma deve essere sviluppata nei seguenti 3 ambienti secondo l'ordine indicato: prima \glo{Locale}, poi \glo{Testing} e dopo \glo{Staging}. & Capitolato \\

	\rvin{O} & La piattaforma deve funzionare correttamente nel browser Google Chrome per desktop dalla versione 88.0.4324.182 in poi. & Interna \\

	\rvin{O} & La piattaforma deve funzionare correttamente nel browser Mozilla Firefox per desktop dalla versione 86.0.0 in poi. & Interna \\

	\rvin{O} & La piattaforma deve funzionare correttamente nel browser Microsoft Edge per desktop dalla versione 88.0.705.68 in poi. & Interna \\

	\rvin{O} & La piattaforma deve funzionare correttamente nel browser Safari per desktop dalla versione 14.0.2 in poi. & Interna \\

	\rvin{O} & La piattaforma deve funzionare correttamente nel browser Google Chrome per mobile dalla versione 88.0.4324.181 in poi. & Interna \\

	\rvin{O} & La piattaforma deve funzionare correttamente nel browser Mozilla Firefox per mobile dalla versione 86.1.1 in poi. & Interna \\

	\rvin{O} & La piattaforma deve funzionare correttamente nel browser Microsoft Edge per mobile dalla versione 46.01.4.5140 in poi. & Interna \\

	\rvin{O} & La piattaforma deve funzionare correttamente nel browser Safari per mobile dalla versione 14.4 in poi. & Interna \\
	
	\rvin{D} & È desiderabile avere anche lo stadio di \glo{Production}. & Capitolato \\
	
\end{longtable}


\subsection{Requisiti prestazionali} \label{ReqPrest}
\rowcolors{2}{white}{celeste} 
\renewcommand{\arraystretch}{1.5}

\addcontentsline{lot}{table}{Requisiti prestazionali}


\begin{longtable}{c C{9.5cm} C{2.5cm}} 
	
	\rowcolor{darkblue}
	\textcolor{white}{\textbf{Codice Requisito}}&
	\textcolor{white}{\textbf{Descrizione}}&
	\textcolor{white}{\textbf{Fonte}}\\

	\rpre{O} & Tempo di riposta di tutte le pagine, escluso il pagamento, sotto i 10 secondi. & Interna \\

\end{longtable}


\newpage
