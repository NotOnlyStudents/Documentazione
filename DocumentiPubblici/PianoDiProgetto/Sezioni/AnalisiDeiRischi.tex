\section{Analisi dei rischi}
\label{analisi_dei_rischi}
Durante lo sviluppo di un progetto così complesso e di grandi dimensioni, la possibilità di incontrare delle problematiche è alta. Per cercare di evitare il più possibile queste criticità si è svolta un'attenta analisi sui possibili rischi del progetto. La procedura utilizzata per l'analisi dei rischi si suddivide nelle seguenti parti:
\begin{itemize}
    \item \textbf{Individuazione:} Attività iniziale di identificazione di possibili problematiche che potrebbero interferire e minare il corretto avanzamento del progetto;
    \item \textbf{Analisi:} Attività di analisi dei singoli rischi con l'obiettivo di evidenziare la probabilità di occorrenza, l'indice di gravità e le conseguenze che  questi potrebbero causare;
    \item \textbf{Pianificazione di Controllo:} Attività di pianificazione delle misure da adottare per cercare di impedire il verificarsi di tali rischi o per gestirli al meglio qualora si verifichino;
    \item \textbf{Monitoraggio:} Attività di controllo costante con lo scopo di evitare il verificarsi di problematiche e permettere di attuare le strategie di contenimento individuate dal team per limitarne i danni.
    %Ho visto che il prof l'ha messo nelle valutazioni ma secondo me è un punto inutile
    %\item \textbf{Attualizzazione:} Descrizione dell'approccio adottato dal gruppo quando il rischio è effettivamente in corso durante l'attività progettuale.
\end{itemize}

Abbiamo suddiviso i principali fattori di rischio nelle seguenti categorie:
\begin{itemize}
    \item \textbf{Rischi legati alle tecnologie;}
    \item \textbf{Rischi legati ai membri del gruppo;}
    \item \textbf{Rischi legati agli strumenti;}
    \item \textbf{Rischi legati all'orgranizzazione del lavoro;}
    \item \textbf{Rischi legati ai requisiti.}
\end{itemize}

\rowcolors{2}{white}{celeste}
\renewcommand{\arraystretch}{1.5}
\section{Tecnologie, framework e librerie impiegate}\label{Tecnologie}
Di seguito vengono descritte tecnologie, framework e servizi di terze parti utilizzati per il progetto.
\subsection{Swagger}
Swagger è un linguaggio di descrizione dell'interfaccia per descrivere le \glo{API} \glo{RESTful} espresse utilizzando \glo{JSON}. Swagger viene utilizzato insieme a una serie di strumenti software open source per progettare, creare, documentare e utilizzare i servizi Web RESTful.
\subsection{AWS Cognito}
AWS Cognito fornisce autenticazione, autorizzazione e gestione degli utenti per le applicazioni Web e mobili. Gli utenti possono accedere direttamente con un nome utente e una password, oppure tramite terze parti, ad esempio Facebook, Amazon, Google o Apple.
I due componenti principali di Amazon Cognito da utilizzare separatamente o insieme sono i pool di utenti, directory utente che forniscono opzioni di registrazione e di accesso, e i pool di identità che consentono di concedere agli utenti l'accesso ad altri servizi AWS.
\subsection{AWS DynamoDB} 
Amazon DynamoDB è un database \glo{NoSQL} che offre prestazioni veloci e predicibili, in particolare si tratta di un database completamente gestito che consente di scaricare gli oneri amministrativi legati al funzionamento e al ridimensionamento di un database distribuito in modo da non doversi preoccupare del provisioning, dell'installazione e della configurazione dell'hardware, della replica, delle patch del software o del ridimensionamento del cluster eliminando il carico operativo e la complessità coinvolti nella protezione dei dati sensibili. 

\subsection{AWS API Gateway}
Amazon API Gateway semplifica per gli sviluppatori la creazione, la pubblicazione, la manutenzione, il monitoraggio e la protezione delle API RESTful su qualsiasi scala. Le API fungono da “porta di entrata” per consentire l’accesso delle applicazioni ai dati, alla logica aziendale o alle funzionalità dai servizi back-end. 
API Gateway gestisce tutte le attività di accettazione ed elaborazione relative a centinaia di migliaia di chiamate API simultanee, inclusi gestione del traffico, controllo di accessi e autorizzazioni, monitoraggio e gestione delle versioni delle API. 

\subsection{AWS Lambda}
AWS Lambda è un servizio di elaborazione serverless che permette di eseguire il codice senza effettuare il provisioning o gestire i server, creare una logica di dimensionamento dei cluster in funzione dei carichi di lavoro, mantenere integrazioni degli eventi o gestire i runtime. Con Lambda è possibile eseguire codice per qualsiasi tipo di applicazione o servizio di back-end, senza alcuna amministrazione.  È possibile scrivere le funzioni Lambda nel linguaggio preferiro (Node.js, Python, Go, Java e altri ancora) e utilizzare strumenti per creare, testare e distribuire le funzioni.
\subsection{Amazon S3 Bucket}
È un servizio di storage di oggetti che offre scalabilità, disponibilità dei dati, sicurezza e prestazioni all'avanguardia nel settore offrendo caratteristiche di gestione semplici da utilizzare che consentono di organizzare i dati e di configurare controlli di accesso ottimizzati per soddisfare requisiti aziendali, di pianificazione e di conformità specifici.
\subsection{Amazon SNS}
Amazon Simple Notification Service (Amazon SNS) è un servizio di messaggistica completamente gestito per la comunicazione application-to-person (A2P) e application-to-application (A2A).
Le comunicazioni avvengono in modo asincrono, con un punto di accesso logico e un canale di comunicazione. 
\subsection{Amazon SQS}
Amazon Simple Queue Service (SQS) è un servizio di accodamento messaggi completamente gestito che consente la separazione e la scalabilità di microservizi, sistemi distribuiti e applicazioni serverless. Con SQS, è possibile inviare, memorizzare e ricevere qualsiasi volume di messaggi tra componenti software senza perdite e senza dover impiegare altri servizi per mantenere la disponibilità. 
SQS offre due tipi di code di messaggi: le code standard offrono throughput massimo, ordinamento semplificato e distribuzione di tipo at-least-once mentre le code FIFO sono progettate per garantire che i messaggi vengano elaborati esattamente una sola volta, nell'ordine in cui sono inviati.
\subsection{Stripe}
Stripe è una piattaforma esterna per i pagamenti online che consente all'e-commerce di accettare pagamenti con bancomat, carta ricaricabile o carta di credito. È una soluzione sicura e rapida, i pagamenti vengono elaborati utilizzando strumenti ad hoc sviluppati per la gestione dei flussi di pagamento con controlli anti-frode.
\subsection{NodeJs}
Node.js è un runtime system open source multipiattaforma orientato agli eventi per l'esecuzione di codice JavaScript, ha un'architettura orientata agli eventi che rende possibile l’I/O asincrono. Questo design punta ad ottimizzare il Throughput e la scalabilità nelle applicazioni web con molte operazioni di input/output. Il modello di networking su cui si basa Node.js è I/O event-driven: ciò vuol dire che Node richiede al sistema operativo di ricevere notifiche al verificarsi di determinati eventi, e rimane quindi in sleep fino alla notifica stessa: solo in tale momento torna attivo per eseguire le istruzioni previste nella funzione di callback, così chiamata perché da eseguire una volta ricevuta la notifica che il risultato dell'elaborazione del sistema operativo è disponibile.
\subsection{Npm}
Npm è un package manager per il linguaggio di programmazione JavaScript, il predefinito per l'ambiente di runtime JavaScript Node.js. Consiste in un client da linea di comando, chiamato anch'esso npm, e un database online di moduli pubblici e privati che offrono diverse funzionalità: dalla gestione dell’upload di file, ai database MySQL o a Redis, attraverso framework, sistemi di template e la gestione della comunicazione in tempo reale con i visitatori.
\subsection{TypeScript}
TypeScript è un linguaggio di programmazione open source sviluppato da Microsoft che estende la sintassi di JavaScript, aggiungendo o rendendo più flessibili e potenti varie sue caratteristiche, in modo che qualunque programma scritto in JavaScript sia anche in grado di funzionare con TypeScript senza nessuna modifica. 
\subsection{Next.js}
Next.js è un framework JavaScript back-end per applicazioni React che non richiede alcun setup e che consente il rendering automatico lato server (SSR, server side rendering).
Con Next.js si possono sviluppare applicazioni web, app mobile, desktop e web app progressive: è costruito secondo il principio di “Build once, run anywhere“.
Altre caratteristiche di Next.js sono suddivisione automatica del codice, routing automatico, hot code reloading (viene ricaricato solo il codice modificato) ed esportazione statica (con un solo comando può esportare un sito statico).
\subsection{Jest.js}
Jest.js è un framework di test JavaScript con particolare attenzione alla semplicità e al supporto per applicazioni web di grandi dimensioni.  Fornisce diverse funzionalità come la creazione automatizzata di mock, l'esecuzione di test in parallelo per aumentarne la velocità e la possibilità di testare il codice asincrono in modo sincrono. Jest trova automaticamente i test da eseguire nel codice sorgente, e funziona su progetti JS che includono React, Babel, TypeScript, Node, Angular, Vue.
\subsection{ESLint}
ESLint è uno strumento di analisi del codice statico per identificare pattern problematici o codice che non rispetta certe linee guida predefinite nel codice JavaScript senza eseguirlo. Le regole in ESLint sono configurabili e le regole personalizzate possono essere definite e caricate. ESLint è scritto usando Node.js per fornire un ambiente a runtime veloce e di facile installazione attraverso npm.
\subsection{Material-UI}
Material-UI è una libreria di componenti React per sviluppare il design del proprio progetto o per poter utilizzare tutta una serie di componenti stabili predefiniti.
\subsection{Serverless Framework}
Serverless Framework è un framework Web gratuito e open source scritto utilizzando Node.js, sviluppato per la creazione di applicazioni su AWS Lambda. Il Serverless Framework è costituito da una \glo{CLI} open source e da una dashboard che insieme, forniscono una gestione completa del ciclo di vita delle applicazioni serverless.

\newpage

\subsection{Rischi legati ai membri del gruppo}


    \begin{table}[H]
        \begin{tabular}{|c|p{10cm}|}
        \hline
        \rowcolor{darkblue}
        \multicolumn{2}{|c|}{\textcolor{white}{\textbf{RG1 - Contrasti tra i Componenti}}} \\
        \hline
         Descrizione & Il gruppo deve cooperare con professionalità nonostante la poca esperienza e non si conoscessero in precedenza, questo può causare tensioni o contrasti.\\ 
         \hline
         Conseguenze & Tensioni o contrasti rallentano e danneggiano il corretto svolgimento del progetto.\\
         \hline
         Probabilità di Occorrenza & Bassa.\\
         \hline
         Pericolosità & Alta.\\
         \hline
         Precauzioni & Ogni elemento del gruppo cercherà di limitare eventuali tensioni a favore del collettivo.\\
         \hline
         Piano di Contingenza & Il {\Responsabile} riassegnerà i compiti per limitare la vicinanza delle parti interessate, eventualmente insieme al resto del team cercherà di sanare le incomprensioni. In casi estremi verrà chiamato in causa il \VT.\\ 
         \hline
        \end{tabular}
        \caption{\label{tab:RG1}Analisi dei rischi per contrasti tra i componenti.}
    \end{table}


    \begin{table}[H]
        \begin{tabular}{|c|p{10cm}|}
        \hline
        \rowcolor{darkblue}
        \multicolumn{2}{|c|}{\textcolor{white}{\textbf{RG2 - Disponibilità dei Membri}}} \\
        \hline
         Descrizione & Ogni membro ha impegni universitari e personali oltre all'attività di progetto; possono inoltre insorgere problemi di salute e familiari che potrebbero rendere i membri improduttivi per alcuni periodi.\\ 
         \hline
         Conseguenze & Possibilità di ritardi su attività individuali o collettive.\\
         \hline
         Probabilità di Occorrenza & Media.\\
         \hline
         Pericolosità & Medio-Alta.\\
         \hline
         Precauzioni & Ogni membro del gruppo è tenuto a compilare un calendario condiviso e deve far presente agli altri componenti del team di eventuali periodi di improduttività. Il {\Responsabile} è così in grado di organizzare al meglio il lavoro.\\
         \hline
         Piano di Contingenza & In caso di mancanze prolungate che provocherebbero pesanti ritardi, il {\Responsabile} si occuperà di ridistribuire i compiti da svolgere ai restanti membri del gruppo di lavoro.\\ 
         \hline
        \end{tabular}
        \caption{\label{tab:RG2}Analisi dei rischi per disponibilità dei membri.}
    \end{table}


    \begin{table}[H]
        \begin{tabular}{|c|p{10cm}|}
        \hline
        \rowcolor{darkblue}
        \multicolumn{2}{|c|}{\textcolor{white}{\textbf{RG3 - Inesperienza Gestionale}}} \\
        \hline
         Descrizione & Il team non ha mai affrontato un progetto di tali dimensioni, nè a livello di carico di lavoro nè per grandezza del gruppo di lavoro. Inoltre, ciascun componente del gruppo non ha esperienza di lavoro che richieda il coordinamento di un gruppo di sette o più persone.\\ 
         \hline
         Conseguenze & Ogni singolo membro del team è disabituato a relazionarsi con un gruppo di persone. I componenti del gruppo non hanno familiarità con i ruoli che devono impersonare e con i compiti che è necessario svolgere portando così possibili ritardi nello sviluppo del prodotto.\\
         \hline
         Probabilità di Occorrenza & Alta.\\
         \hline
         Pericolosità & Alta.\\
         \hline
         Precauzioni & Una frequente comunicazione interna di eventuali difficoltà, specialmente rivolta al {\Responsabile}, aiuterà a mitigare l'insorgere di problemi.\\
         \hline
         Piano di Contingenza & Qualunque difficoltà sarà notificata al {\Responsabile} e affrontata con la collaborazione di tutti. Il {\Responsabile} riassegnerà ai membri più in difficoltà dei compiti più adatti alle loro competenze. Il componente in questione dovrà eseguire un’autoanalisi per comprendere le motivazioni dei suoi problemi e come migliorare.\\ 
         \hline
        \end{tabular}
        \caption{\label{tab:RG3}Analisi dei rischi per inesperienza gestionale.}
    \end{table}

    \begin{table}[H]
        \begin{tabular}{|c|p{10cm}|}
        \hline
        \rowcolor{darkblue}
        \multicolumn{2}{|c|}{\textcolor{white}{\textbf{RG4 - Scarsa Comunicazione}}} \\
        \hline
         Descrizione & Ciascun membro del team non ha molta esperienza nel lavoro di gruppo, questo può portare a una cattiva comunicazione tra i vari membri del gruppo.\\ 
         \hline
         Conseguenze & Un'insufficiente comunicazione tra i membri del gruppo può portare a ritardi nello sviluppo del progetto o addirittura a contrasti tra i componenti stessi.\\
         \hline
         Probabilità di Occorrenza & Bassa.\\
         \hline
         Pericolosità & Media.\\
         \hline
         Precauzioni & Il {\Responsabile} provvederà a monitorare e promuovere un adeguato livello di comunicazione attiva all'interno del team. Non dovranno insorgere situazioni dove, per mancata o scarsa comunicazione, ci siano ritardi nelle decisioni progettuali importanti con una conseguente insorgenza di situazioni di tensione.\\
         \hline
         Piano di Contingenza & Nel caso si rilevi una scarsa comunicazione da parte di un membro del gruppo sarà compito del {\Responsabile} ricercarne il motivo, se questo si dimostra non lecito provvederà prima ad un richiamo informale (a voce o con gli strumenti di comunicazione interni). Se il comportamento venisse reiterato il {\Responsabile} programmerà una riunione interna al gruppo per discutere la situazione.\\ 
         \hline
        \end{tabular}
        \caption{\label{tab:RG4}Analisi dei rischi per scarsa comunicazione.}
    \end{table}
\newpage

\subsection{Rischi legati agli strumenti}

\begin{table}[H]
    \begin{tabular}{|c | p{10cm}|}
    \hline
    \rowcolor{darkblue}
    \multicolumn{2}{|c|}{\textbf{RS1 - Inesperienza del Gruppo}} \\
    \hline
    Descrizione & Il gruppo fa utilizzo di software di terze parti, per cui il loromalfunzionamento non dipende dal team.\\ 
    \hline
    Conseguenze & Il loro malfunzionamento causerebbe pesanti perdite di datie ritardi.\\
    \hline
    Probabilità di Occorrenza & Bassa.\\
    \hline
    Pericolosità & Alta.\\
    \hline
    Precauzioni & Il gruppo cercherà di effettuare un backup il più frequentemente possibile in modo da minimizzare possibili danni.\\ 
    \hline
    Piano di Contingenza & Nel caso si rilevino dei malfunzionamenti da parte di serviziesterni ogni membro del gruppo lo comunicherà a tutto ilteam. Nel caso peggiore ilResponsabile di Progettosi im-pegnerà per passare ad altri software il più simili possibilea quelli utilizzati.\\ 
    \hline
    \end{tabular}
    \caption{\label{tab:RS1}Analisi dei rischi per inesperienza del gruppo.}
    
\end{table}
\newpage

\subsection{Rischi legati all'orgranizzazione}

\begin{table}[H]
    \begin{tabular}{|c | p{10cm}|}
    \hline
    \multicolumn{2}{|c|}{\textbf{RO1 - Costi delle Attività}} \\
    \hline
    Descrizione & Per ogni attività viene calcolato il costo, questo calcolo può essere non corretto (sottostimato o sovrastimato) a causa dell'inesperienza del gruppo\\ 
    \hline
    Conseguenze & Una sottostima provocherebbe ritardi nella pianificazione;una sovrastima porterebbe ad uno spreco di tempo. Entrambi i casi richiederebbero poi una nuova pinificazione delle attività\\
    \hline
    Probabilità di Occorrenza & Medio-Alta.\\
    \hline
    Pericolosità & Medio-Alta.\\
    \hline
    Precauzioni & Il gruppo cercherà di effettuare un backup il più frequentemente possibile in modo da minimizzare possibili danni.\\ 
    \hline
    Piano di Contingenza & Nel caso si rilevino dei malfunzionamenti da parte di serviziesterni ogni membro del gruppo lo comunicherà a tutto ilteam. Nel caso peggiore ilResponsabile di Progettosi im-pegnerà per passare ad altri software il più simili possibilea quelli utilizzati.\\ 
    \hline
    \end{tabular}
    \caption{\label{tab:RO1}Analisi dei rischi per i costi delle attività.}
    
\end{table}
\newpage

\subsection{Rischi legati ai requisiti}
\begin{table}[H]
    \begin{tabular}{|c|p{11.5cm}|}
    \rowcolor{darkblue} \hline
    \multicolumn{2}{|c|}{\textcolor{white}{\textbf{RR1 - Analisi dei requisiti imperfetta}}}\\ \hline
    Descrizione & Data l'inesperienza dei componenti del gruppo nell'analisi dei requisiti, è possibile una comprensione errata dei requisiti portando alla stesura di un'{\AdR} insoddisfacente o incompleta.\\ \hline
    Conseguenze & La mancata correttezza dell'{\AdR} potrebbe portare a un prodotto inadeguato o a una cattiva pianificazione seguita quindi da conseguenti perdite di tempo.\\ \hline
    Probabilità di occorrenza & Media.\\ \hline
    Pericolosità & Alta.\\ \hline
    Precauzioni & Il gruppo cercherà di stabilire una buona comunicazione con {\Proponente} cercando di chiarire ogni dubbio o, in caso fosse necessario, chiedere ulteriori spiegazioni.\\ \hline
    Piano di contingenza & Qualsiasi errore segnalato da {\Proponente} verrà corretto con la massima priorità.\\ \hline
    \end{tabular}
    \caption{\label{tab:RR1}Analisi dei rischi per analisi dei requisiti imperfetta.}
\end{table}

\begin{table}[H]
    \begin{tabular}{|c|p{11.5cm}|}
    \rowcolor{darkblue} \hline
    \multicolumn{2}{|c|}{\textcolor{white}{\textbf{RR2 - Modifica dei requisiti}}}\\ \hline
    Descrizione & Nonostante {\Proponente} sia stata chiara nello stilare i requisiti, può accadere che decida di cambiarli in corso d'opera.\\ \hline
    Conseguenze & In caso di modifica parziale o di requisiti opzionali il danno al progetto sarebbe contenuto e facilmente risolvibile con eventuali lievi ritardi; in caso di cambiamenti importanti il rischio è di pesanti ritardi e perdita del lavoro già svolto fino a quel momento.\\ \hline
    Probabilità di occorrenza & Bassa.\\ \hline
    Pericolosità & Alta.\\ \hline
    Precauzioni & Il gruppo cercherà il più possibile di avere degli incontri con il proponente e ognuno di questi sarà verbalizzato.\\ \hline
    Piano di contingenza & Nel caso di piccoli cambiamenti saranno attuati il più velocemente possibile dal team; ridimensionamenti pesanti saranno discussi con il proponente per trovare un possibile accordo.\\ \hline
    \end{tabular}
    \caption{\label{tab:RR2}Analisi dei rischi per modifica dei requisiti.}
\end{table}
\newpage
