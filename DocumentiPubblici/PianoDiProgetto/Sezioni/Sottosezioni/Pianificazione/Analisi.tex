\subsection{Analisi}
\label{analisi}
Nella fase di analisi, si forma il gruppo che si prepara alla \textit{Revisione dei Requisiti} studiando i capitolati proposti e svolgendo un'analisi preliminare dei requisiti e la pianificazione globale del lavoro da svolgere.
\textbf{Periodo:} dal 2020-11-26 al 2021-01-14\\

Le precondizioni sono:
\begin{itemize}
    \item Formazione del gruppo;
    \item Presentazione dei Capitolati.
\end{itemize}

Le postcondizioni sono:
\begin{itemize}
    \item Scelta del nome e del logo del gruppo;
    \item Creazione della mail e di una repository GitHub;
    \item Scelta del capitolato;
    \item Redazione dei documenti quali: \textit{\SdF}, \textit{\NdP}, \textit{\PdP}, \textit{Glossario}, \textit{Lettera di Presentazione}, \textit{\PdQ}, \textit{\AdR} ed i verbali;
    \item Verifica e approvazione di quanto redatto.
\end{itemize}

Questa fase è composta da sette attività che corrispondono ai documenti prodotti:
\begin{itemize}
    \item \textbf{\textit{\SdF}}: Vengono analizzati i vari capitolati per evidenziarne gli aspetti
negativi e positivi. Dopo un periodo di studio e confronto {\Gruppo} dà la preferenza in quello più adatto. L'attività è bloccante per l'\textit{\AdR};

    \item \textbf{\textit{\NdP}}: Vengono definite tutte le regole che il gruppo {\Gruppo} dovrà seguire per la stesura dei documenti e per lo sviluppo del progetto;

    \item \textbf{\textit{Glossario}}: Contiene tutti i termini che possono risultare ambigui durante lo svolgimento del progetto; di essi viene fornita una definizione sintetica ed esaustiva;

    \item \textbf{\textit{Lettera di Presentazione}}: Documento in cui il gruppo {\Gruppo} si candida al capitolato scelto come fornitore del prodotto software richiesto;

    \item \textbf{\textit{\PdP}}: Nel presente documento sono descritte le attività, i rischi del progetto e viene calcolao il preventivo per la realizzazione di progetto. Presenta poi la suddivisione del lavoro tra i membri del gruppo {\Gruppo} e il calcolo del preventivo;

    \item \textbf{\textit{\AdR}}: Vengono studiati e analizzati nel dettaglio i requisiti del capitolato scelto nello \textit{\SdF};

    \item \textbf{\textit{\PdQ}}: Si individuano metodi e procedure per garantire la qualità del prodotto.
\end{itemize}

La pianificazione di questa fase è stata organizzata nelle seguenti sotto-fasi:
\begin{enumerate}
\item \textbf{2020-11-26 - 2020-12-10}:
Inizio della stesura delle \textit{\NdP} per avere delle regole condivise dal gruppo per svolgere il lavoro. Durante la stesura di queste, il gruppo si confronta riguardo ai vari capitolati esponendo le preferenze personali di ogni membro del \glo{team}. Viene analizzato ogni capitolato focalizzando l'attenzione principalmente verso i loro aspetti positivi e negativi e dei possibili rischi che si possono incontrare per ognuno. Dopo gli iniziali confronti si è potuto iniziare uno \textit{\SdF} generico per tutti i capitolati per poi approfondirlo per i capitolati che suscitavano più interesse al gruppo e iniziare a stilare il \textit{Glossario} per la registrazione dei termini, usati nei documenti, che potrebbero creare ambiguità. Durante questo periodo sono state prese decisioni tecniche e organizzative come: nome del gruppo, logo, indirizzo mail, creazione repository e i vari mezzi di comunicazione. Vengono trascritti i verbali interni relativi alle riunioni del gruppo durante questo arco di tempo;
\item \textbf{2020-12-10 - 2020-12-22}:
Inizio della stesura del \textit{\PdP}, contenente la pianificazione del lavoro da svolgere e la suddivisione dei ruoli tra i membri del gruppo. Stesa una prima bozza della \textit{Lettera di Presentazione}, il cui completamento sarà fatto dopo la conclusione \textit{\PdP} con l'aggiunta del prospetto economico finale. Integrato il \textit{\Glossario} quando necessario. In data 2020-12-22 il gruppo ha fissato una milestone per il completamento dei documenti iniziati in queste prime fasi, ad eccezione dell'ultima sezione del \textit{\PdP} destinata al consuntivo finale della fase successiva. Iniziano le attività di verifica incrementale per i documenti in corso di stesura. Stesi i verbali interni relativi agli incontri svoltesi durante questa sotto-fase;
\item \textbf{2020-12-22 - 2021-01-12}:
    Inizio della stesura dell'\textit{\AdR} e del \textit{\PdQ}, con l'esposizione dei criteri di valutazione della qualità scelti dal gruppo e le rispettive metriche di calcolo. Continuano le attività di verifica incrementale per i documenti in corso di stesura;
\item \textbf{2021-01-12 - 2021-01-14}
    Il gruppo svolge le attività di verifica sugli ultimi documenti, completa il \textit{\Glossario} e uniforma tutti i prodotti alle regole stabilite nelle \textit{\NdP} se necessario.
\end{enumerate}

\subsubsection{Diagramma di Gantt: Analisi}

\begin{figure}[ht]
    \centering
    \includegraphics[width=\textwidth]{../../Immagini/Analisi}
    \caption{Diagramma di Gantt dell'avvitià di Analisi}
\end{figure}
