\subsection{Progettazione Architetturale}
\label{progettazione_architetturale}
\textbf{Periodo:} dal 2020-01-18 al 2020-03-08
\\La fase inizia appena conclusa la precedente e termina con la Revisione di Progettazione.
\\Le precondizioni sono:

\begin{itemize}
    \item Le \glo{postcondizioni} della fase precedente sono state soddisfatte;
    \item La candidatura del gruppo al progetto {\NomeProgetto} è stata accolta.
\end{itemize}
    Le postcondizioni sono:
\begin{itemize}
    \item Aggiornamento e correzione dei documenti già prodotti;
    \item Produzione del \textit{Proof of Concept} e dell'Allegato Tecnico;
    \item Completamento della progettazione ad alto livello del software;
    \item Consegna dei documenti richiesti in entrata alla \textit{Revisione di Progettazione};
    \item Ultimata preparazione della presentazione da esporre in sede di revisione.
\end{itemize}

Le attività da svolgere durante il periodo sono:
\begin{itemize}
    \item \textbf{Incremento e Verifica:} i documenti già prodotti vengono migliorati e aggiornati se necessario (\textit{\NdP}, \textit{\PdP}, \textit{Glossario},\textit{\PdQ}, \textit{\AdR});
    \item \textbf{Technology Baseline:} viene fatta un'analisi ad alto livello del software e viene redatto l'Allegato Tecnico dove vengono individuati i design pattern che verranno adottati per lo sviluppo. Infine viene codificato il \textit{Proof of Concept}, il quale viene presentato o condiviso al committente e proponente in una data da definirsi.
\end{itemize}

\subsubsection{Diagramma di Gantt: Progettazione Architetturale}
