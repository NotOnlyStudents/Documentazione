\subsection{Rischi legati ai membri del gruppo}
\begin{table}[H]
	\begin{tabular}{|c|p{10cm}|}
	\rowcolor{darkblue} \hline
	\multicolumn{2}{|c|}{\textcolor{white}{\textbf{RG1 - Contrasti tra i componenti}}}\\ \hline
	 Descrizione & Il gruppo deve cooperare con professionalità nonostante la poca esperienza e la mancata conoscenza precedente, questo può causare tensioni o contrasti.\\ \hline
	 Conseguenze & Tensioni o contrasti rallentano e danneggiano il corretto svolgimento del progetto.\\ \hline
	 Probabilità di occorrenza & Bassa.\\ \hline
	 Pericolosità & Alta.\\ \hline
	 Precauzioni & Ogni elemento del gruppo cercherà di limitare eventuali tensioni a favore del collettivo.\\ \hline
	 Piano di contingenza & Il {\Responsabile} riassegnerà i compiti per limitare la vicinanza delle parti interessate, eventualmente insieme al resto del team cercherà di sanare le incomprensioni. In casi estremi verrà chiamato in causa il \VT{}.\\ \hline
	\end{tabular}
	\caption{\label{tab:RG1}Analisi dei rischi per contrasti tra i componenti.}
\end{table}

\begin{table}[H]
    \begin{tabular}{|c|p{11.5cm}|}
    \rowcolor{darkblue} \hline
    \multicolumn{2}{|c|}{\textcolor{white}{\textbf{RG2 - Disponibilità dei membri}}}\\ \hline
     Descrizione & Ogni membro ha impegni universitari e personali oltre all'attività di progetto; possono inoltre insorgere problemi di salute e familiari che potrebbero rendere i membri improduttivi per alcuni periodi.\\ \hline
     Conseguenze & Possibilità di ritardi su attività individuali o collettive.\\ \hline
     Probabilità di occorrenza & Media.\\ \hline
     Pericolosità & Medio-Alta.\\ \hline
     Precauzioni & Ogni membro del gruppo è tenuto a compilare un calendario condiviso e deve far presente agli altri componenti del team di eventuali periodi di improduttività. Il {\Responsabile} è così in grado di organizzare al meglio il lavoro.\\ \hline
     Piano di contingenza & In caso di mancanze prolungate che provocherebbero pesanti ritardi, il {\Responsabile} si occuperà di ridistribuire i compiti da svolgere ai restanti membri del gruppo di lavoro.\\ \hline
    \end{tabular}
    \caption{\label{tab:RG2}Analisi dei rischi per disponibilità dei membri.}
\end{table}

\begin{table}[H]
    \begin{tabular}{|c|p{11.5cm}|}
    \rowcolor{darkblue} \hline
    \multicolumn{2}{|c|}{\textcolor{white}{\textbf{RG3 - Inesperienza gestionale}}}\\ \hline
     Descrizione & Il team non ha mai affrontato un progetto di tali dimensioni, nè a livello di carico di lavoro nè per grandezza del gruppo di lavoro. Inoltre, ciascun componente del gruppo non ha esperienza di lavoro che richieda il coordinamento di un gruppo di sette o più persone.\\ \hline
     Conseguenze & Ogni singolo membro del team è disabituato a relazionarsi con un gruppo di persone. I componenti del gruppo non hanno familiarità con i ruoli che devono impersonare e con i compiti che è necessario svolgere portando così possibili ritardi nello sviluppo del prodotto.\\ \hline
     Probabilità di occorrenza & Alta.\\ \hline
     Pericolosità & Alta.\\ \hline
     Precauzioni & Una frequente comunicazione interna di eventuali difficoltà, specialmente rivolta al {\Responsabile}, aiuterà a mitigare l'insorgere di problemi.\\ \hline
     Piano di contingenza & Qualunque difficoltà sarà notificata al {\Responsabile} e affrontata con la collaborazione di tutti. Il {\Responsabile} riassegnerà ai membri più in difficoltà dei compiti più adatti alle loro competenze. Il componente in questione dovrà eseguire un’autoanalisi per comprendere le motivazioni dei suoi problemi e come migliorare.\\ \hline
    \end{tabular}
    \caption{\label{tab:RG3}Analisi dei rischi per inesperienza gestionale.}
\end{table}

\begin{table}[H]
    \begin{tabular}{|c|p{11.5cm}|}
    \rowcolor{darkblue} \hline
    \multicolumn{2}{|c|}{\textcolor{white}{\textbf{RG4 - Scarsa comunicazione}}}\\ \hline
     Descrizione & Ciascun membro del team non ha molta esperienza nel lavoro di gruppo, questo può portare a una cattiva comunicazione tra i vari membri del gruppo.\\ \hline
     Conseguenze & Un'insufficiente comunicazione tra i membri del gruppo può portare a ritardi nello sviluppo del progetto o addirittura a contrasti tra i componenti stessi.\\ \hline
     Probabilità di occorrenza & Bassa.\\ \hline
     Pericolosità & Media.\\ \hline
     Precauzioni & Il {\Responsabile} provvederà a monitorare e promuovere un adeguato livello di comunicazione attiva all'interno del team. Non dovranno insorgere situazioni dove, per mancata o scarsa comunicazione, ci siano ritardi nelle decisioni progettuali importanti con una conseguente insorgenza di situazioni di tensione.\\ \hline
     Piano di contingenza & Nel caso si rilevi una scarsa comunicazione da parte di un membro del gruppo sarà compito del {\Responsabile} ricercarne il motivo, se questo si dimostra non lecito provvederà prima ad un richiamo informale (a voce o con gli strumenti di comunicazione interni). Se il comportamento venisse reiterato il {\Responsabile} programmerà una riunione interna al gruppo per discutere della situazione.\\ \hline
    \end{tabular}
    \caption{\label{tab:RG4}Analisi dei rischi per scarsa comunicazione.}
\end{table}