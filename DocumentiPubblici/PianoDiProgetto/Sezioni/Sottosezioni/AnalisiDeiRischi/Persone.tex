\subsection{Rischi legati ai membri del gruppo}


    \begin{table}[H]
        \begin{tabular}{|c|p{10cm}|}
        \hline
        \rowcolor{darkblue}
        \multicolumn{2}{|c|}{\textcolor{white}{\textbf{RG1 - Contrasti tra i Componenti}}} \\
        \hline
         Descrizione & Il gruppo deve cooperare con professionalità nonostante la poca esperienza e non si conoscessero in precedenza, questo può causare tensioni o contrasti.\\ 
         \hline
         Conseguenze & Tensioni o contrasti rallentano e danneggiano il corretto svolgimento del progetto.\\
         \hline
         Probabilità di Occorrenza & Bassa.\\
         \hline
         Pericolosità & Alta.\\
         \hline
         Precauzioni & Ogni elemento del gruppo cercherà di limitare eventuali tensioni a favore del collettivo.\\
         \hline
         Piano di Contingenza & Il {\Responsabile} riassegnerà i compiti per limitare la vicinanza delle parti interessate, eventualmente insieme al resto del \glo{team} cercherà di sanare le incomprensioni.  In casi estremi verrà chiamato in causa il \VT.\\ 
         \hline
        \end{tabular}
        \caption{\label{tab:RG1}Analisi dei rischi per contrasti tra i componenti.}
    \end{table}


    \begin{table}[H]
        \begin{tabular}{|c|p{10cm}|}
        \hline
        \rowcolor{darkblue}
        \multicolumn{2}{|c|}{\textcolor{white}{\textbf{RG2 - Disponibilità dei Membri}}} \\
        \hline
         Descrizione & Ogni membro ha impegni universitari e personali oltre all’attività di progetto; possono inoltre insorgere problemi di salute e familiari che potrebbero rendere i membri improduttivi per alcuni periodi.\\ 
         \hline
         Conseguenze & Possibilità di ritardi su attività individuali o collettive.\\
         \hline
         Probabilità di Occorrenza & Media.\\
         \hline
         Pericolosità & Medio-Alta.\\
         \hline
         Precauzioni & Ogni membro del gruppo è tenuto a compilare un calendario condiviso e deve far presente agli altri componenti del team di eventuali periodi di improduttività. Il {\Responsabile} è così in grado di organizzare al meglio il lavoro.\\
         \hline
         Piano di Contingenza & In caso di mancanze prolungate che provocherebbero pesanti ritardi, il {\Responsabile} si occuperà di ridistribuire i compiti da svolgere ai restanti membri del gruppo di lavoro.\\ 
         \hline
        \end{tabular}
        \caption{\label{tab:RG2}Analisi dei rischi per disponibilità dei membri.}
    \end{table}


    \begin{table}[H]
        \begin{tabular}{|c|p{10cm}|}
        \hline
        \rowcolor{darkblue}
        \multicolumn{2}{|c|}{\textcolor{white}{\textbf{RG3 - Inesperienza Gestionale}}} \\
        \hline
         Descrizione & Il team non ha mai affrontato un progetto di tali dimensioni, nè a livello di carico di lavoro nè per grandezza del gruppo di lavoro. Inoltre, ciascun componente del gruppo non ha esperienza di lavoro che richieda il coordinamento di un gruppo di sette o più persone.\\ 
         \hline
         Conseguenze & Ogni singolo membro del team è disabituato a relazionarsi con un gruppo di persone. I membri del gruppo non hanno familiarità con i ruoli che devono impersonare e con i compiti che devono svolgere portando così possibili ritardi nello sviluppo del prodotto.\\
         \hline
         Probabilità di Occorrenza & Alta.\\
         \hline
         Pericolosità & Alta.\\
         \hline
         Precauzioni & Una frequente comunicazione interna di eventuali difficoltà, specialmente rivolta al {\Responsabile}, aiuterà a mitigare l'insorgere di problemi.\\
         \hline
         Piano di Contingenza & Qualunque difficoltà sarà notificata al {\Responsabile} e affrontata con la collaborazione di tutti. Il {\Responsabile} riassegnerà ai membri più in difficoltà dei compiti  più adatti alle loro competenze. Il componente in questione dovrà eseguire un’autoanalisi per comprendere le motivazioni dei suoi problemi e come migliorare.\\ 
         \hline
        \end{tabular}
        \caption{\label{tab:RG3}Analisi dei rischi per inesperienza gestionale.}
    \end{table}

    \begin{table}[H]
        \begin{tabular}{|c|p{10cm}|}
        \hline
        \rowcolor{darkblue}
        \multicolumn{2}{|c|}{\textcolor{white}{\textbf{RG4 - Scarsa Comunicazione}}} \\
        \hline
         Descrizione & Ciascun membro del gruppo non ha molta esperienza nel \glo{teamwork}, questo può portare a una cattiva comunicazione tra i membri del gruppo.\\ 
         \hline
         Conseguenze & Un'insufficiente comunicazione tra i membri del gruppo può portare a ritardi nello sviluppo del progetto o addirittura a contrasti tra i componeneti stessi.\\
         \hline
         Probabilità di Occorrenza & Bassa.\\
         \hline
         Pericolosità & Media.\\
         \hline
         Precauzioni & Il {\Responsabile} provvederà a monitorare e promuovere un adeguato livello di comunicazione attiva all'interno del team. Non dovranno insorgere situazioni dove, per mancata o scarsa comunicazione, ci siano ritardi nelle decisioni progettuali importanti  con una conseguente insorgenza di situazioni di tensione.\\
         \hline
         Piano di Contingenza & Nel caso si rilevi una scarsa comunicazione da parte di un membro del gruppo sarà compito del {\Responsabile} ricercarne il motivo, se questo si dimostra non lecito provvederà prima ad un richiamo informale (a voce o con gli strumenti di comunicazione interni). Se il comportamento venisse reiterato il {\Responsabile} programmerà una riunione interna al gruppo per discutere la situazione.\\ 
         \hline
        \end{tabular}
        \caption{\label{tab:RG4}Analisi dei rischi per scarsa comunicazione.}
    \end{table}