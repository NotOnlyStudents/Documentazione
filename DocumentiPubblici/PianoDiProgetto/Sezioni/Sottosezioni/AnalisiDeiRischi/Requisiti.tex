\subsection{Rischi legati ai Requisiti}

\begin{table}[H]
    \begin{tabular}{|c | p{10cm}|}
    \hline
    \rowcolor{darkblue}
    \multicolumn{2}{|c|}{\textcolor{white}{\textbf{RR1 - Analisi dei Requisiti Imperfetta}}} \\
    \hline
    Descrizione & Data l'inesperienza dei componenti del gruppo nell'analisi dei requisiti, è possibile una comprensione errata dei requisiti portando alla stesura di un' \textit{\AdR}   insoddisfacente o incompleta.\\ 
    \hline
    Conseguenze & La mancata correttezza dell'\textit{\AdR} potrebbe portare a un prodotto inadeguato o a una cattiva pianificazione seguita quindi da conseguenti perdite di tempo.\\
    \hline
    Probabilità di Occorrenza & Media.\\
    \hline
    Pericolosità & Alta.\\
    \hline
    Precauzioni & Il gruppo cercherà di stabilire una buona comunicazione con {\Proponente} cercando di chiarire ogni dubbio o, in caso fosse necessario, chiedere ulteriori spiegazioni.\\ 
    \hline
    Piano di Contingenza & Qualsiasi errore segnalato da {\Proponente} verrà corretto con la massima priorità.\\ 
    \hline
    \end{tabular}
    \caption{\label{tab:RR1}Analisi dei rischi per Analisi dei requisiti imperfetta.}
    
\end{table}


\begin{table}[H]
    \begin{tabular}{|c | p{10cm}|}
    \hline
    \rowcolor{darkblue}
    \multicolumn{2}{|c|}{\textcolor{white}{\textbf{RR2 - Modifica dei Requisiti}}} \\
    \hline
    Descrizione & Nonostante {\Proponente} sia stata chiara nello stilare i requisiti, può accadere che decida di cambiarli in corso d'opera.\\ 
    \hline
    Conseguenze & In caso di modifica parziale o di requisiti opzionali il danno al progetto sarebbe contenuto e facilmente risolvibile con eventuali lievi ritardi; in caso di cambiamenti importanti il rischio è di pesanti ritardi e perdita del lavoro già svolto fino a quel momento.\\
    \hline
    Probabilità di Occorrenza & Bassa.\\
    \hline
    Pericolosità & Alta.\\
    \hline
    Precauzioni & Il gruppo cercherà il più possibile di avere degli incontri con il proponente e ognuno di questi sarà verbalizzato.\\ 
    \hline
    Piano di Contingenza & Nel caso di piccoli cambiamenti saranno attuati il più velocemente possibile dal \glo{team}; ridimensionamenti pesanti saranno discussi con il proponente per trovare un possibile accordo.\\ 
    \hline
    \end{tabular}
    \caption{\label{tab:RR2}Analisi dei rischi per Modifica dei Requisiti.}
    
\end{table}