\subsection{Rischi legati all'orgranizzazione}

\begin{table}[H]
    \begin{tabular}{|c | p{10cm}|}
    \hline
    \rowcolor{darkblue}
    \multicolumn{2}{|c|}{\textcolor{white}{\textbf{RO1 - Costi delle Attività}}} \\
    \hline
    Descrizione & Data l'inesperienza del team nella pianificazione di progetto e la messa in pratica della stessa su un arco di tempo medio-lungo, può verificarsi una sottostima o una sovrastima dei costi e dei tempi necessari alla realizzazione del progetto.\\ 
    \hline
    Conseguenze & Una sottostima provocherebbe ritardi nella pianificazione; una sovrastima porterebbe ad uno spreco di tempo. Entrambi i casi richiederebbero poi una nuova pianificazione delle attività.\\
    \hline
    Probabilità di Occorrenza & Medio-Alta.\\
    \hline
    Pericolosità & Alta.\\
    \hline
    Precauzioni & Ogni componente del gruppo controllerà periodicamente lo stato della propria attività rispetto alla tabella di marcia imposta nel documento \textit{\PdP}.\\ 
    \hline
    Piano di Contingenza & Per ogni attività è previsto un tempo di \glo{slack}, in modo da arginare il più possibile il ritardo nell'avanzamento del progetto.\\ 
    \hline
    \end{tabular}
    \caption{\label{tab:RO1}Analisi dei rischi per i costi delle attività.}
    
\end{table}