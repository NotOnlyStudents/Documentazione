\subsection{Rischi legati all'organizzazione del lavoro}
\begin{table}[H]
    \begin{tabular}{|c|p{11.5cm}|}
    \rowcolor{darkblue} \hline
    \multicolumn{2}{|c|}{\textcolor{white}{\textbf{RO1 - Costi delle attività}}}\\ \hline
    Descrizione & Data l'inesperienza del team nella pianificazione di progetto e la messa in pratica della stessa su un arco di tempo medio-lungo, può verificarsi una sottostima o una sovrastima dei costi e dei tempi necessari alla realizzazione del progetto.\\ \hline
    Conseguenze & Una sottostima provocherebbe ritardi nella pianificazione; una sovrastima porterebbe ad uno spreco di tempo. Entrambi i casi richiederebbero poi una nuova pianificazione delle attività.\\ \hline
    Probabilità di occorrenza & Medio-Alta.\\ \hline
    Pericolosità & Alta.\\ \hline
    Precauzioni & Ogni componente del gruppo controllerà periodicamente lo stato della propria attività rispetto alla tabella di marcia imposta nel documento {\PdP}.\\ \hline
    Piano di contingenza & Per ogni attività è previsto un tempo di \glo{slack}, in modo da arginare il più possibile il ritardo nell'avanzamento del progetto.\\ \hline
    \end{tabular}
    \caption{\label{tab:RO1}Analisi dei rischi per i costi delle attività.}
\end{table}
\begin{table}[H]
	\begin{tabular}{|c|p{11.5cm}|}
		\rowcolor{darkblue} \hline
		\multicolumn{2}{|c|}{\textcolor{white}{\textbf{RO2 - Scadenze}}}\\ \hline
		Descrizione & Per svolgere in modo corretto ed \glo{efficiente} un progetto occorre suddividere il lavoro tra i membri del gruppo e questo porta a fissare delle scadenze per permettere ad ognuno di svolgere i propri compiti.\\ \hline
		Conseguenze & Se le scadenze assegnate sono troppo stringenti si aumenta il rischio di insuccesso mentre avendo delle scadenze troppo lasche si rallenterebbe notevolmente il lavoro.\\ \hline
		Probabilità di occorrenza & Media.\\ \hline
		Pericolosità & Media.\\ \hline
		Precauzioni & Ad ogni incontro, tutti i membri del gruppo devono comunicare eventuali difficoltà incontrate nello svolgimento dei compiti assegnati. I componenti con più disponibilità possono aiutare quelli in difficoltà così da poter terminare i compiti assegnati entro la scadenza prefissata.\\ \hline
		Piano di contingenza & I compiti vengono assegnati in base alle conoscenze e all'esperienza di ogni componente cercando in questo modo di evitare il più possibile il ritardo nell'avanzamento del progetto.\\ \hline
	\end{tabular}
	\caption{\label{tab:RO2}Analisi dei rischi per le scadenze.}
\end{table}