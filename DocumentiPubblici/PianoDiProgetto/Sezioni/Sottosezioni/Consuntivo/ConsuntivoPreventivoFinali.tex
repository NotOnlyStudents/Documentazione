\subsection{Consuntivo complessivo} \label{ConsuntivoComplessivo}
\subsubsection{Prospetto orario}
Di seguito si riporta la tabella che descrive il totale delle ore di lavoro rendicontate svolte da ciascun componente per la realizzazione del progetto dal periodo di Progettazione architetturale compreso.
\begin{table}[H]
	\begin{center}
		\begin{tabular}{ |c c c c c c c c| }
			\rowcolor{darkblue} 
			\textcolor{white}{\textbf{Nominativo}} & \textcolor{white}{\textbf{Re}} & \textcolor{white}{\textbf{Am}} & \textcolor{white}{\textbf{An}} & \textcolor{white}{\textbf{Pt}} & \textcolor{white}{\textbf{Pr}} & \textcolor{white}{\textbf{Ve}} & \textcolor{white}{\textbf{Ore}} \\ \hline
			\BL 	& 4  	& 2(-2)  	& 8 	& 32(+1) 	& 27(+3) 	& 27 	& 100(+2) \\ \hline
			\FF 	& 6 	& 4 	& 2(+2)		& 46 	& 29 	& 12(-1) 	& 100(+1) \\ \hline
			\MM 	& 4  	& 9(-2)  	& 14(+4) 	& 23 	& 24 	& 25(-1)  	& 99(+1) \\ \hline
			\PC 	& 9 	& 4  	& 10	& 13(+2) 	& 28	& 36 	& 100(+2) \\ \hline
			\TG 	& 5  	& 8(-1) 	& 12(+2) 	& 19 	& 27 	& 28 	& 99(+1) \\ \hline
			\TL 	& 5  	& 6 	& 6(+6) 	& 19 	& 30 	& 34(-4) 	& 100(+2) \\ \hline
			\VD 	& 13(-4)  	& 12  	& 5(+5)		& 24 	& 23 	& 22	& 99(+1) \\ \hline
			\textbf{Ore totali} & \textbf{46(-4)} & \textbf{45(-5)} & \textbf{57(+19)} & \textbf{176(+3)} & \textbf{186(+3)} & \textbf{186(-6)} & \textbf{696(+10)} \\ \hline
		\end{tabular}
		\caption{Distribuzione delle ore rendicontate}
	\end{center}
\end{table}
\subsubsection{Prospetto economico}
Il costo totale rendicontato per ogni ruolo è il seguente:
\begin{table}[H]
	\begin{center}
		\begin{tabular}{ |c c c| }
			\rowcolor{darkblue} 
			\textcolor{white}{\textbf{Ruolo}} & \textcolor{white}{\textbf{Ore}} & \textcolor{white}{\textbf{Costo in €}}\\ \hline
			{\Responsabile} 			& 46	& 1380 \\ \hline
			{\Amministratore}			& 45	& 900 \\ \hline
			\textit{Analista} 			& 57	& 1425\\ \hline
			\textit{Progettista} 		& 176	& 3872 \\ \hline
			\textit{Programmatore} 		& 186 	& 2790 \\ \hline
			\textit{Verificatore} 		& 186	& 2790 \\ \hline
			\textbf{Totale} & \textbf{696} & \textbf{13157} \\ \hline
		\end{tabular}
		\caption{Prospetto delle ore rendicontate}
	\end{center}
\end{table}
\subsubsection{Conclusioni}
Valutando i dati ottenuti dalle precedenti tabelle, è possibile affermare che il ruolo in cui si sono presentati più problemi e che ha causato più ritardi è il ruolo di \textit{Analista}.
Il ruolo in questione rappresenta uno dei ruoli fondamentali per la realizzazione del progetto in quanto il suo corretto svolgimento determina cosa sarà possibile svolgere nella piattaforma realizzata. La nostra poca esperienza ha causato diversi ritardi che non erano stati previsti e causati principalmente durante il periodo di progettazione architetturale. Il documento di \AdRv{} presentava gravi errori, da noi individuati e\_o segnalati dopo la \glo{RR}, che dovevano essere sanati velocemente per poter procedere correttamente con la progettazione di \NomeProgetto\, per questo motivo è stato svolto un importante lavoro di ristrutturazione del documento che ha impegnato diversi componenti del gruppo e di conseguenza causato dei ritardi e lo svolgimento di ore di lavoro non preventivate. Proprio per questi importanti errori derivanti dalla nostra inesperienza, è stato necessario rimodulare il volume di impiego orario nei periodi successivi, in particolare per il periodo di validazione e collaudo, permettendo così al gruppo di rispettare il preventivo concordato con la \glo{RR}. 
La pianificazione delle ore nei periodi di progettazione di dettaglio e validazione è stata rispettata grazie al maggiore controllo effettuato sullo svolgimento delle attività. Per quanto riguarda le ore di lavoro svolte da ciascun componente rispetto a quanto preventivato dopo la rimodulazione delle ore del periodo di validazione, 98 ore ciascuno, ogni componente ha svolto 1/2 ore in più rispetto a quanto previsto, quindi in linea generale tutti i membri hanno partecipato equamente allo svolgimento del progetto.
\subsection{Preventivo a finire}\label{Paf}
Viene qui mostrata una tabella contenente l'attuale preventivo a finire. I valori relativi all'analisi e al consolidamento dei requisiti sono stati inseriti puramente a scopo riassuntivo, questi infatti non saranno conteggiati nel calcolo delle ore rendicontate. Se il valore del consuntivo non fosse ancora presente, verrà usato il valore del preventivo.
\begin{table}[H]
	\centering
	\begin{tabular}{|c|c|c|}
		\rowcolor{darkblue} 
		\textcolor{white}{Periodo}		&\textcolor{white}{Preventivo €}&	\textcolor{white}{Consuntivo €}\\ \hline
		Analisi							&	4575,00						&	4595,00 \\ \hline
		Consolidamento dei requisiti	&	954,00						&	954,00 \\ \hline
		\rowcolor{darkblue} \multicolumn{3}{|c|}{\textcolor{white}{Rendicontato}}\\ \hline
		Consolidamento delle tecnologie	&	4327,00						&	4503,00 \\ \hline
		Progettazione e codifica		&	6338,00						&	6438,00 \\ \hline
		Validazione e collaudo			&	2498,00						&	2216,00 \\ \hline
		\rowcolor{darkblue}				&\textcolor{white}{Preventivo €}&	\textcolor{white}{Preventivo a finire €}\\ \hline
		Totale							&	18692,00					&	18706,00 \\ \hline
		Rendicontato					&	13163,00					&	13157,00 \\ \hline
	\end{tabular}
	\caption{Preventivo a finire}
\end{table}