\subsection{Periodo di progettazione architetturale}\label{ConsuntivoPArchitetturale}
\subsubsection{Consuntivo di periodo}
Di seguito è presente la tabella contenente i dati del consuntivo per il periodo di progettazione architetturale.
\begin{table}[H]
	\centering
	\begin{tabular}{|c|c|c|c|c|}
		\rowcolor{darkblue} 
		&\multicolumn{2}{c|}{\textcolor{white}{Ore}}&\multicolumn{2}{c|}{\textcolor{white}{Costo in €}}\\ \hline
		Ruolo			&	Preventivo				&	Consuntivo		&	Preventivo	&	Consuntivo\\ \hline
		{\Responsabile}		&	18					&	18				&	540,00		&	540,00 \\ \hline
		{\Amministratore}	&	20					&	20				&	400,00		&	400,00 \\ \hline
		\textit{Analista}	&	45(+7)					&	38				&	1125,00(+175)		&	1125,00 \\ \hline
		\textit{Progettista}& 	59(-2)					&	61 				& 	1298,00(-44)		&  	1298,00 \\ \hline
		\textit{Programmatore}& 21					& 	21				& 	315			&  	315 \\ \hline
		\textit{Verificatore}&	47(-5)					&	52				&	705,00(-75)		&	705,00 \\ \hline
		Totale				&	210					&	210				&	4383,00(+56)		&	4383,00 \\ \hline
		Differenza			& 	\multicolumn{2}{c|}{-} 			&\multicolumn{2}{c|}{-}\\ \hline
	\end{tabular}
	\caption{Prospetto orario ed economico a consuntivo nella periodo di progettazione architetturale}
\end{table}
\subsubsection{Specifica dei consuntivi}
Come per i preventivi si riportano i consuntivi suddivisi per le attività individuate nella sezione \S\ref{progettazione_architetturale}
\myparagraph{Incremento e verifica documentazione}
Di seguito è presente la tabella contenente i dati del consuntivo dal 2021\_02\_01 al 2021\_02\_11.
\begin{table}[H]
	\centering
	\begin{tabular}{|c|c|c|c|c|}
		\rowcolor{darkblue} 
		&\multicolumn{2}{c|}{\textcolor{white}{Ore}}&\multicolumn{2}{c|}{\textcolor{white}{Costo in €}}\\ \hline
		Ruolo			&	Preventivo				&	Consuntivo		&	Preventivo	&	Consuntivo\\ \hline
		{\Responsabile}		&	9					&	9				&	270,00		&	270,00 \\ \hline
		{\Amministratore}	&	13					&	13				&	260,00		&	260,00 \\ \hline
		\textit{Analista}	&	23					&	29(+6)			&	575,00		&	725,00(+150) \\ \hline
		\textit{Progettista}& 	-					&	- 				& 	-		    &  	- \\ \hline
		\textit{Programmatore}& -					& 	-				& 	-			&  	- \\ \hline
		\textit{Verificatore}&	18					&	14(-4)			&	270,00		&	210,00(-60) \\ \hline
		Totale				&	63					&	65				&	1375,00		&	1465,00 \\ \hline
		Differenza			& 	\multicolumn{2}{c|}{-2} 			&\multicolumn{2}{c|}{-90}\\ \hline
	\end{tabular}
	\caption{Prospetto orario ed economico a consuntivo dal 2021\_02\_01 al 2021\_02\_11}
\end{table}
\subsubsection{Conclusione}
In questo sottoperiodo si è proceduto a correggere i documenti a seguito di quanto segnalatoci con la Revisione dei requisiti.
\subsubsection{Preventivo a finire}
Per poter correggere le mancanze individuate nell'\AdRv{1.0.0}\ a seguito dell'incontro con il \CR\, sono state necessarie ben 6 ore in più nel ruolo di \textit{Analista} rispetto a quanto preventivato ma riorganizzando il metodo di verifica è stato possibile recuperare 4 ore nel ruolo di \textit{Verificatore}.

\myparagraph{\glo{TB} e \glo{PoC} - Incremento 1}
Di seguito è presente la tabella contenente i dati del consuntivo dal 2021\_02\_10 al 2021\_02\_20.
\begin{table}[H]
	\centering
	\begin{tabular}{|c|c|c|c|c|}
		\rowcolor{darkblue} 
		&\multicolumn{2}{c|}{\textcolor{white}{Ore}}&\multicolumn{2}{c|}{\textcolor{white}{Costo in €}}\\ \hline
		Ruolo			&	Preventivo				&	Consuntivo		&	Preventivo	&	Consuntivo\\ \hline
		{\Responsabile}		&	5					&	5				&	150,00		&	150,00 \\ \hline
		{\Amministratore}	&	5					&	5				&	100,00		&	100,00 \\ \hline
		\textit{Analista}	&	10					&	14(+4)				&	250,00		&	350,00(+100) \\ \hline
		\textit{Progettista}& 	32					&   32 				& 	704,00		&  	704,00 \\ \hline
		\textit{Programmatore}& 12					& 	12				& 	180,00		&  	180,00 \\ \hline
		\textit{Verificatore}&	13					&	10(-3)				&	195,00		&	150,00(-45) \\ \hline
		Totale				&	77					&	77				&	1579,00		&	1579,00 \\ \hline
		Differenza			& 	\multicolumn{2}{c|}{-1} 			        &\multicolumn{2}{c|}{-55}\\ \hline
	\end{tabular}
	\caption{Prospetto orario ed economico a consuntivo dal 2021\_02\_10 al 2021\_02\_20}
\end{table}
\subsubsection{Conclusione}
In questo periodo si è iniziata la stesura del \glo{PoC} e si è continuato con la correzione dell'\AdR e l'aggiornamento di alcuni documenti.
\subsubsection{Preventivo a finire}
Anche in questo periodo sono state necessarie più ore nel ruolo di \textit{Analista} rispetto a quanto preventivato, 4 ore in più, e meno nel ruolo di \textit{Verificatore} grazie al migliore procedimento di verifica già attuato anche nelle settimane precedenti. Dati i risultati ottenuti per evitare nuovi sforamenti, si è deciso di ripianificare la distribuzione oraria nei vari ruoli nelle settimane successive.

\myparagraph{TB e PoC - Incremento 2}
Di seguito è presente la tabella contenente i dati del consuntivo dal 2021\_02\_21 al 2021\_02\_27.
\begin{table}[H]
	\centering
	\begin{tabular}{|c|c|c|c|c|}
		\rowcolor{darkblue} 
		&\multicolumn{2}{c|}{\textcolor{white}{Ore}}&\multicolumn{2}{c|}{\textcolor{white}{Costo in €}}\\ \hline
		Ruolo			&	Preventivo				&	Consuntivo		&	Preventivo	&	Consuntivo\\ \hline
		{\Responsabile}		&	4					&	4				&	120,00		&	120,00 \\ \hline
		{\Amministratore}	&	2					&	2				&	40,00		&	40,00 \\ \hline
		\textit{Analista}	&	12(+7)					&	12				&	300,00(+175)		&	300,00 \\ \hline
		\textit{Progettista}& 	27(-2)					&   27 				& 	594,00(-44)		&  	594,00 \\ \hline
		\textit{Programmatore}& 9					& 	9				& 	135,00		&  	135,00 \\ \hline
		\textit{Verificatore}&	16(-5)					&	16				&	240,00(-75)		&	240,00 \\ \hline
		Totale				&	70					&	70				&	1429,00(+56)		&	1429,00 \\ \hline
		Differenza			& 	\multicolumn{2}{c|}{-} 			&\multicolumn{2}{c|}{-}\\ \hline
	\end{tabular}
	\caption{Prospetto orario ed economico a consuntivo dal 2021\_02\_21 al 2021\_02\_27}
\end{table}
%\subsubsection{Conclusione}
%\subsubsection{Preventivo a finire}