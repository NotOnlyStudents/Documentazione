\subsection{Periodo di analisi}
L'analisi è considerata come un periodo di investimento. Il consuntivo viene esposto solamente a scopo informativo e non viene conteggiato nel preventivo a finire.
\subsubsection{Consuntivo di periodo}
Di seguito è presente la tabella contenente i dati del consuntivo per il periodo di analisi.
\begin{table}[H]
	\centering
	\begin{tabular}{|c|c|c|c|c|}
		\rowcolor{darkblue} 
		&\multicolumn{2}{c|}{\textcolor{white}{Ore}}&\multicolumn{2}{c|}{\textcolor{white}{Costo in €}}\\ \hline
		Ruolo				&	Preventivo			&	Consuntivo		&	Preventivo	&	Consuntivo\\ \hline
		{\Responsabile}		&	24					&	24				&	720,00		&	720,00 \\ \hline
		{\Amministratore}	&	35					&	34 (+1)			&	700,00		&	600,00 (+100,00) \\ \hline
		\textit{Analista}	&	89					&	95 (-6)			&	2225,00		&	2375,00 (-150,00) \\ \hline
		\textit{Progettista}& 	-					&	- 				& 	-			&  	- \\ \hline
		\textit{Programmatore}& -					& 	-				& 	-			&  	- \\ \hline
		\textit{Verificatore}&	62					&	60 (+2)			&	930,00		&	900,00 (+30,00) \\ \hline
		Totale				&	210					&	213				&	4575,00		&	4595,00 \\ \hline
		Differenza			& 	\multicolumn{2}{c|}{-3} 				& \multicolumn{2}{c|}{-20,00}\\ \hline
	\end{tabular}
	\caption{Prospetto orario ed economico a consuntivo nella periodo di analisi}
\end{table}
\subsubsection{Conclusione}
Nello svolgimento dell'\glo{attività} di analisi si è stati costretti ad impiegare più tempo del previsto per il ruolo di \textit{Analista} e, invece, a ridurre il consumo di ore nel ruolo di {\Amministratore}. Questo è dovuto ad una sottostima del carico di lavoro richiesto per la stesura dell'\AdRv{1.0.0}. Il risultato di questo primo periodo è complessivamente di 3 ore lavorative oltre il previsto con una spesa aggiuntiva di \textbf{20,00€}.
\subsubsection{Preventivo a finire}
Dal momento che il risultato del bilancio è negativo, pur trattandosi di un periodo non rendicontato e di una differenza minima rispetto a quanto previsto, il gruppo {\Gruppo} ha deciso di adottare delle azioni correttive per ridurre il rischio di ulteriori ritardi nel proseguo del lavoro.
\begin{itemize}
	\item Migliorare il procedimento di verifica continua mediante l'uso di \glo{GitHub} Actions con controlli più stringenti sulla correttezza ortografica e dei comandi presenti nei documenti \LaTeX{};
	\item Rendere gli incontri interni del gruppo più frequenti e regolari;
	\item Utilizzo di \glo{Slack} per la comunicazione così da avere sempre una visione generale dell'andamento del progetto grazie all'integrazione con GitHub e Zoom;
	\item Coinvolgere maggiormente \Proponente\ per le scelte relative ai requisiti e alle tecnologie da implementare.
\end{itemize}