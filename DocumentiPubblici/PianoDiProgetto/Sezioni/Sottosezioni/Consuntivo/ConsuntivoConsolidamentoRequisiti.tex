\subsection{Periodo di consolidamento dei requisiti}
Il consolidamento dei requisiti è considerato come un periodo di investimento. Il consuntivo viene esposto solamente a scopo informativo e non viene conteggiato nel preventivo a finire.
\subsubsection{Consuntivo di periodo}
Di seguito è presente la tabella contenente i dati del consuntivo per il periodo di consolidamento dei requisiti.
\begin{table}[H]
	\centering
	\begin{tabular}{|c|c|c|c|c|}
		\rowcolor{darkblue} 
		&\multicolumn{2}{c|}{\textcolor{white}{Ore}}&\multicolumn{2}{c|}{\textcolor{white}{Costo in €}}\\ \hline
		Ruolo			&	Preventivo				&	Consuntivo		&	Preventivo	&	Consuntivo\\ \hline
		{\Responsabile}		&	5					&	5				&	150,00		&	150,00 \\ \hline
		{\Amministratore}	&	6					&	6				&	120,00		&	120,00 \\ \hline
		\textit{Analista}	&	17					&	17				&	425,00		&	425,00 \\ \hline
		\textit{Progettista}& 	7					&	7 				& 	154,00		&  	154,00 \\ \hline
		\textit{Programmatore}& -					& 	-				& 	-			&  	- \\ \hline
		\textit{Verificatore}&	7					&	7				&	105,00		&	105,00 \\ \hline
		Totale				&	42					&	42				&	954,00		&	954,00 \\ \hline
		Differenza			& 	\multicolumn{2}{c|}{-} 			&\multicolumn{2}{c|}{-}\\ \hline
	\end{tabular}
	\caption{Prospetto orario ed economico a consuntivo nella periodo di consolidamento dei requisiti}
\end{table}
\subsubsection{Conclusione}
Trattandosi di un periodo non rendicontato e non avendo riscontrato un bilancio negativo, il gruppo {\Gruppo} ritiene che non sia necessario agire in alcun modo nel preventivo orario ed economico.
\subsubsection{Preventivo a finire}
Il preventivo a finire rimane in linea con quanto previsto dal momento che le ore preventivate sono state rispettate.