\section{Consuntivo e preventivo a finire}
\label{consuntivo_preventivo_a_finire}
Vengono di seguito mostrati i \glo{consuntivi} dei vari periodi con una breve valutazione degli stessi. Sarà inoltre presente un \glo{preventivo} a finire che tiene conto dei soli periodi rendicontati. I valori inseriti nel consuntivo saranno di tue tipologie:
\begin {itemize}
	\item \textbf{Positivi:} se il valore del preventivo è superiore al valore del consuntivo;
	\item \textbf{Negativi:} se il valore del preventivo è inferiore al valore del consuntivo.
\end{itemize}
\subsection{Periodo di analisi}
L'analisi è considerata come un periodo di investimento. Il consuntivo viene esposto solamente a scopo informativo e non viene conteggiato nel preventivo a finire.
\subsubsection{Consuntivo di periodo}
Di seguito è presente la tabella contenente i dati del consuntivo per il periodo di analisi.
\begin{table}[H]
	\centering
	\begin{tabular}{|c|c|c|c|c|}
		\rowcolor{darkblue} 
		&\multicolumn{2}{c|}{\textcolor{white}{Ore}}&\multicolumn{2}{c|}{\textcolor{white}{Costo in €}}\\ \hline
		Ruolo			&	Preventivo			&	Consuntivo		&	Preventivo	&	Consuntivo\\ \hline
		{\Responsabile}		&	24					&	24				&	720,00		&	720,00 \\ \hline
		{\Amministratore}	&	35					&	34 (+1)			&	700,00		&	600,00 (+100,00) \\ \hline
		\textit{Analista}	&	89					&	95 (-6)			&	2225,00		&	2375,00 (-150,00) \\ \hline
		\textit{Progettista}& 	-					&	- 				& 	-			&  	- \\ \hline
		\textit{Programmatore}& -					& 	-				& 	-			&  	- \\ \hline
		\textit{Verificatore}&	62					&	60 (+2)			&	930,00		&	900,00 (+30,00) \\ \hline
		Totale				&	210					&	213				&	4575,00		&	4595,00 \\ \hline
		Differenza			& 	\multicolumn{2}{c|}{-3} 			&\multicolumn{2}{c|}{-20,00}\\ \hline
	\end{tabular}
	\caption{Prospetto orario ed economico a consuntivo nella periodo di analisi}
\end{table}
\subsubsection{Conclusione}
Nello svolgimento dell'attività di analisi si è stati costretti ad impiegare più tempo del previsto per il ruolo di \textit{Analista}, si è invece riusciti a ridurre il consumo di ore nel ruolo di {\Amministratore}. Questo è dovuto ad una sottostima del carico di lavoro richiesto per la stesura dell'\AdRv{1.0.0}. Il risultato di questo primo periodo è complessivamente di 3 ore lavorative oltre il previsto con una spesa aggiuntiva di 20,00€.\\Dal momento che il risultato del bilancio è negativo, pur trattandosi di un periodo non rendicontato e di una differenza minima rispetto a quanto previsto, il gruppo {\Gruppo} ha deciso di adottare delle azioni correttive per ridurre il rischio di ulteriori ritardi nel proseguo del lavoro.
\begin{itemize}
	\item Migliorare il procedimento di verifica continua mediante \glo{GitHub} Actions;
	\item Rendere gli incontri interni del gruppo più frequenti e regolari;
	\item Utilizzo di \glo{Slack} per la comunicazione così da avere sempre una visione generale dell'andamento del progetto grazie all'integrazione con GitHub e Zoom;
	\item Coinvolgere maggiormente \Proponente\ per le scelte progettuali.
\end{itemize}
\subsection{Periodo di consolidamento dei requisiti}
Il consolidamento dei requisiti è considerato come un periodo di investimento. Il consuntivo viene esposto solamente a scopo informativo e non viene conteggiato nel preventivo a finire.
\subsubsection{Consuntivo di periodo}
Di seguito è presente la tabella contenente i dati del consuntivo per il periodo di consolidamento dei requisiti.
\begin{table}[H]
	\centering
	\begin{tabular}{|c|c|c|c|c|}
		\rowcolor{darkblue} 
		&\multicolumn{2}{c|}{\textcolor{white}{Ore}}&\multicolumn{2}{c|}{\textcolor{white}{Costo in €}}\\ \hline
		Ruolo			&	Preventivo				&	Consuntivo		&	Preventivo	&	Consuntivo\\ \hline
		{\Responsabile}		&	5					&	5				&	150,00		&	150,00 \\ \hline
		{\Amministratore}	&	6					&	6				&	120,00		&	120,00 \\ \hline
		\textit{Analista}	&	17					&	17				&	425,00		&	425,00 \\ \hline
		\textit{Progettista}& 	7					&	7 				& 	154,00		&  	154,00 \\ \hline
		\textit{Programmatore}& -					& 	-				& 	-			&  	- \\ \hline
		\textit{Verificatore}&	7					&	7				&	105,00		&	105,00 \\ \hline
		Totale				&	42					&	42				&	954,00		&	954,00 \\ \hline
		Differenza			& 	\multicolumn{2}{c|}{-} 			&\multicolumn{2}{c|}{-}\\ \hline
	\end{tabular}
	\caption{Prospetto orario ed economico a consuntivo nella periodo di consolidamento dei requisiti}
\end{table}
\subsubsection{Conclusione}
Trattandosi di un periodo non rendicontato e non avendo riscontrato un bilancio negativo, il gruppo {\Gruppo} ritiene che non sia necessario agire in alcun modo nel preventivo orario ed economico.
\subsubsection{Impatto sul preventivo a finire}
Il preventivo a finire rimane in linea con quanto previsto dal momento che le ore preventivate sono state rispettate.
\subsection{Preventivo a finire}
Viene qui mostrata una tabella contenente l'attuale preventivo a finire. I valori relativi all'analisi e al consolidamento dei requisiti sono stati inseriti puramente a scopo riassuntivo. Questi infatti non saranno conteggiati nel calcolo delle ore rendicontate. Se il valore del consuntivo non fosse ancora presente, verrà usato il valore del preventivo.
\begin{table}[H]
	\centering
	\begin{tabular}{|c|c|c|}
	\rowcolor{darkblue} 
		\textcolor{white}{Periodo}		&\textcolor{white}{Preventivo €}&	\textcolor{white}{Consuntivo €}\\ \hline
		Analisi							&	4575,00						&	4595,00 \\ \hline
		Consolidamento dei requisiti	&	954,00						&	954,00 \\ \hline
		\rowcolor{darkblue} \multicolumn{3}{|c|}{\textcolor{white}{Rendicontato}}\\ \hline
		Consolidamento delle tecnologie	&	4327,00						&	Non presente \\ \hline
		Progettazione e codifica		&	6338,00						&	Non presente \\ \hline
		Validazione e collaudo			&	2498,00						&	Non presente \\ \hline
		\rowcolor{darkblue}				&\textcolor{white}{Preventivo €}&	\textcolor{white}{Preventivo a finire €}\\ \hline
		Totale							&	18692,00					&	18712,00 \\ \hline
		Rendicontato					&	13163,00					&	13183,00 \\ \hline
	\end{tabular}
	\caption{Preventivo a finire}
\end{table}