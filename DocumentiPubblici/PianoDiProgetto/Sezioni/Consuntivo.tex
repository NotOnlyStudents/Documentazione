\section{Consuntivo e preventivo a finire}
\label{consuntivo_preventivo_a_finire}
Vengono di seguito mostrati i \glo{consuntivi} dei vari periodi con una breve valutazione degli stessi. Sarà inoltre presente un preventivo a finire che tiene conto dei soli periodi rendicontati. I valori inseriti nel consuntivo saranno di tue tipologie:
\begin {itemize}
	\item \textbf{Positivi:} se il valore del preventivo è superiore al valore del consuntivo;
	\item \textbf{Negativi:} se il valore del preventivo è inferiore al valore del consuntivo.
\end{itemize}
\subsection{Periodo di Analisi}
L'Analisi è considerata come un periodo di investimento. Il consuntivo viene esposto solamente a scopo informativo e non viene conteggiato nel preventivo a finire.
\subsubsection{Consuntivo}
Di seguito è presente la tabella contenente i dati del consuntivo per il periodo di Analisi.
\begin{table}[H]
	\centering
	\begin{tabular}{|c|c|c|c|c|}
	\rowcolor{darkblue} 
	 	 			&	\multicolumn{2}{c|}{\textcolor{white}{Ore}} 		& 	\multicolumn{2}{c|}{\textcolor{white}{Costo in €}}  \\ \hline
			Ruolo		&	Preventivo	&	Consuntivo	&	Preventivo	&	Consuntivo 	\\ \hline
		Responsabile		&	24		&	24		&	720,00	&	720,00  	\\ \hline
		Amministratore	&	35		&	34 (+1)	&	700,00	&	600,00 (+100,00)  \\ \hline
		Analista		&	89		&	95 (-6)	&	2225,00	&	2375,00 (-150,00)  \\ \hline
		Progettista		& 			&	 		& 			&  			\\ \hline
		Programmatore	& 			& 			& 			&  			\\ \hline
		Verificatore		&	62		&	60 (+2)	&	930,00	&	900,00 (+30,00) 	\\ \hline
		Totale			&	210		&	213		&	4575,00	&	4595,00  	\\ \hline
		Differenza		& 	\multicolumn{2}{c|}{-3 Ore} 	& 	\multicolumn{2}{c|}{-20,00 €} 	\\ \hline
	\end{tabular}
	\caption{Prospetto orario ed economico a consuntivo nella periodo di Analisi}
\end{table}
\subsubsection{Conclusione}
Nello svolgimento dell'Analisi si è stati costretti ad impiegare più tempo del previsto per il ruolo di \textit{Analista}, si è invece riusciti a ridurre il consumo di ore nel ruolo di \Amministratore. Questo è dovuto ad una sottostima del cario di lavoro richiesto per la stesura dell'\AdRv{}. Il risultato di questo primo periodo è complessivamente di 3 ore lavorative oltre il previsto con una spesa aggiuntiva di 20,00€.
\subsection{Preventivo a finire}
Viene qui mostrata una tabella contenente l'attuale preventivo a finire. I valori relativi all'Analisi e al Consolidamento dei Requisiti sono stati inseriti puramente a scopo riassuntivo. Questi infatti non saranno conteggiati nel calcolo delle ore rendicontate. Se il valore del consuntivo non fosse ancora presente, verrà usato il valore del preventivo.
\begin{table}[H]
	\centering
	\begin{tabular}{|c|c|c|}
	\rowcolor{darkblue} 
		\textcolor{white}{Periodo}	&\textcolor{white}{Preventivo €}	&	\textcolor{white}{Consuntivo €} \\ \hline
		Analisi					&	4575,00				&	4595,00  \\ \hline
		Consolidamento dei requisiti	&	954,00				&	Non presente  \\ \hline
		\rowcolor{darkblue} \multicolumn{3}{|c|}{\textcolor{white}{Rendicontato}}  \\ \hline
		Consolidamento delle tecnologie	&	4327,00				&	Non presente  \\ \hline
		Progettazione e codifica		&	6338,00				&	Non presente  \\ \hline
		Validazione e collaudo		&	2498,00				&	Non presente  \\ \hline
		\rowcolor{darkblue}		&\textcolor{white}{Preventivo €}	&	\textcolor{white}{Preventivo a finire €} \\ \hline
		Totale					&	18692,00				&	18712,00 \\ \hline
		Rendicontato			&	13119,00				&	13417,00 \\ \hline
	\end{tabular}
	\caption{Preventivo a finire - periodo di Analisi}
\end{table}