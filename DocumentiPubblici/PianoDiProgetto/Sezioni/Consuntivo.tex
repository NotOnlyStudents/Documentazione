\section{Consuntivo e preventivo a finire}
\label{consuntivo_preventivo_a_finire}
Vengono di seguito mostrati i consuntivi dei vari periodi con una breve valutazione degli stessi. Sarà inoltre presente un preventivo a finire che tiene conto dei soli periodi rendicontati. I valori inseriti nel consuntivo saranno di due tipologie:
\begin{itemize}
	\item \textbf{Positivi:} se il valore del preventivo è superiore al valore del consuntivo;
	\item \textbf{Negativi:} se il valore del preventivo è inferiore al valore del consuntivo.
\end{itemize}
\subsection{Periodo di analisi}
L'analisi è considerata come un periodo di investimento. Il consuntivo viene esposto solamente a scopo informativo e non viene conteggiato nel preventivo a finire.
\subsubsection{Consuntivo di periodo}
Di seguito è presente la tabella contenente i dati del consuntivo per il periodo di analisi.
\begin{table}[H]
	\centering
	\begin{tabular}{|c|c|c|c|c|}
		\rowcolor{darkblue} 
		&\multicolumn{2}{c|}{\textcolor{white}{Ore}}&\multicolumn{2}{c|}{\textcolor{white}{Costo in €}}\\ \hline
		Ruolo			&	Preventivo			&	Consuntivo		&	Preventivo	&	Consuntivo\\ \hline
		{\Responsabile}		&	24					&	24				&	720,00		&	720,00 \\ \hline
		{\Amministratore}	&	35					&	34 (+1)			&	700,00		&	600,00 (+100,00) \\ \hline
		\textit{Analista}	&	89					&	95 (-6)			&	2225,00		&	2375,00 (-150,00) \\ \hline
		\textit{Progettista}& 	-					&	- 				& 	-			&  	- \\ \hline
		\textit{Programmatore}& -					& 	-				& 	-			&  	- \\ \hline
		\textit{Verificatore}&	62					&	60 (+2)			&	930,00		&	900,00 (+30,00) \\ \hline
		Totale				&	210					&	213				&	4575,00		&	4595,00 \\ \hline
		Differenza			& 	\multicolumn{2}{c|}{-3} 			&\multicolumn{2}{c|}{-20,00}\\ \hline
	\end{tabular}
	\caption{Prospetto orario ed economico a consuntivo nella periodo di analisi}
\end{table}
\subsubsection{Conclusione}
Nello svolgimento dell'attività di analisi si è stati costretti ad impiegare più tempo del previsto per il ruolo di \textit{Analista}, si è invece riusciti a ridurre il consumo di ore nel ruolo di {\Amministratore}. Questo è dovuto ad una sottostima del carico di lavoro richiesto per la stesura dell'\AdRv{1.0.0}. Il risultato di questo primo periodo è complessivamente di 3 ore lavorative oltre il previsto con una spesa aggiuntiva di 20,00€.\\Dal momento che il risultato del bilancio è negativo, pur trattandosi di un periodo non rendicontato e di una differenza minima rispetto a quanto previsto, il gruppo {\Gruppo} ha deciso di adottare delle azioni correttive per ridurre il rischio di ulteriori ritardi nel proseguo del lavoro.
\begin{itemize}
	\item Migliorare il procedimento di verifica continua mediante \glo{GitHub} Actions;
	\item Rendere gli incontri interni del gruppo più frequenti e regolari;
	\item Utilizzo di \glo{Slack} per la comunicazione così da avere sempre una visione generale dell'andamento del progetto grazie all'integrazione con GitHub e Zoom;
	\item Coinvolgere maggiormente \Proponente\ per le scelte progettuali.
\end{itemize}
\subsection{Periodo di consolidamento dei requisiti}
Il consolidamento dei requisiti è considerato come un periodo di investimento. Il consuntivo viene esposto solamente a scopo informativo e non viene conteggiato nel preventivo a finire.
\subsubsection{Consuntivo di periodo}
Di seguito è presente la tabella contenente i dati del consuntivo per il periodo di consolidamento dei requisiti.
\begin{table}[H]
	\centering
	\begin{tabular}{|c|c|c|c|c|}
		\rowcolor{darkblue} 
		&\multicolumn{2}{c|}{\textcolor{white}{Ore}}&\multicolumn{2}{c|}{\textcolor{white}{Costo in €}}\\ \hline
		Ruolo			&	Preventivo				&	Consuntivo		&	Preventivo	&	Consuntivo\\ \hline
		{\Responsabile}		&	5					&	5				&	150,00		&	150,00 \\ \hline
		{\Amministratore}	&	6					&	6				&	120,00		&	120,00 \\ \hline
		\textit{Analista}	&	17					&	17				&	425,00		&	425,00 \\ \hline
		\textit{Progettista}& 	7					&	7 				& 	154,00		&  	154,00 \\ \hline
		\textit{Programmatore}& -					& 	-				& 	-			&  	- \\ \hline
		\textit{Verificatore}&	7					&	7				&	105,00		&	105,00 \\ \hline
		Totale				&	42					&	42				&	954,00		&	954,00 \\ \hline
		Differenza			& 	\multicolumn{2}{c|}{-} 			&\multicolumn{2}{c|}{-}\\ \hline
	\end{tabular}
	\caption{Prospetto orario ed economico a consuntivo nella periodo di consolidamento dei requisiti}
\end{table}
\subsubsection{Conclusione}
Trattandosi di un periodo non rendicontato e non avendo riscontrato un bilancio negativo, il gruppo {\Gruppo} ritiene che non sia necessario agire in alcun modo nel preventivo orario ed economico.
\subsubsection{Impatto sul preventivo a finire}
Il preventivo a finire rimane in linea con quanto previsto dal momento che le ore preventivate sono state rispettate.
\subsection{Periodo di progettazione architetturale}\label{ConsuntivoPArchitetturale}
\subsubsection{Consuntivo di periodo}
Di seguito è presente la tabella contenente i dati del consuntivo per il periodo di progettazione architetturale.
\begin{table}[H]
	\centering
	\begin{tabular}{|c|c|c|c|c|}
		\rowcolor{darkblue} 
		&\multicolumn{2}{c|}{\textcolor{white}{Ore}}&\multicolumn{2}{c|}{\textcolor{white}{Costo in €}}\\ \hline
		Ruolo			&	Preventivo				&	Consuntivo		&	Preventivo	&	Consuntivo\\ \hline
		{\Responsabile}		&	18					&	14(-4)			&	540,00		&	420,00(-120) \\ \hline
		{\Amministratore}	&	20					&	15(-5)			&	400,00		&	300,00(-100) \\ \hline
		\textit{Analista}	&	38					&	53(+15)			&	950,00		&	1325,00(+375) \\ \hline
		\textit{Progettista}& 	61					&	64(+3)			& 	1342,00		&  	1408,00(+66) \\ \hline
		\textit{Programmatore}& 21					& 	24(+3)			& 	315			&  	360,00(+45) \\ \hline
		\textit{Verificatore}&	52					&	46(-6)			&	780,00		&	690,00(-90) \\ \hline
		Totale				&	210					&	216				&	4327,00		&	4503,00 \\ \hline
		Differenza			& 	\multicolumn{2}{c|}{+6} 				&\multicolumn{2}{c|}{+176}\\ \hline
	\end{tabular}
	\caption{Prospetto orario ed economico a consuntivo nella periodo di progettazione architetturale}
\end{table}
\subsubsection{Specifica dei consuntivi}
Come per i preventivi si riportano i consuntivi suddivisi per le attività individuate nella sezione \S\ref{progettazione_architetturale}
\myparagraph{Incremento e verifica documentazione}
Di seguito è presente la tabella contenente i dati del consuntivo dal 2021\_02\_01 al 2021\_02\_11.
\begin{table}[H]
	\centering
	\begin{tabular}{|c|c|c|c|c|}
		\rowcolor{darkblue} 
		&\multicolumn{2}{c|}{\textcolor{white}{Ore}}&\multicolumn{2}{c|}{\textcolor{white}{Costo in €}}\\ \hline
		Ruolo			&	Preventivo				&	Consuntivo		&	Preventivo	&	Consuntivo\\ \hline
		{\Responsabile}		&	9					&	7 (-2)			&	270,00		&	210,00(-60) \\ \hline
		{\Amministratore}	&	13					&	9 (-4)			&	260,00		&	180,00(-80) \\ \hline
		\textit{Analista}	&	23					&	29(+6)			&	575,00		&	725,00(+150) \\ \hline
		\textit{Progettista}& 	-					&	- 				& 	-		    &  	- \\ \hline
		\textit{Programmatore}& -					& 	-				& 	-			&  	- \\ \hline
		\textit{Verificatore}&	18					&	14(-4)			&	270,00		&	210,00(-60) \\ \hline
		Totale				&	63					&	59				&	1375,00		&	1325,00 \\ \hline
		Differenza			& 	\multicolumn{2}{c|}{-4} 				&\multicolumn{2}{c|}{-50}\\ \hline
	\end{tabular}
	\caption{Prospetto orario ed economico a consuntivo dal 2021\_02\_01 al 2021\_02\_11}
\end{table}
\subsubsection{Conclusione}
In questo sottoperiodo si è proceduto a correggere i documenti a seguito di quanto segnalatoci con la \glo{\textbf{RR}} e si è iniziata la ristrutturazione dell'{\AdR}.
\subsubsection{Preventivo a finire}
A seguito dell'incontro sostenuto con il \CR\ per poter discutere delle mancanze individuate nell'\AdRv{1.0.0}\, sono emersi diversi errori non considerati dal gruppo, dovendo così ristrutturare in modo sostanziale il documento. 
In questa prima parte sono state necessarie ben 6 ore in più nel ruolo di \textit{Analista} rispetto a quanto preventivato. Riorganizzando il metodo di verifica è stato possibile recuperare 4 ore nel ruolo di \textit{Verificatore}.

\myparagraph{\glo{TB} e PoC - Incremento 1}
Di seguito è presente la tabella contenente i dati del consuntivo dal 2021\_02\_10 al 2021\_02\_20.
\begin{table}[H]
	\centering
	\begin{tabular}{|c|c|c|c|c|}
		\rowcolor{darkblue} 
		&\multicolumn{2}{c|}{\textcolor{white}{Ore}}&\multicolumn{2}{c|}{\textcolor{white}{Costo in €}}\\ \hline
		Ruolo			&	Preventivo				&	Consuntivo		&	Preventivo	&	Consuntivo\\ \hline
		{\Responsabile}		&	5					&	4(-1)			&	150,00		&	120,00(-30) \\ \hline
		{\Amministratore}	&	5					&	5				&	100,00		&	100,00 \\ \hline
		\textit{Analista}	&	10					&	15(+5)			&	250,00		&	375,00(+125) \\ \hline
		\textit{Progettista}& 	32					&   32 				& 	704,00		&  	704,00 \\ \hline
		\textit{Programmatore}& 12					& 	12				& 	180,00		&  	180,00 \\ \hline
		\textit{Verificatore}&	13					&	10(-3)			&	195,00		&	150,00(-45) \\ \hline
		Totale				&	77					&	78				&	1579,00		&	1629,00 \\ \hline
		Differenza			& 	\multicolumn{2}{c|}{+1} 			    &\multicolumn{2}{c|}{+50}\\ \hline
	\end{tabular}
	\caption{Prospetto orario ed economico a consuntivo dal 2021\_02\_10 al 2021\_02\_20}
\end{table}
\subsubsection{Conclusione}
In questo periodo si è iniziata la stesura del \glo{PoC} e si è continuato con la correzione dell'\AdR{} e l'aggiornamento di alcuni documenti.
\subsubsection{Preventivo a finire}
Anche in questo sottoperiodo sono state necessarie più ore nel ruolo di \textit{Analista} rispetto a quanto preventivato, 5 ore in più, e meno nel ruolo di \textit{Verificatore} grazie al migliore procedimento di verifica già attuato anche nelle settimane precedenti.

\myparagraph{TB e PoC - Incremento 2}
Di seguito è presente la tabella contenente i dati del consuntivo dal 2021\_02\_21 al 2021\_02\_27.
\begin{table}[H]
	\centering
	\begin{tabular}{|c|c|c|c|c|}
		\rowcolor{darkblue} 
		&\multicolumn{2}{c|}{\textcolor{white}{Ore}}&\multicolumn{2}{c|}{\textcolor{white}{Costo in €}}\\ \hline
		Ruolo			&	Preventivo				&	Consuntivo		&	Preventivo	&	Consuntivo\\ \hline
		{\Responsabile}		&	4					&	3(-1)			&	120,00		&	90,00(-30) \\ \hline
		{\Amministratore}	&	2					&	1(-1)			&	40,00		&	20,00(-20) \\ \hline
		\textit{Analista}	&	5					&	7(+2)			&	125,00		&	175,00(+50) \\ \hline
		\textit{Progettista}& 	29					&   29 				& 	638,00		&  	638,00 \\ \hline
		\textit{Programmatore}& 9					& 	9				& 	135,00		&  	135,00 \\ \hline
		\textit{Verificatore}&	21					&	18(-3)			&	315,00		&	270,00(-45) \\ \hline
		Totale				&	70					&	67				&	1373,00		&	1328,00 \\ \hline
		Differenza			& 	\multicolumn{2}{c|}{-3} 				&\multicolumn{2}{c|}{-45}\\ \hline
	\end{tabular}
	\caption{Prospetto orario ed economico a consuntivo dal 2021\_02\_21 al 2021\_02\_27}
\end{table}
\subsubsection{Conclusione}
Si è proseguito con la stesura del \glo{PoC} e con la correzione dell'\AdR{}.
\subsubsection{Preventivo a finire}
A causa della pesante riorganizzazione in corso all'{\AdR} sono state svolte più ore nel ruolo di \textit{Analista} rispetto a quanto preventivato in precedenza. Grazie ad una migliore organizzazione interna al gruppo e ad una suddivisone efficace delle attività legate al progetto; si è riusciti a diminuire il carico di lavoro nei ruoli di \textit{Verificatore}, di {\Responsabile} e di {\Amministratore}.

\myparagraph{Variazione nella pianificazione}
Di seguito è presente la tabella contenente i dati del consuntivo dal 2021\_02\_28 al 2021\_03\_05.
\begin{table}[H]
	\centering
	\begin{tabular}{|c|c|c|c|c|}
		\rowcolor{darkblue} 
		&\multicolumn{2}{c|}{\textcolor{white}{Ore}}&\multicolumn{2}{c|}{\textcolor{white}{Costo in €}}\\ \hline
		Ruolo			&	Preventivo				&	Consuntivo		&	Preventivo	&	Consuntivo\\ \hline
		{\Responsabile}		&	-					&	-				&	-			&	- \\ \hline
		{\Amministratore}	&	-					&	-				&	-			&	- \\ \hline
		\textit{Analista}	&	-					&	2				&	-			&	50,00 \\ \hline
		\textit{Progettista}& 	-					&   3 				& 	-			&  	66,00 \\ \hline
		\textit{Programmatore}& -					& 	3				& 	-			&  	45,00 \\ \hline
		\textit{Verificatore}&	-					&	4				&	-			&	60,00 \\ \hline
		Totale				&	-					&	12				&	-			&	221,00 \\ \hline
		Differenza			& 	\multicolumn{2}{c|}{+12} 				&\multicolumn{2}{c|}{+221,00}\\ \hline
	\end{tabular}
	\caption{Prospetto orario ed economico a consuntivo dal 2021\_02\_28 al 2021\_03\_05}
\end{table}
\subsubsection{Conclusione}
Si è terminato con la stesura del \glo{PoC} e con la correzione dell'\AdR{}. A causa della numerosità delle tecnologie da studiare e da implementare nello sviluppo del progetto e la sovrapposizione della sessione invernale d'esami con il periodo di progettazione, si è stati costretti a ritardare la consegna dei vari documenti e la presentazione della \glo{TB}.
\subsubsection{Preventivo a finire}
Anche in questa ultima parte del periodo di \textbf{progettazione architetturale} sono stati commessi ritardi durante l'esecuzione delle varie attività relative al progetto a causa dei vari impegni universitari che hanno coinvolto i membri del gruppo {\Gruppo}. portando ad una conseguente necessità di ricoprire per più ore i ruoli di \textit{Analista}, \textit{Progettista}, \textit{Programmatore} e \textit{Verificatore}. Queste sono tutte ore che non erano state preventivate avendo un impatto pesante sui costi totali del progetto.
Il risultato di questo primo periodo è complessivamente di 6 ore lavorative oltre il previsto con una spesa aggiuntiva di \textbf{176,00€}.
\subsection{Preventivo a finire}\label{Paf}
Viene qui mostrata una tabella contenente l'attuale preventivo a finire. I valori relativi all'analisi e al consolidamento dei requisiti sono stati inseriti puramente a scopo riassuntivo, questi infatti non saranno conteggiati nel calcolo delle ore rendicontate. Se il valore del consuntivo non fosse ancora presente, verrà usato il valore del preventivo.
\begin{table}[H]
	\centering
	\begin{tabular}{|c|c|c|}
	\rowcolor{darkblue} 
		\textcolor{white}{Periodo}		&\textcolor{white}{Preventivo €}&	\textcolor{white}{Consuntivo €}\\ \hline
		Analisi							&	4575,00						&	4595,00 \\ \hline
		Consolidamento dei requisiti	&	954,00						&	954,00 \\ \hline
		\rowcolor{darkblue} \multicolumn{3}{|c|}{\textcolor{white}{Rendicontato}}\\ \hline
		Consolidamento delle tecnologie	&	4327,00						&	Non presente \\ \hline
		Progettazione e codifica		&	6338,00						&	Non presente \\ \hline
		Validazione e collaudo			&	2498,00						&	Non presente \\ \hline
		\rowcolor{darkblue}				&\textcolor{white}{Preventivo €}&	\textcolor{white}{Preventivo a finire €}\\ \hline
		Totale							&	18692,00					&	18712,00 \\ \hline
		Rendicontato					&	13163,00					&	13183,00 \\ \hline
	\end{tabular}
	\caption{Preventivo a finire}
\end{table}