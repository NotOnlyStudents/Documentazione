\section{Informazioni generali}
\begin{itemize}
\item \textbf{Luogo:} Incontro Zoom;
\item \textbf{Data:} \Data;
\item \textbf{Ora:} 12:30 - 13:00;
\item \textbf{Partecipanti:}
	\begin{itemize}
		\item \BL{}; 
		\item \FF{};
		\item \PC{};
		\item \TG{};
		\item \TL{};
		\item \VD{};
		\item \CR{}.
	\end{itemize}
\item \textbf{Segretario:} \TG{}.
\end{itemize}

\section{Ordine del giorno}
\begin{enumerate}
	\item Discussione sul significato dell'inclusione;
	\item Discussione sul corretto metodo di individuazione e classificazione dei casi d'uso;
	\item Discussione sul corretto significato di Proof of Concept.
\end{enumerate}

\section{Resoconto}
\subsection{Discussione sul significato dell'inclusione}
A seguito di quanto segnalato nella valutazione della \textbf{Revisione dei requisiti}, si è ritenuto opportuno chiedere un chiarimento rispetto al significato di "inclusione" nei diagrammi dei casi d'uso. I diagrammi erano stati realizzati in modo errato dando un significato temporale all'esecuzione dei vari casi d'uso coinvolti.
\subsection{Discussione sul corretto metodo di individuazione e classificazione dei casi d'uso}
Sono stati chiesti dei chiarimenti sulla corretta individuazione e classificazione dei casi d'uso. I casi d'uso generici individuati seppur corretti devono comunque essere specializzati in base a precise precondizioni e postcondizioni. Di conseguenza anche la loro denominazione deve essere più specifica.
\subsection{Discussione sul corretto significato di Proof of Concept}
Il gruppo ha chiesto un chiarimento rispetto al \textbf{Proof of Concept} che ha portato ad una discussione sul significato dello stesso. Il prototipo deve permettere al gruppo di capire il funzionamento corretto delle tecnologie richieste e la loro integrazione.

\section{Registro delle decisioni}
\begin{itemize}
  \item \textbf{VE\_\Data.1} I diagrammi che presentano "inclusioni" vanno rivisti secondo il corretto significato delle stesse.
   \item \textbf{VE\_\Data.2} I casi d'uso generici vengono suddivisi in casi d'uso più specifici e vengono corretti i nomi dei casi d'uso eccessivamente imprecisi.
   \item \textbf{VE\_\Data.3} È opportuno approfondire lo studio delle tecnologie coinvolte per poter comprendere come realizzare il Proof of Concept.
\end{itemize}