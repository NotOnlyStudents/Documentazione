\section{Informazioni Generali}
\begin{itemize}
\item \textbf{Luogo:} Incontro Zoom;
\item \textbf{Data:} \Data;
\item \textbf{Ora:} 12:00-13:00;
\item \textbf{Partecipanti:}
	\begin{itemize}
		\item \BL{}; 
		\item \FF{};
		\item \MM{};
		\item \TG{};
		\item \TL{};
		\item \VD{};
		\item \VT{}.
	\end{itemize} 
\item \textbf{Segretario:} \MM{}.
\end{itemize}

\section{Ordine del Giorno}
\begin{enumerate}
	\item Chiarimenti dubbi riguardo alla correzione della \textbf{RR}:
	\begin{itemize}
        \item Cambio di versione dei file;
        \item Scelta di adottare modello incrementale;
        \item Utilizzo preventivo a finire e consuntivo;
        \item Considerazione sull'\AdR{};
        \item Insufficiente allineamento norme e obiettivi di qualità nel \PdQ{}.
    \end{itemize}
	\item Consigli da parte del professore per migliorare.
\end{enumerate}

\section{Resoconto}
\subsection{Chiarimenti dubbi riguardo alla correzione della RR:}
\subsubsection{Cambio di versione dei file}
Come sottolineato dal \VT{}, le regole cha abbiamo adottato per dare il numero di versione dei documenti confondono il lettore.
Aumentare la versione a ogni modifica verificata e le ulteriori verifiche complessive complica capire l'effettiva validità dei contenuti e indica un approccio a tentativi.
Il professore invita a riflettere sulle regole adottate da \textbf{Chromium} e \textbf{LibreOffice}.
Il primo presenta sempre l'ultima versione stabile e incrementa ogni qualvolta il prodotto è funzionante, la grandezza dell'incremento di versione è determinata dalla dimensione della modifica.
Il secondo si impegna di presentare al pubblico una versione stabile rilasciata annualmente, L'incremento sostanziale di versione viene fatto in queste occasioni mantenendo fede alla promessa.
Il gruppo rifletterà su questi due approcci adattandoli alla validità dei documenti per opinare il problema.
\subsubsection{Scelta di adottare il modello incrementale come obiettivo}
Il professore ci fa notare che è prematuro e incorretto definire il nostro \textbf{way of working} incrementale.
Un gruppo alle prime armi fa iterazione, il modello incrementale è solo un obiettivo raggiungibile attraverso il miglioramento.
Per raggiungere il nostro obiettivo il \VT{} ci invita a fare pianificazione in un periodo breve per comprenderla meglio e poi lentamente allungarla in base a quanto il gruppo riduce le iterazioni.
\subsubsection{Utilizzo preventivo a finire e consuntivo}
Il professore sottolinea come l'utilizzo effettuato dal gruppo del preventivo a finire e del consuntivo è un mero calcolo matematico, mentre questo dovrebbe aiutarci a valutare come sia andata la pianificazione e agire in caso qualcosa non abbia funzionato come pensato. Sottolinea nuovamente l'importanza di una buona pianificazione a breve termine.
\subsubsection{Considerazione sull'Analisi dei requisiti}
Il professore descrive come i requisiti di processo non dovrebbe essere scritti all'interno dell'\AdR{} che è un documento rigido e va assolutamente rispettato, ma di definirli in un documento più flessibile come le \NdP{}.
Evidenzia poi che sviluppare una "Single-Page application" rende necessario specificare le versione dei browser compatibili con i \glo{framework} richiesti dal capitolato.
\subsubsection{Insufficiente allineamento norme e obiettivi di qualità nel Piano di qualifica}
Il professore evidenzia lo scopo degli obiettivi di qualità: distinguere i vari fornitori, in quanto un proponente decide a chi affidare il capitolato non riguardo all'\AdR\, ma secondo gli obiettivi di qualità che vengono dichiarati. Il \textbf{way of working} descritto e approfondito nelle \NdP{} deve poi supportare quanto scritto nel \PdQ\, altrimenti questo risulta poco credibile.
\subsection{Consigli da parte del professore per migliorare}
Ginnastica intellettuale: mettere in discussione quanto fatto, capire cosa si è sbagliato per poter correggere gli errori e migliorare il modo di lavoro non solo dopo le revisioni di avanzamento.

\section{Registro delle decisioni}
\begin{itemize}
	\item \textbf{VE\_\Data.1} Riflettere e aggiornare la numerazione relativa alle versioni;
	\item \textbf{VE\_\Data.2} Porre il modello incrementale come obiettivo aggiungendone la difesa nel \PdP{};
	\item \textbf{VE\_\Data.3} Concentrarsi sul breve termine per la pianificazione ed il preventivo all'interno del \PdP{};
	\item \textbf{VE\_\Data.4} Spostare i requisiti di processo all'interno delle \NdP\ e aggiungere le versioni dei browser alla \AdR{};
	\item \textbf{VE\_\Data.6} Allineare \NdP\ e \PdQ{};
	\item \textbf{VE\_\Data.7} Mettere in discussione quanto fatto fino ad ora per decidere cosa cambiare e cosa mantenere invariato.
\end{itemize}