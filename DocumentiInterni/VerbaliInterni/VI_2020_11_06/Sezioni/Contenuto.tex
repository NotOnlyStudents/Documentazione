\section{Informazioni Generali}
\begin{itemize}
\item \textbf{Luogo:} Incontro \textit{Zoom};
\item \textbf{Data:} \Data;
\item \textbf{Ora:} 18:00 - 19:00;
\item \textbf{Partecipanti:}
	\begin{itemize}
		\item \BL{}; 
		\item \FF{};
		\item \MM{}; 
		\item \PC{};
		\item \TG{};
		\item \TL{};
		\item \VD{}.
	\end{itemize} 
\item \textbf{Segretario:} \PC{}.
\end{itemize}

\section{Ordine del Giorno}
\begin{enumerate}
 \item Presentazione membri del gruppo;
 \item Discussione sui canali di comunicazione da usare;
 \item Confronto sulla frequenza degli incontri.
\end{enumerate}

\section{Resoconto}
\subsection{Presentazione membri del gruppo}
I membri del gruppo si sono presentati e si è potuto capire la situazione con gli esami arretrati, gli impegni di ogni membro e le relative esperienze lavorative e scolastiche.

\subsection{Discussione sui canali di documentazione da usare}
Dopo una discussione nella quale sono stati confrontati diversi software, sono stati scelti \glo{\textit{Telegram}} per la comunicazione testuale e \textit{Zoom} per la comunicazione audio e video.

\subsection{Confronto sulla frequenza degli incontri}
Il gruppo ha deciso che in queste settimane che precedono la scelta definitiva del \glo{capitolato} si riunirà dopo aver partecipato a ciascun seminario tenuto dai vari proponenti per analizzarne i suoi contenuti e discutere su eventuali dubbi. Scelto il capitolato, il gruppo si riunirà per valutare il procedere del lavoro in base alla disponibilità dei propri membri con cadenza non superiore alle due settimane. Si è inoltre creato un documento con gli impegni inderogabili di ogni singolo membro del gruppo per poter decidere quando effettuare le chiamate.

\section{Registro delle decisioni}
\begin{itemize}
  \item \textbf{VI\_\Data.1} Presentazione membri;
  \item \textbf{VI\_\Data.2} \textit{Zoom} e \glo{\textit{Telegram}} scelti come mezzi di comunicazione;
  \item \textbf{VI\_\Data.3} Riunioni del gruppo a cadenza al più bisettimanale nella fase iniziale del progetto.
\end{itemize}



