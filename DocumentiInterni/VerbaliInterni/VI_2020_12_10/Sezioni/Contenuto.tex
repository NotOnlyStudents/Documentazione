\section{Informazioni Generali}
\begin{itemize}
\item \textbf{Luogo:} Incontro \textit{Zoom};
\item \textbf{Data:} \Data;
\item \textbf{Ora:} 18:00 - 19:00;
\item \textbf{Partecipanti:}
	\begin{itemize}
		\item \BL{}; 
		\item \FF{};
		\item \MM{}; 
		\item \PC{};
		\item \TG{};
		\item \TL{};
		\item \VD{}.
	\end{itemize} 
\item \textbf{Segretario:} \PC{}.
\end{itemize}

\section{Ordine del Giorno}
\begin{enumerate}
 \item Impostazione \glo{repository} \glo{\textit{GitHub}};
 \item Organizzazione \glo{milestones};
 \item Assegnazione priorità ai documenti.
\end{enumerate}

\section{Resoconto}
\subsection{Impostazione repository GitHub}
È stato deciso di creare una \glo{repository} dedicata alla documentazione e alla sua manutenzione. Si è inoltre deciso di utilizzare il workflow \glo{\textit{gitflow}} per la repository. La repository dedicata alla documentazione sarà suddivisa nelle sottocartelle \textit{DocumentazioneInterna}, \textit{DocumentazionePubblica} e \textit{Utilita}.

\subsection{Organizzazione delle milestone}
 \begin{itemize}
  \item \textbf{2021-01-11:} Consegna documentazione;
  \item \textbf{2021-01-07:} Completamento documenti;
  \item \textbf{2021-12-30:} Verifica dei documenti redatti e mancanti alla verifica precedente;
  \item \textbf{2021-12-22:} Stesura \textit{\PdP}. Continuare la redazione degli altri documenti iniziati.
 \end{itemize}

\subsection{Assegnazione priorità ai documenti}
Si è scelto di iniziare subito con le \textit{\NdP}, già cominciate a scrivere nelle settimane precedenti, così come per lo \textit{\SdF}. I prossimi documenti da produrre sono l'\textit{\AdR} ed il \textit{\PdP}. 

\section{Registro delle decisioni}
\begin{itemize}
  \item \textbf{VI\_\Data.1} Realizzate le \glo{repository} \glo{\textit{GitHub}} e l'uso di \glo{\textit{gitflow}};
  \item \textbf{VI\_\Data.2} Approvate le \glo{milestones} interne;
  \item \textbf{VI\_\Data.3} Verificare lo {\SdF}. Continuare con la stesura delle {\NdP}. Iniziare a scrivere l'{\AdR} e il \PdP.
\end{itemize}




