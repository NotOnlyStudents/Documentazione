\section{Informazioni generali}
\begin{itemize}
\item \textbf{Luogo:} Incontro Zoom;
\item \textbf{Data:} \Data;
\item \textbf{Ora:} 11:00 - 12:40;
\item \textbf{Partecipanti:}
	\begin{itemize}
		\item \BL{}; 
		\item \FF{};
		\item \MM{}; 
		\item \PC{};
		\item \TG{};
		\item \TL{};
		\item \VD{}.
	\end{itemize} 
\item \textbf{Segretario:} \TL{}.
\end{itemize}

\section{Ordine del giorno}
\begin{enumerate}
 \item Discussione sulle soluzioni agli errori emersi dalla valutazione \textbf{RR};
 \item Aggiornare la schedulazione delle \glo{attività} fino alla \textbf{RP};
 \item Cambiare piattaforma di comunicazione.
\end{enumerate}

\section{Resoconto}
\subsection{Discussione sulle soluzioni agli errori emersi dalla valutazione RR}
Si è discusso riguardo i problemi indicati dal \VT\ e si sono trovate le seguenti soluzioni:
\begin{itemize}
	\item \textbf{Problema dello scatto del versionamento}: Si potrebbe rimuovere il terzo valore progressivo incrementando il secondo solo dopo le verifiche complessive;
	\item \textbf{Problema del preventivo a finire}: È possibile aggiungere una colonna alla tabella per confrontare facilmente i risultati finali con quanto preventivato;
	\item \textbf{Problema del modello di sviluppo}: Si è deciso di confermare la scelta di continuare con il modello incrementale, progettando la documentazione in anticipo e aggiungendo la "difesa" dello sviluppo incrementale all'interno dell'analisi dei rischi nel \PdP, opportuno confrontarsi con il \VT;
	\item \textbf{Problemi inclusioni Analisi dei Requisiti}: È necessario confrontarsi con il \CR\ per la modifica delle inclusioni presenti nei vari \glo{diagrammi UML};
	\item \textbf{Incoerenza tra Norme di Progetto e Piano di Qualifica}: Si deve limitare il contenuto del \PdQ\ alle metriche e riorganizzare le sezioni della qualità di processo e di prodotto dividendole nei diversi tipi di processo (primario, di supporto e organizzativo) individuati nelle \NdP;
	\item \textbf{Riduzione delle parole da scrivere in corsivo}: Utilizzare il corsivo solo per i nomi propri di persona, nomi dei documenti, nome del progetto e per i ruoli;
	\item \textbf{Aggiornare lista dei controlli da fare quando si verifica un documento}: Si deve aggiungere il controllo della presenza e dell'aggiornamento delle risorse (es. grafici). Si è discusso sulla possibilità di implementare anche il controllo automatico del glossario tramite uno script in \glo{Python}.
	\item \textbf{Riduzione del numero di termini nel glossario}: È necessario ridurre il numero delle voci riguardanti le tecnologie che non utilizziamo e sono ben conosciute, ad esempio \textbf{CSV}; oltre a rimuovere parole superflue come \textbf{prodotto}.
\end{itemize}
\subsection{Aggiornare la schedulazione delle attività fino alla RP}
Dato il periodo sovrapposto alla sessione invernale e l'aumento di tipi d'\glo{attività} da compiere è necessario rivedere la schedulazione del \PdP.
\subsection{Cambiare piattaforma di comunicazione}
Dopo una breve discussione tra i vari membri del gruppo, si è deciso di migrare le comunicazioni interne da \glo{Telegram} a \glo{Slack}.

\section{Registro delle decisioni}
\begin{itemize}
  \item \textbf{VI\_\Data.1} Applicare le soluzioni individuate dopo un confronto con \VT\ e \CR;
  \item \textbf{VI\_\Data.2} \MM\ e \TL\ si occupano di aggiornare la schedulazione per prepararsi all'ingresso in \textbf{RP};
  \item \textbf{VI\_\Data.3} \FF\ crea lo spazio di lavoro su \glo{Slack} e invita gli altri membri del gruppo.
\end{itemize}