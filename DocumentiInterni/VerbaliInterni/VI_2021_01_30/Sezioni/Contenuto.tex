\section{Informazioni Generali}
\begin{itemize}
\item \textbf{Luogo:} Incontro \textit{Zoom};
\item \textbf{Data:} \Data;
\item \textbf{Ora:} 11:00 - 12:40;
\item \textbf{Partecipanti:}
	\begin{itemize}
		\item \BL{}; 
		\item \FF{};
		\item \MM{}; 
		\item \PC{};
		\item \TG{};
		\item \TL{};
		\item \VD{}.
	\end{itemize} 
\item \textbf{Segretario:} \TL{}.
\end{itemize}

\section{Ordine del Giorno}
\begin{enumerate}
 \item Discussione sulle soluzioni agli errori emersi dalla valutazione RR.
 \item Aggiornare la schedulazione delle attività fino alla RP. 
 \item Cambiare piattaforma di comunicazione.
\end{enumerate}

\section{Resoconto}
\subsection{Discussione sulle soluzioni agli errori emersi dalla valutazione RR}
Il gruppo ha discusso le trovato le seguenti soluzioni ai problemi indicati dal \VT{}.
\begin{itemize}
	\item \textbf{Problema dello scatto del versionamento}: Rimuovere il terzo valore progressivo, e incrementare il secondo solo dopo le verifiche complessive.

	\item \textbf{Problema del preventivo a finire}: aggiungere una colonna alla tabella per confrontare facilmente i risultati finali con quelli previsti

	\item \textbf{Problema del modello di sviluppo}: confermare la scelta di seeguire il modello incrementale, progettando la documentazione in anticipo e aggiungendo la difficoltà dello sviluppo incrementale all'analisi dei rischi nel \PdP{}, opportuno confrontarsi con.

	\item \textbf{Problemi inclusioni analisi dei requisiti}: confrontarsi con il \CR{} per la modifica delle inclusioni.

	\item \textbf{Incoerenza tra \NdP{} e \PdQ{}}: Limitare il contenuto del \PdQ{} alle metriche, e riorganizzare le sezioni della qualità di processo e prodotto dividendole.

	\item \textbf{Riduzione delle parole da scrivere in corsivo}: Utilizzare il corsivo solo per i nomi propri di persona, nomi dei documenti e nome del progetto, e ruoli.

	\item \textbf{Aggiornare lista dei controlli da fare quando si verifica un documento}: aggiungere il controllo della presenza e dell'aggiornamento delle risorse (es. grafici). Aggiungere anche il controllo automatico del glossario.

	\item \textbf{Riduzione del numero di termini nel glossario}: eliminare le voci riguardanti alle tecnologie che non utilizziamo e sono ben conosciute (es. Java), rimuovere parole superflue(es. prodotto).
\end{itemize}

\subsection{Aggiornare la schedulazione delle attività fino alla RP.}
Dato il periodo sovrapposto alla sessione  invernale e l'aumento di tipi d'attività da compiere è necessario rivedere la schedulazione del PdP.

\subsection{Cambiare piattaforma di comunicazione}
I membri hanno deciso di migrare le comunicazioni interne da \glo{Telegram} a \glo{Slack}.

\section{Registro delle decisioni}
\begin{itemize}
  \item \textbf{VI\_\Data.1} Applicare le soluzioni individuate dopo un confronto con \VT{} e \CR{}.
  \item \textbf{VI\_\Data.2} \MM{} e \TL{} si occupano dell'aggiornare la schedulazione.
  \item \textbf{VI\_\Data.3} \FF{} crea lo spazio di lavoro su \glo{Slack} e invita gli altri membri del gruppo.
\end{itemize}
