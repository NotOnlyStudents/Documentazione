\section{Informazioni generali}
\begin{itemize}
\item \textbf{Luogo:} Incontro Zoom;
\item \textbf{Data:} \Data;
\item \textbf{Ora:} 18:00 - 19:30;
\item \textbf{Partecipanti:}
	\begin{itemize}
		\item \BL{}; 
		\item \FF{};
		\item \MM{}; 
		\item \PC{};
		\item \TG{};
		\item \TL{};
		\item \VD{}.
	\end{itemize} 
\item \textbf{Segretario:} \PC{}.
\end{itemize}

\section{Ordine del giorno}
\begin{enumerate}
 \item Scelta nome del gruppo;
 \item Scelta logo del gruppo;
 \item Scelta della mail del gruppo;
 \item Scelta degli strumenti collaborativi per la documentazione;
 \item Analisi dei \glo{capitolati};
 \item Decisione struttura dello \SdF.
\end{enumerate}

\section{Resoconto}
\subsection{Nome del gruppo}
Dopo una lunga riflessione individuale nella quale ogni membro ha proposto un nome per il gruppo, è stato scelto {\Gruppo}.
\subsection{Logo del gruppo}
É stato approvato il logo disegnato da \TG{} all'unanimità che verrà utilizzato nella pagina principale di ogni documento e come intestazione delle altre.
\subsection{Mail del gruppo}
Il gruppo ha discusso sulla creazione di un'email (indirizzo: nos.unipd@gmail.com) che possa essere utile per eventuali comunicazioni con il proponente del \glo{capitolato}, in particolare si è scelto di creare un indirizzo email associato con la suite di funzionalità di Google per poter utilizzare lo spazio di archiviazione del servizio Google Drive ad essa associato per eventuali scambi di documenti utili al gruppo.
\subsection{Strumenti collaborativi per la documentazione}
Si è deciso di utilizzare \glo{Git} come sistema di \glo{versionamento} del progetto e \glo{GitHub} come piattaforma di hosting. I componenti del gruppo che non conoscono questo sistema si impegneranno a formarsi sulle sue funzionalità di base in queste settimane che precedono la scelta del capitolato. Al fine di redarre in modo professionale tutti i documenti che saranno prodotti, il gruppo ha scelto di utilizzare il linguaggio \LaTeX. Sono stati creati dei template utili per la redazione delle future lettere di presentazione e per i documenti che saranno redatti durante il progetto.
\subsection{Analisi dei capitolati}
Il gruppo ha svolto una prima riflessione sui capitolati presentati, si sono valutati gli aspetti positivi e negativi di ciascun progetto considerando le competenze e le esperienze di ciascun componente e si è stabilita una prima iniziale preferenza.
Abbiamo deciso di svolgere i primi approfondimenti sulla fattibilità dei capitolati C2, C5 e C4 dato che erano quelli per cui sono state espresse più preferenze.
\subsection{Decisione struttura dello Studio di Fattibilità}
Tutte le norme riguardanti lo \SdF\ sono state decise, appuntate e approvate.
\section{Registro delle decisioni}
\begin{itemize}
  \item \textbf{VI\_\Data.1} Scelto {\Gruppo} come nome del gruppo;
  \item \textbf{VI\_\Data.2} Approvato il logo del gruppo;
  \item \textbf{VI\_\Data.3} Individuata \textbf{nos.unipd@gmail.com} come email del gruppo;
  \item \textbf{VI\_\Data.4} Approvato \glo{GitHub} come sistema di \glo{versionamento} e \LaTeX\ per la stesura della documentazione;
  \item \textbf{VI\_\Data.5} Scelti \glo{capitolati} C2, C5 e C4 come prime tre possibilità e conseguente redazione delle relative sezioni presenti nello \SdF;
  \item \textbf{VI\_\Data.6} Deciso lo scheletro dello \SdF.
\end{itemize}