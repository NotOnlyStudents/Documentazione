\section{Informazioni generali}
\begin{itemize}
	\item \textbf{Luogo:} incontro Zoom;
	\item \textbf{Data:} \Data;
	\item \textbf{Ora:} 10:30 - 12:00;
	\item \textbf{Partecipanti:}
	\begin{itemize}
		\item \BL{}; 
		\item \FF{};
		\item \MM{}; 
		\item \PC{};
		\item \TG{};
		\item \TL{};
		\item \VD{}.
	\end{itemize} 
	\item \textbf{Segretario:} \FF{}.
\end{itemize}

\section{Ordine del giorno}
\begin{enumerate}
	\item Discussione sulle soluzioni agli errori emersi dalla valutazione \glo{\textbf{RP}};
	\item Aggiornamento della suddivisione dei compiti.
\end{enumerate}

\section{Resoconto}
\subsection{Discussione sulle soluzioni agli errori emersi dalla valutazione RP}
Si è discusso riguardo i problemi indicati dal \VT\ e si sono pensate le seguenti soluzioni:
\begin{itemize}
	\item \textbf{Problema dei casi d'uso generici:} i diagrammi generici inseriti nell'\AdRv{2.0} non sono corretti perché non sono associati ad alcun caso d'uso e per questa ragione devono essere rimossi;
	\item \textbf{Ruolo dell'identity manager:} si è discusso sul ruolo dell'identity manager rispetto alla piattaforma, i componenti hanno idee differenti per questo motivo si ritiene opportuno discutere con i committenti prima di procedere;
	\item \textbf{Problema di UC2/UC3 e UC24.4/UC26.6:} valutando quanto sottolineatoci si procede con l'unificazione dei casi d'uso, eliminando ogni riferimento ad acquirente e venditore;
	\item \textbf{Immagine del modello di sviluppo incrementale concettualmente errata:} si sostituisce la relativa immagine con un'immagine coerente con il significato del modello;
	\item \textbf{Associazione incremento-impegno:} è possibile inserire delle tabelle per il tracciamento tra obiettivi di un incremento e costo orario pianificato;
	\item \textbf{Sforamento del preventivo:} viene fatta una ripianificazione dell'impegno orario per questo ed il successivo periodo per poter rientrare nel costo del preventivo concordato;
	\item \textbf{Contenuti della sezione \S{}A del \PdP\ da rendere proattivi:} i rischi vengono individuati preventivamente per i periodi successivi;
	\item \textbf{Interpretazione errata ciclo PDCA:} è necessario approfondire lo studio del ciclo PDCA e deve essere corretta la sezione \S{}B delle \NdP.
\end{itemize}
\subsection{Aggiornamento della suddivisione dei compiti}
In seguito alla pubblicazione dell'esito della \glo{RP}, si è deciso di suddividere tra i vari componenti le correzioni da apportare alla documentazione esistente, la redazione dei nuovi documenti da presentare per la \glo{RQ} ed il lavoro di sviluppo relativo al prodotto.

\section{Registro delle decisioni}
\begin{itemize}
	\item \textbf{VI\_\Data.1} Applicare le soluzioni individuate dopo un confronto con {\Proponente};
	\item \textbf{VI\_\Data.2} Il gruppo sarà suddiviso nel seguente modo:
	\begin{itemize}
		\item {\TL},{\BL} e {\FF} si occuperanno principalmente del \glo{back end};
		\item {\MM},{\PC},{\TG} e {\VD} si occuperanno prevalentemente del \glo{front end}.
	\end{itemize}
\end{itemize}