\section{Valutazione del capitolato scelto: C2 - \NomeProgetto}
\subsection{Informazioni sul capitolato}
\begin{itemize}
	\item \textbf{Nome:} \textit{\NomeProgetto}
	\item \textbf{Proponente:} \textit{RedBabel}
	\item \textbf{Committenti:} \textit{\VT{} e \CR{}}
\end{itemize}

\subsection{Descrizione del capitolato}
Il \glo{capitolato} richiede lo sviluppo di una piattaforma di \glo{e-commerce} generico.

\subsection{Scopo del capitolato}
Lo scopo del capitolato consiste nello sviluppo di una piattaforma e-commerce con architettura basata su \glo{micro-servizi} e \glo{Serverless}.
La piattaforma consisterà in un sito web che dovrà essere composto dalle seguenti parti:
\begin{itemize}
	\item home page;
	\item pagine di elenco dei prodotti;
	\item pagine di descrizione del prodotto;
	\item carrello della spesa;
	\item pagina di checkout;
	\item pagina di account.
\end{itemize}

Dovrà essere in grado di supportare i seguenti ruoli:
\begin{itemize}
	\item Amministratore, l'utente con questo ruolo può:
		\begin{itemize}
			\item distribuire l'applicazione nel cloud;
			\item gestire la configurazione delle integrazioni di terze parti.
		\end{itemize}
	\item Commerciante, l'utente con questo ruolo può:
		\begin{itemize}
			\item avere una panoramica di tutti gli ordini;
			\item gestire totalmente i vari prodotti da vendere.
		\end{itemize}
	\item Cliente, l'utente con questo ruolo può:
	\begin{itemize}
		\item cercare, filtrare e aggiungere al carrello come ospite o utente;
		\item se connesso può aggiornare le informazioni del suo profilo;
		\item eliminare l'account creato;
		\item procedere con il pagamento dei prodotti selezionati solo se ha fatto il login.
	\end{itemize}
\end{itemize}

\subsection{Tecnologie coinvolte}
L'azienda consiglia di utilizzare:
\begin{itemize}
	\item \glo{Typescript} con approccio \glo{async-await}, come linguaggio principale sia per il \glo{front-end} e sia per il \glo{back-end};
	\item \glo{typescript-eslint} per l'analisi statica del codice;
	\item \glo{AWS Lambda} come tecnologia Serverless con Typescript;
	\item \glo{Amazon CloudWatch} come servizio di monitoraggio, anche se non è obbligatorio;
	\item \glo{stripe} come fornitore del servizio di pagamento, anche se non è obbligatorio.
\end{itemize}

\subsection{Vincoli}
Il progetto deve operare in 3 ambienti: \glo{Locale}, \glo{Test} e \glo{Staging}.
L'ambiente di Staging deve essere accessibile pubblicamente.

Per raggiungere gli obiettivi minimi del progetto viene richiesto che:
\begin{itemize}
	\item il front-end sia implementato in \glo{Next.js};
	\item il back-end sia implementato in Serverless usando \glo{Typescript};
	\item l'integrazione di servizi di terze parti deve essere implementata con tecnologia Serveless usando Typescript;
	\item il progetto deve essere \glo{production-ready}, predisponendo anche un ambiente opzionale \glo{production};
	\item ci sia l'integrazione con un fornitore di servizi di pagamento.
\end{itemize}

\subsection{Aspetti positivi}
\begin{itemize}
	\item 3 componenti del gruppo hanno già sviluppato e conoscono \glo{Javascript}.
	\item Le tecnologie proposte stimolano i componenti del gruppo ad imparare ed a lavorare con esse.
	\item Le tecnologie proposte stanno sempre di più prendendo spazio nell'ambito del lavoro, specialmente lo sviluppo Serverless e lo sviluppo WEB con Typescript.
	\item L'uso di un'architettura a micro-servizi e Serverless favorisce la parallelizzazione dello sviluppo del back-end.
	\item L'azienda proponente è molto orientata al dialogo e alla comunicazione.
\end{itemize}

\subsection{Aspetti critici}
\begin{itemize}
	\item Nessuno dei componenti nel gruppo ha mai utilizzato \glo{AWS}.
	\item Nessuno dei componenti nel gruppo ha mai sviluppato un back-end Serverless.
\end{itemize}

\subsection{Conclusioni}
Il seguente capitolato è stato accolto da tutti i componenti con grande entusiasmo. Le tecnologie proposte, il tipo di progetto e la disponibilità comunicativa data dall'azienda ha generato molto interesse e voglia di lavorare. Per questi motivi è stato approvato come prima scelta.