\section{Valutazione del capitolato scelto: C2 - \NomeProgetto}\label{C2}
\subsection{Informazioni sul capitolato}
\begin{itemize}
	\item \textbf{Nome:} \textit{\NomeProgetto}.
	\item \textbf{\glo{Proponente}:} \textit{RedBabel}.
	\item \textbf{Committenti:} \textit{\VT{} e \CR{}}.
\end{itemize}

\subsection{Descrizione del capitolato}
Il \glo{capitolato} richiede lo sviluppo di una piattaforma di \glo{e-commerce} generico, in grado di favorire l'incontro tra commercianti e clienti. All'interno di \NomeProgetto\ sarà possibile controllare la gestione dei prodotti da parte dei commercianti registrati, facilitando la ricerca e l'acquisto da parte dei clienti che accedono alla piattaforma come ospiti o utenti registrati.

\subsection{Scopo del capitolato}
Lo scopo del capitolato consiste nello sviluppo di una piattaforma e-commerce con architettura basata su \glo{microservizi} e \glo{\textit{Serverless}}.
La piattaforma consisterà in un sito web che dovrà essere composto dalle seguenti parti:
\begin{itemize}
	\item Home page;
	\item Pagine di elenco dei prodotti;
	\item Pagine di descrizione del prodotto;
	\item Carrello della spesa;
	\item Pagina di \glo{checkout};
	\item Pagina di account.
\end{itemize}

Dovrà essere in grado di supportare i seguenti ruoli:
\begin{itemize}
	\item \textbf{Amministratore}: l'utente con questo ruolo può:
		\begin{itemize}
			\item Distribuire l'applicazione nel \glo{cloud};
			\item Gestire la configurazione delle integrazioni di terze parti.
		\end{itemize}
	\item \textbf{Commerciante}: l'utente con questo ruolo può:
		\begin{itemize}
			\item Avere una panoramica di tutti gli ordini;
			\item Gestire totalmente i vari prodotti da vendere.
		\end{itemize}
	\item \textbf{Cliente}: l'utente con questo ruolo può:
	\begin{itemize}
		\item Cercare, filtrare e aggiungere al carrello prodotti come ospite o utente;
		\item Se connesso può aggiornare le informazioni del suo profilo;
		\item Eliminare l'account creato;
		\item Procedere con il pagamento dei prodotti selezionati solo se ha fatto il login.
	\end{itemize}
\end{itemize}

\subsection{Tecnologie coinvolte}
L'azienda consiglia di utilizzare:
\begin{itemize}
	\item \textit{\glo{Typescript}} con approccio \textit{\glo{async-await}}, come linguaggio principale sia per il \glo{back end} e sia per il \glo{front end};
	\item \textit{\glo{typescript-eslint}}per l'analisi statica del codice;
	\item \textit{\glo{AWS Lambda}} come tecnologia \textit{Serverless} con \textit{Typescript};
	\item \textit{\glo{Amazon CloudWatch}} come servizio di monitoraggio, anche se non è obbligatorio;
	\item \textit{\glo{Stripe}} come fornitore del servizio di pagamento, anche se non è obbligatorio.
 \end{itemize}

\subsection{Vincoli}
Il progetto deve operare in 3 ambienti: \glo{Locale}, \glo{Test} e \glo{Staging}. L'ambiente di Staging deve essere accessibile pubblicamente.

Per raggiungere gli obiettivi minimi del progetto viene richiesto che:
\begin{itemize}
	\item Il front end sia implementato in \textit{\glo{Next.js}};
	\item Il back end sia implementato in \textit{Serverless} usando Typescript;
	\item L'integrazione di servizi di terze parti deve essere implementata con tecnologia Serveless usando Typescript;
	\item Il progetto deve essere \glo{production-ready}, predisponendo anche un ambiente opzionale \glo{production};
	\item Ci sia l'integrazione con un fornitore di servizi di pagamento.
\end{itemize}

\subsection{Aspetti positivi}
\begin{itemize}
	\item 3 componenti del gruppo hanno già sviluppato e conoscono \textit{\glo{Javascript}}.
	\item Le tecnologie proposte stimolano i componenti del gruppo a imparare e a lavorare con esse.
	\item Le tecnologie proposte stanno sempre di più prendendo spazio nell'ambito del lavoro, specialmente lo sviluppo \textit{Serverless} e lo sviluppo WEB con \textit{Typescript}.
	\item L'uso di un'architettura a microservizi e \textit{Serverless} favorisce la parallelizzazione dello sviluppo del back end.
	\item Il prodotto da realizzare è concreto rispetto alla realtà, oggi gli e-commerce vengono sempre più utilizzati.
	\item L'azienda proponente è molto orientata al dialogo e alla comunicazione.
\end{itemize}

\subsection{Aspetti critici}
\begin{itemize}
	\item Nessuno dei componenti nel gruppo ha mai utilizzato \textit{\glo{AWS}}.
	\item Nessuno dei componenti nel gruppo ha mai sviluppato un back end \textit{Serverless}.
\end{itemize}

\subsection{Conclusioni}
Il seguente capitolato è stato accolto da tutti i componenti con grande entusiasmo. Le tecnologie proposte, il tipo di progetto e la disponibilità comunicativa data dall'azienda ha generato molto interesse e voglia di lavorare. Per questi motivi è stato approvato come prima scelta.