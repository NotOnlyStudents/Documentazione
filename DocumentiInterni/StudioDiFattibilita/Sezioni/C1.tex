\section{Capitolato C1 - BlockCOVID}\label{C1}
\subsection{Informazioni sul capitolato}
\begin{itemize}
	\item \textbf{Nome:} \textit{BlockCOVID: supporto digitale al contrasto della pandemia}.
	\item \textbf{\glo{Proponente}:} \textit{Imola informatica}.
	\item \textbf{Committenti:} \textit{\VT{} e \CR{}}.
\end{itemize}

\subsection{Descrizione del capitolato}
Il \glo{capitolato} richiede lo sviluppo di un'applicazione con relativo server, in grado di tracciare in modo immutabile, certificato e in \glo{real-time}, le presenze alle postazioni di lavoro di una stanza, lo stato di pulizia di esse, la prenotazione di una postazione da remoto e la consultazione dei locali utilizzati dall'ultima igienizzazione.

\subsection{Scopo del capitolato}
La scopo del capitolato consiste nello sviluppo di un'applicazione (android o IOS) che permetterà le seguenti operazioni:
\begin{itemize}
	\item Recupero lista delle postazioni libere;
	\item Prenotazione di una postazione;
	\item Tracciamento in tempo reale tramite la scanerizzazione del \glo{tag RFID} relativo ad una specifica postazione;
	\item Segnalazione della pulizia autonoma di una postazione;
	\item Storico delle postazioni occupate;
	\item Storico delle postazioni igienizzate.
\end{itemize}

Il server, dotato di \glo{UI}, dovrà ricevere i vari tag RFID dall'applicazione e permetterà di:
\begin{itemize}
	\item Sapere in ogni momento se la postazione \'e occupata, prenotata oppure da pulire;
	\item Controllare quali postazioni sono prenotate (da chi) e bloccare le prenotazioni per una determinata stanza;
	\item Mostrare una vista a calendario, con l’elenco delle postazioni prenotate e in quali giorni;
	\item Salvare su memoria immutabile e certificata tutti i cambiamenti di stato relativi alla pulizia della postazione, nonch\'e le informazioni su chi l'ha igienizzata;
	\item Prenotare una postazione con granularit\'a di 1 ora;
	\item Creare, modificare o eliminare le utenze ai dipendenti e agli addetti delle pulizie;
	\item Creare, modificare o eliminare la definizione delle stanze e delle postazioni.
\end{itemize}

\subsection{Tecnologie coinvolte}
L'azienda consiglia di utilizzare:
\begin{itemize}
\item \textit{\glo{Java}} (versione 8 o superiori), \textit{\glo{Python} o \glo{Node.js}} per lo sviluppo del server \glo{back-end};
\item Protocolli \glo{asincroni} per le comunicazioni app mobile-server;
\item Sistema \glo{\textit{Blockchain}} per salvare con opponibilit\'a a terzi i dati di sanificazione;
\item \textit{\glo{IAAS Kubernetes} o di \glo{PAAS}, \glo{Openshift} o \glo{Rancher}}, per il rilascio delle componenti del server e la gestione della \glo{scalabilit\'a orizzontale}.
\end{itemize}

\subsection{Vincoli}
Per raggiungere gli obiettivi minimi del progetto viene richiesto di:
\begin{itemize}
\item Avere il server che esponga, in aggiunta a eventuali altri protocolli per l’interazione con il servizio specifico, delle \glo{API} \textit{\glo{REST}} attraverso le quali sia possibile utilizzare l'applicativo. In alternativa \'e possibile utilizzare \glo{gRPC} come soluzione alternativa a \textit{REST};
\item Scansione dei tag nel tempo sufficiente a certificare la presenza della persona in postazione;
\item Avere le componenti applicative correlate da \glo{test d'unità} e \glo{test d’integrazione}. Viene inoltre richiesto che il sistema sia testato nella sua interezza tramite \glo{test end-to-end} e che la copertura dei test d'unità sia $\geq$ 80\%.
\end{itemize}

\subsection{Aspetti positivi}
\begin{itemize} 
	\item 4 componenti del gruppo hanno già sviluppato e conoscono \glo{\textit{Java}}.
	\item Le tecnologie proposte dal capitolato stimolano i componenti del gruppo a imparare e a lavorare con esse.
	\item Il capitolato può essere molto utile e interessante per una prospettiva lavorativa, data la complessità, le tecnologie e il tipo di progetto.
\end{itemize}

\subsection{Aspetti critici}
\begin{itemize}
\item \'E richiesto sia lo sviluppo di un'interfaccia WEB per il server sia lo sviluppo di un'applicazione mobile per i vari tipo di utenti.
\end{itemize}

\subsection{Conclusioni}
Dopo un'attenta valutazione e discussione, il gruppo era molto propenso verso la presa in considerazione di questo capitolato come prima scelta, ma data la complessità e la competizione degli altri gruppi si è preferito non sceglierlo.
