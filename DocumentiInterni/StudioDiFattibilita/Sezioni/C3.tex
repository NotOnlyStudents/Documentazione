\section{Capitolato C3 - Gathering Detection Platform}\label{C3}
\subsection{Informazioni sul capitolato}
\begin{itemize}
	\item \textbf{Nome:} \textit{GDP - Gathering Detection Platform} 
	\item \textbf{Proponente:} \textit{SyncLab}
	\item \textbf{Committenti:} \textit{\VT{} e \CR{}}
\end{itemize}

\subsection{Descrizione del capitolato}
Il \glo{capitolato} ha come obiettivo finale la realizzazione di un \glo{prototipo} software in grado di acquisire, \glo{monitorare}, utilizzare, \glo{correlare} tra loro tutti i dati e le informazioni generate dai sistemi e dai dispositivi installati ed operativi in specifiche zone, con l’intento di identificare i possibili eventi che concorrono all’insorgere di variazioni di flussi di utenti.

\subsection{Scopo del capitolato}
Lo scopo del progetto consiste nello sviluppo di un sistema che riesca a predire il numero di persone che saranno presenti nei prossimi 10 minuti in un dato negozio oppure in un mezzo di trasporto. I vari dati elaborati dovranno essere disponibili in \glo{real-time} all'utente attraverso un'applicazione web di dashboard. \\
Essa dovrà fornire:
\begin{itemize}
	\item Una previsione del flusso di utenti negli intervalli futuri;
	\item Un confronto e una correlazione tra dati provenienti da flussi diversi, come strade, mezzi di trasporto e fermate degli autobus.
\end{itemize}

\subsection{Tecnologie coinvolte}
L'azienda consiglia di utilizzare:
\begin{itemize}
	\item \glo{Java} e \glo{Angular} per lo sviluppo delle parti di \glo{back-end} e \glo{front-end} della componente \\glo{Web Application} del sistema;
	\item Il \glo{framework} \glo{Leaflet} per la gestione delle mappe;
	\item Utilizzo di protocolli \glo{asincroni} per le comunicazioni tra le diverse componenti;
	\item Utilizzo del \glo{pattern} \glo{Publisher}/\glo{Subscriber} e adozione del protocollo \glo{MQTT}. 
\end{itemize}

\subsection{Vincoli}
Per raggiungere gli obiettivi minimi del progetto viene richiesto di:
\begin{itemize}
	\item Avere il server, completo di \glo{UI}, in grado di soddisfare i requisiti espressi;
	\item Test che dimostrano il corretto funzionamento dei servizi e delle funzionalit\'a previste (copertura di test $\geq$ 80\% correlata di report);
	\item Presentare la documentazione su scelte implementative e progettuali effettuate con relative motivazioni e documentazione su problemi aperti e eventuali soluzioni proposte.
\end{itemize}

\subsection{Aspetti positivi}
\begin{itemize}
	\item L'azienda mette a disposizione figure di diverso livello, in modo da supportare al meglio tutte le esigenze degli studenti. In caso di bisogno, fornirà dei server nei quali gli studenti potranno effettuare le installazioni dei componenti applicativi sviluppati.
	\item L'analisi in real-time e la predizione dei dati stimola molto il gruppo, dato l'interesse sempre cresente verso questi argomenti. 
	\item 4 componenti del gruppo hanno già sviluppato e conoscono Java.
\end{itemize}

\subsection{Aspetti critici}
\begin{itemize}
	\item La raccolta dei dati necessari per sviluppare il progetto pu\'o essere ritenuto uno dei principali punti critici.
	\item Nessun componente del gruppo ha mai sviluppato con il framework Leaflet e Angular.
\end{itemize}

\subsection{Conclusioni}
Dopo un'attenta valutazione e discussione, il gruppo ha escluso il seguente capitolato in quanto non è stato sufficientemente apprezzato e la difficoltà nella raccolta dei dati è molto elevata.
