\section{Capitolato C6 - Realtime Gaming Platform}
\label{C6}
\subsection{Informazioni sul capitolato}
\begin{itemize}
	\item \textbf{Nome:} RGP: Realtime Gaming Platform;
	\item \textbf{Proponente:} zero12;
	\item \textbf{Committenti:} \VT{} e \CR{}.
\end{itemize}

\subsection{Descrizione del capitolato}
Il \glo{capitolato} richiede lo sviluppo di un videogioco a scorrimento verticale utilizzabile sia in modalità \glo{multiplayer}, da 2 a 6 giocatori, che single-player. Il progetto si focalizza maggiormente sulla componente server che, tramite tecnologie \glo{AWS}, permette di gestire la comunicazione in tempo reale tra i diversi dispositivi collegati nella stessa sessione di gioco.

\subsection{Scopo del capitolato}
Lo scopo del capitolato consiste nello sviluppo di un gioco per iOS o Android che soddisfi i seguenti obiettivi:
\begin{itemize}
	\item Deve essere multiplayer da 2 a 6 giocatori;
	\item Deve avere una modalità fantasma;
	\item Devono essere presenti gli stessi \glo{powerup} e nemici durante il multiplayer;
	\item Il gioco deve essere in modalità infinita;
	\item Il server deve essere realizzato con tecnologie \glo{Serverless}.
\end{itemize}

\subsection{Tecnologie coinvolte}
Per questo progetto sono state raccomandate le seguenti tecnologie:
\begin{itemize}
	\item \glo{AWS GameLift}: servizio di hosting per i giochi online;
	\item \glo{AWS appsync}: servizio gestito per lo sviluppo rapido di \glo{API} GraphQL;
	\item \glo{AWS DynamoDB}: servizio di database \glo{NoSQL} rapido e flessibile per il salvataggio dei dati in \glo{cloud};
	\item \glo{Node.js} come linguaggio utilizzato per lo sviluppo di componenti server;
	\item \glo{Swift} e \glo{Kotlin} come linguaggi di programmazione per lo sviluppo mobile, il primo nel caso si scelga la realizzazione dell'applicativo su iOS mentre il secondo se si opta per Android.
\end{itemize}

\subsection{Vincoli}
Per raggiungere gli obiettivi minimi del progetto viene richiesto di:
\begin{itemize}
	\item Svolgere un'\glo{attività} di analisi per scegliere la migliore tecnologia basata su AWS su cui sviluppare la componente server del progetto fornendo una giustificazione di tale scelta;
	\item Sviluppare le API preferibilmente in Node.js;
	\item Sviluppare l'intero progetto a \glo{microservizi};
	\item Sviluppare un'applicazione mobile per testare la correttezza delle dinamiche del gioco in \glo{real-time}.
\end{itemize}

 \subsection{Aspetti positivi}
\begin{itemize}
	\item Utilizzo di vari servizi AWS sempre più diffusi.
	\item L'uso di un'architettura a microservizi e Serverless favorisce la parallelizzazione dello sviluppo del \glo{back end}.
\end{itemize}

 \subsection{Aspetti critici}
\begin{itemize}
	\item Sviluppare un gioco per iOS è condizionato dal possedere un MacBook o un iMac, inoltre Swift è conosciuto solo da pochi componenti del gruppo e non in modo approfondito;
	\item Tutti i componenti del gruppo non hanno mai sviluppato un'applicazione;
	\item Lo sviluppo di un'applicazione non stimola l'interesse di alcun componente del gruppo.
\end{itemize}

\subsection{Conclusioni}
Dopo un'attenta valutazione e discussione, si è deciso che questo capitolato, nonostante tocchi delle tecnologie piuttosto interessanti, richiede un dispendio troppo elevato di tempo per l'apprendimento dei linguaggi di programmazione.