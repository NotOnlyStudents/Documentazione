\section{Capitolato C5 - PORTACS}\label{C5}
\subsection{Informazioni sul capitolato}
\begin{description}
	\item \textbf{Nome:} \textit{PORTACS: piattaforma di controllo mobilità autonoma}.
	\item \textbf{\glo{Proponente}:} \textit{Sanmarco Informatica}.
	\item \textbf{Committente:} \textit{\VT{} e \CR{}}.
\end{description}

\subsection{Descrizione del capitolato}
Il \glo{capitolato} richiede lo sviluppo di un sistema \glo{real-time} in grado di gestire più entità caratterizzate da un punto di partenza, una velocità di crociera ed una lista di \glo{punti di interesse} da raggiungere.

\subsection{Scopo del capitolato}
Lo scopo del capitolato consiste nello sviluppare un motore real-time che si occupa di trovare il migliore percorso da far percorrere alla singola unità.
La componente server deve segnalare ad ogni unità la prossima azione da fare in base ai prossimi punti e alle posizioni delle altre entità. Deve essere presente una mappa real-time che consenta la visualizzazione di ogni unità e dei loro movimenti all'interno della struttura.
Ogni unità dovrà implementare un'interfaccia grafica dove vengono indicate le informazioni caratterizzanti.

\subsection{Tecnologie coinvolte}
Il capitolato offre libera scelta su come implementare l'applicazione, l'unico limite è nell'utilizzo della tecnologia \textit{\glo{Docker}} per il rilascio. Il gruppo aveva pensato di utilizzare le seguenti tecnologie:
\begin{itemize}
	\item Motore real-time: si pensava di usare \textit{\glo{Java}}, per la facilità di gestione degli oggetti e dei \glo{thread};
	\item Monitor real-time: utilizzo di \textit{\glo{Tomcat} e \glo{Angular}} per l'interfaccia utente;
	\item Simulatori delle unità operative: si ipotizzava di optare per l'uso di \textit{Java} così da facilitare la connessione, attraverso \glo{socket}, al motore real-time.
\end{itemize}

\subsection{Vincoli}
Per raggiungere gli obiettivi minimi del progetto viene richiesto che:
\begin{itemize}
\item Il sistema sviluppato sia in real-time;
\item Sia presente una scacchiera o mappa con la definizione delle percorrenze, relativi vincoli e punti di interesse;
\item Le N unità dovranno essere definite da:
	\begin{itemize}
	\item Identificativo di sistema;
	\item Velocità massima;
	\item Posizione iniziale;
	\item Lista dei punti di interesse da attraversare.
	\end{itemize}
\end{itemize}

\subsection{Aspetti positivi}
\begin{itemize}
	\item Non ci sono restrizioni particolari sulle tecnologie da usare, quindi possiamo scegliere quelle più conosciute dal gruppo.
	\item Il capitolato è sembrato molto interessante perchè offre un buon punto di partenza per entrare nel mondo della guida autonoma.
	\item Il capitolato offre la possibilità di utilizzo pratico di ciò che abbiamo studiato durante il corso di Ricerca Operativa.
\end{itemize}

\subsection{Aspetti critici}
\begin{itemize}
	\item Il gruppo ha scarse conoscenze riguardo le tecnologie \textit{Docker}.
	\item Si ritiene difficile realizzare un prodotto veramente \glo{efficace}.
	\item La quantità di studio nel rendere efficace l'invio dei dati e su come temporizzare l'acquisizione dei dati è molto elevata.
	\item Per tutti i membri del gruppo, sia il documento del capitolato che la successiva presentazione sono risultati dispersivi e poco chiari. 
\end{itemize}

\subsection{Conclusioni}
Dopo un'attenta valutazione e discussione, il gruppo ha valutato la complessità dell'intero progetto come troppo elevata e per questo motivo si è deciso di escludere il capitolato e di orientarsi verso alternative più stimolanti.
