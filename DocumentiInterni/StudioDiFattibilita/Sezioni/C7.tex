\section{Capitolato C7 - SSD}\label{C7}
\subsection{Informazioni sul capitolato}
\begin{itemize}
	\item \textbf{Nome:} SSD: Soluzioni di Sincronizzazione Desktop;
	\item \textbf{Proponente:} Zextras;
	\item \textbf{Committenti:} \VT{} e \CR{}.
\end{itemize}

\subsection{Descrizione del capitolato}
Il \glo{capitolato} richiede la realizzazione di un algoritmo solido ed efficiente in grado di garantire il salvataggio in \glo{cloud} del lavoro e contemporaneamente la sincronizzazione dei cambiamenti avvenuti.

\subsection{Scopo del capitolato}
Lo scopo del capitolato consiste nello sviluppo di un'applicazione desktop che soddisfi i seguenti obiettivi:
\begin{itemize}
	\item Sviluppo di un algoritmo solido ed efficiente: deve essere in grado di garantire il salvataggio in cloud del lavoro e contemporaneamente la sincronizzazione dei cambiamenti presenti in cloud;
	\item Sviluppo di un’interfaccia grafica \glo{cross-platform}: deve fornire un'interfaccia grafica per interagire con le funzionalità offerte dall'algoritmo di sincronizzazione e deve essere utilizzabile sui tre sistemi operativi maggiormente diffusi, ovvero Linux, MacOS, Windows. La soluzione sviluppata non deve necessitare dell'installazione da parte dell'utente di \glo{framework} di terze parti per il proprio funzionamento;
	\item Integrazione con \glo{Zextras Drive}: l'algoritmo di sincronizzazione e l'interfaccia grafica devono essere integrati con il servizio di gestione file Zextras Drive.
\end{itemize}
Altre funzionalità che dovrebbero essere aggiunte sono:
\begin{itemize}
	\item Configurazione e autenticazione dell'utente;
	\item Gestione dei file da sincronizzare o ignorare sia dal lato cloud che dal lato desktop e la possibilità di modificare in ogni momento la configurazione;
	\item Sincronizzazione costante dei cambiamenti, sia locali che remoti;
	\item Sistema di notifica dei cambiamenti;
	\item Funzionalità presenti in altre soluzioni attualmente disponibili:
	\begin{itemize}
		\item Gestione delle condivisioni;
		\item Integrazione con il prodotto web.
	\end{itemize}
\end{itemize}

\subsection{Tecnologie coinvolte}
Le seguenti tecnologie possono essere impiegate nello sviluppo del progetto:
\begin{itemize}
	\item \glo{Framework} Qt consigliato per lo sviluppo dell'interfaccia grafica;
	\item \glo{Python} consigliato come linguaggio per lo sviluppo della \glo{Business Logic}.
\end{itemize}

\subsection{Vincoli}
Per raggiungere gli obiettivi minimi del progetto viene richiesto che:
\begin{itemize}
	\item L'algoritmo di sincronizzazione e l'interfaccia grafica utente devono essere cross-platform senza l'installazione manuale di ulteriori prodotti;
	\item Forte distinzione tra interfaccia utente e Business Logic;
	\item L'applicazione desktop deve essere sviluppata seguendo il design pattern \glo{MVC}.
\end{itemize}

\subsection{Aspetti positivi}
\begin{itemize}
	\item Le tecnologie proposte dal capitolato suscitano interesse nel gruppo;
	\item Permette di approfondire gli algoritmi per la collaborazione a distanza, molto attuali e sempre più diffusi;
	\item Il team ha già familiarità con il framework Qt ed il linguaggio Python.
\end{itemize}

\subsection{Aspetti critici}
\begin{itemize}
	\item Si ha la paura di incontrare parecchi problemi durante lo sviluppo cross-platform.
\end{itemize}

\subsection{Conclusioni}
Dopo un'attenta valutazione e discussione, il progetto ha entusiasmato l'intero gruppo, ma non in maniera tale da porsi come prima scelta.