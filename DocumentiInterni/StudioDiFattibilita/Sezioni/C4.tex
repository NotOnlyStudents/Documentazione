\section{Capitolato C4 - HD VIZ}\label{C4}
\subsection{Informazioni sul capitolato}
\begin{itemize}
	\item \textbf{Nome:} \textit{HD VIZ: visualizzazione di dati con molte dimensioni}.
	\item \textbf{\glo{Proponente}:} \textit{Zucchetti Software}.
	\item \textbf{Committenti:} \textit{\VT{} e \CR{}}.
\end{itemize}

\subsection{Descrizione del capitolato}
Il \glo{capitolato} richiede lo sviluppo di un sito WEB per visualizzare dati con molte dimensioni a supporto della \glo{fase} esplorativa dell'analisi dei dati.

\subsection{Scopo del capitolato}
Lo scopo del capitolato consiste nello sviluppo di un sito WEB che mostrerà dei dati presenti in un database o caricati da un file \glo{CSV}, nelle seguenti visualizzazioni:
\begin{itemize}
	\item Scatter plot Matrix (fino ad un massimo di 5 dimensioni);
	\item Force Field;
	\item Heat Map;
	\item Proiezione Lineare Multi Asse.
\end{itemize}
Altre funzionalità che dovrebbero essere aggiunte sono:
\begin{itemize}
	\item Visualizzazione dei dati con altri grafici con più di tre dimensioni;
	\item Utilizzo di funzioni di calcolo della distanza, diverse dalla \glo{distanza Euclidea} in tutte le visualizzazioni che dipendono da tale concetto;
	\item Utilizzo di \glo{funzioni di forza} diverse da quelle previste in automatico dal grafico \glo{force based} di \textit{\glo{D3.js}};
	\item L'analisi automatica per evidenziare situazioni di particolare interesse;
	\item L'utilizzo di algoritmi di preparazione del dato per la visualizzazione, cioè anziché eseguire la trasformazione direttamente nella visualizzazione far precedere un passo di trasformazione.
\end{itemize}

\subsection{Tecnologie coinvolte}
Lo sviluppo avverrà prevalentemente con tecnologie \textit{\glo{HTML}/\glo{CSS}/\glo{JavaScript}} utilizzando la libreria \textit{D3.js}.
La parte server, di supporto alla presentazione nel browser e utile all'esecuzione delle query ad un database \textit{\glo{SQL} o \glo{NoSQL}}, potr\'a essere sviluppata in \textit{\glo{Java} con \glo{Tomcat}} o in \textit{Javascript} con \textit{\glo{Node.js}}.

\subsection{Vincoli}
Per raggiungere gli obiettivi minimi del progetto viene richiesto che:
\begin{itemize}
	\item I dati da visualizzare dovranno poter avere almeno fino a 15 dimensioni, ma deve essere possibile visualizzare anche dati con meno dimensioni;
	\item I dati devono poter essere forniti al sistema di visualizzazione sia con query ad un database che da file in formato \glo{CSV} preparati precedentemente;
	\item Dovrà presentare almeno le seguenti visualizzazioni:
	\begin{itemize}
		\item Scatter plot Matrix: presentazione a riquadri disposti a matrice di tutte le combinazioni di scatter plot;
		\item Force Field: traduce le distanze nello spazio a molte dimensioni di forze di attrazione e repulsione tra i punti proiettati nello spazio bidimensionale;
		\item Heat map: trasforma la distanza tra i punti di colori più o meno intensi;
		\item posiziona i punti dello spazio multidimensionale in un piano
		cartesiano.
	\end{itemize}
	\item Ordinare i punti nel grafico “Heat map” per evidenziare i \glo{cluster} presenti nei dati.
\end{itemize}

\subsection{Aspetti positivi}
\begin{itemize}
	\item Tutti i componenti conoscono \textit{HTML, CSS e JavaScript} impiegate nell'ambito dello sviluppo WEB.
	\item Tratta un ambito moderno come lo studio dei \glo{BigData}.
\end{itemize}

\subsection{Aspetti critici}
\begin{itemize}
	\item Il tema trattato e le tecnologie proposte dal capitolato non suscitano l'interesse del gruppo.
\end{itemize}

\subsection{Conclusioni}
Dopo un'attenta valutazione e discussione, il gruppo ha deciso di non prendere in considerazione il seguente capitolato dato lo scarso interesse per il tema e le tecnologie proposte.
