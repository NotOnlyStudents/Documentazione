\section{Capitolato C4 - HD VIZ}\label{C4}
\subsection{Informazioni sul capitolato}
\begin{itemize}
	\item \textbf{Nome:} HD VIZ: visualizzazione di dati con molte dimensioni;
	\item \textbf{Proponente:} Zucchetti Software;
	\item \textbf{Committenti:} \VT{} e \CR{}.
\end{itemize}

\subsection{Descrizione del capitolato}
Il \glo{capitolato} richiede lo sviluppo di un sito web per visualizzare dati con molte dimensioni a supporto della \glo{fase} esplorativa dell'analisi dei dati.

\subsection{Scopo del capitolato}
Lo scopo del capitolato consiste nello sviluppo di un sito web che mostrerà dei dati presenti in un database o caricati da un file CSV, nelle seguenti visualizzazioni:
\begin{itemize}
	\item Scatter plot matrix (fino ad un massimo di 5 dimensioni);
	\item Force field;
	\item Heat map;
	\item Proiezione lineare multi asse.
\end{itemize}
Altre funzionalità che dovrebbero essere aggiunte sono:
\begin{itemize}
	\item Visualizzazione dei dati con altri grafici con più di tre dimensioni;
	\item Utilizzo di funzioni di calcolo della distanza, diverse dalla distanza Euclidea in tutte le visualizzazioni che dipendono da tale concetto;
	\item Utilizzo di \glo{funzioni di forza} diverse da quelle previste in automatico dal grafico \glo{force based} di \glo{D3.js};
	\item L'analisi automatica per evidenziare situazioni di particolare interesse;
	\item L'utilizzo di algoritmi di preparazione del dato per la visualizzazione, cioè anziché eseguire la trasformazione direttamente nella visualizzazione far precedere un passo di trasformazione.
\end{itemize}

\subsection{Tecnologie coinvolte}
Lo sviluppo avverrà prevalentemente con tecnologie \glo{HTML}/\glo{CSS}/\glo{JavaScript} utilizzando la libreria D3.js.
La parte server, di supporto alla presentazione nel browser e utile all'esecuzione delle query ad un database \glo{SQL} o \glo{NoSQL}, potrà essere sviluppata in \glo{Java} con \glo{Tomcat} o in Javascript con \glo{Node.js}.

\subsection{Vincoli}
Per raggiungere gli obiettivi minimi del progetto viene richiesto che:
\begin{itemize}
	\item I dati da visualizzare dovranno poter avere almeno fino a 15 dimensioni, ma deve essere possibile visualizzare anche dati con meno dimensioni;
	\item I dati devono poter essere forniti al sistema di visualizzazione sia con query ad un database che da file in formato CSV preparati precedentemente;
	\item Dovrà presentare almeno le seguenti visualizzazioni:
	\begin{itemize}
		\item \textbf{Scatter plot matrix}: presentazione a riquadri disposti a matrice di tutte le combinazioni di scatter plot;
		\item \textbf{Force field}: traduce le distanze nello spazio a molte dimensioni di forze di attrazione e repulsione tra i punti proiettati nello spazio bidimensionale;
		\item \textbf{Heat map}: trasforma la distanza tra i punti di colori più o meno intensi;
		\item \textbf{Proiezione lineare multi asse}:posiziona i punti dello spazio multidimensionale in un piano
		cartesiano.
	\end{itemize}
	\item Ordinare i punti nel grafico "Heat map" per evidenziare i \glo{cluster} presenti nei dati.
\end{itemize}

\subsection{Aspetti positivi}
\begin{itemize}
	\item Tutti i componenti conoscono HTML, CSS e JavaScript impiegate nell'ambito dello sviluppo web;
	\item Tratta un ambito moderno come lo studio dei \glo{BigData}.
\end{itemize}

\subsection{Aspetti critici}
\begin{itemize}
	\item Il tema trattato e le tecnologie proposte dal capitolato non suscitano l'interesse del gruppo.
\end{itemize}

\subsection{Conclusioni}
Dopo un'attenta valutazione e discussione, il gruppo ha deciso di non prendere in considerazione il seguente capitolato dato lo scarso interesse per il tema e le tecnologie proposte.