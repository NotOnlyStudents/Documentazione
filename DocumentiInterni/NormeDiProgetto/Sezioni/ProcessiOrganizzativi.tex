\section{Processi Organizzativi}
\label{PO}
\subsection{Descrizione}\label{PO_Descrizione}
I Processi Organizzativi comprendono tutti i processi utilizzati per definire e gestire una struttura costituita dal personale e dai processi del ciclo di vita del prodotto, al fine di migliorarne continuamente i processi e la struttura stessa. 
In questa sezione vengono analizzati:
\begin{itemize}
	\item \textbf{Gestione di processo};
	\item \textbf{Gestione dell'infrastruttura};
	\item \textbf{Miglioramento del processo};
	\item \textbf{Formazione del personale}.
\end{itemize}

\subsection{Gestione di processo}
\subsubsection{Scopo del processo}\label{PO_GestioneProcesso_Scopo}
Come stabilito dallo \glo{standard} \glo{ISO/IEC 12207:1995}, comprende le attività e i compiti generici per la gestione dei rispettivi processi. 
\subsubsection{Descrizione della sezione}
La sezione definisce le norme stabilite dal gruppo per la Pianificazione e il Coordinamento del lavoro dei componenti del gruppo.
\subsubsection{Aspettative}
Le aspettative del gruppo per questo processo sono:
\begin{itemize}
	\item Pianificare in modo ragionevole le attività da seguire;
	\item Gestire i componenti del gruppo assegnando ruoli e compiti;
	\item Monitorare l'andamento del progetto.
\end{itemize}
\subsubsection{Pianificazione}
\myparagraph{Scopo dell'attività}
Lo scopo dell'attività di Pianificazione è di definire un metodo di lavoro stabilendone fattibilità e assicurandone la sua esecuzione.
\myparagraph{Descrizione della sezione}
Nella sezione vengono stabiliti compiti e ruoli di progetto, gestione dei rischi. 
\myparagraph{Aspettative}
Con l'attività di Pianificazione il gruppo intende definire il metodo di lavoro che viene seguito durante la realizzazione del progetto.
\myparagraph{Ruoli del progetto}
I ruoli previsti dal progetto sono:
\paragraph*{Responsabile}
Coordina e organizza il lavoro e rappresenta il gruppo presso il proponente e i committente per tutta la durata del progetto. \\
I suoi compiti principali sono:
\begin{itemize}
	\item Predisporre l'attività di Pianificazione e assicurarne l'esecuzione;
	\item Approvare i documenti;
	\item Redigere \textit{Organigramma} e \PdPv;
	\item Coordinare le attività, le risorse e i componenti del gruppo.
\end{itemize}

\paragraph*{Amministratore}
Gestisce, controlla e cura gli strumenti che il gruppo utilizza per il proprio lavoro. \\
I suoi compiti principali sono:
\begin{itemize}
	\item Gestire i problemi legati alla gestione dei processi;
	\item Gestire la documentazione del progetto;
	\item Controlla versioni e configurazioni del prodotto;
	\item Redigere le \NdPv.
\end{itemize}

\paragraph*{Analista}
Svolge tutte le attività necessarie all'Analisi dei Requisiti.\\
I suoi compiti principali sono:
\begin{itemize}
	\item Studiare il dominio del problema;
	\item Definire tutti i requisiti del prodotto richiesto;
	\item Redigere \SdFv\ e \AdRv.
\end{itemize}

\paragraph*{Progettista}
È responsabile della progettazione del prodotto partendo da quanto svolto dall'\textit{Analista}. \\
I suoi compiti principali sono:
\begin{itemize}
	\item Seguire lo sviluppo del prodotto;
	\item Influire sulle scelte tecniche e tecnologiche;
	\item Redigere la documentazione dell'architettura del prodotto.
\end{itemize}

\paragraph*{Programmatore}
È responsabile delle attività di codifica del prodotto e delle componenti necessarie per la sua verifica e validazione. \\
I suoi compiti principali sono:
\begin{itemize}
	\item Codificare quanto previsto dal \textit{Progettista};
	\item Codificare i componenti ausiliari di supporto alla verifica e alla validazione;
	\item Redigere i \textit{Manuali}.
\end{itemize}

\paragraph*{Verificatore}
È responsabile dell'attività di verifica. \\
I suoi compiti principali sono:
\begin{itemize}
	\item Controllare che le attività di processo non abbiano causato errori;
	\item Redigere la parte retrospettiva del \PdQv\ che descrive l'esito delle verifiche e delle prove effettuate.
\end{itemize}

\myparagraph{Gestione dei compiti}\label{GestioneCompiti}
Durante lo sviluppo del progetto spetta al singolo componente stabilire in piena autonomia quali sono i suoi compiti a seconda del proprio ruolo.
Per permettere di gestire i compiti in modo agevole è stato scelto di utilizzare l'\glo{Issue Tracking System} fornito da \glo{GitHub}: il \glo{repository} usato dal gruppo per il \glo{versionamento} del progetto.
Il componente del gruppo in base al proprio ruolo crea delle \glo{Issue} che rappresentano dei singoli compiti, che assegna a se stesso o ad altri componenti in seguito a riunioni e secondo accordi, comprendenti le seguenti informazioni:
\begin{itemize}
	\item \textbf{Titolo:} nome del compito da eseguire;
	\item \textbf{Descrizione:} descrizione dettagliata del compito da eseguire;
	\item \textbf{Assegnatari:} persone a cui compete lo svolgimento del compito;
	\item \textbf{Project:} cruscotto di progetto in cui il compito sarà monitorato;
	\item \textbf{Milestone:} data entro la quale lo svolgimento del compito deve essere completato.
\end{itemize}
Ogni Issue attraversa degli stati che permettono di monitorare l'avanzamento nello svolgimento del compito che essa rappresenta, questi sono:
\begin{itemize}
	\item \textbf{To do:} compito da svolgere;
	\item \textbf{In progress:} compito in svolgimento;
	\item \textbf{Done:} compito svolto.
\end{itemize}

\myparagraph{Gestione dei rischi}\label{GestioneRischi}
Tutti i rischi che si possono presentare durante la realizzazione del progetto vengono inseriti all'interno del \PdPv\ redatto dal \Responsabile.
Viene seguita la seguente procedura:
\begin{itemize}
	\item Individuazione di nuovi problemi e monitoraggio dei rischi già individuati;
	\item Aggiunta dei nuovi rischi nel \PdPv;
	\item Ridefinizione se necessario delle strategie di gestione.
\end{itemize}
Per consentire l'identificazione univoca ogni rischio viene individuato con un codice secondo il seguente schema:
\begin{center}
	\textbf{R[Tipologia][Numero]}
\end{center}
dove: 
\begin{itemize}
	\item \textbf{Tipologia}: identifica uno dei seguenti tipi di rischio:
		\begin{itemize}
			\item \textbf{T}: rischio legato alla tecnologia;
			\item \textbf{G}: rischio legato ai membri del gruppo;
			\item \textbf{S}: rischio legato agli strumenti;
			\item \textbf{O}: rischio legato all'organizzazione del lavoro;
			\item \textbf{R}: rischio legato ai requisiti.
		\end{itemize}
	\item \textbf{Numero}: incrementale a partire da 1, per ogni tipologia si riparte da 1.
\end{itemize}
Nel \PdPv  per ogni rischio è presente una tabella indicante:
\begin{itemize}
	\item \textbf{Codice identificativo e titolo}: secondo quanto indicato precedentemente e un breve titolo;
	\item \textbf{Descrizione};
	\item \textbf{Conseguenze} causate dal verificarsi del rischio;
	\item \textbf{Probabilità di occorrenza};
	\item \textbf{Pericolosità};
	\item \textbf{Precauzioni};
	\item \textbf{Piano di contingenza};
\end{itemize}

\myparagraph{Metriche}

\subsubsection{Coordinamento}
\myparagraph{Scopo dell'attività}\label{PO_Coordinamento_Scopo}
L'attività di Coordinamento ha lo scopo di definire come vengono gestite le comunicazioni all'interno del gruppo \Gruppo e tra il gruppo e soggetti esterni.
\myparagraph{Descrizione della sezione}
La sezione presenta le norme che il gruppo segue per le comunicazioni e le riunioni.
\myparagraph{Aspettative}
L'aspettativa del gruppo per questa attività è di stabilire metodi chiari di comportamento da seguire durante lo svolgimento del progetto.
\myparagraph{Comunicazioni}
 I membri del gruppo sono tenuti a seguire quanto indicato per tutta la durata del progetto.
\paragraph*{Comunicazioni interne}
I membri utilizzano per le comunicazioni interne il gruppo Telegram appositamente creato. Per tutte le comunicazioni importanti viene preferito utilizzare Zoom effettuando quindi delle videochiamate che permettono un feedback immediato.
\paragraph*{Comunicazioni esterne}
I soggetti esterni con cui il gruppo può comunicare sono: 
\begin{itemize}
	\item \textbf{I committenti} \VT\ e \CR;
	\item \textbf{L'azienda \Proponente} rappresentata da ;
	\item \textbf{Gruppi concorrenti} che lavorano allo stesso progetto \NomeProgetto\ individuati in \textit{Milo Ertola} e \textit{Alessandro Maccagnan};
\end{itemize}
Per le comunicazioni con il proponente si rimanda alla sezione \S\ref{Rapporti RedBabel}. \\
Le comunicazioni con i committenti avvengono esclusivamente tramite l'email del gruppo
\begin{center}
	 \textbf{\Mail} 
\end{center} ed eventualmente tramite videochiamate Zoom.\\
Le comunicazioni con i gruppi concorrenti avvengono in modo informale tramite Telegram.
\myparagraph{Riunioni}
In caso di necessità, i membri possono proporre di organizzare una riunione per discutere di eventuali problemi o dubbi. Ogni incontro dovrà essere fissato in accordo con i partecipanti in base alle loro disponibilità, se un componente è impossibilitato a partecipare per un periodo eccessivamente esteso accetta quanto stabilito dagli altri componenti. Gli incontri possono avvenire tra i soli membri del gruppo o tra il gruppo e i committenti/proponenti. 
Tutto ciò che viene stabilito durante le riunioni viene inserito in un Verbale da un segretario incaricato. Per la struttura dei Verbali si rimanda alla sezione \S\ref{StrutturaDocumenti}.
\paragraph*{Incontri interni}
Agli incontri interni partecipano solamente i membri del gruppo in modalità virtuale tramite viedochiamate Zoom di gruppo.
Se e quando la situazione attuale legata all'emergenza sanitaria migliorerà e sarà possibile accedere liberamente alle aule dell'Università di Padova, il gruppo può decidere se svolgere le riunioni in presenza presso le aule di Torre Archimede - Dipartimento di Matematica "Tullio Levi Civita" (Via Trieste 63, 35121 Padova(PD)).
Tra un incontro e un altro viene creato un documento condiviso in Google Drive nel quale ogni componente può inserire gli argomenti che vuole trattare nel prossimo incontro, questi costiuiranno l'ordine del giorno della riunione.
\paragraph*{Incontri esterni}
Negli incontri esterni, assieme ai membri del gruppo, sono coinvolti anche uno o più rappresentanti dell'azienda proponente.
Gli incontri esterni con l'azienda proponente avvengono secondo gli accordi (vedi sezione \S\ref{Rapporti RedBabel}). Se si necessita di discutere con i committenti il \Responsabile\ ha il compito di contattarli e organizzare l'incontro secondo le disponibilità del \VT\ e del \CR.
\myparagraph{Norme generali}\label{NormeGenerali}
I componenti del gruppo si impegnano a:
\begin{itemize}
	\item Essere il più possibile reperibili;
	\item Informare tempestivamente il gruppo in caso di promblemi di salute e/o famigliare che non cosentono il corretto svolgimento dei propri compiti;
	\item Aggiornare il \Responsabile\ a intervalli regolari rispetto il lavoro in svolgimento;
	\item Informare tempestivamente il gruppo se si hanno difficoltà nello svolgimento dei compiti di propria competenza.
\end{itemize}

\subsection{Gestione dell'infrastruttura}
\subsubsection{Scopo del processo}\label{PO_GestioneInfrastruttura_Scopo}
Il Processo di Gestione dell'infrastruttura ha lo scopo di stabilire tutti gli strumenti utilizzati per gestire l'infrastruttura necessaria per tutti gli altri processi.
\subsubsection{Descrizione della sezione}
La sezione indica quali strumenti vengono utilizzati dal gruppo per il coordinamento e la pianificazione del lavoro durante la realizzazione del progetto.
\subsubsection{Aspettative}
I componenti del gruppo utilizzano quanto previsto di seguito per operare senza ambiguità e coerentemente con quanto stabilito.
\subsubsection{Strumenti utilizzati}
Gli strumenti utilizzati dal gruppo sono:
\begin{itemize}
	\item \textbf{\textit{Zoom}}: strumento utilizzato per le video-conferenze tra i membri del gruppo.
	\item \textbf{\textit{Telegram}}: strumento di messaggistica istantanea basato su cloud, utilizzato dal gruppo \Gruppo\ per le comunicazioni tra i vari membri.
	\item \textbf{\textit{Gmail}}: strumento open-source di posta elettronica, utilizzato per gestire la casella e-mail del gruppo.
	\item \textbf{\textit{Google Drive}}: servizio web basato su cloud per la memorizzazione e condivisione di file, utilizzato dal gruppo per condividere file utili ai propri membri.
	\item \textbf{\textit{Google Calendar}}: servizio web basato su cloud utilizzato per avere accesso ad un calendario come in cui segnare incontri e impegni personali.
	\item \textbf{\textit{Slack}}: strumento collaborativo aziendale, utilizzato per inviare messaggi in modo istantaneo ai membri di un team di lavoro. Come indicato nella sezione \S\ref{Rapporti RedBabel} viene utilizzato per eventuali comunicazioni con il proponente \Proponente.
	\item \textbf{\textit{Google Meet}}: strumento utilizzato per le video-conferenze tra i membri del gruppo e il proponente \Proponente.
	\item \textbf{GanttProject}: utilizzato per la Pianificazione e la realizzazione dei \glo{diagrammi di Gantt}. 
\end{itemize}

\subsection{Miglioramento del processo}
\subsubsection{Scopo del processo}
Il Processo di Miglioramento ha lo scopo di stabilire, valutare, misurare, controllare e migliorare il ciclo di vita del software.
\subsubsection{Descrizione della sezione}
La sezione analizza le tre attività principali del Processo di Miglioramento, ovvero:
\begin{itemize}
	\item \textbf{Istituzione del processo};	
	\item \textbf{Valutazione del processo};
	\item \textbf{Miglioramento del processo}.
\end{itemize}
\subsubsection{Aspettative}
Il gruppo si pone come aspettativa di questo processo la pianificazione del lavoro nel rispetto del principio di miglioramento continuo.
\subsubsection{Norme generali}\label{PO_NormeGenerali}
All'interno del \PdQv{}\ vengono inseriti gli obiettivi e le metriche di qualità per valutare l'andamento delle attività di ogni processo. Qualora si presentassero problemi il membro del gruppo deve avvertire tempestivamente il gruppo che procederà ad una eventuale modifica di quanto pianificato. Periodicamente vengono calcolate le qualità previste.

\subsection{Formazione del personale}
\subsubsection{Scopo del processo}
Il Processo di Formazione consente di pianificare la preparazione dei componenti del gruppo rispetto a tecnologie e strumenti utilizzati durante la realizzazione del progetto.
\subsubsection{Descrizione della sezione}
La sezione comprende le norme stabilite dal gruppo per l'autonoma fomazione dei componenti.
\subsubsection{Aspettative}
Attraverso questo processo il gruppo intende acquisire le competenze necessarie per poter realizzare il prodotto con la qualità stabilita in modo efficace ed efficiente.
\subsubsection{Decisioni sulla formazione}\label{DecisioniFormazione}
I membri del gruppo provvedono ad approfondire le proprie conoscenze e a colmare le proprie lacune sulle tecnologie e sugli strumenti che vengo utilizzati durante tutti i processi. I membri più esperti hanno il compito di condividere le proprie conoscenze per velocizzare il Processo di Formazione secondo il principio di miglioramento continuo. 