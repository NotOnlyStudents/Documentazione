\subsection{Documentazione}
\subsubsection{Scopo dell'attività} 
Quando si realizza un progetto è di fondamentale importanza documentare processi e attività significative per lo sviluppo dello stesso. \\
La Documentazione permette di:
\begin{itemize}
	\item Disciplinare ciò che verrà fatto durante il ciclo di vita software;
	\item Garantire che tutte le informazioni d'interesse siano facilmente accessibili agli stakeholders;
	\item Assicurare che quanto previsto dalla Documentazione sia rispettato per garantire la qualità attesa.
\end{itemize}

\subsubsection{Descrizione della sezione} 
La sezione comprende le norme relative alla documentazione associata al progetto che tutti i componenti devono seguire nella redazione dei documenti di loro competenza.

\subsubsection{Aspettative}
L'obiettivo del gruppo è di ottenere dei prodotti documentali coerenti e validi dal punto di vista tipografico e formale.
I documenti che prodotti dal gruppo devono:
\begin{itemize}
	\item Essere facilmente aggiornabili e modificabili;
	\item Contenere solo ciò che è essenziale;
	\item Essere facili da leggere.
\end{itemize}

\subsubsection{Ciclo di vita di un documento}
Tutti i documenti prodotti sono sottoposti a queste attività:
\begin{itemize}
	\item \textbf{Creazione: }il documento viene creato secondo il template presente nella repository;
	\item \textbf{Stesura: }il documento viene redatto dai componenti incaricati, tutti i documenti devono contenere:
	\begin{itemize}
		\item Registro delle Modifiche;
		\item Indice dei Contenuti;
	\end{itemize}
	\item \textbf{Revisione: }le sezioni del documento vengono regolarmente revisionate da uno o più componenti del gruppo diversi dai redattori delle stesse;
	\item \textbf{Approvazione: }il \Responsabile stabilisce la validità del documento che ora può essere rilasciato. 
\end{itemize}

\subsubsection{Classificazione dei documenti}
I documenti vengo suddivisi in:
\begin{itemize}
	\item \textbf{Documenti Pubblici: }comprendono i documenti di interesse per il gruppo, il proponente e il commitente. 
	Sono documenti pubblici:
	\begin{itemize}
		\item \textbf{Glossario: }elenco ordinato di termini utilizzati dal gruppo nei loro documenti e che possono presentare ambiguità nel loro significato;
		\item \textbf{Studio di Fattibilità: }analisi dei capitolati proposti, con le valutazioni da parte dei membri del gruppo;
		\item \textbf{Piano di Progetto: }espone la pianificazione delle attività di progetto e relative previsioni rispetto a impiego orario dei membri, preventivo spese e consuntivi di periodo;
		\item \textbf{Piano di Qualifica: }presenta i criteri di valutazione della qualità utilizzati dal gruppo;
		\item \textbf{Analisi dei Requisiti: }contiene tutti i requisiti e le caratteristiche del prodotto finale individuati dal gruppo.
	\end{itemize}
	\item Documenti Interni: comprendono i documenti di interesse principalmente per i componenti del gruppo.
	Sono documenti interni:
	\begin{itemize}
		\item \textbf{Norme di Progetto: } raccolta delle norme stabilite dal gruppo che tutti i componenti devono seguire durante la realizzazione del progetto;
		\item \textbf{Verbali: } si suddividono a loro volta in:
		\begin{itemize}
			\item \textbf{Interni: } contengono un resoconto sintetico relativo alle riunioni svolte dal gruppo;
			\item \textbf{Interni: } contengono un resoconto relativo alle riunioni del gruppo con il proponente e/o i commitenti.
		\end{itemize}
	\end{itemize}
\end{itemize}

\subsubsection{Nomenclatura dei documenti} \label{NomenclaturaDoc}
Tutti i documenti ad eccezione dei Verbali seguiranno il seguente schema per la nomenclatura:

\begin{center}
	\textbf{[NomeDocumento]\_v[X].[Y].[Z]}
\end{center}
dove:
\begin{itemize}
	\item \textbf{[NomeDocumento]} corrisponde al nome ufficiale del documento, non deve presentare spazi e viene utilizzata la notazione \textit{CamelCase}.
	\item \textbf{v[X].[Y].[Z]} dove v sta per "versione" e X, Y, Z rispettano quanto descritto nella sezione \S\ref{Versionamento}
\end{itemize}

Per quanto riguarda la nomenclatura dei Verbali si segue il seguente schema:
\begin{center}
	\textbf{V[I/E]\_[YYYY]\_[MM]\_[DD]}
\end{center}
dove:
\begin{itemize}
	\item \textbf{V: } indica che il documento è un Verbale;
	\item \textbf{[I/E]: } indica se il Verbale è interno o esterno;
	\item \textbf{[YYYY]\_[MM]\_[DD]: } indica la data in cui si è svolto l'incontro indicata come anno\_mese\_giorno.
\end{itemize}

\subsubsection{Directory dei documenti}
Ogni documento viene inserito in una directory il cui nome coincide con il NomeDocumento dello stesso, posizionata a sua volta secondo la tipologia del documento contenuto nella directory \textbf{documenti-interni} o \textbf{documenti-pubblici}.

\subsubsection{Struttura dei documenti}\label{Struttura documenti}
Tutti i documenti contengono prima delle pagine con il contenuto previsto:
\paragraph*{Frontespizio}
La prima pagina di ogni documento comprende in sequenza i seguenti elementi:
\begin{itemize}
	\item \textbf{Logo del gruppo; }
	\item \textbf{Nome del documento;}
	\item \textbf{Nome del gruppo e del Progetto;}
	\item \textbf{Email del gruppo;}
	\item \textbf{Informazioni sul documento}, che includono:
		\begin{itemize}
			\item \textbf{Versione;}
			\item \textbf{Approvatore;}
			\item \textbf{Redattori;}
			\item \textbf{Verificatori;}
			\item \textbf{Uso} se interno o pubblico;
			\item \textbf{Distribuzione}, cioè i destinatari del documento.
		\end{itemize}
	\item \textbf{Descrizione} breve del contenuto del documento.
\end{itemize}

\paragraph*{Registro delle Modifiche}
Sotto forma di tabella viene presentato un registro delle modifiche contenente le modifiche effettuate al documento durante il suo ciclo di vita e le varie verifiche effettuate.\\
Ciascuna voce della tabella riporta:
\begin{itemize}
	\item \textbf{Versione} del documento dopo la modifica/verifica;
	\item \textbf{Data} della modifica/verifica;
	\item \textbf{Nominativo} del componente che ha effettuato la modifica/verifica;
	\item \textbf{Ruolo} del componente che ha effettuato la modifica/verifica;
	\item \textbf{Descrizione} della modifica/verifica effettuata.
\end{itemize}
Per garantire la consistenza del Registro delle Modifiche le sezioni che vengono eliminate sono indicate con il loro titolo.

\paragraph*{Indice}
Dopo il registro delle modifiche viene inserito un indice dei contenuti utile a chi utilizza il documento per orientarsi velocemente all'interno dello stesso.

\paragraph*{Corpo dei documenti}
Dopo l'indice dei contenuti è presente il corpo del documento. Gli argomenti sono divisi in sezioni e sottosezioni raggruppati in modo coerente e coeso.

Il corpo dei Verbali deve contenere le seguenti sezioni:
\begin{itemize}
	\item \textbf{Informazioni Generali}, che includono:
		\begin{itemize}
			\item \textbf{Luogo} in cui si è svolta la riunione;
			\item \textbf{Data} in cui si è svolta la riunione;
			\item \textbf{Ora}, comprendente orario di inzio e di fine;
			\item \textbf{Partecipanti}, membri del gruppo ed eventualmente committenti e/o proponente;
			\item \textbf{Segretario}, componente del gruppo che ha redatto il verbale.
		\end{itemize}
	\item \textbf{Ordine del giorno: }argomenti principali di cui si è trattato.
	\item \textbf{Resoconto: }per ogni argomento trattato è presente una breve descrizione della discussione avvenuta;
	\item \textbf{Registro delle decisioni:} vengono presentate sotto forma di elenco le decisioni prese dal gruppo.
\end{itemize}

\subsubsection{Formattazione e norme tipografiche}
\myparagraph{Template}
Il gruppo utilizza per la stesura dei documenti il linguaggio \LaTeX. Al fine di mantenere una coerenza nella formattazione dei documenti vengono utilizzati due template permettendo così l'automatizzazione delle scelte tipografiche. \\
I template utilizzati sono:
\begin{itemize}
	\item \textbf{StileLettera: }template esclusivamente per le Lettere di Presentazione;
	\item \textbf{StileTemplate: }template per tutti i restanti documenti.
\end{itemize}

\myparagraph{Termini di glossario}
I termini il cui significato potrebbe essere ambiguo sono individuati con un apice \ap{G} alla fine della parola, salvo che il suo significato non venga subito spiegato. Tutti i termini da glossario sono stati inseriti nel documento \Glossariov.

\myparagraph{Stile del testo}
Gli stili di testo adottati nei documenti sono:
\begin{itemize}
	\item \textbf{grassetto:} per titoli, sottotitoli, nome degli oggetti negli elenchi puntati, termini importanti secondo il \textit{Redattore};
	\item \textbf{corsivo:} per i nomi propri, nomi dei ruoli, nome del progetto, nome dei documenti, nomi delle tecnologie, degli strumenti, delle formule matematiche, termini specifici e citazioni.
	\item \textbf{maiuscolo:} per i nomi dei file, delle directory e dei processi (convenzione \textit{CamelCase}).	
\end{itemize}

\myparagraph{Formatazione di date e orari}
Per l'indicazione delle date e degli orari si segue quanto sancito dallo standard ISO \glo{8601}. \\
Le date vengono indicate secondo il seguente formato:
\begin{center}
	\textbf{[YYYY]\_[MM]\_[DD]} 
\end{center}
dove:
\begin{itemize}
	\item \textbf{[YYYY]} corrisponde all'anno;
	\item \textbf{[MM]} corrisponde al mese;
	\item \textbf{[DD]} corrisponde al giorno.
\end{itemize}

Gli orari vengono indicati secondo il seguente formato:
\begin{center}
	\textbf{[HH]:[MM]} 
\end{center}
dove:
\begin{itemize}
	\item \textbf{[HH]} rappresenta l'ora;
	\item \textbf{[MM]} rappresenta i minuti;
\end{itemize}

\myparagraph{Formattazione elementi grafici}
La sezione definisce le norme per gli elementi grafici all'interno dei documenti.
\paragraph*{Immagini}
Le figure presenti nei documenti vanno centrate rispetto al testo e accompagnate da una didascalia coerente.
\paragraph*{Grafici UML}
I grafici in linguaggio \glo{UML} sono inseriti come immagini.
\paragraph*{Tabelle}
Le tabelle vanno accompagnate da una didascalia opportuna.\\
Nella didascalia viene indicato l'identificativo 
\begin{center}
	\textbf{Tabella [X]} 
\end{center}
dove [X] inidica il numero assoluto della tabella all'interno del documento; segue poi il testo della didascalia.
A questa prassi fanno eccezione le tabelle del registro delle modifiche.

\subsubsection{Strumenti} 
Gli strumenti utilizzati per la stesura dei documenti sono:
\begin{itemize}
	\item Linguaggio \LaTeX
\end{itemize}

\subsubsection{Metriche} 
\subsubsection{Verifica ortografica} 
