\subsection{Documentazione}
\subsubsection{Scopo dell'attività} \label{PSup_Documentazione_Scopo}
Quando si realizza un progetto è di fondamentale importanza documentare processi e \glo{attività} significative per lo sviluppo dello stesso.\\
La documentazione permette di:
\begin{itemize}
	\item Disciplinare ciò che verrà fatto durante il ciclo di vita del software;
	\item Garantire che tutte le informazioni d'interesse siano facilmente accessibili agli \glo{stakeholders};
	\item Assicurare che quanto previsto dalla documentazione sia rispettato per garantire la qualità attesa.
\end{itemize}

\subsubsection{Descrizione della sezione} 
La sezione comprende le norme relative alla documentazione associata al progetto che tutti i componenti devono seguire nella redazione dei documenti di loro competenza.

\subsubsection{Aspettative}\label{AspettativeDocumentazione}L'obiettivo del gruppo è di ottenere dei documenti coerenti e validi dal punto di vista tipografico e formale.
I documenti redatti dal gruppo devono:
\begin{itemize}
	\item Essere facilmente aggiornabili e modificabili;
	\item Contenere solo ciò che è essenziale;
	\item Essere facili da leggere.
\end{itemize}
Per assicurare il raggiungimento degli obiettivi di qualità previsti dal \PdQ\ il gruppo intende:
\begin{itemize}
	\item Utilizzare un metodo di \glo{versionamento} per il tracciamento delle versioni;
	\item Rendere disponibile solo versioni di documenti verificati e quindi consistenti;
	\item Individuare regole di impaginazione comune;
	\item Creazione di un glossario per chiarire il significato di termini che possono risultare ambigui;
	\item Assicurare che tutto ciò che sia di interesse degli \glo{stakeholders} sia presente;
	\item Controllare attentamente l'ortografia e il livello di comprensione di quanto scritto.
\end{itemize}

\subsubsection{Ciclo di vita di un documento}\label{CicloDocumentazione}
Tutti i documenti redatti sono sottoposti a queste attività:
\begin{itemize}
	\item \textbf{Creazione:} il documento viene creato secondo il template presente nella \glo{repository};
	\item \textbf{Stesura:} il documento viene redatto dai componenti incaricati, tutti i documenti devono contenere:
	\begin{itemize}
		\item Registro delle modifiche;
		\item Indice dei contenuti.
	\end{itemize}
	\item \textbf{Revisione:} le sezioni del documento vengono regolarmente revisionate da uno o più componenti del gruppo diversi dai redattori delle stesse;
	\item \textbf{Approvazione:} il \Responsabile\ stabilisce la validità del documento che può in seguito essere rilasciato. 
\end{itemize}

\subsubsection{Classificazione dei documenti}\label{ClassificazioneDocumenti}
I documenti vengo suddivisi in:
\begin{itemize}
	\item \textbf{Documenti Pubblici:} comprendono i documenti di interesse per il gruppo, il proponente e il committente. 
	Sono documenti pubblici:
	\begin{itemize}
		\item \textbf{Glossario:} elenco ordinato di termini utilizzati dal gruppo nei loro documenti e che possono presentare ambiguità nel loro significato;
		\item \textbf{Studio di fattibilità:} analisi dei \glo{capitolati} proposti, con le valutazioni da parte dei membri del gruppo;
		\item \textbf{Piano di progetto:} espone la pianificazione delle attività di progetto e relative previsioni rispetto a impiego orario dei membri, preventivo e consuntivi di periodo;
		\item \textbf{Piano di qualifica:} presenta i criteri di valutazione della qualità utilizzati dal gruppo;
		\item \textbf{Analisi dei requisiti:} contiene tutti i requisiti e le caratteristiche del prodotto finale individuati dal gruppo.
	\end{itemize}
	\item \textbf{Documenti Interni:} comprendono i documenti di interesse principalmente per i componenti del gruppo.
	Sono documenti interni:
	\begin{itemize}
		\item \textbf{Norme di progetto:} raccolta delle norme stabilite dal gruppo che tutti i componenti devono seguire durante la realizzazione del progetto;
		\item \textbf{Verbali:} si suddividono a loro volta in:
		\begin{itemize}
			\item \textbf{Interni:} contengono un resoconto sintetico relativo alle riunioni svolte dal gruppo;
			\item \textbf{Esterni:} contengono un resoconto relativo alle riunioni del gruppo con il proponente e/o i committenti.
		\end{itemize}
	\end{itemize}
\end{itemize}

\subsubsection{Nomenclatura dei documenti} \label{NomenclaturaDoc}
Tutti i documenti ad eccezione dei verbali seguiranno il seguente schema per la nomenclatura:
\begin{center}
	\textbf{[NomeDocumento]\_v[X].[Y]}
\end{center}
dove:
\begin{itemize}
	\item \textbf{[NomeDocumento]} corrisponde al nome ufficiale del documento, non deve presentare spazi e viene utilizzata la notazione \textbf{CamelCase}.
	\item \textbf{v[X].[Y]} dove v sta per "versione" e X, Y rispettano quanto descritto in \S\ref{Versionamento}
\end{itemize}
Per quanto riguarda la nomenclatura dei verbali si segue il seguente schema:
\begin{center}
	\textbf{V[I/E]\_[YYYY]\_[MM]\_[DD]}
\end{center}
dove:
\begin{itemize}
	\item \textbf{V:} indica che il documento è un verbale;
	\item \textbf{[I/E]:} indica se il verbale è interno o esterno;
	\item \textbf{[YYYY]\_[MM]\_[DD]:} indica la data in cui si è svolto l'incontro indicata come anno\_mese\_giorno.
\end{itemize}

\subsubsection{Directory dei documenti}
Ogni documento viene inserito in una directory il cui nome coincide con il NomeDocumento dello stesso, posizionata a sua volta secondo la tipologia del documento contenuto nella directory \textbf{DocumentiInterni} o \textbf{DocumentiPubblici}.

\subsubsection{Struttura dei documenti}\label{StrutturaDocumenti}
Tutti i documenti contengono prima delle pagine con il contenuto previsto:
\begin{itemize}
\item \textbf{Frontespizio:} la prima pagina di ogni documento comprende in ordine i seguenti elementi:
\begin{itemize}
	\item \textbf{Logo del gruppo;}
	\item \textbf{Nome del documento;}
	\item \textbf{Nome del gruppo e del progetto;}
	\item \textbf{Email del gruppo;}
	\item \textbf{Informazioni sul documento}, che includono:
		\begin{itemize}
			\item \textbf{Versione;}
			\item \textbf{Approvatore;}
			\item \textbf{Redattori;}
			\item \textbf{Verificatori;}
			\item \textbf{Uso} se interno o pubblico;
			\item \textbf{Distribuzione}, cioè i destinatari del documento.
		\end{itemize}
	\item \textbf{Descrizione} breve del contenuto del documento.
\end{itemize}

\item \textbf{Registro delle modifiche:} sotto forma di tabella viene presentato un diario delle modifiche contenente i cambiamenti effettuati al documento durante il suo ciclo di vita e le varie verifiche effettuate.\\
Ciascuna voce della tabella riporta:
\begin{itemize}
	\item \textbf{Versione} del documento dopo la modifica/verifica;
	\item \textbf{Data} della modifica/verifica;
	\item \textbf{Nominativo} del componente che ha effettuato la modifica/verifica;
	\item \textbf{Ruolo} del componente che ha effettuato la modifica/verifica;
	\item \textbf{Descrizione} della modifica/verifica effettuata.
\end{itemize}
Per garantire la consistenza del registro le sezioni sono etichettate con il comando \LaTeX\ \textbf{label}, le sezioni che vengono eliminate sono indicate con il loro titolo.

\item \textbf{Indice:} dopo il registro delle modifiche viene inserito un indice dei contenuti utile a chi utilizza il documento per orientarsi velocemente all'interno dello stesso.

\item \textbf{Corpo dei documenti:} dopo l'indice dei contenuti è presente il corpo del documento. Gli argomenti sono divisi in sezioni e sottosezioni raggruppati in modo coerente e coeso.\\
Il corpo dei verbali deve contenere le seguenti sezioni:
\begin{itemize}
	\item \textbf{Informazioni generali}, che includono:
		\begin{itemize}
			\item \textbf{Luogo} in cui si è svolta la riunione;
			\item \textbf{Data} in cui si è svolta la riunione;
			\item \textbf{Ora}, comprendente orario di inizio e di fine;
			\item \textbf{Partecipanti}, membri del gruppo ed eventualmente committenti e/o proponente;
			\item \textbf{Segretario}, componente del gruppo che ha redatto il verbale.
		\end{itemize}
	\item \textbf{Ordine del giorno:} argomenti principali di cui si è trattato;
	\item \textbf{Resoconto:} per ogni argomento trattato è presente una breve descrizione della discussione avvenuta;
	\item \textbf{Registro delle decisioni:} vengono presentate sotto forma di elenco le decisioni prese dal gruppo.
\end{itemize}
\end{itemize}

\subsubsection{Formattazione e norme tipografiche}
\myparagraph{Template}
Il gruppo utilizza per la stesura dei documenti il linguaggio \LaTeX. Al fine di mantenere una coerenza nella formattazione dei documenti vengono utilizzati due template permettendo così l'automatizzazione delle scelte tipografiche.\\
I template utilizzati sono:
\begin{itemize}
	\item \textbf{Stilelettera:} template esclusivamente per le \textit{lettere di presentazione};
	\item \textbf{Stiletemplate:} template per tutti i restanti documenti.
\end{itemize}

\myparagraph{Termini di glossario}
I termini il cui significato potrebbe essere ambiguo sono individuati con un apice \ap{G} alla fine della parola, salvo che il suo significato non venga subito spiegato. Tutti i termini da glossario sono stati inseriti nel documento \Glossario. Di ogni termine solamente la prima occorrenza per ogni sezione va evidenziata con l'apice \ap{G}.

\myparagraph{Stile del testo}
Gli stili di testo adottati nei documenti sono:
\begin{itemize}
	\item \textbf{grassetto:} per titoli, sottotitoli, nome degli oggetti negli elenchi puntati, termini importanti secondo il \textit{Redattore};
	\item \textbf{corsivo:} per i nomi propri di persona, nomi dei ruoli, nome del progetto, nome dei documenti;
	\item \textbf{maiuscolo:} per i nomi dei file, delle directory e dei processi (convenzione \textbf{CamelCase}).	
\end{itemize}
Ogni volta che si utilizza il nome di un documento come titolo o in una voce di elenco seguita da una descrizione, tale nome deve seguire solo la prima convenzione senza utilizzare il corsivo.

\myparagraph{Formattazione di date e orari}
Per l'indicazione delle date e degli orari si segue quanto sancito dallo \glo{standard} ISO 8601.\\
Le date vengono indicate secondo il seguente formato:
\begin{center}
	\textbf{[YYYY]\_[MM]\_[DD]} 
\end{center}
dove:
\begin{itemize}
	\item \textbf{[YYYY]} corrisponde all'anno;
	\item \textbf{[MM]} corrisponde al mese;
	\item \textbf{[DD]} corrisponde al giorno.
\end{itemize}
Gli orari vengono indicati secondo il seguente formato:
\begin{center}
	\textbf{[HH]:[MM]} 
\end{center}
dove:
\begin{itemize}
	\item \textbf{[HH]} rappresenta l'ora;
	\item \textbf{[MM]} rappresenta i minuti.
\end{itemize}

\myparagraph{Formattazione elementi grafici}
La sezione definisce le norme per gli elementi grafici all'interno dei documenti.
\begin{itemize}
\item \textbf{Immagini:} le figure presenti nei documenti vanno centrate rispetto al testo e accompagnate da una didascalia coerente;
\item \textbf{Grafici UML:} i grafici in linguaggio \glo{UML} sono inseriti come immagini;
\item \textbf{Tabelle:} le tabelle vanno accompagnate da una didascalia opportuna.\\
Nella didascalia viene indicato l'identificativo 
\begin{center}
	\textbf{Tabella [X]} 
\end{center}
dove [X] indica il numero assoluto della tabella all'interno del documento, segue poi il testo della didascalia.
A questa prassi fanno eccezione le tabelle del registro delle modifiche.
\end{itemize}

\subsubsection{Metriche}\label{MDocumentazione}
Di seguito vengono presentate le metriche utilizzate per garantire il controllo sulla qualità. Per una descrizione sullo standard di riferimento e un elenco completo di tutte le metriche applicate si rimanda a \S\ref{9126} dell'appendice. 
\begin{itemize}
	\item\textbf{MPR02: Indice di Gulpease}\hypertarget{MGulpease} \\
	L'indice permette di stabilire la difficoltà di lettura di un testo.\\
	Viene calcolato nel seguente modo:
	\begin{center}
		\(
		GULP = 89+\frac{300\times\# frasi-10\times\#lettere}{\#parole}
		\)
	\end{center}
	L'indice può assumere i seguenti valori:
	\begin{itemize}
		\item \textbf{GULP<80}: indica una leggibilità difficile per un utente con licenza elementare;
		\item \textbf{GULP<60}: indica una leggibilità difficile per un utente con licenza media;
		\item \textbf{GULP<40}: indica una leggibilità difficile per un utente con licenza superiore.
	\end{itemize}
	\item\textbf{MPR03: Indice di correttezza ortografica} \hypertarget{MOrtografia}\\
	Indica la correttezza ortografica in un documento.
	Viene calcolato nel seguente modo:
	\begin{center}
		\textbf{CORT = \# numero di errori ortografici}
	\end{center}
	Un risultato maggiore di zero indica la presenza di errori nel documento.
\end{itemize}

\subsubsection{Strumenti} \label{Documentazione_Strumenti}
Di seguito sono riportati gli strumenti utilizzati durante la redazione della documentazione.
\begin{itemize}
	\item \textbf{\LaTeX};
	\item \textbf{GanttProject:} utilizzato per la realizzazione dei \glo{diagrammi di Gantt} all'interno del documento \PdPv{3.0};
	\item \textbf{Microsoft Excel:} utilizzato per realizzare i diagrammi all'interno dei documenti;
	\item \textbf{Diagrams.net:} utilizzato per la realizzazione dei \glo{diagrammi UML} all'interno del documento \AdRv{2.0}.
\end{itemize}