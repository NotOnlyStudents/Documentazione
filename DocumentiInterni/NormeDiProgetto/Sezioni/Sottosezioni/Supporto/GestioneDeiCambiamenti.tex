\subsection{Gestione dei cambiamenti}\label{GestioneCambiamenti}
\subsubsection{Scopo dell'attività} 
L'attività ha lo scopo di garantire una corretta gestione dei problemi che si possono verificare durante il progetto.

\subsubsection{Descrizione della sezione} 
La sezione descrive la prassi generale che il gruppo intende seguire nel caso di complicazioni o malfunzionamenti.

\subsubsection{Aspettative}
Il gruppo si aspetta di affrontare tempestivamente attraverso quanto previsto dall'attività, gli eventuali problemi che si possono presentare.

\subsubsection{Norme generali} \label{GCambiamenti_Norme}
Come descritto nella sezione \S\ref{GestioneCompiti}, ogni componente gestisce autonomamente i compiti che gli spettano. Periodicamente viene effettuata l'attività di verifica secondo quanto previsto dalla sezione \S\ref{Verifica}.\\ È di fondamentale importanza tracciare tutte le osservazioni che ci vengono fatte notare da committenti e proponente per poter correggere tempestivamente eventuali problemi. 
Tutti i cambiamenti che verranno apportarti ad attività o prodotti sono riportati nel {\PdQ}.

\subsubsection{Denominazione dei cambiamenti}\label{NomeCambiamenti}
Per facilitare il tracciamento dei cambiamenti, vengono classificati secondo il seguente schema:
\begin{center}
	\textbf{C[P/A][Attività/Prodotto][ID]}
\end{center} 
dove:
\begin{itemize}
	\item \textbf{C} indica cambiamento;
	\item \textbf{P/A} indicano rispettivamente:
	\begin{itemize}
		\item \textbf{P} se riguarda il prodotto;
		\item \textbf{A} se riguarda un'attività.
	\end{itemize}
	\item \textbf{Attività/Prodotto} possono assumere i seguenti valori:
	\begin{itemize}
		\item Se si tratta di un'attività:
		\begin{itemize}
			\item \textbf{An} se riguarda l'attività di analisi;
			\item \textbf{Pg} se riguarda l'attività di progettazione;
			\item \textbf{Cd} se riguarda l'attività di codifica;
			\item \textbf{Do} se riguarda l'attività di documentazione;
			\item \textbf{Cg} se riguarda l'attività di configurazione;
			\item \textbf{Qu} se riguarda l'attività di gestione della qualità;
			\item \textbf{Ve} se riguarda l'attività di verifica;
			\item \textbf{Pn} se riguarda l'attività di pianificazione.
		\end{itemize}
		\item Se si tratta di un prodotto:
		\begin{itemize}
			\item \textbf{D} se riguarda un prodotto documentale;
			\item \textbf{C} se riguarda un prodotto di codifica.
		\end{itemize}
	\end{itemize}
	\item \textbf{ID} è un numero incrementale che parte da 1.
\end{itemize}