\subsection{Gestione della configurazione}
\subsubsection{Scopo dell'attività} \label{PSup_GestioneConf_Scopo}
La Gestione della configurazione ha lo scopo di gestire ordinatamente e sistematicamente la produzione dei documenti e del codice.

\subsubsection{Descrizione della sezione} 
La sezione presenta le norme che il gruppo intende seguire per il controllo della documentazione e del codice.

\subsubsection{Aspettative}
Con queste norme il gruppo intende assicurare una corretta gestione dei documenti e del codice rendendo sistematica la loro produzione e garantendo la chiara individuazione delle versioni prodotte.

\subsubsection{Identificazione di configurazione}
Ogni \glo{configuration item} ha un'identità unica che ne consente la chiara individuazione. Per una trattazione precisa si rimanda alla sezione \S\ref{NomenclaturaDoc}.

\subsubsection{Controllo di baseline}
Per garantire riproducibilità e tracciabilità viene definita una successione di \glo{baseline} associate a una serie di \glo{milestone} specifiche per obiettivi di avanzamento e coerenti con la strategia di progetto. Data l'importanza di controllare lo svolgimento del progetto il numero delle milestone sarà superiore al numero delle Revisioni.

\subsubsection{Controllo di versione}\label{Versionamento}
\myparagraph{Codice di versione per documenti e software}
Il codice di una versione è formato: 
\begin{center}
	\textbf{[X].[Y].[Z]}
\end{center}
dove 
\begin{itemize}
	\item \textbf{[X]} corrisponde ad una versione del documento approvata dal \Responsabile\ e pronta al rilascio, la numerazione parte da 0;
	\item \textbf{[Y]} corrisponde ad una versione del documento verificata, inizia da 0 e riparte da questo valore ad ogni incremento di X;
	\item \textbf{[Z]} parte da zero e viene incrementato ad ogni modifica del documento, la numerazione riparte da zero ad ogni incremento di X o Y.
\end{itemize}
Il Controllo di versione si appoggia su un repository creato in \textit{Github} la cui struttura viene spiegata nella sezione seguente.

\subsubsection{Struttura della repository}\label{StrutturaRepo}
La repository del gruppo \Gruppo\ è accessibile all'indirizzo
\begin{center}
	\textbf{https://github.com/NotOnlyStudents}
\end{center}
L'utilizzo di questo strumento permette a ciascun componente di lavorare sui \glo{configuration items} senza rischio di sovrascritture consentendo allo stesso tempo di controllare l'andamento del lavoro. 
All'interno del repository viene divisa la documentazione dal codice software.
\myparagraph{Repository per la documentazione}\label{RepoDoc}
Il repository per la documentazione è composto da diversi \textbf{branch}:
\begin{itemize}
	\item \textbf{master} su cui vengono riportati i documenti approvati dal \Responsabile\ e pronti nella loro versione definitiva ad essere presentati;
	\item \textbf{develop} ramo in cui vengono inserite la versioni dei documenti verificati ma non ancora approvati;
	\item rami dedicati ai singoli documenti in sviluppo non verificati.
\end{itemize}
All'interno del repository troviamo le cartelle così organizzate:
\begin{itemize}
	\item \textbf{Immagini}: contiene le immagini utilizzate all'interno dei documenti;
	\item \textbf{Utilita}: contiene file di utilità per la produzione di tutti i documenti;
	\item \textbf{DocumentiInterni}: contiene le cartelle dei documenti di utilità per il gruppo e marginale per proponente e committenti;
	\item \textbf{DocumentiPubblici}: contiene le cartelle dei documenti di utilità per il gruppo, per il proponente e per i committenti.
\end{itemize}
\myparagraph{Modifiche al repository}\label{ModificaRepo}
Nel caso fosse necesario modificare file all'interno del ramo master è necessaria una \textit{pull request} con conseguente approvazione da almeno un altro elemento del gruppo.