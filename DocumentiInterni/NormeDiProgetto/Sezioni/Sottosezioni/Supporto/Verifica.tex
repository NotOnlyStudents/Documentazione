\subsection{Verifica}\label{Verifica}
\subsubsection{Scopo dell'attività} \label{PSup_Verifica_Scopo}
L'attività di verifica ha lo scopo di accertare che quanto eseguito durante il periodo preso in considerazione non abbia introdotto errori e quindi che quanto realizzato sia corretto e completo.

\subsubsection{Descrizione della sezione} 
La sezione descrive come avviene l'attività di verifica da parte del gruppo \Gruppo\ per la documentazione e il codice.

\subsubsection{Aspettative}
Utilizzando le norme presenti il gruppo si aspetta di ottimizzare e uniformare l'attività di verifica. L'attività di verifica viene attuata attraverso l'\textbf{analisi statica} e l'\textbf{analisi dinamica}.

\subsubsection{Analisi statica}
L'analisi statica studia la documentazione e il codice accertando che siano conformi alle regole stabilite, non vi siano difetti e presentino le proprietà desiderate.
Il \textit{Verificatore} seguirà i seguenti metodi:
\begin{itemize}
	\item \glo{\textbf{Walkthrough:}} procede ad una lettura critica del prodotto in esame alla ricerca di difetti;
	\item \glo{\textbf{Inspection:}} procede ad una lettura critica del prodotto in esame, focalizzando la ricerca su problemi che si presentano più frequentemente indicati nella lista di controllo all'interno del \PdQ.
\end{itemize}

\myparagraph{Verifica della documentazione}\label{VerificaDocumentazione}Ogniqualvolta si redige una sezione o si apporta una modifica al documento, viene effettuata la verifica. Il \textit{Verificatore} incaricato procede nel seguente modo:
\begin{itemize}
	\item Rilegge la sezione, nel caso emergano sostanziali errori si valuta l'aggiornamento della lista di controllo;
	\item Successivamente viene applicato il metodo di Inspection concentrandosi sulle criticità presenti nella lista di controllo;
	\item Se riscontra problemi logici, si accorda con il redattore del documento per la correzione;
	\item Chiude la pull request collegata e automaticamente l'issue associata passa in "VERIFIED".
\end{itemize}

\myparagraph{Verifica del codice}\label{VerificaCodice}Il \textit{Verificatore} incaricato procede nel seguente modo:
\begin{itemize}
	\item Rilegge interamente il codice scritto al fine di:
	\begin{itemize}
		\item Accertare che il codice esegua nella sequenza specificata e sia ben strutturato;
		\item Localizzare codice non raggiungibile;
		\item Identificare parti di codice che possano non terminare;
		\item Accertare che nessun cammino acceda a variabili prive di valore;
		\item Rilevare anomalie;
		\item Provare la correttezza del codice sorgente rispetto ai requisiti;
		\item Assicurare che sia stato seguito quanto scritto nelle \NdPv\ in \S\ref{CodificaConvenzioni}.
	\end{itemize}
	Nel caso emergano sostanziali errori si valuta l'aggiornamento della lista di controllo;
	\item Successivamente ad ogni verifica viene applicato il metodo di Inspection concentrandosi sulle criticità presenti nella lista di controllo;
	\item Se riscontra problemi, si accorda con il redattore del documento per la correzione. 
	\item Chiude la pull request collegata e automaticamente l'issue associata passa in "VERIFIED".
\end{itemize}

\subsubsection{Analisi dinamica}
L'analisi dinamica procede a verificare il programma durante la sua esecuzione. Per procedere a questa attività vengono utilizzati GitHub Actions e dei test ripetibili e automatizzati con specifici:
\begin{itemize}
	\item \textbf{Ambiente di esecuzione};
	\item \textbf{Input e Output};
	\item \textbf{Procedure};
\end{itemize}

\myparagraph{Tipologia di test effettuati}\label{Tests}I test permettono di valutare se il programma svolge i tasks per cui è stato sviluppato. Il \PdQv{}\ specifica quali e quanti test effettuare.
\paragraph*{Test d'unità}
I \glo{test d'unità} consistono nell'isolare e testare la parte più piccola di software, chiamata appunto unità. Lo scopo è di valutare la logica interna del codice d'unità in modo da poterne stabilire il corretto funzionamento. 
\paragraph*{Test d'integrazione}
Applicati alle componenti della progettazione architetturale, hanno lo scopo di valutare che l'aggregazione delle parti del sistema non introduca errori nel software prodotto.
\paragraph*{Test di regressione}
Questi test sono necessari a verificare che la modifica di una parte P di S non causi errori in P, in S o in una parte in relazione con essi. Le modifiche apportate non devono pregiudicare le funzionalità già verificate. 
\paragraph*{Test di sistema}
Vengono effettuati per valutare che il prodotto soddisfi tutti i requisiti software previsti.

\myparagraph{Procedura di verifica}\label{ProceduraVerifica}Il \textit{Verificatore} incaricato procede nel seguente modo:
\begin{itemize}
	\item Esegue tutti i test previsti;
	\item Se riscontra problemi, si accorda con il redattore del codice per la correzione. 
	\item Chiude la pull request collegata e automaticamente l'issue associata passa in "VERIFIED".
	\item Avverte i redattori incaricati ad aggiornare il \PdQv\ con i risultati ottenuti.
\end{itemize}