\subsection{Gestione della qualità}
\subsubsection{Scopo dell'attività} \label{PSup_GestioneQualità_Scopo}
L'attività ha lo scopo di assicurare al proponente che il prodotto realizzato soddisfi i requisiti e la qualità previsti e stabiliti con il proponente.

\subsubsection{Descrizione della sezione} 
La sezione presenta il processo utilizzato dal gruppo per gestire il \glo{Sistema Qualità}, indicando procedure e strumenti utili che i componenti devono seguire.

\subsubsection{Aspettative}
Il gruppo si aspetta da questo processo di conseguire la qualità del prodotto prevista secondo le aspettative del proponente attraverso un'azione preventiva. Il documento \PdQv{1.0.0} individua quanto stabilito indicando gli obiettivi di qualità da raggiungere e le \glo{metriche} da utilizzare per la valutazione della qualità.

\subsubsection{Istanziazione di un processo}
Perseguendo la qualità attraverso un'azione preventiva, il gruppo deve compiere le proprie scelte considerando le caratteristiche di qualità ricercate. 
Le seguenti norme devono essere considerate con l'istanziazione di ogni processo:
\begin{itemize}
	\item Ogni processo deve puntare a perseguire un unico obiettivo;
	\item Ad ogni istanziazione di un processo devono essere analizzati i rischi connessi e prevista una pianificazione del lavoro per il loro contrasto;
	\item Ad ogni istanziazione di un processo le risorse vengono assegnate considerando gli altri processi già in esecuzione;
	\item Ogni processo deve avere una durata definita.
\end{itemize}

\subsubsection{Svolgimento di un processo}
Durante lo svolgimento di un processo, le attività vengono svolte perseguendo il raggiungimento degli obiettivi individuati all'interno del \PdQv{1.0.0} e le seguenti caratteristiche di qualità:
\begin{itemize}
	\item \textbf{Qualità di prodotto}: 
	\begin{itemize}
		\item \textbf{Funzionalità};
		\item \textbf{Efficienza};
		\item \textbf{Usabilità};
		\item \textbf{Affidabilità};
		\item \textbf{Manutenibilità}.
	\end{itemize}
\end{itemize}
Per una descrizione degli standard di riferimento utilizzati si rimanda alle sezioni \S\ref{spice},\S\ref{PDCA} e \S\ref{9126} dell'appendice.

\subsubsection{Classificazione degli obiettivi}
Gli obiettivi di qualità presenti nel \PdQv{1.0.0} vengono classificati secondo il seguente formato:
\begin{center}
	\textbf{O[Categoria][D/S][X]}
\end{center}
dove 
\begin{itemize}
	\item \textbf{O} indica che si tratta di un obiettivo di qualità;
	\item \textbf{Categoria} può assumere i seguenti valori:
	\begin{itemize}
		\item \textbf{PR:} l'obiettivo riguarda un processo;
		\item \textbf{PD:} l'obiettivo riguarda un prodotto.
	\end{itemize}
	\item \textbf{D/S} indicano rispettivamente se l'obiettivo riguarda un documento o un prodotto software, è presente solo per gli obiettivi relativi ai prodotti;
	\item \textbf{X} è un numero progressivo che inizia da 1.
\end{itemize}

\subsubsection{Classificazione delle metriche}\label{ClassificazioneMetriche}
Le metriche utilizzate per la valutazione della qualità presenti nel \PdQv{1.0.0}\ vengono classificate secondo il seguente formato:
\begin{center}
	\textbf{M[Categoria][D/S][X]}
\end{center}
dove 
\begin{itemize}
	\item \textbf{M} indica che si tratta di una metrica di qualità;
	\item \textbf{Categoria} può assumere i seguenti valori:
	\begin{itemize}
		\item \textbf{PR:} la metrica riguarda un processo;
		\item \textbf{PD:} la metrica riguarda un prodotto.
	\end{itemize}
	\item \textbf{D/S} indicano rispettivamente se l'obiettivo riguarda un documento o un prodotto software,è presente solo per gli obiettivi relativi ai prodotti;
	\item \textbf{X} è un numero progressivo che inizia da 1.
\end{itemize}
Per una descrizione approfondita delle metriche utilizzate si rimanda alla sezione \S\ref{Metriche} dell'Appendice.