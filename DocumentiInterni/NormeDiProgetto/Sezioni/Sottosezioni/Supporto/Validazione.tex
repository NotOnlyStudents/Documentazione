\subsection{Validazione}\label{PSup_Validazione}
\subsubsection{Scopo dell'attività}\label{PSup_Validazione_Scopo}
La validazione stabilisce se il prodotto soddisfa i compiti per cui è stato creato, in particolare permette di garantire che quanto richiesto e stabilito da \Proponente\ sia soddisfatto.

\subsubsection{Descrizione della sezione} 
La sezione descrive la procedura seguita per la validazione di quanto viene svolto durante il progetto.

\subsubsection{Aspettative}
Il gruppo si aspetta di procedere alla validazione dei documenti in modo chiaro. 

\subsubsection{Validazione dei documenti}\label{ValidazioneDoc}
Il \Responsabile\ procede nel seguente modo:
\begin{itemize}
	\item Se non ancora creata, crea l'issue in GitHub e la sposta in "IN PROGRESS";
	\item Se riscontra problemi, avverte il redattore del documento e il verificatore. L'issue torna in "TO DO" in attesa delle modifiche. Quando queste vengono apportate si torna al primo punto;
	\item Sposta l'issue in "VERIFIED" approvando il documento;
	\item Avverte i redattori incaricati di aggiornare il \PdQv{} con i risultati ottenuti.
\end{itemize}

\subsubsection{Validazione del codice}\label{ValidazioneCodice}
Il \textit{Verificatore} incaricato procede nel seguente modo:
\begin{itemize}
	\item Se non ancora creata, crea l'issue in GitHub;
	\item Sposta l'issue in "IN PROGRESS";
	\item Esegue i test descritti nella sezione \S\ref{Tests} così da garantire un software pronto al rilascio;
	\item Sposta l'issue in "VERIFIED" approvando il documento.
\end{itemize}

\subsubsection{Piano di Qualifica e Test}
All'interno del \PdQ\ sono inseriti tutti i test pianificati e implementati dai \textit{Progettisti}, eseguiti durante il progetto.\\
Per ogni test devono essere specificati:
\begin{itemize}
	\item Entità sottoposte a verifica;
	\item Task e obiettivi della validazione;
	\item Membri che hanno eseguito i test;
	\item Resoconto dei risultati ottenuti.
\end{itemize}

\subsubsection{Nomenclatura dei test}
Ogni test viene descritto con:
\begin{itemize}
	\item Codice identificativo;
	\item Descrizione;
\end{itemize}
Il codice identificativo viene così assegnato:
\begin{center}
	\textbf{T[Tipo][ID]}
\end{center}
dove:
\begin{itemize}
	\item \textbf{T} indica test;
	\item\textbf{Tipo}: assume i seguenti valori:
	\begin{itemize}
		\item \textbf{I}: test d'integrazione;
		\item \textbf{U}: test d'unità;
		\item \textbf{R}: test di regressione;
		\item \textbf{V}: test di validazione;
	\end{itemize}
	\item \textbf{ID}: per i test d'unità, integrazione e regressione rappresenta un numero progressivo, per i test di sistema identifica il codice di un requisito presente all'interno dell'\AdRv{} come indicato nella sezione \S\ref{ClassificazioneRequisiti};
\end{itemize}