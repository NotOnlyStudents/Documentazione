\subsubsection{Analisi dei requisiti}
\myparagraph{Scopo dell'attività} 
L'attività di analisi dei requisti viene svolta dagli \textit{Analisti} incaricati per individuare i requisiti che il prodotto deve soddisfare.
I requisiti possono essere esplicitamente richiesti dal proponente o individuati implicitamente tramite l'attività di analisi, possono essere individuati direttamente o indirettamente perché dipendono da altri requisiti. \\
L'individuazione dei requisiti consente di:
\begin{itemize}
	\item Realizzare un prodotto software che soddisfi i bisogni del proponente;
	\item Facilitare le revisioni del codice;
	\item Fornire delle linee guida per le attività di test.
\end{itemize}

\myparagraph{Descrizione della sezione}
La sezione contiene le norme relative all'attività di analisi dei requisiti e del relativo documento \AdRv\ che i componenti del gruppo, in particolare gli \textit{Analisti}, devono seguire.

\myparagraph{Aspettative} 
L'attività di analisi dei requisiti ha come obiettivo la redazione del documento di \AdRv, che comprende in modo formale tutti i requisiti che il prodotto software deve o è desiderabile soddisfi.

\myparagraph{Struttura del documento}
La struttura del documento comprenderà:
\begin{itemize}
	\item \textbf{Descrizione:} Definisce le caratteristiche, i vincoli e gli obiettivi del prodotto;
	\item \textbf{Casi d'uso:} Contiene tutte le informazioni utili per descrivere gli scenari che l'utente del prodotto può incontrare, comprendendo anche la loro rappresentazione attraverso i diagrammi UML;
	\item \textbf{Requisiti:} Viene inserita una tabella contenente:
	\begin{itemize}
		\item Codice identificativo del requisito;
		\item Descrizione del requisito;
		\item Fonte dalla quale è stato individuato il requisito. 
	\end{itemize} 
\end{itemize}

\myparagraph{Classificazione dei requisiti}\label{ClassificazioneRequisiti}
Tutti i requisti sono individuati da un codice identificativo univoco, non modificabile dopo l'assegnazione, rappresentato nel seguente modo:

\begin{center}
	\textbf{R[Tipologia][Rilevanza][Numero][\_Codice]}
\end{center}

\begin{itemize}
	\item \textbf{Tipologia:} rappresenta il tipo di requisito, può assumere i seguenti valori:
	\begin{itemize}
		\item \textbf{V:} requisito di \textit{Vincolo};
		\item \textbf{F:} requisito \textit{Funzionale};
		\item \textbf{P:} requisito \textit{Prestazionale};
		\item \textbf{Q:} requisito di \textit{Qualità}.
	\end{itemize}

	\item \textbf{Rilevanza:} rappresenta l'utilità del requisito che va negoziata e concordata con il committente, può assumere i seguenti valori:
	\begin{itemize}
		\item \textbf{O:} requisito \textit{Obbligatorio}, deve essere soddisfatto perché irrinunciabile per qualcuno degli \glo{stakeholder};
		\item \textbf{D:} requisito \textit{Desiderabile}, non strettamente necessario ma aumenta la completezza del prodotto. Questi requisiti vengono negoziati con l'azienda \Proponente;
		\item \textbf{Z:} requisito \textit{Opzionale}, relativamente utile anch'esso ed è contrattabile con l'azienda \Proponente.
	\end{itemize}

	\item \textbf{Numero:} numero progressivo che parte da 1.
	
	\item \textbf{Codice:} opzionale, identifica il caso d'uso generico e gli eventuali sotto-casi ad esso associati, è rappresentato da:
	\begin{center}
		\textbf{[NumeroCasoBase](.NumeroSottoCaso)*}
	\end{center}
	dove NumeroCasoBase e NumeroSottoCaso sono rappresentati da numeri progressivi. \\
	I sotto-casi possono ramificarsi ulteriormente portandoli ad avere a loro volta altri sotto-casi.
\end{itemize}

\begin{table}[h]
	\centering
	\caption{Esempio di classificazione requisito} 
	
\rowcolors{2}{white}{celeste} 
\renewcommand{\arraystretch}{1.5}
\begin{tabular}{|c c c|} 
	
	\rowcolor{darkblue}
	\textcolor{white}{\textbf{Codice Requisito}}&
	\textcolor{white}{\textbf{Descrizione}}&
	\textcolor{white}{\textbf{Fonte}}\\	

	RFO1\_1.1 & ......... & Capitolato\\

\end{tabular}
\end{table}

\myparagraph{Classificazione dei Casi d'uso}
Tutti i casi d'uso analizzati sono individuati da un codice identificativo univoco, non modificabile dopo l'assegnazione, rappresentato nel seguente modo:

\begin{center}
	\textbf{UC[NumeroCasoBase](.NumeroSottoCaso)*}
\end{center}
dove:
\begin{itemize}
	\item \textbf{NumeroCasoBase:} è costituito da un numero progressivo che indica il caso d'uso generico;
	\item \textbf{NumeroSottoCaso} è costituito da un numero progressivo opzionale che indica il sotto-caso d'uso del caso d'uso generico.
\end{itemize}  
I sotto-casi possono avere a loro volta altri sotto-casi.\\ 
I casi d'uso analizzati comprenderanno:
\begin{itemize}
	\item \textbf{Codice identificativo:} assegnato secondo quanto stabilito precedentemente;
	\item \textbf{Nome:} titolo assegnato al caso d'uso e indicato dopo il codice identificativo; 
	\item \textbf{Rappresentazione grafica:} descrizione grafica del caso d'uso attraverso lo standard \glo{\textit{UML}};
	\item \textbf{Descrizione:} breve descrizione testuale del caso d'uso;
	\item \textbf{Attori:} rappresentano gli utenti che hanno un'interazione con il sistema, si dividono in:
	\begin{itemize}
		\item \textbf{Attori primari:} svolgono attivamente il caso d'uso;
		\item \textbf{Attori secondari:} sono entità estranee al sistema che supportano gli attori primari nelle loro attività.
	\end{itemize}
	\item \glo{\textbf{Precondizioni:}} condizioni del sistema prima degli eventi che determinano il caso d'uso;
	\item \glo{\textbf{Postcondizioni:}} condizioni del sistema dopo il verificarsi degli eventi che hanno determinato il caso d'uso;
	\item \textbf{Trigger:} opzionale, evento scatenante del caso d'uso;
	\item \textbf{Scenario principale:} elenco del flusso di eventi del caso d'uso;
	\item \textbf{Scenario alternativo:} opzionale, elenco delle azioni che si verificano allo scatenarsi di un evento non previsto rispetto allo scenario principale del caso d'uso;
	\item \textbf{Inclusioni:} opzionali, si ha inclusione quando un caso d'uso è incondizionatamente incluso nell'esecuzione del caso d'uso in esame;	
	\item \textbf{Estensioni:} opzionali, si ha estensione quando l'esecuzione di un caso d'uso interrompe l'esecuzione del caso d'uso in esame;
	\item \textbf{Generalizzazioni:} opzionali, si ha generalizzazione quando si intende aggiungere o modificare caratteristiche base di un caso d'uso ad un altro caso d'uso.
\end{itemize}

\myparagraph{Qualità dell'analisi dei requisiti}\label{QualitàAnalisi}
Come indicato dallo standard \glo{IEEE 830-1998}, la specifica deve essere:
\begin{itemize}
	\item \textbf{Priva di ambiguità:} i requisiti devono rispettare chiaramente i bisogni dell'utente finale del prodotto;
	\item \textbf{Corretta:} i requisiti devono essere coerenti rispetto alle richieste degli utenti finali;
	\item \textbf{Completa:} i requisiti presenti permettono la completa comprensione del dominio del problema;
	\item \textbf{Verificabile:} deve essere possibile verificare il soddisfacimento dei requisiti da parte del prodotto;
	\item \textbf{Consistente:} i requisiti non devono essere contraddittori tra loro;
	\item \textbf{Modificabile:} i requisiti devono poter essere modificati senza perdita di consistenza e completezza;
	\item \textbf{Tracciabile:} l'origine dei requisiti è chiara e facilmente rintracciabile;
	\item \textbf{Ordinata per rilevanza:} i requisiti devono essere classificati secondo la loro rilevanza rispetto a quanto contrattato con il proponente.
\end{itemize}