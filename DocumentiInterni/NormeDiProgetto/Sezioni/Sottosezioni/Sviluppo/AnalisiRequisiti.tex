\subsubsection{Analisi dei requisiti}
\myparagraph{Scopo dell'attività} 
L'attività di analisi dei requisti viene svolta dagli \textit{Analisti} incaricati per individuare i requisiti che il prodotto deve soddisfare.
I requisiti possono essere esplicitamente richiesti dal proponente o individuati implicitamente tramite l'attività di analisi, possono essere individuati direttamente o indirettamente perché dipendono da altri requisiti.\\
L'individuazione dei requisiti consente di:
\begin{itemize}
	\item Realizzare un prodotto software che soddisfi i bisogni del proponente;
	\item Facilitare le revisioni del codice;
	\item Fornire delle linee guida per le attività di test.
\end{itemize}

\myparagraph{Descrizione della sezione}
La sezione contiene le norme relative all'attività di analisi dei requisiti e del relativo documento \AdR\ che i componenti del gruppo, in particolare gli \textit{Analisti}, devono seguire.

\myparagraph{Aspettative}\label{AspettativeAnalisi}L'attività di analisi dei requisiti ha come obiettivo la redazione del documento di \AdR{} che comprende in modo formale tutti i requisiti che il prodotto software deve o è desiderabile soddisfi. Per assicurare il raggiungimento degli obiettivi di qualità previsti dal \PdQ\ il gruppo intende:
\begin{itemize}
	\item Coinvolgere frequentemente proponente e committente;
	\item Studiare prodotti simili a quello richiesto;
	\item Valutare i possibili scenari immedesimandosi nell'utente del prodotto;
	\item Individuare requisiti che possono migliorare la qualità del prodotto anche se non obbligatori;
	\item Assicurare che l'implementazione del prodotto soddisfi i requisiti individuati in modo coerente rispetto al loro significato.
\end{itemize}

\myparagraph{Struttura del documento}
La struttura del documento comprenderà:
\begin{itemize}
	\item \textbf{Descrizione:} definisce le caratteristiche, i vincoli e gli obiettivi del prodotto;
	\item \textbf{Casi d'uso:} contiene tutte le informazioni utili per descrivere gli scenari che l'utente del prodotto può incontrare, comprendendo anche la loro rappresentazione attraverso i diagrammi UML;
	\item \textbf{Requisiti:} viene inserita una tabella contenente:
	\begin{itemize}
		\item Codice identificativo del requisito;
		\item Descrizione del requisito;
		\item Fonte dalla quale è stato individuato il requisito. 
	\end{itemize} 
\end{itemize}

\myparagraph{Classificazione dei requisiti}\label{ClassificazioneRequisiti}Tutti i requisti sono individuati da un codice identificativo univoco rappresentato nel seguente modo:

\begin{center}
	\textbf{R[Tipologia][Rilevanza][Numero][\_Codice]}
\end{center}

\begin{itemize}
	\item \textbf{Tipologia:} rappresenta il tipo di requisito, può assumere i seguenti valori:
	\begin{itemize}
		\item \textbf{V:} requisito di \textbf{vincolo};
		\item \textbf{F:} requisito \textbf{funzionale};
		\item \textbf{P:} requisito \textbf{prestazionale};
		\item \textbf{Q:} requisito di \textbf{qualità}.
	\end{itemize}
	\item \textbf{Rilevanza:} rappresenta l'utilità del requisito che va negoziata e concordata con il committente, può assumere i seguenti valori:
	\begin{itemize}
		\item \textbf{O:} requisito \textbf{obbligatorio}, deve essere soddisfatto perché irrinunciabile per qualcuno degli \glo{stakeholder};
		\item \textbf{D:} requisito \textbf{desiderabile}, non strettamente necessario ma aumenta la completezza del prodotto. Questi requisiti vengono negoziati con l'azienda \Proponente;
		\item \textbf{Z:} requisito \textbf{opzionale}, relativamente utile anch'esso ed è contrattabile con l'azienda \Proponente.
	\end{itemize}
	\item \textbf{Numero:} numero progressivo che parte da 1.
	\item \textbf{Codice:} opzionale, identifica il caso d'uso generico e gli eventuali sotto-casi ad esso associati, è rappresentato da:
	\begin{center}
		\textbf{[NumeroCasoBase](.NumeroSottoCaso)*}
	\end{center}
	dove NumeroCasoBase e NumeroSottoCaso sono rappresentati da numeri progressivi.\\
	I sotto-casi possono ramificarsi ulteriormente portandoli ad avere a loro volta altri sotto-casi.
\end{itemize}

\begin{table}[h]
	\centering
	\caption{Esempio di classificazione requisito} 
	
\rowcolors{2}{white}{celeste} 
\renewcommand{\arraystretch}{1.5}
\begin{tabular}{|c c c|} 
	
	\rowcolor{darkblue}
	\textcolor{white}{\textbf{Codice Requisito}}&
	\textcolor{white}{\textbf{Descrizione}}&
	\textcolor{white}{\textbf{Fonte}}\\	

	RFO1\_1.1 & ......... & Capitolato\\
	\hline
\end{tabular}
\end{table}

\myparagraph{Requisiti di processo}\label{RequisitiProcesso}Dopo un confronto con il \VT\ si è ritenuto opportuno riportare i requisiti presenti nell'\AdR\ ritenuti da noi requisiti di processo all'interno di questo documento.\\
\begin{table}[h]
	\centering
	\caption{Requisiti di processo} 
\rowcolors{2}{white}{celeste} 
\renewcommand{\arraystretch}{1.5}
\begin{tabular}{|C{9.5cm} C{2.5cm}|} 
	
	\rowcolor{darkblue}
	\textcolor{white}{\textbf{Descrizione}}&
	\textcolor{white}{\textbf{Fonte}}\\	

	La piattaforma deve essere sviluppata \glo{production-ready}. & Capitolato\\
	Il codice sorgente deve essere realizzato con un sistema di \glo{versionamento} e caricato in una \glo{repository} su \glo{GitHub}. & Capitolato\\
	Il codice sorgente deve essere sottoposto ad analisi statica attraverso lo strumento \glo{Typescript-eslint}. & Capitolato\\
	Devono essere realizzati test di integrazione per verificare l’integrazione con i sottosistemi. & Capitolato\\
	Devono essere realizzati test d’unità per verificare le singole componenti del prodotto. & Interna\\
	\hline
\end{tabular}
\end{table}

\myparagraph{Classificazione dei casi d'uso}
Tutti i casi d'uso analizzati sono individuati da un codice identificativo univoco rappresentato nel seguente modo:

\begin{center}
	\textbf{UC[NumeroCasoBase](.NumeroSottoCaso)*}
\end{center}
dove:
\begin{itemize}
	\item \textbf{NumeroCasoBase:} è costituito da un numero progressivo che indica il caso d'uso generico;
	\item \textbf{NumeroSottoCaso} è costituito da un numero progressivo opzionale che indica il sotto-caso d'uso del caso d'uso generico.
\end{itemize}  
I sotto-casi possono avere a loro volta altri sotto-casi.\\ 
I casi d'uso analizzati comprenderanno:
\begin{itemize}
	\item \textbf{Codice identificativo:} assegnato secondo quanto stabilito precedentemente;
	\item \textbf{Nome:} titolo assegnato al caso d'uso e indicato dopo il codice identificativo;
	\item \textbf{Rappresentazione grafica:} descrizione grafica del caso d'uso attraverso lo standard \glo{UML};
	\item \textbf{Descrizione:} breve descrizione testuale del caso d'uso;
	\item \textbf{Attori:} rappresentano gli utenti che hanno un'interazione con il sistema, si dividono in:
	\begin{itemize}
		\item \textbf{Attori primari:} svolgono attivamente il caso d'uso;
		\item \textbf{Attori secondari:} sono entità estranee al sistema che supportano gli attori primari nelle loro attività.
	\end{itemize}
	\item \glo{\textbf{Precondizioni:}} condizioni del sistema prima degli eventi che determinano il caso d'uso;
	\item \glo{\textbf{Postcondizioni:}} condizioni del sistema dopo il verificarsi degli eventi che hanno determinato il caso d'uso;
	\item \textbf{Trigger:} opzionale, evento scatenante del caso d'uso;
	\item \textbf{Scenario principale:} elenco del flusso di eventi del caso d'uso;
	\item \textbf{Scenario alternativo:} opzionale, elenco delle azioni che si verificano allo scatenarsi di un evento non previsto rispetto allo scenario principale del caso d'uso;
	\item \textbf{Inclusioni:} opzionali, si ha inclusione quando un caso d'uso è incondizionatamente incluso nell'esecuzione del caso d'uso in esame;
	\item \textbf{Estensioni:} opzionali, si ha estensione quando l'esecuzione di un caso d'uso interrompe l'esecuzione del caso d'uso in esame;
	\item \textbf{Generalizzazioni:} opzionali, si ha generalizzazione quando si intende aggiungere o modificare caratteristiche base di un caso d'uso a un altro caso d'uso.
\end{itemize}

\myparagraph{Qualità dell'analisi dei requisiti}\label{QualitàAnalisi}Come indicato dallo standard IEEE 830-1998, la specifica deve essere:
\begin{itemize}
	\item \textbf{Priva di ambiguità:} i requisiti devono rispettare chiaramente i bisogni dell'utente finale del prodotto;
	\item \textbf{Corretta:} i requisiti devono essere coerenti rispetto alle richieste degli utenti finali;
	\item \textbf{Completa:} i requisiti presenti permettono la completa comprensione del dominio del problema;
	\item \textbf{Verificabile:} deve essere possibile verificare il soddisfacimento dei requisiti da parte del prodotto;
	\item \textbf{Consistente:} i requisiti non devono essere contraddittori tra loro;
	\item \textbf{Modificabile:} i requisiti devono poter essere modificati senza perdita di consistenza e completezza;
	\item \textbf{Tracciabile:} l'origine dei requisiti è chiara e facilmente rintracciabile;
	\item \textbf{Ordinata per rilevanza:} i requisiti devono essere classificati secondo la loro rilevanza rispetto a quanto contrattato con il proponente.
\end{itemize}

\myparagraph{Metriche}\label{MAnalisi}Di seguito vengono presentate le metriche utilizzate per garantire il controllo sulla qualità. Per una descrizione sullo standard di riferimento e un elenco completo di tutte le metriche applicate si rimanda alla sezione \S\ref{9126} dell'appendice.
\begin{itemize}
\item\textbf{MPR01: Soddisfacimento requisiti obbligatori}\hypertarget{MSoddRequisiti}\\
Indica la percentuale di requisiti obbligatori soddisfatti nel momento in cui viene calcolato.\\
Viene calcolato nel seguente modo:
	\begin{center}
		\textbf{RO=\(\frac{ROC}{RO}\)*100}
	\end{center}
	dove:
	\begin{itemize}
		\item \textbf{RO} sta per \textbf{requisiti obbligatori};
		\item \textbf{ROC} sta per \textbf{requisiti obbligatori coperti dall'implementazione}.
	\end{itemize}
\end{itemize}