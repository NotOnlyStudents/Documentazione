\subsubsection{Codifica software}
\myparagraph{Scopo dell'attività} 
L'attività di codifica svolta dai \textit{Programmatori} ha lo scopo di trasformare in codice quanto previsto dall'attività di progettazione svolta dai \textit{Progettisti}, procedendo all'effettiva realizzazione del prodotto software richiesto.

\myparagraph{Descrizione della sezione}
La sezione tratta le norme da seguire per ogni linguaggio di programmazione utilizzato durante lo svolgimento del progetto.

\myparagraph{Aspettative}
Con le norme individuate il gruppo si aspetta la realizzazione di codice uniforme.

\myparagraph{Qualità della codifica}
La codifica è di qualità se: 
\begin{itemize}
	\item Il codice è facilmente leggibile;
	\item I costrutti del linguaggio sono utilizzati in modo chiaro e coerente;
	\item La compilazione non presenta errori fatali o potenziali.
\end{itemize}
Queste caratteristiche sono in grado di agevolare manutenzione, verifica e validazione e di conseguenza migliorare la qualità di prodotto.

\myparagraph{Convenzioni generiche}\label{CodificaConvenzioni}Vengono di seguito riportate le norme generiche stabilite per la codifica.
\paragraph*{Intestazione}
I file sorgenti consegnati presentano la seguente intestazione inserita in un blocco di commento.
\paragraph*{Nomenclatura}
Tutte le classi, i metodi, le variabili devono avere un nome univoco, esplicativo e scritto in lingua inglese, in particolare:
\begin{itemize}
	\item Per i nomi di cartelle, file e classi viene seguita la convenzione \textit{CamelCase};
	\item I nomi di variabili e metodi hanno iniziale minuscola, se composti da più parole la prima lettera delle parole successive alla prima è maiuscola;
	\item Le costanti vengono scritte in maiuscolo.
\end{itemize}
\paragraph*{Parentesi}
I blocchi di codice vanno inseriti tra parentesi graffe, anche se il blocco è vuoto o costituito da una sola riga di codice.
\paragraph*{Verbosità}
Una riga di codice deve essere lunga al massimo 140 caratteri.
Se possibile è desiderabile definire metodi brevi evitando la ricorsione.


