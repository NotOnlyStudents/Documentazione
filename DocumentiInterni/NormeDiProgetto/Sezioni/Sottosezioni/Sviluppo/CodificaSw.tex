\subsubsection{Codifica Software}
\myparagraph{Scopo dell'attività} 
L'attività di Codifica svolta dai \textit{Programmatori} ha lo scopo di trasformare in codice quanto previsto dall'attività di Progettazione svolta dai \textit{Progettisti}, procedendo all'effetiva realizzazione del prodotto software richiesto.

\myparagraph{Descrizione della sezione}
La sezione tratta le norme da seguire per ogni linguaggio di programmazione utilizzato durante lo svolgimento del progetto.

\myparagraph{Aspettative}
Con le norme individuate il gruppo si aspetta la realizzazione di codice uniforme.

\myparagraph{Qualità della Codifica}
La Codifica è di qualità se: 
\begin{itemize}
	\item Il codice è facilmente leggibile;
	\item I costrutti del linguaggio sono utilizzati in modo chiaro e coerente;
	\item la compilazione non presenta errori fatali o potenziali.
\end{itemize}
Queste caratteristiche sono in grado di agevolare manutenzione, verifica e validazione e di conseguenza migliorare la qualità di prodotto.

\myparagraph{Convenzioni generiche}\label{CodificaConvenzioni}
Di seguito si riportano le norme generiche stabilite per la codifica.
\paragraph*{Intestazione}
I file sorgenti consegnati presentano la seguente intestazione inserita in un blocco di commento.
\paragraph*{Nomenclatura}
Tutte le classi, i metodi, le variabili devono avere un nome univoco, esplicativo e scritto in lingua inglese, in particolare:
\begin{itemize}
	\item Per i nomi di cartelle, file e classi viene seguita la convenzione CamelCase;
	\item I nomi di variabili e metodi hanno iniziale minuscola, se composti da più parole la prima lettera delle parole successive alla prima è maiuscola;
	\item Le costanti vengono scritte in maiuscolo;
\end{itemize}
\paragraph*{Indentazioni}
I blocchi di codice vanno correttamente innestati, i membri impostano a (4) spazi la tabulazione nel proprio editor o \glo{IDE}.
\paragraph*{Parentesi}
I blocchi di codice vanno inseriti tra parentesi grafe, anche se il blocco è vuoto o costituito da una sola riga di codice. \\
Le parentesi vanno inserite in linea.
\paragraph*{Verbosità}
Una riga di codice deve essere lunga al massimo 140 caratteri.
Se possibile è desiderabile definire metodi brevi evitando la ricorsione.


