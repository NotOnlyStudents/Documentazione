\subsubsection{Codifica software}
\myparagraph{Scopo dell'attività} 
L'attività di codifica svolta dai \textit{Programmatori} ha lo scopo di trasformare in codice quanto previsto dall'attività di progettazione svolta dai \textit{Progettisti}, procedendo all'effettiva realizzazione del prodotto software richiesto.

\myparagraph{Descrizione della sezione}
La sezione tratta le norme da seguire per ogni linguaggio di programmazione utilizzato durante lo svolgimento del progetto.

\myparagraph{Aspettative}\label{AspettativeCodifica}Con le norme individuate il gruppo si aspetta la realizzazione di codice uniforme. Per assicurare il raggiungimento degli obiettivi di qualità previsti dal \PdQ\ il gruppo intende:
\begin{itemize}
	\item Utilizzare i linguaggi sfruttando adeguatamente le loro capacità;
	\item Ridurre il numero di righe di codice che possono causare errori;
	\item Redarre il codice affinché risulti essere chiaro nel tempo e facilmente comprensibile anche da altri programmatori.
\end{itemize}

\myparagraph{Qualità della codifica}
La codifica è di qualità se: 
\begin{itemize}
	\item Il codice è facilmente leggibile;
	\item I costrutti del linguaggio sono utilizzati in modo chiaro e coerente;
	\item La compilazione non presenta errori fatali o potenziali.
\end{itemize}
Queste caratteristiche sono in grado di agevolare manutenzione, verifica e validazione e di conseguenza migliorare la qualità di prodotto.

\myparagraph{Convenzioni generiche}\label{CodificaConvenzioni}
Vengono di seguito riportate le norme generiche stabilite per la codifica.
\paragraph*{Intestazione}
I file sorgenti consegnati presentano la seguente intestazione inserita in un blocco di commento.
\paragraph*{Nomenclatura}
Tutte le classi, i metodi, le variabili devono avere un nome univoco, esplicativo e scritto in lingua inglese, in particolare:
\begin{itemize}
	\item Per i nomi di cartelle, file e classi viene seguita la convenzione \textbf{CamelCase};
	\item I nomi di variabili e metodi hanno iniziale minuscola, se composti da più parole la prima lettera delle parole successive alla prima è maiuscola;
	\item Le costanti vengono scritte in maiuscolo.
\end{itemize}
\paragraph*{Parentesi}
I blocchi di codice vanno inseriti tra parentesi graffe, anche se il blocco è vuoto o costituito da una sola riga di codice.
\paragraph*{Verbosità}
Una riga di codice deve essere lunga al massimo 140 caratteri. Se possibile è desiderabile definire metodi brevi evitando la ricorsione.

\myparagraph{Metriche}\label{MCodifica}Di seguito vengono presentate le metriche utilizzate per garantire il controllo sulla qualità. Per una descrizione sullo standard di riferimento e un elenco completo di tutte le metriche applicate si rimanda a \S\ref{9126} dell'appendice. 
\begin{itemize}
	\item\textbf{MPDS01: Completezza dell'implementazione}\hypertarget{MCImplementazione}\\
	L'indice stabilisce la completezza del prodotto software realizzato nel momento in cui viene calcolato rispetto a tutti i requisiti previsti dall'\AdRv{2.0}.\\
	Viene calcolato nel seguente modo:
	\begin{center}
		\textbf{CI=\(1-\frac{FNI}{FI}\)*100}
	\end{center}
	dove:
	\begin{itemize}
		\item \textbf{FNI} indica il numero di funzionalità non implementate;
		\item \textbf{FI} indica il numero di funzionalità implementate.
	\end{itemize}
\item\textbf{MPDS02: Densità errori} \hypertarget{MErrori}\\
Rappresenta la percentuale di righe di codice che possono causare errori.\\
Viene calcolato nel seguente modo:
\begin{center}
	\textbf{DE=\(\frac{RB}{RTOT}\)*100}
\end{center}
dove:
\begin{itemize}
	\item \textbf{RB} indica il numero di linee di codice che possono causare imprevisti o comportamenti diversi da quelli desiderati;
	\item \textbf{RTOT} indica il numero di linee di codice totali.
\end{itemize}

\item\textbf{MPDS07: Facilità di comprensione} \hypertarget{MFComprensione}\\
Il valore indica la facilità di comprensione del codice in base al numero di righe di commento presenti.\\
Viene calcolato nel seguente modo:
\begin{center}
	\textbf{FC=\(\frac{LCOD}{LCOM}\)*100}
\end{center}
dove
\begin{itemize}
	\item \textbf{LCOM} indica il numero di linee di commento;
	\item \textbf{LCOD} indica il numero di linee di codice.
\end{itemize}

\item\textbf{MPDS08: Semplicità delle funzioni}\hypertarget{MSFunzioni}\\
Il valore stabilisce la semplicità di un metodo basandosi sul numero di parametri a lui necessari.

\item\textbf{MPDS09: Semplicità delle classi}\hypertarget{MSClassi}\\
Il valore stabilisce la semplicità di una classe basandosi sul numero di metodi al suo interno.
\end{itemize}