\section{Introduzione}\label{Introduzione}
\subsection{Scopo del Documento}\label{ScopoDocumento}
Questo documento contiene le linee guida che tutti i componenti del gruppo sono tenuti a rispettare durante lo svolgimento del progetto, comprende quindi norme, tecnologie e strumenti che il gruppo \Gruppo\ intende utilizzare per la realizzazione del prodotto finale. 
Le norme presenti potranno subire dei cambiamenti quali aggiunte, rimozioni e modifiche che saranno comunicate dal \Responsabile\ a ciascun componente del gruppo.

\subsection{Scopo generale del prodotto}
Il progetto {\NomeProgetto} ha come scopo quello di rendere disponibile un servizio e-commerce sfruttando però tutti i vantaggi di un'architettura Serverless:
\begin{itemize}
    \item Gli sviluppatori potranno concentrare la propria attenzione sullo sviluppo del prodotto finale invece di focalizzarsi sulla gestione e sul funzionamento di server e di runtime, che siano nel cloud o in locale.
    \item Comodità nel costruire un insieme di chiamate asincrone che rispondono a diversi clienti contemporaneamente.
    \item Minori costi di sviluppo e di produzione.
    \item Semplicità nel suddividere il progetto in un insieme di microservizi.
\end{itemize}

\subsection{Glossario}\label{Glossario}
Al fine di rendere il documento più chiaro e leggibile si fornisce un \textit{Glossario}. I termini che possono assumere un significato ambiguo sono indicati da una 'G' ad apice.

\subsection{Riferimenti}
\subsubsection{Normativi}\label{RiferimentiNormativi}
\begin{itemize}
\item \url{https://www.math.unipd.it/~tullio/IS-1/2020/Progetto/C2.pdf}
\end{itemize}

\subsubsection{Informativi}
\begin{itemize}
\item \PdPv{}
\item \PdQv{}
\end{itemize}