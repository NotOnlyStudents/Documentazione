\section{Introduzione}\label{Introduzione}
\subsection{Scopo del documento}\label{ScopoDocumento}
Questo documento contiene le linee guida che tutti i componenti del gruppo sono tenuti a rispettare durante lo svolgimento del progetto, comprende quindi norme, tecnologie e strumenti che il gruppo \Gruppo\ intende utilizzare per la realizzazione del \glo{prodotto} finale. 
Le norme presenti potranno subire dei cambiamenti quali aggiunte, rimozioni e modifiche che saranno comunicate dal \Responsabile\ a ciascun componente del gruppo.

\subsection{Scopo generale del prodotto}
Il progetto {\NomeProgetto} ha come scopo quello di rendere disponibile un servizio \glo{e-commerce} sfruttando però tutti i vantaggi di un'architettura \glo{Serverless}:
\begin{itemize}
    \item Gli sviluppatori potranno concentrare la propria attenzione sullo sviluppo del prodotto finale invece di focalizzarsi sulla gestione e sul funzionamento di server e di runtime, che siano nel \glo{cloud} o in locale;
    \item Comodità nel costruire un insieme di chiamate \glo{asincrone} che rispondono a diversi clienti contemporaneamente;
    \item Minori costi di sviluppo e di produzione;
    \item Semplicità nel suddividere il progetto in un insieme di \glo{microservizi}.
\end{itemize}

\subsection{Glossario}\label{Glossario}
Al fine di rendere il documento più chiaro e leggibile si fornisce un \textit{Glossario}. I termini che possono assumere un significato ambiguo sono indicati da una 'G' ad apice.

\subsection{Riferimenti}
\subsubsection{Normativi}\label{RiferimentiNormativi}
\begin{itemize}
\item \textbf{Capitolato d'Appalto C2:} \\ \url{https://www.math.unipd.it/~tullio/IS-1/2020/Progetto/C2.pdf}
\end{itemize}

\subsubsection{Informativi}\label{RiferimentiInformativi}
\begin{itemize}
	\item \textbf{Processi SW - Materiale del corso di Ingegneria del Software}:
	\begin{itemize}
		\item \url{https://www.math.unipd.it/~tullio/IS-1/2020/Dispense/L03.pdf}
		\item \url{https://www.math.unipd.it/~tullio/IS-1/2009/Approfondimenti/ISO_12207-1995.pdf}
	\end{itemize}
	\item \textbf{Gestione di progetto - Materiale del corso di Ingegneria del Software}:\\
	\url{https://www.math.unipd.it/~tullio/IS-1/2020/Dispense/L06.pdf}
	\item \textbf{Analisi dei Requisiti - Materiale del corso di Ingegneria del Software}:\\
	\url{https://www.math.unipd.it/~tullio/IS-1/2020/Dispense/L07.pdf}
	\item \textbf{Progettazione - Materiale del corso di Ingegneria del Software}:\\
	\url{https://www.math.unipd.it/~tullio/IS-1/2020/Dispense/L09.pdf}
	\item \textbf{Verifica - Materiale del corso di Ingegneria del Software}:
	\begin{itemize}
		\item \url{https://www.math.unipd.it/~tullio/IS-1/2020/Dispense/L15.pdf}
		\item \url{https://www.math.unipd.it/~tullio/IS-1/2020/Dispense/L16.pdf}
	\end{itemize}
	\item \textbf{Standard ISO/IEC 12207:1995}:\\
	\url{https://www.math.unipd.it/~tullio/IS-1/2009/Approfondimenti/ISO_12207-1995.pdf}
	\item \textbf{Diagrammi UML - Materiale del corso di Ingegneria del Software}:
	\begin{itemize}
		\item \url {https://www.math.unipd.it/%7Ercardin/swea/2021/Diagrammi%20delle%20Classi_4x4.pdf}
		\item \url {https://www.math.unipd.it/%7Ercardin/swea/2021/Diagrammi%20dei%20Package_4x4.pdf}
		\item \url {https://www.math.unipd.it/%7Ercardin/swea/2021/Diagrammi%20di%20Attività_4x4.pdf}
		\item \url {https://www.math.unipd.it/%7Ercardin/swea/2021/Diagrammi%20di%20Sequenza_4x4.pdf}
		\item \url{https://www.math.unipd.it/%7Ercardin/swea/2021/Diagrammi%20Use%20Case_4x4.pdf}		
	\end{itemize}
	\item \textbf{Standard ISO/IEC 15504}: 
	\begin{itemize}
			\item \url{http://museo.tecnoteca.it/sezioni/qualita/norme/iso15504}
			\item \url{http://www.spice121.com/cms/en/about-spice-1-2-1.html}
	\end{itemize}
	\item \textbf{Standard ISO/IEC 9126}:\\
	\url{http://www.colonese.it/00-Manuali_Pubblicatii/07-ISO-IEC9126_v2.pdf}
	\item \textbf{PDCA - Ciclo di Deming}:\\
	\url{https://www.ionos.it/startupguide/produttivita/pdca/}
\end{itemize}