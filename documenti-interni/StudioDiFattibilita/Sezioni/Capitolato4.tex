\documentclass[a4paper, oneside, dvipsnames, table]{article}
%\usepackage{../../utils/Stiletemplate}
%\usepackage{hyperref}
%\usepackage{fancyhdr}
\usepackage[italian]{babel}
\usepackage[utf8]{inputenc}

\begin{document}
\section{Capitolato C4}
\subsection{Informazioni sul capitolato}
\begin{itemize}
\item Titolo: $HD$ $VIZ:$ $visualizzazione$ $di$ $dati$ $con$ $molte$ $dimensioni$
\item Proponente: Zucchetti Software
\item Committenti: Proff. Tullio e Cardin
\end{itemize}

\subsection{Descrizione del capitolo}
Il presente capitolato ha come obiettivo finale la realizzazione di un’applicazione di visualizzazione di dati con molte dimensioni a supporto della fase esplorativa dell’analisi dei dati attraverso l'utilizzo della libreria D3.js

\subsection{Tecnologie coinvolte}
Lo sviluppo sarà prevalentemente in tecnologia HTML/CSS/JavaScript utilizzando la libreria D3.js.
La parte server di supporto alla presentazione nel browser e alle query ad un database SQL o NoSQL potr\'a essere sviluppata in Java con server Tomcat o in Javascript con server Node.js.

\subsection{Vincoli}
Per raggiungere gli obiettivi minimi del progetto viene richiesto:

\begin{itemize}
\item I dati da visualizzare dovranno poter avere almeno fino a 15
dimensioni, ma deve essere possibile visualizzare anche dati con meno dimensioni
\item I dati devono poter essere forniti al sistema di visualizzazione sia con query ad un database che da file in formato CSV preparati precedentemente
\item dovr\'a presentare almeno le seguenti visualizzazioni:
\begin{itemize}
\item Scatter plot Matrix: presentazione a riquadri disposti a matrice di tutte le combinazioni di scatter plot
\item Force Field: traduce le distanze nello spazio a molte dimensioni di forze di attrazione e repulsione tra i punti proiettati nello spazio bidimensionale
\item Heat map: trasforma la distanza tra i punti di colori più o meno intensi
\item  posiziona i punti dello spazio multidimensionale in un piano
cartesiano: posiziona i punti dello spazio multidimensionale in un piano
cartesiano
\end{itemize}
\item Ordinare i punti nel grafico “Heat map” per evidenziare i “cluster” presenti nei dati.
\end{itemize}

\subsection{Aspetti positivi}
\begin{itemize}
\item Componenti del gruppo che hanno già competene in:
\begin{itemize}
    \item JavaScript: 7/7
    \item HTML: 7/7
    \item CSS: 7/7
\end{itemize}
\end{itemize}

\subsection{Aspetti critici}
\begin{itemize}
\item Componenti del gruppo che conoscono la libreria D3.js 0/7
\item La complessità delle tecnologie da sviluppare dovuta dall'alta specializzazione del progetto
\end{itemize}

\subsection{Conclusioni}
Nonostante il capitolato piacesse alla magioranza del gruppo, è stato escluso a causa dell'elevata difficoltà nello studio degli argomenti da sviluppare.
\end{document}