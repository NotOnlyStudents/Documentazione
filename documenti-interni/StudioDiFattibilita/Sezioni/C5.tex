\section{Capitolato C5}
\subsection{Titolo del capitolato}
Il capitolato in questione si chiama \textit{"PORTACS: piattaforma di controllo mobilità autonoma"}, il proponente \`e l'azienda \textit{SANMARCO INFORMATICA} e i committenti sono \VT{} e \CR{}.

\subsection{Descrizione del capitolo}
Il capitolato ha come obiettivo finale la realizzazione di una piattaforma che visualizzi in real time un mappa che rappresenta uno spazio in cui delle unit\'a possono muoversi (che possono rappresentare un robot, un muletto o un’automobile). Ogni unit\'a deve inviare al sistema centrale costantemente la propria posizione, direzione e velocit\'a, in modo tale che il sistema centrale piloti e coordini tutte le unit\'a per evitare incidenti e ingorghi. 

\subsection{Prerequisiti e tecnologie coinvolte}
L'azienda non ha espresso preferenze riguardo alle tecnologie da utilizzare.

\subsection{Vincoli}
Per raggiungere gli obiettivi minimi del progetto viene richiesto che il software accetti:

\begin{itemize}
\item il Sistema sviliuppato deve essere in real-time;
\item una scacchiera o mappa con definizione di percorrenze, relativi vincoli e Point of Interest;
\item definizione delle N unità (Identificativo di sistema, velocità massima, posizione iniziale lista dei punti di interesse da attraversare.
\end{itemize}

\subsection{Aspetti positivi}
\begin{itemize}
\item Non ci sono restrizioni particolari sulle tecnologie da usare, quindi possiamo scegliere quelle più conosciute dal gruppo
\item Componenti del gruppo che esprimono Preferenza per questo capitolato: 7/7.
\end{itemize}

\subsection{Aspetti critici}
\begin{itemize}
\item Componenti del gruppo che hanno lavorato con sistemi in real-time: 1/7.
\end{itemize}

\subsection{Conclusioni}