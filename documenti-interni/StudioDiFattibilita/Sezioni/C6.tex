\section{Capitolato C6 - Realtime Gaming Platform}
\subsection{Informazioni sul capitolato}
\begin{itemize}
	\item \textbf{Nome:} \textit{RGP: Realtime Gaming Platform}
	\item \textbf{Proponente:} \textit{zero12}
	\item \textbf{Committenti:} \textit{\VT{} e \CR{}}
\end{itemize}

\subsection{Descrizione del capitolato}
Il capitolato richiede lo sviluppo di un videogioco a scorrimento verticale utilizzabile sia in modalità multiplayer, da 2 a 6 giocatori, che single-player. Il progetto si focalizza maggiormente sulla componente server che, tramite tecnologie Amazon Web Services, permette di gesitre la comunicazione in tempo reale tra i diversi device collegati nella stessa sessione di gioco.

\subsection{Scopo del capitolato}
Lo scopo del capitolato consiste nello sviluppo di un gioco per IOS o Android che soddisfi i seguenti obiettivi:
\begin{itemize}
	\item deve essere multiplayer da 2 a 6 giocatori;
	\item deve avere una modalità fantasma;
	\item devono essere presenti gli stessi powerup e nemici durante il multiplayer;
	\item il gioco deve essere infinito;
	\item il server deve essere realizzato con tecnologie serverless.
\end{itemize}

\subsection{Tecnologie coinvolte}
Per questo progetto sono state raccomandate le seguenti tecnologie:
\begin{itemize}
	\item AWS GameLift: servizio di hosting per i giochi online;
	\item AWS appsync: servizio gestito per lo sviluppo rapido di API GraphQL;
	\item AWS DynamoDB: servizio di database NoSQL rapido e flessibile per il salvataggio dei dati in cloud;
	\item Node.js come linguaggio utilizzato per lo sviluppo di componenti server;
	\item Swift e Kotlin come linguaggi di programmazione per lo sviluppo mobile, il primo nel caso si scelga la realizzazione dell'applicativo su iOS mentre il secondo se si opta per Android.
\end{itemize}

\subsection{Vincoli}
Per raggiungere gli obiettivi minimi del progetto viene richiesto di:
\begin{itemize}
	\item svolgere un'attività di analisi per scegliere la migliore tecnologia basata su AWS su cui sviluppare la componente server del progetto fornendo una giustificazione di tale scelta;
	% \item sviluppare le API in Node.js;
	\item sviluppare l'intero progetto a micro-servizi;
	\item sviluppare un'applicazione mobile per testare la correttezza delle dinamiche del gioco in realtime.
\end{itemize}

 \subsection{Aspetti positivi}
\begin{itemize}
	\item Utilizzo di vari servizi AWS sempre più diffusi.
	\item L'uso di un'architettura a micro-servizi e serverless favorisce la parallelizzazione dello sviluppo del back-end.
\end{itemize}

 \subsection{Aspetti critici}
\begin{itemize}
	\item Sviluppare un gioco per iOS è condizionato dal possedere un MacBook o un iMac, inoltre Swift è conosciuto solo da pochi componenti del gruppo e non in modo approfondito.
	\item Tutti i componenti del gruppo non hanno mai sviluppato un'applicazione.
	\item Il seguente capitolato non stimola l'interesse di alcun componente del gruppo.
\end{itemize}

\subsection{Conclusioni}
Dopo un'attenta valutazione e discussione, si è deciso che questo capitolato, nonostante tocchi delle tecnologie piuttosto interessanti, richiede un dispendio troppo elevato di tempo per l'apprendimento dei linguaggi di programmazione.
