\section{Capitolato C6}
\subsection{Titolo del capitolato}
Il capitolato in questione si chiama \textit{"RGP: Realtime Gaming Platform"}, il proponente \`e l'azienda \textit{ZERO12} e i committenti sono \VT{} e \CR{}.

\subsection{Descrizione del capitolo}
Il capitolato ha come obiettivo finale la realizzazione di un'applicazione mobile che implementa un videogioco a scorrimento verticale con possibilità di giocare in un multiplayer realtime. La parte principale del progetto rappresenta la possibilità di giocare in modalità multiplayer con sfida ad eliminazione, non vi è interazione tra i giocatori. Nella modalità single player il gioco sarà in modalità infinta con livelli di difficoltà crescenti e terminerà quando saranno state perse tutte le vite o quando non si avranno raccolto i powerup per mantenere il proprio oggetto attivo.

\subsection{Prerequisiti e tecnologie coinvolte}
Il progetto si basa sulla tecnologia \glo{AWS}.
Per lo sviluppo di codice è da preferire NodeJS.
Per lo sviluppo mobile il linguaggio richiesto è Swift/SwiftUI (IOS) o Kotlin (Android).

\subsection{Vincoli}
Per raggiungere gli obiettivi minimi del progetto viene richiesto che:

\begin{itemize}
\item l'architettura server sia scalabile;
\end{itemize}


\subsection{Aspetti positivi}
\begin{itemize}
\item Componenti del gruppo che hanno già sviluppato in nodejs: 2/7.
\item Componenti del gruppo che hanno già svilupatto applicazione per l'ambiente Android: 2/7.
\end{itemize}

\subsection{Aspetti critici}
Doppia Applicazione (IOS),AWS.
\begin{itemize}
\item Componenti del gruppo che esprimono Preferenza per questo capitolato: 2/7.
\item \'E richiesto il doppio dello sviluppo per l'applicazione mobile visto che ne serve una versione per IOS e una per Android.
\item Componenti del gruppo che hanno gi\'a sviluppato almeno un'app IOS: 0/7.
\item Componenti del gruppo che hanno gi\'a sviluppato con AWS: 0/7.
\end{itemize}

\subsection{Conclusioni}