\section{Capitolato C7}
\subsection{Titolo del capitolato}
Il capitolato in questione si chiama \textit{"SSD: soluzioni di sincronizzazione desktop"}, il proponente \`e l'azienda \textit{ZEXTRAS} e i committenti sono \VT{} e \CR{}.

\subsection{Descrizione del capitolo}
Il capitolato ha come obiettivi la realizzazione di:
\begin{itemize}
\item un algoritmo solido ed efficiente in grado di garantire il salvataggio in cloud del lavoro e contemporaneamente la sincronizzazione dei cambiamenti presenti in cloud;
\item un’interfaccia multipiattaforma per l’uso dell’algoritmo nei pi\'u importanti sistemi operativi desktop esistenti;
\item utilizzare l’algoritmo sviluppato per richiedere e fornire i cambiamenti ai contenuti in sincronizzazione verso il prodotto Zextras Drive
\end{itemize}

\subsection{Prerequisiti e tecnologie coinvolte}
L'azienda consiglia di utilizzare:

\begin{itemize}
\item \glo{Qt} per lo sviluppo dell'interfaccia e del controller d'architettura;
\item \glo{Python} per lo sviluppo della Business Logic.
\end{itemize}

\subsection{Vincoli}
Per raggiungere gli obiettivi minimi del progetto viene richiesto di:

\begin{itemize}
\item sviluppare la soluzione indipendente dall'installazione di framework terzi per funzionare;
\item scansione dei codici nel tempo sufficiente a certificare la presenza della persona in postazione; 
\item avere le componenti applicative correlate da test unitari e d’integrazione. Inoltre, \'e richiesto che il sistema venga testato nella sua interezza tramite test end-to-end.
\end{itemize}

\subsection{Aspetti positivi}
\begin{itemize}
\item Componenti del gruppo che hanno già sviluppato in C++ Attraverso Qt: 7/7.
\item Componenti del gruppo che hanno già sviluppato un'applicazione in Python: 2/7.
\end{itemize}

\subsection{Aspetti critici}
\begin{itemize}
\item Componenti del gruppo che esprimono Preferenza per questo capitolato: 1/7.
\end{itemize}

\subsection{Conclusioni}