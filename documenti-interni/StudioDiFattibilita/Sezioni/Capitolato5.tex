%\section{Capitolato C5 - PORTACS: piattaforma di controllo mobilità autonoma}

  \subsection{Informazioni generali}
  \begin{description}
    \item[Nome] \textit{PORTACS: piattaforma di controllo mobilità autonoma}
    \item[Proponente] \textit{Sanmarco Informatica}
    \item[Committente] \VT{}, \CR{}
  \end{description}

  \subsection{Descrizione del capitolato}
Il progetto chiede di realizzare di un sistema real time in grado di gestire più entità caratterizzate da un punto di partenza, una velocità di crociera ed una lista di punti di interesse da raggiungere. La componente server deve segnalare ad ogni unità la prossima azione da fare in base ai prossimi punti di interesse e alle posizioni delle altre entità. Inoltre, deve essere presente una mappa real-time che consenta la visualizzazione di ogni unità e dei loro movimenti all'interno della struttura. Viene richiesto di implementare una interfaccia grafica per ogni unità dove vengono indicate le informazioni caratterizzanti.

  \subsection{Finalità del progetto}
Il proponente consiglia di impostare il prodotto nel seguente modo:
  \begin{itemize}
\item sviluppare un motore real-time che si occupa di trovare il migliore percorso da far percorrere alla singola unità;
\item implementare un visualizzatore/monitor real-time che deve fornire la mappa della struttura con le entità attive e gestire la comunicazione delle varie unità;
\item i simulatori delle unità devono ricevere dal server centrale le direzioni su dove spostarsi e informare il server su possibili ostacoli.
  \end{itemize}

  \subsection{Tecnologie coinvolte}
Il capitolato offre libera scelta su come implementare l'applicazione, il gruppo aveva pensato di usare le seguenti tecnologie:
\begin{itemize}
\item motore real-time: si pensava di usare Java, per la facilità di gestione degli oggetti e dei thread;
\item monitor real-time: utilizzo di tomcat e angular per l'interfaccia utente;
\item simulatori delle unità opertive: si ipotizzava di optare per l'uso di Java così da facilitare la connessione, attraverso socket, al motore real-time.
\end{itemize}

  \subsection{Aspetti positivi}
\begin{itemize}
     \item il capitolato è sembrato molto interessante perchè offre un buon punto di partenza per entrare nel mondo della guida autonoma.
\end{itemize}

  \subsection{Criticità e fattori di rischio}
\begin{itemize}
    \item il gruppo ha scarse conoscenze riguardo le tecnologie Docker;
    %\item si ritiene difficile realizzare un prodotto veramente efficace;
    \item la quantità di studio nel rendere efficace l'invio dei dati e su come temporizzare l'acquisizione dei dati è molto elevata 
    \item per tutti i membri del gruppo, sia il documento del capitolato che la successiva presentazione sono risultati dispersivi e poco chiari. 
\end{itemize}

  \subsection{Conclusioni}
Dopo un'attenta valutazione, il team di lavoro ha valutato la complessità dell'intero progetto come troppo elevata e per questo motivo si è deciso di escludere il capitolato e di orientarsi verso alternative più stimolanti.
