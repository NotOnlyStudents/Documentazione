\section{Valutazione del capitolato scelto: C2 - Emporio$\lambda$ambda}
\subsection{Informazioni sul capitolato}
\begin{itemize}
	\item \textbf{Nome:} \textit{Emporio$\lambda$ambda}
	\item \textbf{Proponente:} \textit{RedBabel}
	\item \textbf{Committenti:} \textit{\VT{} e \CR{}}
\end{itemize}

\subsection{Descrizione del capitolato}
Il capitolato richiede lo sviluppo di una piattaforma di e-commerce generico.

\subsection{Scopo del capitolato}
Lo scopo del capitolato consiste nello sviluppo di una piattaforma e-commerce con architettura basata su micro-servizi e serverless.
La piattaforma consisterà in un sito web che dovrà essere composto dalle seguenti parti:
\begin{itemize}
	\item home page;
	\item pagine di elenco dei prodotti;
	\item pagine di descrizione del prodotto;
	\item carrello della spesa;
	\item pagina di checkout;
	\item pagina di account.
\end{itemize}

Dovrà essere in grado di supportare i seguenti ruoli:
\begin{itemize}
	\item \textbf{Amministratore}, l'utente con questo ruolo può:
		\begin{itemize}
			\item distribuire l'applicazione nel cloud;
			\item gestire la configurazione delle integrazioni di terze parti.
		\end{itemize}
	\item \textbf{Commerciante}, l'utente con questo ruolo può:
		\begin{itemize}
			\item avere una panoramica di tutti gli ordini;
			\item gestire totalmente i vari prodotti da vendere.
		\end{itemize}
	\item \textbf{Cliente}, l'utente con questo ruolo può:
	\begin{itemize}
		\item cercare, filtrare e aggiungere al carrello come ospite o utente;
		\item se connesso può aggiornare le informazioni del suo profilo;
		\item eliminare l'account creato;
		\item procedere con il pagamento dei prodotti selezionati solo se ha fatto il login.
	\end{itemize}
\end{itemize}

\subsection{Tecnologie coinvolte}
L'azienda consiglia di utilizzare:
\begin{itemize}
	\item typescript con approccio async-await, come linguaggio principale sia per il Front-end e sia per il Back-end;
	\item typescript-eslint per l'analisi statica del codice;
	\item AWS Lambda come tecnologia serverless con typescript;
	\item Amazon CloudWatch come servizio di monitoraggio, anche se non è obbligatorio;
	\item stripe come fornitore del servizio di pagamento, anche se non è obbligatorio.
\end{itemize}

\subsection{Vincoli}
Il progetto deve operare in 3 ambienti: Locale, Test e Staging.
L'ambiente di Staging deve essere accessibile pubblicamente.

Per raggiungere gli obiettivi minimi del progetto viene richiesto che:
\begin{itemize}
	\item il front-end sia implementato in Next.js;
	\item il back-end sia implementato in Serverless usando Typescript;
	\item l'integrazione di servizi di terze parti deve essere implementata con tecnologia Serveless usando Typescript;
	\item ci sia l'integrazione con un fornitore di servizi di pagamento.
\end{itemize}

\subsection{Aspetti positivi}
\begin{itemize}
	\item 3 componenti del gruppo hanno già sviluppato e conoscono javascript.
	\item Le tecnologie proposte stimolano i componenti del gruppo ad imparare ed a lavorare con esse.
	\item Le tecnologie proposte stanno sempre di più prendendo spazio nell'ambito del lavoro, specialmente lo sviluppo serverless e lo sviluppo WEB con typescript.
	\item L'uso di un'architettura a micro-servizi e serverless favorisce la parallelizzazione dello sviluppo del back-end.
	\item L'azienda proponente è molto orientata al dialogo e alla comunicazione.
\end{itemize}

\subsection{Aspetti critici}
\begin{itemize}
	\item Nessuno dei componenti nel gruppo ha mai utilizzato AWS.
	\item Nessuno dei componenti nel gruppo ha mai sviluppato un back-end serverless.
\end{itemize}

\subsection{Conclusioni}
Il seguente capitolato è stato accolto da tutti i componenti con grande entusiasmo. Le tecnologie proposte, il tipo di progetto e la disponibilità comunicativa data dall'azienda ha generato molto interesse e voglia di lavorare. Per questi motivi è stato approvato come prima scelta.