\section{Capitolato C2}
\subsection{Titolo del capitolato}
Il capitolato in questione si chiama \textit{"EmporioLambda: piattaforma di e-commerce in stile Serverless"}, il proponente \`e l'azienda \textit{RedBabel} e i committenti sono \VT{} e \CR{}.

\subsection{Descrizione del capitolo}
Il capitolato ha come obiettivo finale la realizzazione di una generica piattaforma e-commerce 

\subsection{Prerequisiti e tecnologie coinvolte}
La piattaforma dovr\'a essere sviluppata utilizzando la tecnologia \glo{AWS}.
Per il Back-end e il Front-end viene richiesto l'uso di \glo{Next.js}.

\subsection{Vincoli}
Il progetto deve operare in 3 ambienti: Locale, Test e Staging.
L'ambiente di Staging deve essere accessibile pubblicamente.

Per raggiungere gli obiettivi minimi del progetto viene richiesto che:

\begin{itemize}
\item il Front-end sia implementato in Next.js;
\item il Back-end sia implementato in Serverless usando Typescript;
\item \glo{EML-I} sia implementato in Serveless usando Typescript;
\item \glo{EML-MON} sia implementato usando Amazon CloudWatch.
\end{itemize}

\subsection{Aspetti positivi}
\begin{itemize}
\item Componenti del gruppo che esprimono Preferenza per questo capitolato: 5/7.
\item Componenti del gruppo che hanno già sviluppato usando JavaScript: 3/7.
\end{itemize}

\subsection{Aspetti critici}
\begin{itemize}
\item Componenti del gruppo che hanno già sviluppato con AWS: 0/7.
\item Componenti del gruppo che hanno già sviluppato un Back-End Serverless: 0/7.
\end{itemize}

\subsection{Conclusioni}
