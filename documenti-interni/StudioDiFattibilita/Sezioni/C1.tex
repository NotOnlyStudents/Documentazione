\section{Capitolato C1 - BlockCOVID}
\subsection{Informazioni sul capitolato}
\begin{itemize}
	\item \textbf{Nome:} \textit{BlockCOVID: supporto digitale al contrasto della pandemia}
	\item \textbf{Proponente:} \textit{Imola informatica}
	\item \textbf{Committenti:} \textit{\VT{} e \CR{}}
\end{itemize}

\subsection{Descrizione del capitolato}
Il capitolato richiede lo sviluppo di un'applicazione, con relativo server, in grado di tracciare, in modo immutabile, certificato e in tempo reale, le presenze alle postazioni di lavoro di una stanza, lo stato di pulizia di esse, la prenotazione di una postazione da remoto e la consultazione dei locali utilizzati dall'ultima igienizzazione.

\subsection{Scopo del capitolato}
La scopo del capitolato consiste nello sviluppo di un'applicazione (android o IOS) che permetterà le seguenti operazioni:
\begin{itemize}
	\item recupero lista delle postazioni libere;
	\item prenotazione di una postazione;
	\item tracciamento in tempo reale tramite la scanerizzazione del tag RFID relativo ad una specifica postazione;
	\item segnalazione della pulizia autonoma di una postazione;
	\item storico delle postazioni occupate;
	\item storico delle postazioni igienizzate;
\end{itemize}

Il server, dotato di UI, dovrà ricevere i vari tag RFID dall'applicazione e permetterà di:
\begin{itemize}
	\item sapere in ogni momento se la postazione \'e occupata, prenotata oppure da pulire;
	\item controllare quali postazioni sono prenotate (da chi) e bloccare le prenotazioni per una determinata stanza;
	\item mostrare una vista a calendario, con l’elenco delle postazioni prenotate e in quali giorni;
	\item salvare su memoria immutabile e certificata tutti i cambiamenti di stato relativi alla pulizia della postazione, nonch\'e le informazioni su chi l'ha igienizzato;
	\item prenotare una postazione con granularit\'a di 1 ora;
	\item creare, modificare o eliminare le utente ai dipendenti e agli addetti delle pulizie;
	\item creare, modificare o eliminare la definizione delle stanze e delle postazioni.
\end{itemize}

\subsection{Tecnologie coinvolte}
L'azienda consiglia di utilizzare:
\begin{itemize}
\item Java (versione 8 o superiori), Python o nodejs per lo sviluppo del server back-end;
\item protocolli asincroni per le comunicazioni app mobile-server;
\item sistema blockchain per salvare con opponibilit\'a a terzi i dati di sanificazione;
\item IAAS Kubernetes o di PAAS, Openshift o Rancher, per il rilascio delle componenti del server e la gestione della scalabilit\'a orizzontale.
\end{itemize}

\subsection{Vincoli}
Per raggiungere gli obiettivi minimi del progetto viene richiesto di:
\begin{itemize}
\item avere il server che esponga, in aggiunta a eventuali altri protocolli per l’interazione con il servizio specifico, delle API REST attraverso le quali sia possibile utilizzare l'applicativo. In alternativa \'e possibile utilizzare gRPC come soluzione alternativa a REST;
\item scansione dei codici nel tempo sufficiente a certificare la presenza della persona in postazione;
\item avere le componenti applicative correlate da test unitari e d’integrazione. Inoltre, \'e richiesto che il sistema venga testato nella sua interezza tramite test end-to-end e che la copertura dei test unitari sia $\geq$ 80\%.
\end{itemize}

\subsection{Aspetti positivi}
\begin{itemize} 
	\item 4 componenti del gruppo hanno già sviluppato e conoscono Java.
	\item Le tecnologie proposte dal capitolato stimolano i componenti del gruppo ad imparare ed a lavorare con esse.
	\item Il capitolato può essere molto utile ed interessante per una prospettiva lavorativa, data la complessità, le tecnologie e il tipo di progetto.
\end{itemize}

\subsection{Aspetti critici}
\begin{itemize}
\item \'E richiesto sia lo sviluppo di un'interfaccia WEB per il server e sia lo sviluppo di un'applicazione mobile per i vari tipo di utenti.
\end{itemize}

\subsection{Conclusioni}
Dopo un'attenta valutazione e discussione, il gruppo era molto propenso verso la presa in considerazione di questo capitolato come prima scelta, ma data la complessità e la competizione degli altri gruppi si è preferito di non sceglierlo.
