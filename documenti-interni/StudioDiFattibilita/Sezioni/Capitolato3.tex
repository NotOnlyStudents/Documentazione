\section{Capitolato C3}
\subsection{Informazioni sul capitolato}
\begin{itemize}
\item Nome: $GDP$: $Gathering$ $Detection$ $Platform$ 
\item Proponente: SyncLab
\item Committenti: Proff. Tullio e Cardin
\end{itemize}

\subsection{Descrizione del capitolo}
Il capitolato richiede lo sviluppo di un prototipo software in grado di acquisire, monitorare, utilizzare e correlare tra loro tutti i dati e le informazioni generate dai sistemi e dai dispositivi installati ed operativi in specifiche zone, con l’intento di identificare i possibili eventi che concorrono all’insorgere di variazioni di flussi di utenti.

\subsection{Tecnologie coinvolte}
L'azienda consiglia l'utilizzo delle seguenti tecnologie:
\begin{itemize}
\item Java e Angular per lo sviluppo delle del Back-end e del Front-end della componente Web Application del sistema;
\item il framework Leaflet per la gestione delle mappe;
\item utilizzo di asincroni per le comunicazioni tra le diverse componenti;
\item utilizzo del pattern Publisher/Subscriber, e adozione del protocollo MQTT. 
\end{itemize}

\subsection{Vincoli}
Per raggiungere gli obiettivi minimi del progetto viene richiesto di:

\begin{itemize}
\item avere il server, completo di UI, in grado di soddisfare i requisiti espressi; 
\item test che dimostrino il corretto funzionamento dei servizi e delle funzionalit\'a previste (copertura di test $\geq$ 80\% correlata di report);
\item presentare la documentazione su scelte implementative e progettuali effettuate con relative motivazioni e documentazione su problemi aperti e eventuali soluzioni proposte.
\end{itemize}

\subsection{Aspetti positivi}
\begin{itemize}
\item L'azienda, per il progetto, mette a disposizione figure di diverso livello in modo da supportare al meglio tutte le esigenze degli studenti. Inoltre in caso di bisogno, server nei quali gli studenti potranno effettuare le installazioni dei componenti applicativi sviluppati.

\item Componenti del gruppo che esprimono Preferenza per questo capitolato: 4/7.
\item Componenti del gruppo che hanno già sviluppato usando Java: 3/7.
\end{itemize}

\subsection{Aspetti critici}
La raccolta dei dati necessari per sviluppare il progetto pu\'o essere ritenuto uno dei principali punti critici del progetto:
\begin{itemize}
\item Componenti del gruppo che hanno gi\'a sviluppato usando Angular: 1/7.
\item Componenti del gruppo che hanno gi\'a utilizzato il FrameWork Leaflet: 0/7.
\end{itemize}

\subsection{Conclusioni}
Il capitolato è stato escluso in quanto non sufficientemente aprezzato dai membri del gruppo e per la difficoltà nella raccolta dei dati.