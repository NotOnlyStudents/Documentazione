\section{Capitolato C4}
\subsection{Titolo del capitolato}
Il capitolato in questione si chiama \textit{"HD Viz: visualizzazione di dati multidimensionali"}, il proponente \`e l'azienda \textit{Zucchetti} e i committenti sono \VT{} e \CR{}.

\subsection{Descrizione del capitolo}
Il capitolato ha come obiettivo finale la realizzazione di un'applicazione per visualizzare dati con molte dimensioni a supporto della fase esplorativa dell'analisi dei dati utilizzando a supporto la libreria D3.js.

\subsection{Prerequisiti e tecnologie coinvolte}
Lo sviluppo sarà prevalentemente in tecnologia HTML/CSS/JavaScript utilizzando la libreria D3.js.
La parte server di supporto alla presentazione nel browser e alle query ad un database SQL o NoSQL potr\'a essere sviluppata in Java con server Tomcat o in Javascript con server Node.js.

\subsection{Vincoli}
Per raggiungere gli obiettivi minimi del progetto viene richiesto:

\begin{itemize}
\item I dati devono poter essere forniti al sistema di visualizzazione sia con query ad un database che da file in formato CSV preparati precedentemente;
\item dovr\'a presentare almeno le seguenti visualizzazioni:
\begin{itemize}
\item jb
\end{itemize}
\item Ordinare i punti nel grafico “Heat map” per evidenziare i “cluster” presenti nei dati.
\end{itemize}

\subsection{Aspetti positivi}
\begin{itemize}
\item Componenti del gruppo che esprimono Preferenza per questo capitolato: 5/7.
\item Componenti del gruppo che hanno già sviluppato usando JavaScript: 7/7.
\item Componenti del gruppo che hanno già sviluppato usando HTML: 7/7.
\item Componenti del gruppo che hanno già sviluppato usando CSS: 7/7.
\end{itemize}

\subsection{Aspetti critici}
\begin{itemize}
\item Componenti del gruppo che conoscono la libreria D3.js 0/7
\end{itemize}

\subsection{Conclusioni}