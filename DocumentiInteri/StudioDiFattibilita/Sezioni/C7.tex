\section{Capitolato C7 - SSD}
\subsection{Informazioni sul capitolato}
\begin{itemize}
	\item \textbf{Nome:} \textit{SSD: Soluzioni di Sincronizzazione Desktop}
	\item \textbf{Proponente:} \textit{Zextras}
	\item \textbf{Committenti:} \textit{\VT{} e \CR{}}
\end{itemize}

\subsection{Descrizione del capitolato}
Il capitolato richiede la realizzazione di un algoritmo solido ed efficiente in grado di garantire il salvataggio in cloud del lavoro e contemporaneamente la sincronizzazione
dei cambiamenti avvenuti.

\subsection{Scopo del capitolato}
Lo scopo del capitolato consiste nello sviluppo di un'applicazione desktop che soddisfi i seguenti obiettivi:
\begin{itemize}
	\item sviluppo di un algoritmo solido ed efficiente: deve essere in grado di garantire il salvataggio in cloud del lavoro e contemporaneamente la sincronizzazione dei cambiamenti presenti in cloud;
	\item sviluppo di un’interfaccia multipiattaforma: deve fornire l'interfaccia per interagire con le funzionalità offerte dall'algoritmo di sincronizzazione. Inoltre, deve essere utilizzabile sui tre sistemi operativi maggiormente diffusi, ovvero Linux, MacOS, Windows; la soluzione sviluppata non deve necessitare dell'installazione, da parte dell'utente, di framework di terze parti per il proprio funzionamento;
	\item integrazione con Zextras Drive: l'algoritmo di sincronizzazione e l'interfaccia devono essere integrati con il servizio di gestione file Zextras Drive.
\end{itemize}
Altre funzionalità che dovrebbero essere aggiunte sono:
\begin{itemize}
	\item configurazione e autenticazione dell'utente;
	\item gestione dei file da sincronizzare o ignorare sia dal lato cloud che dal lato desktop e la possibilità di modificare in ogni momento la configurazione;
	\item sincronizzazione costante dei cambiamenti, sia locali che remoti;
	\item sistema di notifica dei cambiamenti;
	\item funzionalità presenti in altre soluzioni attualmente disponibili:
	\begin{itemize}
		\item gestione delle condivisioni;
		\item integrazione con il prodotto web.
	\end{itemize}
\end{itemize}

\subsection{Tecnologie coinvolte}
Le seguenti tecnologie possono essere impiegate nello sviluppo del progetto:
\begin{itemize}
	\item Framework Qt, consigliato per lo sviluppo dell'interfaccia grafica;
	\item Python, consigliato come linguaggio per lo sviluppo della Business Logic.
\end{itemize}

\subsection{Vincoli}
Per raggiungere gli obiettivi minimi del progetto viene richiesto che:
\begin{itemize}
	\item l'algoritmo di sincronizzazione e l'interfaccia utente devono essere cross-platform senza l'installazione manuale di ulteriori prodotti;
	\item forte distinzione tra Interfaccia utente e
	Business Logic;
	\item l'applicazione desktop deve essere sviluppata seguendo il pattern MVC.
\end{itemize}

\subsection{Aspetti positivi}
\begin{itemize}
	\item Le tecnologie proposte dal capitolato suscitano interesse nel gruppo.
	\item Permette di approfondire gli algoritmi per la collaborazione a distanza, i quali sono molto attuali e sempre più diffusi.
	\item Il team ha già familiarità con il framework Qt ed il linguaggio Python.
\end{itemize}

\subsection{Aspetti critici}
\begin{itemize}
	\item Si ha la paura di incontrare tantissimi problemi durante lo sviluppo cross-platform.
\end{itemize}

\subsection{Conclusioni}
Dopo un'attenta valutazione e discussione, il progetto ha entusiasmato l'intero gruppo, ma non in maniera tale da porsi come prima scelta.